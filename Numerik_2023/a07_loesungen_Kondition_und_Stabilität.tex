\documentclass[
				a4paper,
				10pt
			]
			{scrartcl}

\parindent0mm

\usepackage{amsfonts}
\usepackage{amsmath}
\usepackage{amssymb}
\usepackage{amsthm}
\usepackage[ngerman]{babel}
\usepackage{graphicx}
\usepackage{xcolor}

\usepackage[
			pdftex,
			colorlinks,
			breaklinks,
			linkcolor=blue,
			citecolor=blue,
			filecolor=black,
			menucolor=black,
			urlcolor=black,
			pdfauthor={Andreas Weber},
			pdftitle={Aufgaben zu Analysis und Lineare Algebra},
			plainpages=false,
			pdfpagelabels,
			bookmarksnumbered=true
		   ]{hyperref}


%--------------------------------------%
% Math ------------------------------%
%--------------------------------------%

% Mengen (Zahlen)
\newcommand{\N}{\mathbb{N}}
\newcommand{\Q}{\mathbb{Q}}
\newcommand{\R}{\mathbb{R}}
\newcommand{\Z}{\mathbb{Z}}
\newcommand{\C}{\mathbb{C}}

% Mengen (allgemein)
\newcommand{\K}{\mathbb{K}}
\newcommand\PX{{\cal P}(X)}

% Zahlentheorie
\newcommand{\ggT}{\mathrm{ggT}}


% Ableitungen
\newcommand{\ddx}{\frac{d}{dx}}
\newcommand{\pddx}{\frac{\partial}{\partial x}}
\newcommand{\pddy}{\frac{\partial}{\partial y}}
\newcommand{\grad}{\text{grad}}

%--------------------------------------%
% Layout Colors ------------------%
%--------------------------------------%
\newcommand*{\highlightDef}[1]{{\color{lightBlue}#1}}
\newcommand*{\highlight}[1]{{\color{lightBlue}#1}}
% Color Definitions
\definecolor{dhbwRed}{RGB}{226,0,26} 
\definecolor{dhbwGray}{RGB}{61,77,77}
\definecolor{lightBlue}{RGB}{28,134,230}

%
\addtokomafont{section}{\color{dhbwGray}}
\addtokomafont{subsection}{\color{dhbwGray}}


%-------------------------------------------------------------------
\begin{document}

\vspace*{-20mm}
{
	%\usekomafont{title}
	\color{dhbwGray}
	Dr. Moritz Gruber	\hfill Version \today\\
	DHBW Karlsruhe\\
}

\vspace{10mm}
\begin{center}
	{
		\usekomafont{title}
		%\color{lightBlue}
		{ \LARGE 	Übungsaufgaben 7 - Lösungen}\\[3mm]
		{\Large Kondition und Stabilität}
	}
\end{center}

\vspace{5mm}

%\section{Finite Differenzenmethode I}
%%%%
%Bestimmen Sie durch die Finite Differenzen Methode ein approximative Lösung Randwertproblems
%$$
%\begin{cases} \frac{\partial}{\partial x} u(x,y)+\frac{\partial}{\partial y}u(x,y)= 2x+2y & \text{ auf } \Omega=[0,1]^2\\ \hspace{18mm}u_{|\partial \Omega}(x,y)=x^2+y^2 \end{cases}
%$$
%auf dem äquidistanten Gitter mit $\Delta x = \Delta y = \frac{1}{3}$.
%
%\subsection*{Lösung:}
%Die Diskretisierung von $\Omega$ mit $h=\Delta x = \Delta y = \frac{1}{3}$ ergibt die Gitterpunkte $(x_i,y_j)$ für $ i,j \in \{1,2,3,4\}$ mit 
%\begin{align*}
%x_1=y_1=0, \quad x_2=y_2=\frac{1}{3}, \quad x_3=y_3=\frac{2}{3} \quad \text{und } x_4=y_4=1
%\end{align*}
%Damit ergeben sich mit dem zentralen Differenzenquotienten als Approximation der ersten Ableitung die vier Gleichungen
%\begin{align*}
%\frac{4}{3}=2\frac{1}{3}+2\frac{1}{3}=2x_2+2y_2&= \frac{\partial}{\partial x} u(x_2,y_2)+\frac{\partial}{\partial y}u(x_2,y_2)\\
%&=\frac{u(x_3,y_2)-u(x_1,y_2)}{\frac{2}{3}}+\frac{u(x_2,y_3)-u(x_2,y_1)}{\frac{2}{3}}\\
%&=\frac{3}{2}\left(u(x_3,y_2)-u(0,y_2) +u(x_2,y_3) -u(x_2,0)\right)\\
%\end{align*}
%\begin{align*}
%2=2\frac{2}{3}+2\frac{1}{3}=2x_3+2y_2&= \frac{\partial}{\partial x} u(x_3,y_2)+\frac{\partial}{\partial y}u(x_3,y_2)\\
%&=\frac{u(x_4,y_2)-u(x_2,y_2)}{\frac{2}{3}}+\frac{u(x_3,y_3)-u(x_3,y_1)}{\frac{2}{3}}\\
%&=\frac{3}{2}\left(u(1,y_2)-u(x_2,y_2) +u(x_3,y_3) -u(x_3,0)\right)\\
%\end{align*}
%\begin{align*}
%2=2\frac{1}{3}+2\frac{2}{3}=2x_2+2y_3&= \frac{\partial}{\partial x} u(x_2,y_3)+\frac{\partial}{\partial y}u(x_2,y_3)\\
%&=\frac{u(x_3,y_3)-u(x_1,y_3)}{\frac{2}{3}}+\frac{u(x_2,y_4)-u(x_2,y_2)}{\frac{2}{3}}\\
%&=\frac{3}{2}\left(u(x_3,y_3)-u(0,y_3) +u(x_2,1) -u(x_2,y_2)\right)\\
%\end{align*}
%\begin{align*}
%\frac{8}{3}=2\frac{2}{3}+2\frac{2}{3}=2x_3+2y_3&= \frac{\partial}{\partial x} u(x_3,y_3)+\frac{\partial}{\partial y}u(x_3,y_3)\\
%&=\frac{u(x_4,y_3)-u(x_2,y_3)}{\frac{2}{3}}+\frac{u(x_3,y_4)-u(x_3,y_2)}{\frac{2}{3}}\\
%&=\frac{3}{2}\left(u(1,y_3)-u(x_2,y_3) +u(x_3,1) -u(x_3,y_2)\right)
%\end{align*}
%Da die Randbedingung die Werte für $u(0,y), u(x,0), u(1,y)$ und $u(x,1)$ vorgibt, erhalten wir
%\begin{align*}
%u(x_3,y_2)+u(x_2,y_3)&=\frac{8}{9}+y_2^2+x_2^2=\frac{10}{9}\\
%-u(x_2,y_2)+u(x_3,y_3)&=\frac{4}{3}+x_3^2-y_2^2-1=\frac{2}{3}\\
%-u(x_2,y_2)+u(x_3,y_3)&=\frac{4}{3}+y_3^2-x_2^2-1=\frac{2}{3}\\\
%-u(x_3,y_2)-u(x_2,y_3)&=\frac{16}{9}-y_3^2-1-x_3^2-1=-\frac{10}{9}
%\end{align*}
%Die erste und die vierte so wie die zweite und die dritte Gleichung liefern jeweils die gleiche Information, sodass wir nur zwei Gleichungen für 4 Unbekannte haben.\\
%Wir können allerdings die Symmetrie in $x$ und $y$ nutzen, d.h. wir ergänzen die Information
%$$
%u(x_3,y_2)=u(x_2,y_3) 
%$$
%und erhalten damit
%$$
%u(x_3,y_2)=\frac{5}{9}=u(x_2,y_3) 
%$$
%Für die Werte $u(x_2,y_2)$ und $u(x_3,y_3)$ hilft uns das leider nicht weiter. Allerdings können wir mit der vom letzten Übungsblatt bekannten exakten Lösung $u(x,y)=x^2+y^2$ immerhin verifizieren, dass
%$$
%-u(x_2,y_2)+u(x_3,y_3)=-\left(\left(\frac{1}{3}\right)^2+\left(\frac{1}{3}\right)^2\right)+\left(\left(\frac{2}{3}\right)^2+\left(\frac{2}{3}\right)^2\right)=-\frac{2}{9}+\frac{8}{9}=\frac{2}{3}
%$$
%\newpage
%
%%-------------------------------------------------------------------
%\section{Finite Differenzenmethode II}
%%%%
%Diskretisieren Sie nach der Finite Differenzen Methode das Randwertproblem
%$$
%\begin{cases} \frac{\partial^2}{\partial x^2} u(x,y)-\frac{\partial^2}{\partial y^2}u(x,y)= 0 & \text{ auf } \Omega=[0,1]^2\\ \hspace{18mm}u_{|\partial \Omega}(x,y)=\exp(x+y) \end{cases}
%$$
%auf dem äquidistanten Gitter mit $\Delta x = \Delta y = \frac{1}{4}$ und bestimmen Sie das zugehörige LGS.
%
%\subsection*{Lösung:}
%Die Diskretisierung von $\Omega$ mit $h=\Delta x = \Delta y = \frac{1}{4}$ ergibt die Gitterpunkte $(x_i,y_j)$ für $ i,j \in \{1,2,3,4,5\}$ mit 
%\begin{align*}
%x_1=y_1=0, \quad x_2=y_2=\frac{1}{4}, \quad x_3=y_3=\frac{1}{2}, \quad x_4=y_4=\frac{3}{4} \quad \text{und } x_5=y_5=1
%\end{align*}
%Damit ergeben sich mit dem zentralen Differenzenquotienten als Approximation der zweiten Ableitung die Gleichungen
%\begin{align*}
%0&=\frac{\partial^2}{\partial x^2} u(x_i,y_j)-\frac{\partial^2}{\partial y^2}u(x_i,y_j)\\
%&=\frac{u(x_{i+1},y_j)-2u(x_i,y_j)+u(x_{i-1},y_j)}{\frac{1}{16}}-\frac{u(x_{i},y_{j+1})-2u(x_i,y_j)+u(x_{i},y_{j-1})}{\frac{1}{16}}\\
%&=16 \cdot \left(u(x_{i+1},y_j)-u(x_i,y_{j+1})+u(x_{i-1},y_j)-u(x_i,y_{j-1})  \right)
%\end{align*}
%Mit der Kurzschreibweise $u_{i,j}=u(x_i,y_j)$ und unter Berücksichtigung der Randbedingung ergeben sich so die folgenden 9 Gleichungen
%\begin{align*}
%&-u_{2,3}&&+u_{3,2}&&&&&&=\exp(\frac{1}{4})-\exp(\frac{1}{4})\\
%-u_{2,2}&&-u_{2,4}&&+u_{3,3}&&&&&=-\exp(\frac{1}{2})\\
%&-u_{2,3}&&&&+u_{3,4}&&&&=\exp(\frac{5}{4})-\exp(\frac{3}{4})\\
%%
%u_{2,2}&&&&-u_{3,3}&&+u_{4,2}&&&=\exp(\frac{1}{2})\\
%&\hspace{4.5mm}u_{2,3}&&-u_{3,2}&&-u_{3,4}&&+u_{4,3}&&=0\\
%&&u_{2,4}&&-u_{3,3}&&&&+u_{4,4}&=\exp(\frac{3}{2})\\
%%
%&&&\hspace{4.5mm}u_{3,2}&&&&-u_{4,3}&&=-\exp(\frac{5}{4})+\exp(\frac{3}{4})\\
%&&&&u_{3,3}&&-u_{4,2}&&-u_{4,4}&=-\exp(\frac{3}{2})\\
%&&&&&\hspace{4.5mm}u_{3,4}&&-u_{4,3}&&=\exp(\frac{7}{4})-\exp(\frac{7}{4})
%\end{align*}
%Verwendet man nun noch die Nummerierung $u_k=u_{i,j}$ mit $k=(i-2)\cdot 3+(j-1)$ ergibt sich so das folgende LGS in Matrixschreibweise
%\begin{align*}
%\left(\begin{array}{rrrrrrrrr}
%0&-1&0&1&0&0&0&0&0\\
%-1&0&-1&0&1&0&0&0&0\\
%0&-1&0&0&0&1&0&0&0\\
%1&0&0&0&-1&0&1&0&0\\
%0&1&0&-1&0&-1&0&1&0\\
%0&0&1&0&-1&0&0&0&1\\
%0&0&0&1&0&0&0&-1&0\\
%0&0&0&0&1&0&-1&0&-1\\
%0&0&0&0&0&1&0&-1&0
%\end{array}\right)
%\cdot
%\begin{pmatrix}
%u_1\\u_2\\u_3\\u_4\\u_5\\u_6\\u_7\\u_8\\u_9
%\end{pmatrix}
%=
%\left(\begin{array}{rl}
%0\\
%-\exp(\frac{1}{2})\\
%\exp(\frac{5}{4})&-\exp(\frac{3}{4})\\
%\exp(\frac{1}{2})\\
%0\\
%\exp(\frac{3}{2})\\
%-\exp(\frac{5}{4})&+\exp(\frac{3}{4})\\
%-\exp(\frac{3}{2})\\
%0
%\end{array}\right)
%\end{align*}
%\newpage

%-------------------------------------------------------------------
\section{Kondition}
%%%
Berechnen Sie die Kondition des Problem
$$
y=4x^3+x^2-10x+7
$$
und bestimmen Sie eine Stelle $x_g$ an der das Problem gut konditioniert ist und eine Stelle $x_s$ an der es schlecht konidtioniert ist.

\subsection*{Lösung:}
Per Definition ist die Kondition eines (differenzierbaren) Problems $f$ gegeben durch
$$
kond_x(f)=\left|\frac{xf'(x)}{f(x)}\right|
$$
Somit ergibt sich für das obige Problem
$$
kond_x(f)=\left|\frac{xf'(x)}{f(x)}\right|=\left|\frac{x(12x^2+2x-10)}{4x^3+x^2-10x+7}\right|=\left|\frac{12x^3+2x^2-10x)}{4x^3+x^2-10x+7}\right|
$$
Für $x_g=-1$ gilt $kond_{x_g}(f)=0$ und das Problem ist dort somit sehr gut konditioniert.\\ Für $x_s=-2$ gilt $kond_{x_s}(f)=\left|\frac{12\cdot(-8)+2\cdot4-10\cdot (-2)}{4\cdot (-8)+4-10\cdot(-2)+7}\right|=\left|\frac{-96+8+20}{-32+4+20+7}\right|=68$ und das Problem ist dort somit eher schlecht konditioniert.\\
\begin{center}
\includegraphics[scale=0.9]{ProblemGraph.png}
\end{center}
\newpage

%-------------------------------------------------------------------
\section{Normen}
%%%
Zeigen Sie, dass die Zeilensummennorm 
$$
\|A\|_\infty = \max_i(\sum_j |a_{ij}|) \quad \text{für} \ A=(a_{ij}) \in \R^{n\times n}
$$
die von der Maximumsnorm
$$
\|v\|= \max_i(|v_i|) \quad \text{für} \ v=(v_i) \in \R^n
$$
induzierte Operatornorm ist.

\subsection*{Lösung:}
Wir müssen zeigen, dass $\|A\|_\infty=\max \limits_{\|v\|_\infty=1} \|Av\|_\infty$ gilt.\\ (Hierbei steht auf der rechten Seite die Zeilensummennorm und auf der linken Seite die Maximumsnorm.)\\

Sei $v \in \R^n$ mit $\|v\|_\infty =1$ und $A \in \R^{n\times n}$. Dann gilt
\begin{align*}
\|Av\|_\infty&=\max_{i}|(Av)_i|\\
&=\max_{i}|\sum_{j=1}^n a_{ij}v_j| \\
&\stackrel{1)}{\le} \max_{i}\sum_{j=1}^n |a_{ij}v_j| \\
&\le \max_{i} \sum_{j=1}^n |a_{ij}| \cdot \max_j|v_j|\\
&\stackrel{2)}{=}\max_{i} \sum_{j=1}^n |a_{ij}| \\
&=\|A\|_\infty
\end{align*}
wobei bei 1) die Dreiecksungleichung verwendet wurde und bei 2) die Voraussetzung $\|v\|_\infty=1$.\\
Damit gilt $\|A\|_\infty\ge \max \limits_{\|v\|_\infty=1} \|Av\|_\infty$.\\
Sei nun $k$ der Index der Zeile von $A$ mit der größten (absoluten) Zeilensumme, d.h. $\|A\|_\infty=\sum_j |a_{kj}|$ und $w \in \R^n$ mit $w_j=\text{sgn}(a_{kj}), \forall j$. Dann gilt $\|w\|_\infty=1$ und 
\begin{align*}
\max \limits_{\|v\|_\infty=1} \|Av\|_\infty &\ge \|Aw\|_\infty \\
&= \max_{i}|\sum_{j=1}^n a_{ij}w_j|\\
&=\max_{i}|\sum_{j=1}^n a_{ij}\cdot \text{sgn}(a_{kj})|\\
&\stackrel{3)}{=}\sum_j |a_{kj}|\\
&=\|A\|\infty
\end{align*}
wobei bei 3) benutzt wird, dass $k$ der Index der Zeile von $A$ mit der größten (absoluten) Zeilensumme ist.\\
Somit gilt auch $\|A\|_\infty\le \max \limits_{\|v\|_\infty=1} \|Av\|_\infty$. 
\hfill $\square$
\vspace{5mm}
\newpage
%-------------------------------------------------------------------
\section{Stabilität}
%%%
Bringen das homogene LGS mit Hilfe des Gaußalgorithmus
$$
\left(\begin{array}{rrrr} 1 & 0 & 0 & \ \ 1 \\ -1& 1 &0 &1 \\ -1 & -1 & 1 &1 \\ -1 & -1 &-1 &1 \end{array}\right) \cdot \begin{pmatrix}x_1 \\x_2\\x_3\\x_4 \end{pmatrix} = 0
$$
in Stufenform und bestimmen Sie nach jedem Umformungsschritt die Zeilensummennorm der Matrix.

\subsection*{Lösung:}
$$
A_0=\left(\begin{array}{rrrr} 1 & 0 & 0 & \ \ 1 \\ -1& 1 &0 &1 \\ -1 & -1 & 1 &1 \\ -1 & -1 &-1 &1 \end{array}\right)  \text{ mit } \|A_0\|_\infty = 4
$$
Addition der ersten Zeile auf die zweite, dritte und und vierte Zeile ergibt
$$
A_1=\left(\begin{array}{rrrr}  \ \ 1 & 0 & 0 & \ \ 1 \\ 0& 1 &0 &2 \\ 0 & -1 & 1 &2 \\ 0 & -1 &-1 &2 \end{array}\right)  \text{ mit } \|A_1\|_\infty = 4
$$
Addition der zweiten Zeile auf die dritte und vierte Zeile ergibt
$$
A_2=\left(\begin{array}{rrrr}  \ \ 1 & \ \ 0 & 0 & \ \ 1 \\ 0& 1 &0 &2 \\ 0 &0& 1 &4 \\ 0 &0 &-1 &4 \end{array}\right)   \text{ mit } \|A_2\|_\infty = 5
$$
Die Addition der dritten Zeile auf die vierte ergibt schließlich
$$
A_3=\left(\begin{array}{rrrr} 1 & 0 & 0 & 1 \\ 0& 1 &0 &2 \\ 0 &0& 1 &4 \\ 0 &0 &0 &8 \end{array}\right)   \text{ mit } \|A_3\|_\infty = 8
$$
\end{document}