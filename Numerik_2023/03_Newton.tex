\documentclass{beamer}

\usepackage{beamerthemesplit}

\usepackage{amsfonts}
\usepackage{amsmath}
\usepackage{amssymb}
\usepackage{amsthm}
\usepackage{amscd}

\usepackage{stmaryrd} 					%\lightning
\usepackage{algorithm2e}


\usepackage[ngerman]{babel}

\usepackage[utf8]{inputenc}
\usepackage[T1]{fontenc}
\usepackage{textcomp}


% Color Definitions
\definecolor{dhbwRed}{RGB}{226,0,26} 
\definecolor{dhbwGray}{RGB}{61,77,77}
\definecolor{lightBlue}{RGB}{28,134,230}

% Basic Theme
\usetheme{Malmoe}

% Color Re-Definitions
\usecolortheme[named=lightBlue]{structure}
\setbeamercolor*{alerted  text}{fg=dhbwRed, bg=white}
\setbeamercolor*{subsection in toc}{fg=dhbwGray, bg=white}

%\setbeamercolor*{palette primary}{fg=white,bg=lightBlue}
%\setbeamercolor*{palette secondary}{fg=white,bg=gray}
%\setbeamercolor*{palette tertiary}{fg=white,bg=gray}
%\setbeamercolor*{palette quaternary}{fg=white,bg=dhbwRed}

% no navigation symbols
\setbeamertemplate{navigation symbols}{}

% headline, footline
\setbeamertemplate{footline}{\color{dhbwGray} \hfill\insertframenumber\hspace{5mm}\vspace{2mm}}
\setbeamertemplate{headline}{}

% Title Page
\newcommand*{\makeTitlePage}{
	
	\begin{frame}[plain]
		
		\vfill
		\vfill
		\begin{center}
			{
				\usebeamerfont{title}
				\usebeamercolor[fg]{title}
				\Large
				\inserttitle
			}\\[3mm]
			{	
				\usebeamerfont{subtitle}
				\usebeamercolor[fg]{subtitle}
				\large
				\insertsubtitle
			}
		\end{center}
		%
		\vfill
		\vfill
		\vfill
		\vfill
		%
		\begin{columns}
			\begin{column}{0.5\textwidth}
				\begin{flushleft}
					{
						\usebeamerfont{normal text}
						\color{dhbwGray!80}
						\scriptsize
						Dr. Moritz Gruber\\
						DHBW Karlsruhe\\
						
					}
				\end{flushleft}
			\end{column}
			%
			\begin{column}{0.5\textwidth}
				\begin{flushright}
					\includegraphics[scale=0.06]{../DHBW.png}
				\end{flushright}
			\end{column}
		\end{columns}
		%
		\vspace{1mm}
		\begin{columns}
			\begin{column}{0.5\textwidth}
				\begin{flushleft}
					{
						\usebeamerfont{normal text}
						\color{dhbwGray!80}
						\scriptsize
						Version \today
					}
				\end{flushleft}
			\end{column}
			%
			\begin{column}{0.5\textwidth}
				% nothing (just a placeholder to be in line with the columns above
			\end{column}
		\end{columns}
	\end{frame}

}

% Section Divider Page
\newcommand*{\makeSectionDividerPage}{

	\begin{frame}[plain]
		\begin{center}
			\begin{flushleft}
				{				
					\usebeamercolor[fg]{frametitle}
					{\Large \insertsection} \\[3mm]
					{\large \insertsubsection}
				}
			\end{flushleft}
		\end{center}
        \end{frame}
	
}

% itemize
\setbeamertemplate{itemize items}[circle]
\setbeamertemplate{enumerate item}{(\theenumi)}




%--------------------------------------%
% Math ------------------------------%
%--------------------------------------%

% Mengen (Zahlen)
\newcommand{\N}{\mathbb{N}}
\newcommand{\Q}{\mathbb{Q}}
\newcommand{\R}{\mathbb{R}}
\newcommand{\Z}{\mathbb{Z}}
\newcommand{\C}{\mathbb{C}}

% Mengen (allgemein)
\newcommand{\K}{\mathbb{K}}
\newcommand\PX{{\cal P}(X)}

% Zahlentheorie
\newcommand{\ggT}{\mathrm{ggT}}


% Ableitungen
\newcommand{\ddx}{\frac{d}{dx}}
\newcommand{\pddx}{\frac{\partial}{\partial x}}
\newcommand{\pddy}{\frac{\partial}{\partial y}}
\newcommand{\grad}{\text{grad}}

%--------------------------------------%
% Layout Colors ------------------%
%--------------------------------------%
\newcommand*{\highlightDef}[1]{{\color{lightBlue}#1}}
\newcommand*{\highlight}[1]{{\color{lightBlue}#1}} % after theme for colours

%---------------------------------------------%
\title{Numerische Mathematik}
\subtitle{Verfahren zur Bestimmung von Nullstellen}

%---------------------------------------------%
\begin{document}

%---------------------------------------------%
\makeTitlePage

%---------------------------------------------%
\begin{frame}\frametitle{Inhalt}
   \tableofcontents
\end{frame}
%

%---------------------------------------------%
% Folien -----------------------------------%
%---------------------------------------------%
%

%--------------------------------------------
\section{Motivation}
\makeSectionDividerPage
%%%
%
%
\begin{frame}\frametitle{Problemstellung}
Wann ist die nächste Sonnenfinsternis?\\\vspace{1mm}
\includegraphics[scale=0.35]{sonnenfinsternis.png}\\\vspace{1mm} \pause
Nicht-lineare Gleichungen sind in der Regel nur sehr schwer exakt zu lösen. Die numerische Approximation wird oft durch eine \highlightDef{Zielwertsuche} umgesetzt. Dies sind Iterationsverfahren, die in jedem Schritt versuchen den Eingabewert so zu verändern, dass der Ausgabewert näher am Zielwert liegt.

\end{frame}
%
%

%
%--------------------------------------------
\section{Das Newton-Verfahren}
\makeSectionDividerPage
%%%
\subsection{Algorithmus}
%
\begin{frame}\frametitle{Newton-Verfahren}
\begin{columns}[T]
\begin{column}{0.5 \textwidth}
Es sei $f: I \to \R$ eine im Intervall $I$ differenzierbare Funktion $f$. Dann ist das \highlightDef{Newton-Verfahren} das durch
$$
\begin{cases}
x_0 \in I\\
x_{k+1}=x_{k}-\frac{f(x_k)}{f'(x_k)}
\end{cases}
$$
definierte Iterationsverfahren.
\end{column}
\begin{column}{0.5 \textwidth}
\includegraphics[scale=0.5]{Newton.png}
\end{column}
\end{columns}
\end{frame}
%
%
\begin{frame}\frametitle{Beispiel}
Wir suchen eine Nullstelle von
$$
f: [0,5]\to \R, x \mapsto -x^2+5x-2
$$\pause
Um zwei Schritte des Newton-Verfahrens durchzuführen, verwenden wir den Startwert $x_0=1$ und berechnen die Ableitungsfunktion $f'(x)=-2x+5$. \pause Damit
\vspace{10mm}
\Large
\begin{itemize}
\item $x_1=x_0-\frac{f(x_0)}{f'(x_0)}=1-\frac{2}{3}=\frac{1}{3}$\\\quad\\\pause
\item $x_2=x_1-\frac{f(x_1)}{f'(x_1)}=\frac{1}{3}-\frac{-\frac{1}{9}+\frac{5}{3}-2}{-\frac{2}{3}+5}=\frac{1}{3}-\frac{-\frac{4}{9}}{\frac{13}{3}}=\frac{17}{39}$
\end{itemize}
\end{frame}
%
%
%--------------------------------------------
\subsection{Eigenschaften}
%
\begin{frame}\frametitle{Quadratische Konvergenz}
Es sei $f:I \to \R$ eine (beliebig oft) differenzierbare Funktion und $\bar x \in I$ eine einfache Nullstelle von $f$. Dann gilt in \textit{hinreichender} Nähe von $\bar x$:
$$
\exists L >0 \text{ mit } \quad |\bar x - x_{k+1}|\le L \cdot |\bar x - x_k|^2
$$ 
\quad\\\quad\\
(Das Newton-Verfahren ist somit lokal quadratische konvergent)
\end{frame}
%
%
\begin{frame}\frametitle{Beispiel}
$f: [0,1] \to \R, x \mapsto \cos(x\cdot \pi)$ mit Startwerten $x_0=\frac{1}{4}$ und $\tilde x_0=0$.\\\pause
$\Rightarrow \quad f'(x)=-\pi\sin(x) \text{ und } f''(x)=-\pi^2\cos(x)$\pause
\begin{align*}
x_1&=x_0-\frac{f(x_0)}{f'(x_0)}=\frac{1}{4}-\frac{\cos(\frac{\pi}{4})}{-\pi\sin(\frac{\pi}{4})}=\frac{1}{4}+\frac{1}{\pi}\\
&\\
\Rightarrow&\quad |\bar x - x_1|=\left|\frac{1}{2}-(\frac{1}{4}+\frac{1}{\pi})\right|\le \frac{1}{12}\le  \frac{\pi}{2}(\frac{1}{4})^2=\frac{\pi}{2}|\bar x - x_0|^2
\end{align*}\pause
Im Gegensatz dazu ist $\tilde x_0=0$ ein ungeeigneter Startwert, da $f'(0)=-\pi \sin(0)=0$ gilt und somit $\tilde x_1$ nicht definiert ist. Damit konvergiert auch das Newton-Verfahren für diesen Startwert nicht.
\end{frame}
%
\begin{frame}\frametitle{Nachteile des Newton-Verfahrens}
\begin{itemize}
\item Nur nahe einer (einfachen) Nullstelle schnell konvergent
\item Nahe lokalen Extremstellen sehr schlecht konvergent
\item bei $f'(x)=0$ nicht definiert
\item Ableitung wird benötigt und muss ausgewertet werden
\end{itemize}\pause
\vfill
Alternative Verfahren:
\begin{itemize}
\item Sekantenverfahren: langsamer, aber ohne Ableitung
\item Bisektionsverfahren: langssamer, ohne Ableitung, immer konvergent
\end{itemize}
\end{frame}
%
\section{Weitere Verfahren zur Nullstellenbestimmung}
\makeSectionDividerPage
%%%
%
\begin{frame}\frametitle{Sekantenverfahren}
\begin{columns}[T]
\begin{column}{0.5 \textwidth}
Es sei $f: I \to \R$ eine im Intervall $I$ stetige Funktion $f$. Dann ist das \highlightDef{Sekantenverfahren} das durch \vspace{2mm}
$$
\begin{cases}
x_0, x_1 \in I \text{ mit } x_0\ne x_1\\
x_{k+1}=x_{k-1}-\frac{f(x_{k-1})(x_k-x_{k-1})}{f(x_k)-f(x_{k-1})}
\end{cases}
$$
 \vspace{2mm}
definierte Iterationsverfahren.
\end{column}
\begin{column}{0.5 \textwidth}
\includegraphics[scale=0.4]{Sekante.png}
\end{column}
\end{columns}
\pause \vspace{10mm}
Die Geradengleichung der Sekante ist 
$$
g(x)=f(x_{k-1})+(f(x_k)-f(x_{k-1}))\cdot \frac{x-x_{k-1}}{x_k-x_{k-1}}
$$
\end{frame}
%
\begin{frame}\frametitle{Bisektionsverfahren}
Es sei $f: I \to \R$ eine im Intervall $I$ stetige Funktion $f$. Dann ist das \highlightDef{Bisektionsverfahren} das durch \vspace{2mm}
$$
\begin{cases}
a_0, b_0 \in I \text{ mit } sign(f(a_0)f(b_0))=-1\\
c_k=\frac{a_k+b_k}{2}\\
\begin{cases}
\text{wenn } sign(f(a_k)f(c_k))=1, \text{ dann } a_{k+1}=c_k \text{ und } b_{k+1}=b_k\\
\text{wenn } sign(f(b_k)f(c_k))=1, \text{ dann } b_{k+1}=c_k \text{ und } a_{k+1}=a_k\\
\end{cases}
\end{cases}
$$
 \vspace{2mm}
definierte Iterationsverfahren.
\end{frame}


\end{document}