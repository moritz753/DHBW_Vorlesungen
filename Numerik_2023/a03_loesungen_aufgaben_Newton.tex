\documentclass[
				a4paper,
				10pt
			]
			{scrartcl}

\parindent0mm

\usepackage{amsfonts}
\usepackage{amsmath}
\usepackage{amssymb}
\usepackage{amsthm}
\usepackage[ngerman]{babel}

\usepackage[utf8]{inputenc}
\usepackage[T1]{fontenc}
\usepackage{textcomp}

\usepackage{graphicx}
\usepackage{xcolor}

\usepackage[
			pdftex,
			colorlinks,
			breaklinks,
			linkcolor=blue,
			citecolor=blue,
			filecolor=black,
			menucolor=black,
			urlcolor=black,
			pdfauthor={Andreas Weber},
			pdftitle={Aufgaben zu Analysis und Lineare Algebra},
			plainpages=false,
			pdfpagelabels,
			bookmarksnumbered=true
		   ]{hyperref}


%--------------------------------------%
% Math ------------------------------%
%--------------------------------------%

% Mengen (Zahlen)
\newcommand{\N}{\mathbb{N}}
\newcommand{\Q}{\mathbb{Q}}
\newcommand{\R}{\mathbb{R}}
\newcommand{\Z}{\mathbb{Z}}
\newcommand{\C}{\mathbb{C}}

% Mengen (allgemein)
\newcommand{\K}{\mathbb{K}}
\newcommand\PX{{\cal P}(X)}

% Zahlentheorie
\newcommand{\ggT}{\mathrm{ggT}}


% Ableitungen
\newcommand{\ddx}{\frac{d}{dx}}
\newcommand{\pddx}{\frac{\partial}{\partial x}}
\newcommand{\pddy}{\frac{\partial}{\partial y}}
\newcommand{\grad}{\text{grad}}

%--------------------------------------%
% Layout Colors ------------------%
%--------------------------------------%
\newcommand*{\highlightDef}[1]{{\color{lightBlue}#1}}
\newcommand*{\highlight}[1]{{\color{lightBlue}#1}}
% Color Definitions
\definecolor{dhbwRed}{RGB}{226,0,26} 
\definecolor{dhbwGray}{RGB}{61,77,77}
\definecolor{lightBlue}{RGB}{28,134,230}
\usepackage{stix}

%-------------------------------------------------------------------
\begin{document}

\vspace*{-20mm}
{
	%\usekomafont{title}
	\color{dhbwGray}
	Dr. Moritz Gruber	\hfill Version \today\\
	DHBW Karlsruhe\\
}

\vspace{10mm}
\begin{center}
	{
		\usekomafont{title}
		%\color{lightBlue}
		{ \LARGE Lösungen Übungsaufgaben 3}\\[3mm]
		{\Large Verfahren zur Nullstellenbestimmung}
	}
\end{center}

\vspace{5mm}

%-------------------------------------------------------------------



%-------------------------------------------------------------------
\section{Newton-Verfahren}
%%%
Führen Sie vier Schritte des Newton-Verfahren für die Funktion
$$
f: \R \to \R, x \mapsto x^2-3x+3
$$
mit Startwert $x_0=2$ durch.
%-------------------------------------------------------------------
\subsection*{Lösung}
%%%
Als erstes bestimmen wir die Ableitungsfunktion $f'(x)=2x-3$. Damit:
\begin{align*}
x_1&=x_0-\frac{f(x_0)}{f'(x_0)}=2-\frac{1}{1}=1 \\
x_2&=x_1-\frac{f(x_1)}{f'(x_1)}=1-\frac{1}{-1}=2 \\
x_3&=x_2-\frac{f(x_2)}{f'(x_2)}=2-\frac{1}{1}=1 \\
x_4&=x_3-\frac{f(x_3)}{f'(x_3)}=1-\frac{1}{-1}=2 
\end{align*}
Man sieht hier, wie wichtig die Wahl des Startwertes für die Konvergenz des Verfahrens ist.
\newpage
%-------------------------------------------------------------------
\section{Vergleich der Verfahren}
%%%
Gegen ist die Funktion
$$
f: \R \to \R, x \mapsto x^3-x+1
$$
Führen Sie jeweils eines Schritt mit
\begin{itemize}
\item[a)] dem Newton-Verfahren und Startwert $x_0=-1$,
\item[b)] dem Sekantenverfahren und Startwerten $x_0=-1$, $x_1=-2$
\item[c)] dem Bisektionsverfahren und Startwerten $a_0=-2$, $b_0=-1$
\end{itemize}
durch und Vergleichen Sie die Ergebnisse untereinander so wie mit der nächsten exakten Nullstelle $\bar x \thickapprox -1.3247$ von $f$.
%-------------------------------------------------------------------
\subsection*{Lösung}
%%%
\begin{itemize}
\item[a)] Für das Newton-Verfahren bestimmen wir als erstes die Ableitung $f'(x)=3x^2-1$. Damit
$$
x_1=x_0-\frac{f(x_0)}{f'(x_0)}=-1-\frac{1}{2}=-\frac{3}{2}
$$
\item[b)] Die Sekantenverfahren ergibt
$$
x_2=x_0-\frac{f(x_0)\cdot (x_1-x_0)}{f(x_1)-f(x_0)}=-1-\frac{1\cdot (-2-(-1))}{-5-1}=-1-\frac{-1}{-6}=-\frac{7}{6}
$$
\item[c)] Für das Bisektionsverfahren prüfen wir als erstes die Startwerte:
$$
sign(f(a_0)f(b_0))=sign(f(-2)f(-1))=sign(-5 \cdot 1)=-1
$$
Dann berechnen wir $c_0=\frac{a_0+b_0}{2}=\frac{-2+(-1)}{2}=-\frac{3}{2}$.\\
Mit
$$
f(c_0)=\left(-\frac{3}{2}\right)^3-(-\frac{3}{2})+1=-\frac{27}{8}+\frac{12}{8}+\frac{8}{8}=-\frac{7}{8}
$$
sehen wir $sign(f(a_0)f(c_0))=1$ und somit
\begin{align*}
a_1&=c_0=-\frac{3}{2}\\
b_1&=b_0=-1\\
c_1&=\frac{a_1+b_1}{2}=\frac{-\frac{3}{2}+(-1)}{2}=-\frac{5}{4}
\end{align*}
\end{itemize}
Vergleichen wir nun die Approximationen der verschiedenen Verfahren nach einem Schritt, so ergibt sich
\begin{itemize}
\item Fehler des Newton-Verfahren: $|\bar x - (-\frac{3}{2})|\le 0.18 < 0.2$
\item Fehler des Sekantenverfahren: $|\bar x - (-\frac{7}{6})| \le 0.16 < 0.2$
\item Fehler des Bisektionsverfahren: $|\bar x - (-\frac{5}{4})| \le 0.08 < 0.1$
\end{itemize}
%
\end{document}