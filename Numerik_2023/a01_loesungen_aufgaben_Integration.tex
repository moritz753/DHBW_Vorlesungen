\documentclass[
				a4paper,
				10pt
			]
			{scrartcl}

\parindent0mm

\usepackage{amsfonts}
\usepackage{amsmath}
\usepackage{amssymb}
\usepackage{amsthm}
\usepackage[ngerman]{babel}
\usepackage{graphicx}
\usepackage{xcolor}

\usepackage[
			pdftex,
			colorlinks,
			breaklinks,
			linkcolor=blue,
			citecolor=blue,
			filecolor=black,
			menucolor=black,
			urlcolor=black,
			pdfauthor={Andreas Weber},
			pdftitle={Aufgaben zu Analysis und Lineare Algebra},
			plainpages=false,
			pdfpagelabels,
			bookmarksnumbered=true
		   ]{hyperref}


%--------------------------------------%
% Math ------------------------------%
%--------------------------------------%

% Mengen (Zahlen)
\newcommand{\N}{\mathbb{N}}
\newcommand{\Q}{\mathbb{Q}}
\newcommand{\R}{\mathbb{R}}
\newcommand{\Z}{\mathbb{Z}}
\newcommand{\C}{\mathbb{C}}

% Mengen (allgemein)
\newcommand{\K}{\mathbb{K}}
\newcommand\PX{{\cal P}(X)}

% Zahlentheorie
\newcommand{\ggT}{\mathrm{ggT}}


% Ableitungen
\newcommand{\ddx}{\frac{d}{dx}}
\newcommand{\pddx}{\frac{\partial}{\partial x}}
\newcommand{\pddy}{\frac{\partial}{\partial y}}
\newcommand{\grad}{\text{grad}}

%--------------------------------------%
% Layout Colors ------------------%
%--------------------------------------%
\newcommand*{\highlightDef}[1]{{\color{lightBlue}#1}}
\newcommand*{\highlight}[1]{{\color{lightBlue}#1}}
% Color Definitions
\definecolor{dhbwRed}{RGB}{226,0,26} 
\definecolor{dhbwGray}{RGB}{61,77,77}
\definecolor{lightBlue}{RGB}{28,134,230}

%
\addtokomafont{section}{\color{dhbwGray}}
\addtokomafont{subsection}{\color{dhbwGray}}
\usepackage{stix}

%-------------------------------------------------------------------
\begin{document}

\vspace*{-20mm}
{
	%\usekomafont{title}
	\color{dhbwGray}
	Dr. Moritz Gruber	\hfill Version \today\\
	DHBW Karlsruhe\\
}

\vspace{10mm}
\begin{center}
	{
		\usekomafont{title}
		%\color{lightBlue}
		{ \LARGE Lösungen Übungsaufgaben 1}\\[3mm]
		{\Large Integration}
	}
\end{center}

\vspace{5mm}

%-------------------------------------------------------------------



%-------------------------------------------------------------------
\section{Partielle Integration}
%%%
Bestimmen Sie mit Hilfe partieller Integration den Wert folgende Integrale:
\begin{itemize}
	\item[(i)] 
		$$
			\int_0^{\pi} e^{-x}\sin(x)\, dx
		$$
	\item[(ii)]
		$$
			\int_0^1 x^2e^x \, dx
		$$
\end{itemize}

%-------------------------------------------------------------------
\subsection*{Lösung}
%%%
\begin{itemize}
\item[(i)] Doppelte partielle Integration:
		\begin{align*}
			\int_0^{\pi} \underbrace{e^{-x}}_{f(x)} \underbrace{\sin(x)}_{G(x)}\, dx
				&=  [-e^{-x}\sin(x) ]_0^{\pi} - \int_0^{\pi}(-e^{-x})\cos(x)\, dx					\\
				&=  0 + \int_0^{\pi}\underbrace{e^{-x}}_{f(x)}\underbrace{\cos(x)}_{G(x)}\, dx	\\
				&= [-e^{-x}\cos(x)]_0^{\pi} - \int_0^{\pi} (-e^{-x})\cdot(-\sin(x))\, dx			\\
				&= (e^{-\pi} + 1) - \int_0^{\pi} e^{-x}\sin(x)\, dx.
		\end{align*}
		Es folgt:
		$$
			2\int_0^{\pi} e^{-x}\sin(x)\, dx =  (e^{-\pi} + 1) 
		$$
		und somit
		$$
			\int_0^{\pi} e^{-x}\sin(x)\, dx =  \frac{1}{2}(e^{-\pi} + 1).
		$$		
\item[(ii)] Wieder nutzen wir zweimal partielle Integration:
		\begin{align*}
		\int_0^1 \underbrace{x^2}_{G(x)}\underbrace{e^{x}}_{f(x)} \, dx
				&=[x^2e^x]_0^1 - \int_0^1 2xe^x \, dx\\
				&=e - \int_0^1 \underbrace{2x}_{G(x)}\underbrace{e^x}_{f(x)} \, dx\\
				&=e- \left([2xe^x]_0^1 - \int_0^1 2e^x \, dx\right)\\
				&= e - 2e + [2e^x]_0^1\\
				&=-e+2e-2\\
				&=e-2
		\end{align*}
\end{itemize}
\quad\\[10mm]
%-------------------------------------------------------------------
\section{Substitutionsregel}
%%%
Bestimmen Sie den Wert des folgenden Integrals mit Hilfe der Substitutionsregel:

		$$
			\int_0^{\frac{\pi}{4}} \frac{\sin(x)}{\cos(x)}\, dx
		$$

%-------------------------------------------------------------------
\subsection*{Lösung}
%%%
Substitution $u(x) = \cos(x)$. Es folgt $\frac{du}{dx} = \frac{d}{dx}u(x) = -\sin(x)$ und somit $dx=\frac{1}{-\sin(x)}du$.
	\begin{align*}
		\int_0^{\frac{\pi}{4}} \frac{\sin(x)}{\cos(x)}\, dx&= \int_0^{\frac{\pi}{4}} \frac{\sin(x)}{u}\, \frac{du}{-\sin(x)}	\\
				&= -\int_1^{\frac{1}{\sqrt{2}}} \frac{1}{u}\, du	\\
				&= \int_{\frac{1}{\sqrt{2}}}^1 \frac{1}{u}\, du	\\
				&= [\log(u)]_{\frac{1}{\sqrt{2}}}^1	\\
				&= -\log(\frac{1}{\sqrt{2}})
		\end{align*}

%-------------------------------------------------------------------
\newpage
\section{Newton-Cotes-Formeln}
%%%
\begin{itemize}
\item[a)] Berechnen Sie den den approximativen Wert der beiden Integrale aus Aufgabe 1 mit Hilfe der Trapezregel.
\item[b)] Berechnen Sie den den approximativen Wert des Integrals aus Aufgabe 2 mit Hilfe der Simpson-Regel.

\end{itemize}

%-------------------------------------------------------------------
\subsection*{Lösung}
%%%
\begin{itemize}
\item[a)]
	\begin{itemize}
	\item[(i)] 
		$$
			\int_0^{\pi} e^{-x}\sin(x)\, dx \thickapprox \frac{\pi-0}{2}\cdot \left(e^{-0}\sin(0)+e^{-\pi}\sin(\pi) \right)=\frac{\pi}{2}\cdot(1\cdot 0 +e^{-\pi} \cdot 0)=0 
		$$
	\item[(ii)]
		$$
			\int_0^1 x^2e^x \, dx \thickapprox \frac{1-0}{2}\cdot \left(0^2\cdot e^0 + 1^2\cdot e^1 \right)=\frac{1}{2}\cdot(0\cdot 1 + 1 \cdot e)=\frac{e}{2}
		$$
	\end{itemize}
\item[b)]
\begin{align*}
			\int_0^{\frac{\pi}{4}} \frac{\sin(x)}{\cos(x)}\, dx &\thickapprox \frac{\frac{\pi}{4}-0}{6}\cdot \left(\frac{\sin(0)}{\cos(0)}+4\cdot \frac{\sin\left(\frac{0+\frac{\pi}{4}}{2} \right)}{\cos\left(\frac{0+\frac{\pi}{4}}{2} \right)}+\frac{\sin(\frac{\pi}{4})}{\cos(\frac{\pi}{4})} \right)=\frac{\pi}{24}\left(0+4\frac{\sin(\frac{\pi}{8})}{\cos(\frac{\pi}{8})}+ \frac{\sin(\frac{\pi}{4})}{\cos(\frac{\pi}{4})}\right)\\
&=\frac{\pi}{24}(\tan(\frac{\pi}{8})+\tan(\frac{\pi}{4}))=\frac{\pi}{24}((\sqrt{2}-1)+1)=\frac{\pi\sqrt{2}}{24}
\end{align*}
\end{itemize}
\quad\\

\newpage
%-------------------------------------------------------------------
\section{Summierte Newton-Cotes-Formeln}
%%%
Berechnen Sie das Integral
$$
\int_0^\pi \cos(x)^3 dx
$$
näherungsweise unter Verwendung der summierten Simpson-Regel mit $m=2$.

%-------------------------------------------------------------------
\subsection*{Lösung}
%%%
Es gilt mit $m=2$: $h=\frac{\pi}{2}$ und $x_0=0, x_1=\frac{\pi}{2}$ sowie $x_2=\pi$.
\begin{align*}
\int_0^\pi \cos(x)^3 dx & \thickapprox \frac{\frac{\pi}{2}}{6}\cdot \left(\cos(0)^3+4\cos(\frac{0+\frac{\pi}{2}}{2})^3+2\cos(\frac{\pi}{2})^3+4\cos(\frac{\frac{\pi}{2}+\pi}{2})^3+\cos(\pi)^3 \right)\\
&=\frac{\pi}{12}\cdot \left(1^3+4 \cos(\frac{\pi}{4})^3+2\cdot 0^3+4 \cos(\frac{3\pi}{4})^3+(-1)^3 \right)\\
&=0
\end{align*}
%
%
\end{document}