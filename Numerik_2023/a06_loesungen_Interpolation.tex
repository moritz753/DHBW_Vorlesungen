\documentclass[
				a4paper,
				10pt
			]
			{scrartcl}

\parindent0mm

\usepackage{amsfonts}
\usepackage{amsmath}
\usepackage{amssymb}
\usepackage{amsthm}
\usepackage[ngerman]{babel}

\usepackage[utf8]{inputenc}
\usepackage[T1]{fontenc}
\usepackage{textcomp}

\usepackage{graphicx}
\usepackage{xcolor}

\usepackage[
			pdftex,
			colorlinks,
			breaklinks,
			linkcolor=blue,
			citecolor=blue,
			filecolor=black,
			menucolor=black,
			urlcolor=black,
			pdfauthor={Andreas Weber},
			pdftitle={Aufgaben zu Analysis und Lineare Algebra},
			plainpages=false,
			pdfpagelabels,
			bookmarksnumbered=true
		   ]{hyperref}


%--------------------------------------%
% Math ------------------------------%
%--------------------------------------%

% Mengen (Zahlen)
\newcommand{\N}{\mathbb{N}}
\newcommand{\Q}{\mathbb{Q}}
\newcommand{\R}{\mathbb{R}}
\newcommand{\Z}{\mathbb{Z}}
\newcommand{\C}{\mathbb{C}}

% Mengen (allgemein)
\newcommand{\K}{\mathbb{K}}
\newcommand\PX{{\cal P}(X)}

% Zahlentheorie
\newcommand{\ggT}{\mathrm{ggT}}


% Ableitungen
\newcommand{\ddx}{\frac{d}{dx}}
\newcommand{\pddx}{\frac{\partial}{\partial x}}
\newcommand{\pddy}{\frac{\partial}{\partial y}}
\newcommand{\grad}{\text{grad}}

%--------------------------------------%
% Layout Colors ------------------%
%--------------------------------------%
\newcommand*{\highlightDef}[1]{{\color{lightBlue}#1}}
\newcommand*{\highlight}[1]{{\color{lightBlue}#1}}
% Color Definitions
\definecolor{dhbwRed}{RGB}{226,0,26} 
\definecolor{dhbwGray}{RGB}{61,77,77}
\definecolor{lightBlue}{RGB}{28,134,230}


%-------------------------------------------------------------------
\begin{document}

\vspace*{-20mm}
{
	%\usekomafont{title}
	\color{dhbwGray}
	Dr. Moritz Gruber	\hfill Version \today\\
	DHBW Karlsruhe\\
}

\vspace{10mm}
\begin{center}
	{
		\usekomafont{title}
		%\color{lightBlue}
		{ \LARGE 	Übungsaufgaben 6 - Lösungen}\\[3mm]
		{\Large Interpolation}
	}
\end{center}

%\vspace{5mm}

\section{Ansatzfunktionen}
%%%
Interpolieren Sie die Werte
$$
x_0=0, y_0=1, x_1=\frac{\pi}{3}, y_1=2, x_2=\frac{2\pi}{3}, y_2=2 \text{ und }x_3=\pi, y_3=1
$$
mit den Ansatzfunktionen $g_0(x)=1, g_1(x)=\sin(x), g_2(x)=\sin(2x)$ und $g_3(x)=\sin(3x)$.

\subsection*{Lösung:}
Der Ansatz $g(x)=\sum_{k=0}^3 a_kg_k(x)$ mit $g(x_k)=y_k$ ergibt das LGS
$$
\begin{pmatrix}1 & 0 & 0 &0 \\ 1&\frac{\sqrt{3}}{2}& \frac{\sqrt{3}}{2} & 0\\ 1& \frac{\sqrt{3}}{2}&-\frac{\sqrt{3}}{2}&0  \\1 & 0 & 0& 0 \end{pmatrix}\cdot \begin{pmatrix} a_0\\a_1\\a_2\\a_3 \end{pmatrix}= \begin{pmatrix} 1\\2\\2\\1 \end{pmatrix}
$$
Wir subtrahieren die erste Zeile der Gleichung von den den drei anderen und subtrahieren dann die zweite Zeile von der vierten. So erhalten wir
$$
\begin{pmatrix}1 & 0 & 0 &0 \\ 0&\frac{\sqrt{3}}{2}& \frac{\sqrt{3}}{2} & 0\\ 0& 0&-\sqrt{3}&0  \\0 & 0 & 0& 0 \end{pmatrix}\cdot \begin{pmatrix} a_0\\a_1\\a_2\\a_3 \end{pmatrix}= \begin{pmatrix} 1\\1\\0\\0 \end{pmatrix}
$$
Nun addieren wir die Hälfte der dritten Zeile auf die zweite und normieren die Diagonaleinträge. Das ergibt
$$
\begin{pmatrix}1 & 0 & 0 &0 \\ 0&1& 0 & 0\\ 0& 0&1&0  \\0 & 0 & 0& 0 \end{pmatrix}\cdot \begin{pmatrix} a_0\\a_1\\a_2\\a_3 \end{pmatrix}= \begin{pmatrix} 1\\\frac{2}{\sqrt{3}}\\0\\0 \end{pmatrix}
$$
Wir erhalten damit eine Familie $(g_t)_{t \in \R}$ von Lösungsfunktionen
$$
g_t(x)=1+\frac{2}{\sqrt{3}}\sin(x)+t\cdot \sin(3x)
$$
Dies kann nun so interpretiert werden, dass entweder die Ansatzfunktionen nicht geeignet waren, oder aber, dass mögliche (erwartete/bekannte) Schwingungen höherer Frequenz, die durch den Abstand der Messpunkte nicht beobachtbar waren, so berücksichtigt werden können.
%-------------------------------------------------------------------
\section{Splines}
%%%
Interpolieren Sie die Werte
$$
x_0=1, y_0=1, x_1=2, y_1=3 \text{ und } x_2=3, y_3=0
$$
mit Splines vom Grad $2$.
\vspace{5mm}

\subsection*{Lösung:}
Da der Grad der Splines $k=2$ ist, haben wir einen übirgen Freiheitsgrad. Für die Lösung belegen wir ihn mit
$$
S'(x_0)=p_0'(x_0)=a_{01}+2a_{02}x_0=1
$$
Damit erhalten wir das LGS
$$
\begin{pmatrix}1& 1 & 1 & 0 &0&0  \\ 0&0&0&1&2&4 \\ 0&0&0&1&3&9 \\ 0&1&2 & 0&0&0\\1&2&4 &-1&-2&-4\\ 0&1&4 &0&-1&-4 \end{pmatrix}\cdot \begin{pmatrix} a_{00}\\a_{01}\\ a_{02}\\ a_{10}\\ a_{11}\\ a_{12}   \end{pmatrix}=\begin{pmatrix} 1\\3\\0\\1\\0\\0 \end{pmatrix}
$$
Durch Umsortieren der Zeilen erhalten wir
$$
\begin{pmatrix}1& 1 & 1 & 0 &0&0 \\1&2&4 &-1&-2&-4\\ 0&1&4 &0&-1&-4 \\ 0&1&2 & 0&0&0 \\ 0&0&0&1&2&4 \\ 0&0&0&1&3&9 \end{pmatrix}
\cdot \begin{pmatrix} a_{00}\\a_{01}\\ a_{02}\\ a_{10}\\ a_{11}\\ a_{12}   \end{pmatrix}
=\begin{pmatrix} 1\\0\\0\\1 \\3\\0 \end{pmatrix}
$$
Durch einige Zeilenadditionen und Normierung ergibt dies die Stufenform
$$
\begin{pmatrix}1& 1 & 1 & 0 &0&0 \\0&1&3 &-1&-2&-4\\ 0&0&1 &1&1&0 \\ 0&0&0 &1&\frac{3}{2}&2 \\0&0&0&0&1&4 \\ 0&0&0&0&0&1 \end{pmatrix}
\cdot \begin{pmatrix} a_{00}\\a_{01}\\ a_{02}\\ a_{10}\\ a_{11}\\ a_{12}   \end{pmatrix}
=\begin{pmatrix} 1\\-1\\1\\\frac{3}{2} \\3\\-6 \end{pmatrix}
$$
Und schließlich die Gauß-Normalform
$$
\begin{pmatrix}1& 0 & 0 & 0 &0&0 \\0&1&0 &0&0&0\\ 0&0&1 &0&0&0 \\ 0&0&0 &1&0&0 \\ 0&0&0&0&1&0 \\ 0&0&0&0&0&1 \end{pmatrix}
\cdot \begin{pmatrix} a_{00}\\a_{01}\\ a_{02}\\ a_{10}\\ a_{11}\\ a_{12}   \end{pmatrix}
=\begin{pmatrix} 1\\-1\\1\\-27 \\27\\-6 \end{pmatrix}
$$
Somit sind die beiden gesuchten Polynome
\begin{align*}
p_0(x)&=1-x+x^2\\
p_1(x)&=-27+27x-6x^2
\end{align*}
%-------------------------------------------------------------------
\section{Extrapolation}
%%%
Extrapolieren Sie Ihre Ergebnisse für die Aufgaben 1 und 2 für die Stellen $x_{l}=-2$ und $x_r=5$ je einmal konstant und einmal konsistent.
\vspace{5mm}

\subsection*{Lösung:}
\textbf{Zu Aufgabe 1}\\
\begin{itemize}
\item konstant: $f(-2)=1$ und $f(5)=1$
\item konsistent: $f(-2)=f(0)-2f'(0)=1-2(\frac{4}{\sqrt{3}}\cos(0))=1-\frac{8}{\sqrt{3}}$\\
\hspace*{16.5mm} $f(5)=f(\pi)+(5-\pi)f'(\pi)=1+(5-\pi)(\frac{4}{\sqrt{3}}\cos(2\pi))=1+\frac{20-4\pi}{\sqrt{3}}$
\end{itemize}

\textbf{Zu Aufgabe 2}\\
\begin{itemize}
\item konstant: $f(-2)=1$ und $f(5)=0$
\item konsistent: $f(-2)=f(1)-(1-(-2))f'(1)=1-3\cdot 1=-2$ \\
\hspace*{16.0mm} $f(5)=f(3)+(5-3)f'(3)=0+2\cdot (-9)= -18$
\end{itemize}
%


%{{1, 1 , 1 , 0 ,0,0  },{ 0,0,0,1,2,4 },{0,0,0,1,3,9  },{ 0,1,\frac{1}{2} , 0,0,0 },{1,2,4 ,-1,-2,-4 },{ 0,1,1 ,0,-1,-1 }}* {{ a_{00} },{a_{01} },{ a_{02} },{ a_{10} },{ a_{11} },{ a_{12} }}={{ 1 },{3 },{0 },{1 },{0 },{0 }}
%-------------------------------------------------------------------


\end{document}