\documentclass{beamer}

\usepackage{beamerthemesplit}

\usepackage{amsfonts}
\usepackage{amsmath}
\usepackage{amssymb}
\usepackage{amsthm}
\usepackage{amscd}

\usepackage{stmaryrd} 					%\lightning
\usepackage{algorithm2e}


\usepackage[ngerman]{babel}

\usepackage[utf8]{inputenc}
\usepackage[T1]{fontenc}
\usepackage{textcomp}


% Color Definitions
\definecolor{dhbwRed}{RGB}{226,0,26} 
\definecolor{dhbwGray}{RGB}{61,77,77}
\definecolor{lightBlue}{RGB}{28,134,230}

% Basic Theme
\usetheme{Malmoe}

% Color Re-Definitions
\usecolortheme[named=lightBlue]{structure}
\setbeamercolor*{alerted  text}{fg=dhbwRed, bg=white}
\setbeamercolor*{subsection in toc}{fg=dhbwGray, bg=white}

%\setbeamercolor*{palette primary}{fg=white,bg=lightBlue}
%\setbeamercolor*{palette secondary}{fg=white,bg=gray}
%\setbeamercolor*{palette tertiary}{fg=white,bg=gray}
%\setbeamercolor*{palette quaternary}{fg=white,bg=dhbwRed}

% no navigation symbols
\setbeamertemplate{navigation symbols}{}

% headline, footline
\setbeamertemplate{footline}{\color{dhbwGray} \hfill\insertframenumber\hspace{5mm}\vspace{2mm}}
\setbeamertemplate{headline}{}

% Title Page
\newcommand*{\makeTitlePage}{
	
	\begin{frame}[plain]
		
		\vfill
		\vfill
		\begin{center}
			{
				\usebeamerfont{title}
				\usebeamercolor[fg]{title}
				\Large
				\inserttitle
			}\\[3mm]
			{	
				\usebeamerfont{subtitle}
				\usebeamercolor[fg]{subtitle}
				\large
				\insertsubtitle
			}
		\end{center}
		%
		\vfill
		\vfill
		\vfill
		\vfill
		%
		\begin{columns}
			\begin{column}{0.5\textwidth}
				\begin{flushleft}
					{
						\usebeamerfont{normal text}
						\color{dhbwGray!80}
						\scriptsize
						Dr. Moritz Gruber\\
						DHBW Karlsruhe\\
						
					}
				\end{flushleft}
			\end{column}
			%
			\begin{column}{0.5\textwidth}
				\begin{flushright}
					\includegraphics[scale=0.06]{../DHBW.png}
				\end{flushright}
			\end{column}
		\end{columns}
		%
		\vspace{1mm}
		\begin{columns}
			\begin{column}{0.5\textwidth}
				\begin{flushleft}
					{
						\usebeamerfont{normal text}
						\color{dhbwGray!80}
						\scriptsize
						Version \today
					}
				\end{flushleft}
			\end{column}
			%
			\begin{column}{0.5\textwidth}
				% nothing (just a placeholder to be in line with the columns above
			\end{column}
		\end{columns}
	\end{frame}

}

% Section Divider Page
\newcommand*{\makeSectionDividerPage}{

	\begin{frame}[plain]
		\begin{center}
			\begin{flushleft}
				{				
					\usebeamercolor[fg]{frametitle}
					{\Large \insertsection} \\[3mm]
					{\large \insertsubsection}
				}
			\end{flushleft}
		\end{center}
        \end{frame}
	
}

% itemize
\setbeamertemplate{itemize items}[circle]
\setbeamertemplate{enumerate item}{(\theenumi)}




%--------------------------------------%
% Math ------------------------------%
%--------------------------------------%

% Mengen (Zahlen)
\newcommand{\N}{\mathbb{N}}
\newcommand{\Q}{\mathbb{Q}}
\newcommand{\R}{\mathbb{R}}
\newcommand{\Z}{\mathbb{Z}}
\newcommand{\C}{\mathbb{C}}

% Mengen (allgemein)
\newcommand{\K}{\mathbb{K}}
\newcommand\PX{{\cal P}(X)}

% Zahlentheorie
\newcommand{\ggT}{\mathrm{ggT}}


% Ableitungen
\newcommand{\ddx}{\frac{d}{dx}}
\newcommand{\pddx}{\frac{\partial}{\partial x}}
\newcommand{\pddy}{\frac{\partial}{\partial y}}
\newcommand{\grad}{\text{grad}}

%--------------------------------------%
% Layout Colors ------------------%
%--------------------------------------%
\newcommand*{\highlightDef}[1]{{\color{lightBlue}#1}}
\newcommand*{\highlight}[1]{{\color{lightBlue}#1}} % after theme for colours

%---------------------------------------------%
\title{Numerische Mathematik}
\subtitle{Integration}

%---------------------------------------------%
\begin{document}

%---------------------------------------------%
\makeTitlePage

%---------------------------------------------%
\begin{frame}\frametitle{Inhalt}
 \setcounter{tocdepth}{2}
   \tableofcontents
\end{frame}
%

%---------------------------------------------%
% Folien -----------------------------------%
%---------------------------------------------%
%
\section{Themen der Vorlesung}
%%%
\begin{frame}\frametitle{Themen der Vorlesung}
\begin{itemize}
\item Integration und Quadraturformeln
\item Differentialgleichungen
\item Mehrdimensionale Analysis
\item Optimierungsverfahren
\item Interpolation und Extrapolation
\item Partielle Differentialgleichungen
\item Numerische Problemlösungen
\item Geometrie von $\R^n$ und die QR-Zerlegung
\end{itemize}
\end{frame}
%
%---------------------------------------------%
\section{Analytische Integration}
\makeSectionDividerPage
%%%
%
%--------------------------------------------
\subsection{Das Riemann-Integral}
\subsubsection{Motivation}
%
\begin{frame}\frametitle{Motivation}

	Sei $f: [a,b] \to \R$ eine (stetige) Funktion.\\[1mm]

	Wie berechnet sich der Flächeninhalt, der von $f$ und der $x$-Achse 
	zwischen $a$ und $b$ eingeschlossen wird?
	
	\begin{center}
		\includegraphics[scale=0.8]{Grafiken/Integral_und_Flaeche/Integral_und_Flaeche.pdf}
	\end{center}

\end{frame}
%
%
\begin{frame}\frametitle{Allgemeine Situation}
\begin{itemize}
\item $I=[a,b]$\\[5mm]
\item $f:I\to \R$ beschränkte Funktion.
\end{itemize}
\end{frame}
%
\subsubsection{Definition des Riemann-Integral}
%
\begin{frame}\frametitle{Definition: Zerlegung}
Eine endliche Teilmenge $Z=\{x_0,x_1,...,x_n\}\subseteq I$ heißt eine \highlightDef{Zerlegung} von $I$, wenn $a=x_0<x_1<...<x_n=b$ gilt.\\\pause
\vfill
Ist $Z=\{x_0,x_1,...,x_n\}\subseteq I$ eine Zerlegung von $I$, dann definieren wir für $j\in \{0,...,n-1\}$:
\begin{itemize}
\item $I_j:=[x_j,x_{j+1}]$
\item $|I_j|:=x_{j+1}-x_j$
\item $m_j:=\inf(f(I_j))$
\item $M_j:=\sup(f(I_j))$
\end{itemize}\vfill
Die Menge aller Zerlegungen von $I$ bezeichnen wir mit \highlightDef{$\frak{Z}(I)$}.
\end{frame}
%
\begin{frame}\frametitle{Definition: Unter- und Obersumme}
Es sei $Z=\{x_0,x_1,...,x_n\} \in \frak{Z}(I)$. Dann nennt man
\begin{itemize}
\item $s_f(Z):=\sum_{j=0}^{n-1} m_j\cdot |I_j|$ die \highlightDef{Untersumme} von $f$ bzgl. $Z$ und 
\item $S_f(Z):=\sum_{j=0}^{n-1} M_j\cdot |I_j|$ die \highlightDef{Obersumme} von $f$ bzgl. $Z$.
\end{itemize}

\end{frame}
%
\begin{frame}\frametitle{Bemerkung}
Sind $Z_1,Z_2 \in \frak{Z}(I)$, dann ist auch $Z_1\cup Z_2 \in \frak{Z}(I)$.\\
\vfill
Gilt $Z_1 \subseteq Z_2$, so nennt man $Z_2$ eine \highlightDef{Verfeinerung} von $Z_1$.
\vfill
\pause
\highlightDef{Satz}\\
Seien $Z_1,Z_2 \in \frak{Z}(I)$, dann
\begin{itemize}
\item[a)] $Z_1 \subseteq Z_2 \ \Longrightarrow \ s_f(Z_1)\le s_f(Z_2)$ und $S_f(Z_2)\le S_f(Z_1)$. 
\item[b)] $s_f(Z_1) \le S_f(Z_2)$.
\end{itemize}
\end{frame}
%
\begin{frame}\frametitle{Definition}
\begin{itemize}
\item[]
$$
\underline{\int}_{a}^b f(x)dx:=\sup\{s_f(Z) \mid Z \in \frak{Z}(I)\} \ \text{ heißt \highlightDef{unteres Integral} von }f
$$
\vfill
\item[]
$$
\overline{\int}_a^{b} f(x)dx:=\inf\{S_f(Z) \mid Z \in \frak{Z}(I)\} \ \text{ heißt \highlightDef{oberes Integral} von }f
$$
\end{itemize}
\vfill
\pause
Für jede Zerlegung $Z\in \frak{Z}(I)$ gilt
$$
s_f(Z)\le \underline{\int}_{a}^b f(x)dx\le \overline{\int}_a^{b} f(x)dx\le S_f(Z)
$$
\end{frame}
%
\begin{frame}\frametitle{Definition: Riemann-Integral}
$f$ heißt \highlightDef{integrierbar} über $I=[a,b]$, wenn $\underline{\int}_{a}^b f(x)dx = \overline{\int}_a^{b} f(x)dx$ gilt.\\
In diesem Fall heißt 
$$
\int_a^b f(x) dx := \underline{\int}_{a}^b f(x)dx \quad \left(= \overline{\int}_a^{b} f(x)dx \right)
$$
das \highlightDef{Integral} von $f$ über $I$.
\vfill
\pause
\highlightDef{Bemerkung}\\
Existiert eine Folge $(Z_n)$ von Zerlegungen von $I$ mit $\lim_{n\to \infty}s_f(Z_n)=\lim_{n\to \infty}S_f(Z_n)=R$, so ist $f$ integrierbar mit 
$$
\int_a^b f(x)dx=R
$$
\end{frame}
%
\subsubsection{Rechenregeln für Integrale}
%
\begin{frame}\frametitle{Das Integral ist linear}
	Seien $f,g: [a,b] \to\R$ integrierbare Funktionen und $c\in\R$.
	\vspace{5mm}
	\begin{itemize}
	\item[a)]
	Dann ist auch $c\cdot f+ g$ integrierbar mit
	$$
		\int_{a}^{b}c\cdot f(x)+g(x)\, dx = c\cdot \int_{a}^{b}f(x)dx + \int_{a}^{b}g(x)dx. 
	$$
	\item[b)] Ist $f\le g$ auf $[a,b]$, dann gilt $\int_a^b f(x)dx \le \int_a^b g(x)dx$.
	\end{itemize}
	\vspace{5mm}
	\highlight{Beweis}\\
Folgt mit den Rechenregeln für Grenzwerte und den obigen Bemerkungen.
\end{frame}
%
%
\begin{frame}\frametitle{Satz}
Jede auf $[a,b]$ stetige Funktion ist auch über $[a,b]$ integrierbar.\\
\vspace{5mm}
[ohne Beweis]\pause
\vfill
\highlightDef{Bemerkung}\\
Vorsicht: nicht jede integrierbare Funktion ist stetig!
\end{frame}
%
%--------------------------------------------
\subsection{Der Hauptsatz der Differential- und Integralrechnung}
\makeSectionDividerPage
%%%
%
\subsubsection{Stammfunktionen}
%
\begin{frame}\frametitle{Definition: Stammfunktion}
Es sei $f: I \to \R$ eine Funktion. Eine Funktion $F: I \to \R$ ist eine \highlightDef{Stammfunktion} von $f$, wenn $F'(x)=f(x)$ für alle $x \in I$.
\end{frame}
%
\subsubsection{Der Hauptsatz der Differential- und Integralrechnung}
%
\begin{frame}\frametitle{Der Hauptsatz der Differential- und Integralrechnung}
Es sei $f:[a,b] \to \R$ integrierbar über $[a,b]$ und $F$ eine Stammfunktion von $f$. Dann gilt
$$
\int_a^b f(x) dx = F(b)-F(a) \quad \left(:= [F(x)]_{x=a}^{x=b}\right)
$$
\vfill
\highlightDef{Notation}\\
Wegen dieser Eigenschaft der Stammfunktion schreibt man oft auch $\int\hspace{-1mm} fdx$ oder $\int \hspace{-1mm}f$ statt $F$.
\end{frame}
%
\subsection{Weitere Rechenregeln für Integrale}
\makeSectionDividerPage
%%%
%
\subsubsection{Aufteilen}
%
\begin{frame}\frametitle{Aufteilen eines Integrals}
Es sei $f:[a,b] \to \R$ integrierbar (und beschränkt), sowie $c \in (a,b)$. Dann ist $f$ auf $[a,c]$ und $[c,b]$ integrierbar und es gilt
$$
\int_a^b f(x)dx=\int_a^c f(x)dx + \int_c^b f(x)dx
$$
\pause\vfill
\highlightDef{Beispiel}\\
$ \int_{-1}^1 |x|dx = \int_{-1}^0 -x dx+\int_{0}^1 xdx = \frac{1}{2} +\frac{1}{2}=1$
\end{frame}
%
\subsubsection{Dreiecksungleichung}
%
\begin{frame}\frametitle{Dreiecksungleichung für Integrale}

Es sei $f:[a,b] \to \R$ integrierbar. Dann ist auch $|f|$ integrierbar und es gilt:
$$
\left| \int_a^b f(x)dx \right| \le \int_a^b |f(x)|dx
$$
\pause\vfill
\highlightDef{Beispiel}\\
$0=\left| \int_{-1}^1 xdx \right| \le \int_{-1}^1 |x|dx = 1$
\end{frame}
%
\subsection{Integrationstechniken}
\makeSectionDividerPage
%
\subsubsection{Partielle Integration}
%
\begin{frame}\frametitle{Satz (Partielle Integration)}
Es seien $f,g:[a,b] \to \R$ integrierbar und $F$ und $G$ Stammfunktionen von $f$ und $g$. Dann gilt:
$$
\int_a^b F\cdot g\, dx = [F \cdot G]_a^b - \int_a^b f \cdot G\, dx
$$
\end{frame}
%
\subsubsection{Substitutionsregel}
%
\begin{frame}\frametitle{Satz (Substitutionsregel)}
Es sei $f:[a,b] \to \R$ integrierbar mit Stammfunktion $F$, sowie $g:[\alpha,\beta] \to \R$ differenzierbar mit $g([\alpha,\beta])\subseteq [a,b]$. Dann gilt:
$$
\int_\alpha^\beta f(g(t))\cdot g'(t)\, dt = \int_{g(\alpha)}^{g(\beta)} f(x) dx
$$
\vfill
\highlightDef{Konvention}\\
Es ist oben möglich, dass $g(\alpha)>g(\beta)$ gilt. In diesem Fall benutzt man:\\
Für $a,b \in \R$ gilt $\int_a^b fdx = - \int_b^a fdx$. Dies lässt sich aus der Substitutionsregel mit $g(t)=b+a-t$ für $t \in [a,b]$ folgern.
\end{frame}
%
\begin{frame}\frametitle{Bemerkung}
Oft hilft beim Substituieren das folgende Kalkül:
\begin{align*}
&&g(t)&=x \\
 \Longrightarrow &&g'(t)&=\frac{d}{dt}g(t)=\frac{dx}{dt}\\  \Longrightarrow&& dx&=g'(t)dt
\end{align*}
\end{frame}
%

%%
%------------------------------
%------------------------------
%
\section{Numerische Integration}
\makeSectionDividerPage
%
\subsection{Newton-Cotes-Formeln}
%
\begin{frame}\frametitle{Newton-Cotes-Formeln}
\begin{itemize}
\item Mittelpunktregel  $\int_a^b f(x)dx \thickapprox (b-a)\cdot f(\frac{a+b}{2})$\\\includegraphics[scale=0.3]{Grafiken/MPRegel.png} \pause
\item Trapezregel $\int_a^b f(x)dx \thickapprox (b-a) \cdot \frac{f(a)+f(b)}{2}$\\\includegraphics[scale=0.3]{Grafiken/TRegel.png} \pause 
\item Simpson-Regel $\int_a^b f(x)dx \thickapprox \frac{b-a}{6}\cdot (f(a)+4f(\frac{b+a}{2})+f(b))$\\\includegraphics[scale=0.3]{Grafiken/SRegel.png}
\end{itemize}
Allgemein nennt man numerische/approximative Formeln zur Bestimmung von Integralen auch \highlightDef{Quadraturformeln}.
\end{frame}
%
\begin{frame}\frametitle{Genauigkeitsgrad}
Eine Quadraturformel besitzt den \highlightDef{Genauigkeitsgrad} $d$, wenn sie die Monome $1,x,x^2,...,x^d$ exakt integriert, nicht aber $x^{d+1}$.\\\pause
\vfill
\highlightDef{Satz}\\
Die Mittelpunktregel und die Trapezregel haben Genauigkeitsgrad 1.\\
Die Simpson-Regel hat Genauigkeitsgrad 3.
\end{frame}
%
%
\begin{frame}\frametitle{Approximationsfehler}
\footnotesize
\begin{align*}
&\text{\highlightDef{Mittelpunktregel}}\\
&\left|\int_a^b f(x)\ dx - (b-a)\cdot f(\frac{a+b}{2}) \right|&\le \frac{(b-a)^3}{24}\max_{x\in[a,b]}(|f''(x)|)\\
%
&&\\
&\text{\highlightDef{Trapezregel}}\\
&\left|\int_a^b f(x)\ dx - (b-a) \cdot \frac{f(a)+f(b)}{2}  \right|&\le \frac{(b-a)^3}{12}\max_{x\in[a,b]}(|f''(x)|)\\
%
&&\\
&\text{\highlightDef{Simpson-Regel}}\\
&\left|\int_a^b f(x)\ dx -  \frac{b-a}{6}\cdot \left(f(a)+4f(\frac{b+a}{2})+f(b)\right)\right|&\le \frac{(b-a)^5}{2880}\max_{x\in[a,b]}(|f^{(4)}(x)|)
\end{align*}
\vfill
[ohne Beweis]
\end{frame}
%
%--------------------------------------------
\subsection{Summierte Newton-Cotes-Formeln}
%
\begin{frame}\frametitle{Summierte Newton-Cotes-Formeln}
Für bessere Approximationen kann das Intervall $[a,b]$ zerlegt werden und auf jedes Teilintervall eine Quadraturformel angewendet werden. Dies führt zu den \highlightDef{summierten Newton-Cotes-Formeln}.\\\quad\\
\includegraphics[scale=0.7]{Grafiken/SummierteMPRTR.png}\\
\end{frame}
%
\begin{frame}\frametitle{Summierte Newton-Cotes-Formeln}
\highlightDef{Summierte Mittelpunktregel}\\
$m=$ Anzahl der Teilintervalle, $h=\frac{b-a}{m}$, \\Stützstellen $x_k=a+(k-\frac{1}{2})h$ für $k=1,...,m$.
$$
\int_a^b f(x) \ dx \thickapprox h \cdot \sum_{k=1}^m f(x_k)
$$\pause
\highlightDef{Summierte Trapezregel}\\
$m=$ Anzahl der Teilintervalle, $h=\frac{b-a}{m}$, \\Stützstellen $x_k=a+kh$ für $k=0,...,m$.
$$
\int_a^b f(x) \ dx \thickapprox h \cdot \left(\frac{1}{2}f(x_0)+ \sum_{k=1}^{m-1} f(x_k) + \frac{1}{2}f(x_m)\right)
$$

\end{frame}
%
\begin{frame}\frametitle{Summierte Simpson-Regel}
\includegraphics[scale=0.7]{Grafiken/SummierteSR.png}\\
\highlightDef{Summierte Simpson-Regel}\\
$m=$ Anzahl der Teilintervalle, $h=\frac{b-a}{m}$, \\Stützstellen $x_k=a+kh$ für $k=0,...,m$.
\tiny
$$
\int_a^b f(x) \ dx \thickapprox \frac{h}{6} \cdot \left(f(x_0)+ 4f(\frac{x_0+x_1}{2})+2f(x_1)+4f(\frac{x_1+x_2}{2}+2f(x_2)+...+4f(\frac{x_{m-1}+x_m}{2})+f(x_m)\right)
$$
\end{frame}
%
\begin{frame}\frametitle{Beispiel}
$f(x)=\cos(x)^2$ und $[a,b]=[0, \pi]$ sowie $m=2$. \\\pause 
Dann gilt $h=\frac{\pi-0}{2}$, $x_0=0, x_1=\frac{\pi}{2}, x_2=\pi$ und \pause
\footnotesize
\hspace{-1mm}
\begin{align*}
\hspace{-1mm}\int_0^\pi \cos(x)^2\ dx &\thickapprox \frac{\frac{\pi - 0}{2}}{6}\left(\cos(0)^2+4\cos(\frac{\frac{\pi}{2}}{2})^2+2\cos(\frac{\pi}{2})^2+4\cos(\frac{\frac{\pi}{2}+\pi}{2})^2+\cos(\pi)^2 \right)\\
&=\frac{\pi}{12}\left(1^2+4\cdot \cos(\frac{\pi}{4})^2+0+4\cdot \cos(\frac{3\pi}{4})^2+(-1)^2\right)\\
&=\frac{\pi}{12}\left(1+4\cdot (\frac{1}{\sqrt{2}})^2+4\cdot(-\frac{1}{\sqrt{2}})^2+1\right)\\
&=\frac{\pi}{2}
\end{align*}
\vfill
\normalsize
(Der exakte Wert des Integrals ist $\frac{\pi}{2}$.)
\end{frame}
%
%%
%------------------------------
%------------------------------
%
\section{Anwendungsbeispiel: Weglänge}
\makeSectionDividerPage
%
%
\begin{frame}\frametitle{Wege und Weglänge}
Ein \highlightDef{ebener Weg} ist eine Abbildung
$$
\gamma : [a,b] \to \R^2, t \mapsto (\gamma_1(t),\gamma_2(t))
$$
wobei die beiden Koordinaten-Funktionen $\gamma_1$ und $\gamma_2$ stetig sind. \pause
\vfill
Sind $\gamma_1$ und $\gamma_2$ sogar differenzierbar, so definiert man die \highlightDef{Länge} von $\gamma$ als
$$
L(\gamma):=\int_a^b \|\dot \gamma(t)\| \ dt
$$
wobei $\|\dot \gamma(t)\|=\sqrt{\gamma_1'(t)^2+\gamma_2'(t)^2}$.
\end{frame}
%
%
\begin{frame}\frametitle{Beispiel}
Betrachten wir den Weg $\gamma: [0,1] \to \R^2, t \mapsto (t,\exp(t))$. Dann ist die Länge gegeben als
$$
L(\gamma)=\int_0^1 \sqrt{1+\exp(x)^2} \ dt
$$\pause
Dieses Integral ist nicht unbedingt einfach zu berechnen, aber lässt sich durch eine Quadraturformel (z.B. summierte Trapezregel) approximieren:\pause \small
\begin{align*}
L(\gamma)&=\int_0^1 \sqrt{1+\exp(x)^2} \ dt \\
&\thickapprox \frac{1-0}{2}\left(\frac{1}{2}\sqrt{1+\exp(0)^2}+\sqrt{1+\exp(\frac{1}{2})^2} +\frac{1}{2}\sqrt{1+\exp(1)^2}\right)\\
&=\frac{1}{2}\left(\frac{1}{2}\sqrt{2}+\sqrt{1+e}+\frac{1}{2}\sqrt{1+e^2} \right)\\
&\thickapprox 2,042
\end{align*}
\vfill
(Der exakte Wert des Integrals ist $e$.)
\end{frame}
%
%
\begin{frame}\frametitle{Küstenparadoxon}
\includegraphics[scale=0.4]{Grafiken/Britain200km.png} 
\includegraphics[scale=0.4]{Grafiken/Britain50km.png} \\
\ \ Länge ca. 2800km \qquad Länge ca. 3400km
\end{frame}
%
\end{document}