\documentclass[
				a4paper,
				10pt
			]
			{scrartcl}

\parindent0mm

\usepackage{amsfonts}
\usepackage{amsmath}
\usepackage{amssymb}
\usepackage{amsthm}
\usepackage[ngerman]{babel}
\usepackage{graphicx}
\usepackage{xcolor}

\usepackage[
			pdftex,
			colorlinks,
			breaklinks,
			linkcolor=blue,
			citecolor=blue,
			filecolor=black,
			menucolor=black,
			urlcolor=black,
			pdfauthor={Andreas Weber},
			pdftitle={Aufgaben zu Analysis und Lineare Algebra},
			plainpages=false,
			pdfpagelabels,
			bookmarksnumbered=true
		   ]{hyperref}


%--------------------------------------%
% Math ------------------------------%
%--------------------------------------%

% Mengen (Zahlen)
\newcommand{\N}{\mathbb{N}}
\newcommand{\Q}{\mathbb{Q}}
\newcommand{\R}{\mathbb{R}}
\newcommand{\Z}{\mathbb{Z}}
\newcommand{\C}{\mathbb{C}}

% Mengen (allgemein)
\newcommand{\K}{\mathbb{K}}
\newcommand\PX{{\cal P}(X)}

% Zahlentheorie
\newcommand{\ggT}{\mathrm{ggT}}


% Ableitungen
\newcommand{\ddx}{\frac{d}{dx}}
\newcommand{\pddx}{\frac{\partial}{\partial x}}
\newcommand{\pddy}{\frac{\partial}{\partial y}}
\newcommand{\grad}{\text{grad}}

%--------------------------------------%
% Layout Colors ------------------%
%--------------------------------------%
\newcommand*{\highlightDef}[1]{{\color{lightBlue}#1}}
\newcommand*{\highlight}[1]{{\color{lightBlue}#1}}
% Color Definitions
\definecolor{dhbwRed}{RGB}{226,0,26} 
\definecolor{dhbwGray}{RGB}{61,77,77}
\definecolor{lightBlue}{RGB}{28,134,230}

%
\addtokomafont{section}{\color{dhbwGray}}
\addtokomafont{subsection}{\color{dhbwGray}}
\usepackage{stix}

%-------------------------------------------------------------------
\begin{document}

\vspace*{-20mm}
{
	%\usekomafont{title}
	\color{dhbwGray}
	Dr. Moritz Gruber	\hfill Version \today\\
	DHBW Karlsruhe\\
}

\vspace{10mm}
\begin{center}
	{
		\usekomafont{title}
		%\color{lightBlue}
		{ \LARGE Lösungen Übungsaufgaben 2}\\[3mm]
		{\Large Gewöhnliche Differentialgleichungen}
	}
\end{center}

\vspace{5mm}

%-------------------------------------------------------------------



%-------------------------------------------------------------------
\section{Explizites Eulerverfahren}
%%%
Führen Sie einen Schritt des expliziten Eulerverfahren für das Anfangswertproblem
$$
\begin{cases}
f'(x)&=e^{-x^3}-3x^2 f(x)\\
f(0)&=1
\end{cases}
$$
mit Schrittweite $h=1$ durch.
%-------------------------------------------------------------------
\subsection*{Lösung}
%%%
Wir wählen die Bezeichnungen $x_0=0$ und $x_1=x_0+h=1$. Damit:
\begin{align*}
\hat f(1)&=\hat f(x_1)\\
&=\hat f(x_0)+h \left(e^{-x_0^3}-3x_0^2 \hat f(x_0)\right)\\
&=1+1\cdot (e^0-3\cdot 0 \cdot 1)\\
&=1+1\cdot 1\\
&=2
\end{align*}

\newpage
%-------------------------------------------------------------------
\section{Implizites Eulerverfahren}
%%%
Führen Sie einen Schritt des impliziten Eulerverfahren für das Anfangswertproblem
$$
\begin{cases}
f'(x)&=2\sqrt{f(x)}\\
f(0)&=0
\end{cases}
$$
mit Schrittweite $h=1$ durch.


%-------------------------------------------------------------------
\subsection*{Lösung}
%%%
Wir wählen die Bezeichnungen $x_0=0$ und $x_1=x_0+h=1$. Damit:
\begin{align*}
\hat f(1)&=\hat f(x_1)\\
&=\hat f(x_0)+h \left(2\sqrt{\hat f(x_1)}\right)\\
&=0+1\cdot\left(2\sqrt{\hat f(x_1)}\right)\\
&=2\sqrt{\hat f(x_1)}
\end{align*}
Dies lässt sich umstellen zu:
$$
\hat f(x_1)^2=4\cdot \hat f(x_1)
$$
und weiter
$$
\hat f(x_1)=\frac{4\pm 4}{2}
$$
Man hat hier somit die \textit{Wahl} zwischen $\hat f(1)=0$ und $\hat f(1)=4$.
%-------------------------------------------------------------------
\newpage
\section{Runge-Kutta-Verfahren}
%%%
Gegeben ist das Butcher-Tableau
$$
\begin{array}{c|ccc}
0 & 0 & 0 & 0\\
\frac{1}{4} & 1 & 0 & 0\\
\frac{1}{2} & 0 & 1 & 0\\ \hline
 & \frac{1}{2} & 0 &  \frac{1}{2}
\end{array}
$$
Bestimmen Sie daraus die (allgmeine) Formel für $\hat f(x_{i+1})$ sowie die Zwischenstufen $k_1,k_2$ und $k_3$ gemäß dem Runge-Kutta-Verfahren.
%-------------------------------------------------------------------
\subsection*{Lösung}
%%%
\begin{align*}
\hat f(x_{i+1})&=\hat f(x_i)+h\cdot \sum_{j=1}^3 b_jk_j\\
&=\hat f(x_i)+h\cdot \left(\frac{1}{2}k_1 + \frac{1}{2}k_3 \right)\\
&\\
k_1&=G(x_i+c_1h, \hat f(x_i)+h\sum_{m=1}^3 a_{1m}k_m)\\
&=G(x_1, \hat f(x_i)) \\
&\\
k_2&=G(x_i+c_2h, \hat f(x_i)+h\sum_{m=1}^3 a_{2m}k_m)  \\
&=G(x_1+\frac{h}{4}, \hat f(x_i)+hk_1) \\
&\\
k_3&=G(x_i+c_3h, \hat f(x_i)+h\sum_{m=1}^3 a_{3m}k_m)  \\
&=G(x_1+\frac{h}{2}, \hat f(x_i)+hk_2) \\
\end{align*}
%
%
\end{document}