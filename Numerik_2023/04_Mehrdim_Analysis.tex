\documentclass{beamer}

\usepackage{beamerthemesplit}

\usepackage{amsfonts}
\usepackage{amsmath}
\usepackage{amssymb}
\usepackage{amsthm}
\usepackage{amscd}

\usepackage{stmaryrd} 					%\lightning
\usepackage{algorithm2e}


\usepackage[ngerman]{babel}

\usepackage[utf8]{inputenc}
\usepackage[T1]{fontenc}
\usepackage{textcomp}


% Color Definitions
\definecolor{dhbwRed}{RGB}{226,0,26} 
\definecolor{dhbwGray}{RGB}{61,77,77}
\definecolor{lightBlue}{RGB}{28,134,230}

% Basic Theme
\usetheme{Malmoe}

% Color Re-Definitions
\usecolortheme[named=lightBlue]{structure}
\setbeamercolor*{alerted  text}{fg=dhbwRed, bg=white}
\setbeamercolor*{subsection in toc}{fg=dhbwGray, bg=white}

%\setbeamercolor*{palette primary}{fg=white,bg=lightBlue}
%\setbeamercolor*{palette secondary}{fg=white,bg=gray}
%\setbeamercolor*{palette tertiary}{fg=white,bg=gray}
%\setbeamercolor*{palette quaternary}{fg=white,bg=dhbwRed}

% no navigation symbols
\setbeamertemplate{navigation symbols}{}

% headline, footline
\setbeamertemplate{footline}{\color{dhbwGray} \hfill\insertframenumber\hspace{5mm}\vspace{2mm}}
\setbeamertemplate{headline}{}

% Title Page
\newcommand*{\makeTitlePage}{
	
	\begin{frame}[plain]
		
		\vfill
		\vfill
		\begin{center}
			{
				\usebeamerfont{title}
				\usebeamercolor[fg]{title}
				\Large
				\inserttitle
			}\\[3mm]
			{	
				\usebeamerfont{subtitle}
				\usebeamercolor[fg]{subtitle}
				\large
				\insertsubtitle
			}
		\end{center}
		%
		\vfill
		\vfill
		\vfill
		\vfill
		%
		\begin{columns}
			\begin{column}{0.5\textwidth}
				\begin{flushleft}
					{
						\usebeamerfont{normal text}
						\color{dhbwGray!80}
						\scriptsize
						Dr. Moritz Gruber\\
						DHBW Karlsruhe\\
						
					}
				\end{flushleft}
			\end{column}
			%
			\begin{column}{0.5\textwidth}
				\begin{flushright}
					\includegraphics[scale=0.06]{../DHBW.png}
				\end{flushright}
			\end{column}
		\end{columns}
		%
		\vspace{1mm}
		\begin{columns}
			\begin{column}{0.5\textwidth}
				\begin{flushleft}
					{
						\usebeamerfont{normal text}
						\color{dhbwGray!80}
						\scriptsize
						Version \today
					}
				\end{flushleft}
			\end{column}
			%
			\begin{column}{0.5\textwidth}
				% nothing (just a placeholder to be in line with the columns above
			\end{column}
		\end{columns}
	\end{frame}

}

% Section Divider Page
\newcommand*{\makeSectionDividerPage}{

	\begin{frame}[plain]
		\begin{center}
			\begin{flushleft}
				{				
					\usebeamercolor[fg]{frametitle}
					{\Large \insertsection} \\[3mm]
					{\large \insertsubsection}
				}
			\end{flushleft}
		\end{center}
        \end{frame}
	
}

% itemize
\setbeamertemplate{itemize items}[circle]
\setbeamertemplate{enumerate item}{(\theenumi)}




%--------------------------------------%
% Math ------------------------------%
%--------------------------------------%

% Mengen (Zahlen)
\newcommand{\N}{\mathbb{N}}
\newcommand{\Q}{\mathbb{Q}}
\newcommand{\R}{\mathbb{R}}
\newcommand{\Z}{\mathbb{Z}}
\newcommand{\C}{\mathbb{C}}

% Mengen (allgemein)
\newcommand{\K}{\mathbb{K}}
\newcommand\PX{{\cal P}(X)}

% Zahlentheorie
\newcommand{\ggT}{\mathrm{ggT}}


% Ableitungen
\newcommand{\ddx}{\frac{d}{dx}}
\newcommand{\pddx}{\frac{\partial}{\partial x}}
\newcommand{\pddy}{\frac{\partial}{\partial y}}
\newcommand{\grad}{\text{grad}}

%--------------------------------------%
% Layout Colors ------------------%
%--------------------------------------%
\newcommand*{\highlightDef}[1]{{\color{lightBlue}#1}}
\newcommand*{\highlight}[1]{{\color{lightBlue}#1}} % after theme for colours

%---------------------------------------------%
\title{Numerische Mathematik}
\subtitle{Mehrdimensionale Analysis}

%---------------------------------------------%
\begin{document}

%---------------------------------------------%
\makeTitlePage

%---------------------------------------------%
\begin{frame}\frametitle{Inhalt}
   \tableofcontents
\end{frame}
%

%---------------------------------------------%
% Folien -----------------------------------%
%---------------------------------------------%
%

%--------------------------------------------
\section{Funktionen $f:\R^n \to \R^m$}
\makeSectionDividerPage
%%%
%
\subsection{Grundlagen}
%
\begin{frame}\frametitle{Funktionen $f: \R^n \to \R^m$}
In diesem Abschnitt betrachten wir \highlightDef{vektorwertige Funktionen in mehreren Veränderlichen}
$$
f: D \subset \R^n \to \R^m, (x_1,...,x_n) \mapsto \begin{pmatrix}f_1(x_1,...,x_n)\\ f_2(x_1,...,x_n)\\ ...\\f_m(x_1,...,x_n) \end{pmatrix}
$$
\vfill
Solche Funktionen spielen in der Anwendung eine große Rolle, z.B. in der Strömungsmechanik, der Metereologie oder der Finanzmathematik.

\end{frame}
%
%
\begin{frame}\frametitle{Stetigkeit}
Eine Folge $(x^{(k)})_k=(x_1^{(k)},...,x_n^{(k)})_k$ in $\R^n$ konvergiert gegen $x^{(0)}=(x_1^{(0)},...,x_1^{(0)}) \in \R^n$ für $k \to \infty$, wenn jede der Folgen $(x_j^{(k)})_k$ gegen $x_j^{(0)}$ konvergiert, $j \in \{1,...,n\}$. \\\vfill
\pause Damit gilt für eine Funktion $f: D \subset \R^n \to \R^m$\\\vfill
\begin{itemize}
\item $f(x)$ konvergiert für $x \to x_0$ gegen $y_0 \in \R^n$, wenn für alle gegen $x_0$ konvergierenden Folgen $(x^{(k)})$ gilt: $f(x^{(k)}) \to y_0 \ (k \to \infty)$. Man schreibt dann $\lim \limits_{x\to x_0}f(x)=y_0$\\\pause
\item $f$ heißt \highlightDef{stetig in $x_0$}, wenn $\lim \limits_{x\to x_0} f(x)=f(x_0)$.\\\pause
\item $f$ heißt \highlightDef{stetig}, wenn $f$ in allen Punkten $x \in D$ stetig ist.
\end{itemize}

\end{frame} 
%
\begin{frame}\frametitle{Partielle Ableitungen}
Es sei $f: D \subset \R^n \to \R$ eine \textbf{reell-wertige} Funktion und $x_0 \in \R^n$. Dann heißt $f$ \highlightDef{in $x_0$ partiell differenzierbar nach $x_j$}, wenn der Grenzwert
$$
\lim_{t \to 0} \frac{f(x_0+te_j)-f(x_0)}{t}
$$
existiert.\\\pause Man nennt diesen dann die \highlightDef{partielle Ableitung von $f$ nach $x_j$} und notiert diese als $\frac{\partial}{\partial x_j} f(x_0)$.\\\pause
\vfill
Man nennt $f$ \highlightDef{partiell differenzierbar nach $x_j$}, wenn in allen $x \in D$ partiell differenzierbar nach $x_j$ ist.\\\pause
\vfill
Im Fall noch $n=2$ und $n=3$ schreibt man meist $\frac{\partial}{\partial x} f, \frac{\partial}{\partial y} f$ und $\frac{\partial}{\partial z} f$

\end{frame} 
%
%
\begin{frame}\frametitle{Satz von Schwarz}
Es sei $f: \R^n \to \R$ zweimal stetig partiell differenzierbar\footnote{D.h. für alle $i,j \in \{1,...,n\}$ exisitieren die zweiten partiellen Ableitungen $\frac{\partial^2}{\partial x_i \partial x_j}f$ und sind stetig.} Dann gilt für alle $i,j \in \{1,...n\}$:
$$
\frac{\partial^2}{\partial x_i \partial x_j}f = \frac{\partial^2}{\partial x_j \partial x_i}f
$$
Es ist also gleich, ob man zuerst partiell nach $x_j$ und dann nach $x_i$ ableitet oder andersherum.

\end{frame}
%
%
\begin{frame}\frametitle{Die Jacobi-Matrix}
Es sei $f: D \subset \R^n \to \R^m, (x_1,...,x_n) \mapsto \begin{pmatrix}f_1(x_1,...,x_n)\\ f_2(x_1,...,x_n)\\ ...\\f_m(x_1,...,x_n) \end{pmatrix}$, dann ist die \highlightDef{Jacobi-Matrix} von $f$ and der Stelle $x \in D$ definiert als
$$
J_f(x):=\left(\frac{\partial}{\partial x_j}f_i(x)   \right)_{i=1,j=1}^{m,n} 
= \begin{pmatrix} \frac{\partial}{\partial x_1}f_1(x) &\frac{\partial}{\partial x_2}f_1(x) &...&\frac{\partial}{\partial x_n}f_1(x) \\
\frac{\partial}{\partial x_1}f_2(x) & ... & ... & \frac{\partial}{\partial x_n}f_2(x) \\ ...&&&...\\\frac{\partial}{\partial x_1}f_m(x)  & \frac{\partial}{\partial x_2}f_m(x)  & ... &\frac{\partial}{\partial x_n}f_m(x)  \end{pmatrix}
$$
Die Jacobi-Matrix übernimmt die Rolle der ``Ableitung'', wie wir auch im folgenen sehen werden.
\end{frame}
%
%
\begin{frame}\frametitle{Diffenzierbarkeit}
Eine Funktion $f: D \subset \R^n \to \R^m$ heißt \highlightDef{differenzierbar in $x_0$}, wenn es eine Matrix $A \in \R^{m \times n}$ gibt, sodass
$$
\lim_{h \to 0} \frac{f(x_0+h)-f(x_0)-Ah}{\|h\|}=0
$$
gilt.\\
Hierbei ist für $h \in \R^n$ die \textbf{Norm} $\|h\|:= \sqrt{\sum_{i=1}^n h_i^2}$.
\vfill \pause
In diesem Fall ist die Matrix $A$ eindeutig bestimmt und es gilt $A=J_f(x_0)$.
\end{frame}
%
%
\begin{frame}\frametitle{Beispiel}
$f: \R^2 \to \R^2, (x_1,x_2) \mapsto \begin{pmatrix}x_1x_2 \\ x_1+x_2 \end{pmatrix}$\\
Dann:
$$
J_f(x_1,x_2)= \begin{pmatrix} x_2& x_1 \\ 1 & 1 \end{pmatrix}
$$\pause
und für festes $(x_1,x_2) \in \R^2$ und $h=(h_1,h_2)$ gilt
\begin{align*}
&\frac{f(x_1+h_1,x_2+h_2)-f(x_1,x_2)-J_f(x_1,x_2)h}{\|h\|}\\
=&\frac{\begin{pmatrix} (x_1+h_1)(x_2+h_2) \\ x_1+x_2+h_1+h_2 \end{pmatrix}- \begin{pmatrix} x_1 x_2 \\ x_1+x_2 \end{pmatrix}-\begin{pmatrix} x_2& x_1 \\ 1 & 1 \end{pmatrix} \begin{pmatrix} h_1 \\ h_2 \end{pmatrix}}{\|h\|}\\
%=&\frac{\begin{pmatrix} x_1h_2+x_2h_1+h_1h_2) \\ h_1+h_2 \end{pmatrix}- \begin{pmatrix} x_1h_2+x_2h_1+h_1h_2) \\ h_1+h_2 \end{pmatrix}}{\|h\|}\\
=&\frac{\begin{pmatrix}h_1h_2\\ 0 \end{pmatrix}}{\sqrt{h_1^2+h_2^2}}=\frac{1}{\sqrt{\frac{h_1^2}{h_1^2h_2^2}+\frac{h_2^2}{h_1^2h_2^2}}}\begin{pmatrix}1\\ 0 \end{pmatrix} \to 0 \quad (h \to 0)
\end{align*}
	
\end{frame}
%
%
\begin{frame}\frametitle{Kettenregel}
Es seien $f: D \subset \R^m \to \R^n$ und $g: E \subset \R^n \to \R^k$ zwei (in $x_0$) differenzierbare Funktionen mit $f(D) \subset E$. \\
Dann ist $g\circ f : D \subset \R^m \to \R^k$ differenzierbar (in $x_0$) und es gilt
$$
J_{g \circ f}(x_0)= J_g(f(x_0))\cdot J_f(x_0)
$$
\vfill \pause
\highlightDef{Bemerkung:}\\ Da $J_g(f(x_0)) \in \R^{k \times n}$ und $J_f(x_0) \in \R^{n \times m}$ gilt, ist die Reihenfolge im Produkt relevant.
\end{frame}
%
%
\begin{frame}\frametitle{Beispiel}
$f: \R^2 \to \R, (x,y)  \mapsto x+y$ und $g: \R \to \R^3, u \mapsto \begin{pmatrix} 1 \\ u \\ u^2 \end{pmatrix} $
Dann gilt:
$J_f(x,y)=\begin{pmatrix} 1 & 1  \end{pmatrix} \in \R^{1 \times 2}$ und $J_g(u)=\begin{pmatrix} 0 \\ 1 \\ 2u \end{pmatrix} \in \R^{3 \times 1}$\\
\pause Damit folgt: 
\begin{align*}
J_{g\circ f}(x,y)&=\begin{pmatrix} 0 \\ 1 \\ 2f(x,y) \end{pmatrix}\begin{pmatrix} 1 & 1  \end{pmatrix}=\begin{pmatrix} 0 \\ 1 \\ 2x+2y \end{pmatrix}\begin{pmatrix} 1 & 1  \end{pmatrix}\\
&=\begin{pmatrix} 0 & 0 \\ 1 & 1 \\ 2x+2y & 2x+2y \end{pmatrix}
\end{align*}

\end{frame}
%
%
\begin{frame}\frametitle{Beispiel}
$f: \R^2 \to \R, (x,y)  \mapsto x+y$ und $g: \R \to \R^3, u \mapsto \begin{pmatrix} 1 \\ u \\ u^2 \end{pmatrix} $\\
Als Probe berechnen wir die Jacobi-Matrix von 
$$
h=g \circ f : \R^2 \to \R^3, (x,y) \mapsto \begin{pmatrix}1 \\ x+y \\ (x+y)^2 \end{pmatrix}
$$
direkt über die partiellen Ableitungen nach $x$ und $y$:\pause
\begin{align*}
&\frac{\partial}{\partial x}h_1(x,y)=0 &&\frac{\partial}{\partial y}h_1(x,y)=0\\
&\frac{\partial}{\partial x}h_2(x,y)=1 &&\frac{\partial}{\partial y}h_2(x,y)=1 \\
&\frac{\partial}{\partial x}h_3(x,y)=2(x+y)&&\frac{\partial}{\partial y}h_3(x,y)=2(x+y)
\end{align*}
\end{frame}
%
\subsection{Mehrdimensionales Newton-Verfahren}
%
\begin{frame}\frametitle{Problemstellung}
Bestimme (näherungsweise) eine ``Nullstelle'' einer differenzierbaren Funktion
$
f: \R^n \to \R^n
$
d.h. ein $x_0 \in \R^n$ mit $f(x_0)=\begin{pmatrix}f_1(x_0)\\...\\f_n(x_0) \end{pmatrix}=\begin{pmatrix} 0 \\ ... \\ 0 \end{pmatrix}=0 \in \R^n$
\pause
\vfill
Für den Fall $n=1$ wurde früher in der Vorlesung das Newton-Verfahren mit der Iteration 
$$
x^{(k+1)}=x^{(k)} - \frac{f(x^{(k)})}{f'(x^{(k)})}
$$
eingeführt.
\pause
\vfill
Wie kann man das für $n>1$ übertragen?
\end{frame}
%
%
\begin{frame}\frametitle{Mehrdimensionales Newton-Verfahren}
Es sei $f: \R^n \to \R^n$ eine differenzierbare Funktion und $x^{(1)} \in \R^n$. Dann ist die Iteration des \highlightDef{$n$-dimensionalen Newton-Verfahren} gegeben durch
$$
x^{(k+1)}=x^{(k)} - J_f(x^{(k)})^{-1}\cdot f(x^{(k)})
$$


\end{frame}
%
%
\begin{frame}\frametitle{Beispiel}
Bestimmen Sie näherungsweise eine Lösung des \textbf{nicht-linearen} Gleichungssystem
\begin{align*}
x-\frac{1}{5}y^2 &=0\\
x^2+(y-3)^2&=8
\end{align*}
\pause Dies ist äquivalent zu
\begin{align*}
f(x,y)=\begin{pmatrix} 0\\0\end{pmatrix}
\end{align*}
mit $f: \R^2 \to \R^2, (x,y) \mapsto \begin{pmatrix} x-\frac{1}{5}y^2 \\ x^2+(y-3)^2-8\end{pmatrix}$	
\end{frame}
%
\begin{frame}\frametitle{Beispiel}
Wir berechnen einen Schritt mit den $2$-dimensionale Newton-Verfahren zum Startwert $x^{(1)}=(0,0)^t$. Als erstes bestimmen wir dafür die Jacobi-Matrix:
$$
J_f(x,y)=\begin{pmatrix}1 & - \frac{2}{5}y\\ 2x & 2(y-3) \end{pmatrix}
$$\pause
Damit
\begin{align*}
x^{(2)}&=x^{(1)} - J_f(x^{(1)})^{-1}\cdot f(x^{(1)})\\
&=\begin{pmatrix} 0 \\0 \end{pmatrix} - \begin{pmatrix}1 & 0\\0 & -6 \end{pmatrix}^{-1}\cdot \begin{pmatrix} 0 \\1 \end{pmatrix} \\
&=- \begin{pmatrix}1 & 0\\0 & -\frac{1}{6} \end{pmatrix}\cdot \begin{pmatrix} 0 \\1 \end{pmatrix}\\
&=\begin{pmatrix} 0 \\\frac{1}{6} \end{pmatrix}
\end{align*}
\end{frame}

\begin{frame}\frametitle{Beispiel}
\begin{align*}
x^{(3)}&=x^{(2)} - J_f(x^{(2)})^{-1}\cdot f(x^{(2)})\\
&=\begin{pmatrix} 0 \\\frac{1}{6}  \end{pmatrix} - \begin{pmatrix}1 &-\frac{1}{15}\\0  & -\frac{17}{3} \end{pmatrix}^{-1}\cdot \begin{pmatrix} -\frac{1}{180} \\ \frac{1}{36} \end{pmatrix} \\
&=\begin{pmatrix} 0 \\\frac{1}{6}  \end{pmatrix} - \begin{pmatrix}1 & -\frac{1}{85}\\0 & -\frac{3}{17} \end{pmatrix}\cdot \begin{pmatrix} -\frac{1}{180} \\\frac{1}{36} \end{pmatrix}\\
&=\begin{pmatrix} 0 \\\frac{1}{6} \end{pmatrix}-\begin{pmatrix} -\frac{1}{170} \\-\frac{1}{204} \end{pmatrix}\\
&=\begin{pmatrix} \frac{1}{170} \\\frac{35}{204} \end{pmatrix}
\end{align*}
\pause
Damit: $f(x^{(1)})=\begin{pmatrix}0 \\1 \end{pmatrix}$ \pause , $f(x^{(2)})=\begin{pmatrix}-\frac{1}{180} \\ \frac{1}{36} \end{pmatrix} \thickapprox \begin{pmatrix}-0,005 \\ 0,03 \end{pmatrix} $ \pause und $f(x^{(2)})=\begin{pmatrix}- \frac{1}{208080} \\\frac{61}{1040400} \end{pmatrix}\thickapprox \begin{pmatrix}-0,000005 \\ 0,00006 \end{pmatrix} $
\end{frame}

\section{Mehrdimensionale Integration}
\makeSectionDividerPage
%%%
\begin{frame}\frametitle{Motivation}
\begin{columns}[T] % align columns
\begin{column}{.48\textwidth}
		\begin{center}
			\includegraphics[scale=0.75]{Grafiken/3D_Graph/3D_graph.pdf}
		\end{center}
\end{column}%
\hfill%
\begin{column}{.48\textwidth}
Welches (3-dimensionale) Volumen ist zwischen der Fläche und der $x_1$-$x_2$-Ebene eingeschlossen? \\\pause \quad\\\quad\\
$\Rightarrow$ Integration von Funktionen\\ \quad \ $f:\R^n \to \R$.
\end{column}%
\end{columns}

\end{frame}
%
\subsection{Quader}
%
%
\begin{frame}\frametitle{Definition}
Es seien $a, b \in \R^n$ mit $a_j < b_j$ für alle $j \in \{1,...,n\}$. Dann nennt man 
\begin{align*}
Q(a,b)&:= \big\{x \in \R^n \mid a_j \le x_j \le b_j\ \forall j \in \{1,...,n\} \big\}\\
&\ =[a_1,b_1] \times [a_2,b_2] \times ... \times [a_n,b_n]
\end{align*}
einen \highlightDef{$n$-dimensionalen Quader}.\pause \vfill
Sei nun $f: D \subset \R^n \to \R$ eine stetige Funktion, $a,b \in \R^n$ so, dass $Q(a,b) \subset D$. Dann definieren wir das Integral von $f$ über $Q(a,b)$ als
$$
\int_{Q(a,b)} f(x) dx := \int_{a_n}^{b_n}...\left(\int_{a_2}^{b_2}\left(\int_{a_1}^{b_1} f(x_1,...,x_n) dx_1\right)dx_2\right)...dx_n
$$
\end{frame}
%
%
\begin{frame}\frametitle{Beispiel}
$f: \R^2 \to \R, (x_1,x_2) \mapsto x_1^2+x_2^2$ und $Q=[0,1]\times[-1,2]$.\\
Dann:\pause
\begin{align*}
\int_Q f(x)dx&=\int_{-1}^2 \int_0^1 x_1^2+x_2^2 dx_1dx_2\\
&=\int_{-1}^2 \left[\frac{1}{3}x_1^3+x_2^2x_1\right]_{x_1=0}^{x_1=1} dx_2\\
&=\int_{-1}^2 \frac{1}{3}+x_2^2 dx_2\\
&=\left[ \frac{1}{3}x_2+\frac{1}{3}x_2^3\right]_{x_2=-1}^{x_2=2}\\
&=4
\end{align*}
\end{frame}
%
%
\begin{frame}\frametitle{Satz von Fubini für Quader}
Es sei $f: D \subset \R^n \to \R$ eine stetige Funktion, $a,b \in \R^n$ so, dass $Q(a,b) \subset D$. Weiter sei weiter $\sigma \in Sym(n)$ eine beliebige Permutation der Zahlen $1,...,n$.\\
 Dann gilt:
\small
$$
\int_{Q(a,b)}f(x)dx=\int_{a_{\sigma(n)}}^{b_{\sigma(n)}}...\left(\int_{a_{\sigma(2)}}^{b_{\sigma(2)}}\left(\int_{a_{\sigma(1)}}^{b_{\sigma(1)}} f(x_1,...,x_n) dx_{\sigma(1)}\right)dx_{\sigma(2)}\right)...dx_{\sigma(n)}
$$	
\textnormal
D.h. die Reihenfolge der Integration kann beliebig vertauscht werden.	
\end{frame}
%
%
\begin{frame}\frametitle{Beispiel}
Es sei $f: \R^2 \to \R, (x_1,x_2) \mapsto \sin(x_2)(x_1e^{\cos(3x_1)})$ und $Q=[-2,2]\times[-\frac{\pi}{2},\frac{\pi}{2}]$ \pause
\begin{align*}
\int_Q f(x)dx&= \int_{-\frac{\pi}{2}}^{\frac{\pi}{2}}\int_{-2}^2\sin(x_2)(x_1e^{\cos(3x_1)}) dx_1dx_2\\
&=\int_{-2}^2 \int_{-\frac{\pi}{2}}^{\frac{\pi}{2}}\sin(x_2)(x_1e^{\cos(3x_1)}) dx_2dx_1\\
&=\int_{-2}^2  \left[ -\cos(x_2) (x_1e^{\cos(3x_1)})\right]_{x_2=-\frac{\pi}{2}}^{x_2=\frac{\pi}{2}}dx_1\\
&=\int_{-2}^2 0 dx_1\\
&=0
\end{align*}	
\end{frame}
%
\begin{frame}\frametitle{Beispiel}
		\begin{center}
			\includegraphics[scale=0.75]{Grafiken/Fubini_1.png}
		\end{center}
	
	
\end{frame}
%
\subsection{Normalbereiche in $\R^2$}
%
\begin{frame}\frametitle{Definition: Normalbereich in $\R^2$}
Es sei $[a,b] \in \R$ mit $a<b$ und $h_1,h_2: [a,b] \to \R$ stetige Funktionen. Dann nennt man 
$$
A=\{(x,y) \mid a\le x\le b \text{ und } h_1(x) \le y \le h_2(x)\} \subset \R^2
$$
einen \highlightDef{Normalbereich} bezüglich der \textbf{x-Achse}.\\\pause \vfill
Analog nennt man eine Menge 
$$
B=\{(x,y) \mid a\le y\le b \text{ und } h_1(y) \le x \le h_2(y)\}\subset \R^2
$$
einen \highlightDef{Normalbereich} bezüglich der \textbf{y-Achse}.	
	
\end{frame}
%
%
\begin{frame}\frametitle{Integral über einem Normalbereich in $\R^2$}
Es sei $A \subset \R^2$ ein Normalbereich bezüglich der $x$-Achse (wie oben) und $f:A \to \R$ eine stetige Funktion. Dann ist das Integral von $f$ über $A$ definiert als
$$
\int_A f(x,y)d(x,y):= \int_a^b \left(\int_{h_1(x)}^{h_2(x)} f(x,y) dy \right)dx
$$
\pause \vfill
Analog:\\
Es sei $B \subset \R^2$ ein Normalbereich bezüglich der $y$-Achse (wie oben) und $f:B \to \R$ eine stetige Funktione. Dann ist das Integral von $f$ über $B$ definiert als
$$
\int_B f(x,y)d(x,y):= \int_a^b \left(\int_{h_1(y)}^{h_2(y)} f(x,y) dx \right)dy
$$
	
\end{frame}
%
\begin{frame}\frametitle{Beispiel}
$f(x,y)=xy$ und $A=\{(x,y) \mid 0\le x \le 1 \text{ und } x \le y \le \sqrt{x} \}$ \pause
\begin{align*}
\int_A f(x,y) d(x,y) &= \int_0^1 \left(\int_x^{\sqrt{x}} xy dy\right)dx\\
&=\int_0^1 \left[\frac{1}{2}xy^2\right]_{y=x}^{y=\sqrt{x}} dx\\
&=\int_0^1 \frac{1}{2}x^2-\frac{1}{2}x^3)dx\\
&=\left[ \frac{1}{6}x^3 -  \frac{1}{6}x^4\right]_{x=0}^{x=1}\\
&= \frac{1}{6}- \frac{1}{8}\\
&= \frac{1}{24}
\end{align*}
		
\end{frame}
%
\subsection{Normalbereiche in $\R^3$}
%
\begin{frame}\frametitle{Definition: Normalbereich in $\R^3$}
Sei $A \subset \R^2$ ein Normalbereich bezüglich der x-Achse oder der y-Achse und seien $g_1,g_2:A \to \R$ stetige Funktionen. Dann heißt
$$
\Omega=\{(x,y,z) \in \R^3 \mid (x,y) \in A \text{ und } g_1(x,y) \le z \le g_2(x,y)\} \subset \R^3
$$
ein \highlightDef{Normalbereich} bezüglich der \textbf{x-y-Ebene}.
\pause\vfill
Analog definiert man Normalbereiche bezüglich der x-z-Ebene und der y-z-Ebene.
\end{frame}
%
%
\begin{frame}\frametitle{Integral über einem Normalbereich in $\R^3$}
Es sei $\Omega$ ein Normalbereich bezüglich der x-y-Ebene mit Normalbereich $A$ und Funktionen $g_1,g_2$ (wie oben) und $f:\Omega \to \R$ eine stetige Funktion. Dann ist das Integral von $f$ über $\Omega$ definiert als
$$
\int_\Omega f(x,y,z)dx:= \int_A \left(\int_{g_1(x,y)}^{g_2(x,y)} f(x,y,z) dz \right)d(x,y)
$$
\pause \vfill
Analog definiert man das Integral über Normalbereiche bezüglich der x-z-Ebene und der y-z-Ebene.
	
	
\end{frame}
%
\begin{frame}\frametitle{Beispiel}
Es sei $A=\{(x,y) \mid -1 \le x \le 1 \text{ und } -\sqrt{1-x^2} \le y \le \sqrt{1-x^2}\}$ und $\Omega=\{(x,y,z)\mid (x,y) \in A \text{ und } 0 \le z \le 1\}$. 
\begin{align*}
\int_\Omega 1 d(x,y,z) &= \int_A \left( \int_{0}^{1} 1 dz \right) d(x,y)\\
&=\int_{-1}^1 \left( \int_{-\sqrt{1-x^2}}^{\sqrt{1-x^2}} 1 dy\right)dx\\
&=\int_{-1}^1 2 \sqrt{1-x^2} dx\\
&\stackrel{(*)}{=}\pi 
\end{align*}
wobei bei $(*)$ zuerst die Substitution $\sin(u)=x$ durchgeführt und dann im Anschluss zweimal partiell integriert wird.
	
\end{frame}
%



\end{document}