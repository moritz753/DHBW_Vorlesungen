\documentclass[
				a4paper,
				10pt
			]
			{scrartcl}

\parindent0mm

\usepackage{amsfonts}
\usepackage{amsmath}
\usepackage{amssymb}
\usepackage{amsthm}
\usepackage[ngerman]{babel}

\usepackage[utf8]{inputenc}
\usepackage[T1]{fontenc}
\usepackage{textcomp}

\usepackage{graphicx}
\usepackage{xcolor}

\usepackage[
			pdftex,
			colorlinks,
			breaklinks,
			linkcolor=blue,
			citecolor=blue,
			filecolor=black,
			menucolor=black,
			urlcolor=black,
			pdfauthor={Andreas Weber},
			pdftitle={Aufgaben zu Analysis und Lineare Algebra},
			plainpages=false,
			pdfpagelabels,
			bookmarksnumbered=true
		   ]{hyperref}


%--------------------------------------%
% Math ------------------------------%
%--------------------------------------%

% Mengen (Zahlen)
\newcommand{\N}{\mathbb{N}}
\newcommand{\Q}{\mathbb{Q}}
\newcommand{\R}{\mathbb{R}}
\newcommand{\Z}{\mathbb{Z}}
\newcommand{\C}{\mathbb{C}}

% Mengen (allgemein)
\newcommand{\K}{\mathbb{K}}
\newcommand\PX{{\cal P}(X)}

% Zahlentheorie
\newcommand{\ggT}{\mathrm{ggT}}


% Ableitungen
\newcommand{\ddx}{\frac{d}{dx}}
\newcommand{\pddx}{\frac{\partial}{\partial x}}
\newcommand{\pddy}{\frac{\partial}{\partial y}}
\newcommand{\grad}{\text{grad}}

%--------------------------------------%
% Layout Colors ------------------%
%--------------------------------------%
\newcommand*{\highlightDef}[1]{{\color{lightBlue}#1}}
\newcommand*{\highlight}[1]{{\color{lightBlue}#1}}
% Color Definitions
\definecolor{dhbwRed}{RGB}{226,0,26} 
\definecolor{dhbwGray}{RGB}{61,77,77}
\definecolor{lightBlue}{RGB}{28,134,230}


%-------------------------------------------------------------------
\begin{document}

\vspace*{-20mm}
{
	%\usekomafont{title}
	\color{dhbwGray}
	Dr. Moritz Gruber	\hfill Version \today\\
	DHBW Karlsruhe\\
}

\vspace{10mm}
\begin{center}
	{
		\usekomafont{title}
		%\color{lightBlue}
		{ \LARGE 	Übungsaufgaben 5 - Lösungen}\\[3mm]
		{\Large Mehrdimensionale Analysis - Teil 2}
	}
\end{center}

\vspace{5mm}

%-------------------------------------------------------------------


%%-------------------------------------------------------------------
%\section{Jacobi-Matrix}
%%%%
%Bestimmen Sie die Jacobi-Matrix der Funktion
%$$
%f: \R^4 \to \R^3, \begin{pmatrix}x_1 \\ x_2\\x_3\\x_4\end{pmatrix} \mapsto \begin{pmatrix} x_1+x_2x_3 \\ x_4^2 \\ \sin(x_1)\cos(x_4)\end{pmatrix}
%$$
%%---------------------------------
%\subsection*{Lösung}
%%%%
%Die Einträge der Jacobi-Matrix sind die partiellen Ableitungen $\frac{\partial f_i}{\partial x_j}$. Daher muss man als erstes diese bestimmen:
%\begin{align*}
%\frac{\partial f_1}{\partial x_1} &=1  &\frac{\partial f_1}{\partial x_2}&=x_3  &\frac{\partial f_1}{\partial x_3}&=x_2  &\frac{\partial f_1}{\partial x_4}&=0  \\
%\frac{\partial f_2}{\partial x_1} &=0  &\frac{\partial f_2}{\partial x_2}&=0  &\frac{\partial f_2}{\partial x_3}&=0  &\frac{\partial f_2}{\partial x_4}&= 2x_4 \\
%\frac{\partial f_3}{\partial x_1} &=\cos(x_1)\cos(x_4)  &\frac{\partial f_3}{\partial x_2}&=0  &\frac{\partial f_3}{\partial x_3}&=0  &\frac{\partial f_3}{\partial x_4}&=-\sin(x_1)\sin(x_4)  
%\end{align*}
%Damit erhalten wir
%$$
%J_f(x)=\left( \frac{\partial f_i}{\partial x_j}\right)_{i=1,j=1}^{3,4} =\begin{pmatrix} 1 & x_3 & x_2 & 0 \\ 0 & 0& 0& 2x_4 \\ \cos(x_1)\cos(x_4) & 0&0& -\sin(x_1)\sin(x_4)   \end{pmatrix}
%$$
%
%%-------------------------------------------------------------------
%\newpage
%\section{Newton-Verfahren}
%%%%
%Bestimmen Sie näherungsweise eine Stelle $x \in \R^3$ mit $f(x)=(0,0,0)^t \in \R^3$ für die Funktion
%$$
%f: \R^3 \to \R^3, \begin{pmatrix} x_1 \\ x_2 \\x_3\end{pmatrix} \mapsto \begin{pmatrix} 2x_1-x_2 \\ 3x_2-x_3 \\ x_1x_2x_3-1 \end{pmatrix}
%$$
%indem Sie für den Startwert $x^{(0)}=(1,1,0)^t$ einen Schritt des mehrdimensionalen Newton-Verfahrens durchführen.
%
%
%%---------------------------------
%\subsection*{Lösung}
%%%%
%Um die Iteration 
%$$
%x^{(k+1)}=x^{(k)} - J_f(x^{(k)})^{-1}\cdot f(x^{(k)})
%$$
%des mehrdimensionale Newton-Verfahren durchzuführen, benötigen wir die Jacobi-Matrix der Funktion $f$. Diese ergibt sich - wie in Aufgabe 1 - aus den partiellen Ableitungen:
%\begin{align*}
%\frac{\partial f_1}{\partial x_1} &=2  &\frac{\partial f_1}{\partial x_2}&=-1 &\frac{\partial f_1}{\partial x_3}&=0    \\
%\frac{\partial f_2}{\partial x_1} &=0  &\frac{\partial f_2}{\partial x_2}&=3 &\frac{\partial f_2}{\partial x_3}&=-1  \\
%\frac{\partial f_3}{\partial x_1} &=x_2x_3 &\frac{\partial f_3}{\partial x_2}&=x_1x_3 &\frac{\partial f_3}{\partial x_3}&=x_1x_2
%\end{align*}
%Damit erhalten wir
%$$
%J_f(x)=\left( \frac{\partial f_i}{\partial x_j}\right)_{i=1,j=1}^{3,3} =\begin{pmatrix} 2 & -1 & 0 \\ 0 & 3& -1 \\ x_2x_3&x_1x_3&x_1x_2   \end{pmatrix}
%$$
%Als Startwert wurde $x^{(0)}=(1,1,0)^t$ gewählt, und erhalten damit
%$$
%f(x^{(0)})=\begin{pmatrix} 1\\3\\-1 \end{pmatrix}
%$$
%Dann müssen wir noch die Inverse von 
%$$
%J_f(x^{(0)})=\begin{pmatrix} 2 & -1 & 0 \\ 0 & 3& -1 \\ 0&0&1   \end{pmatrix}
%$$
%bestimmen. Dies tun wir mit dem Gauß-Algorithmus
%\begin{align*}
%&\left(
%		\begin{array}{ccc|ccc}
%			2&-1&0 	&1&0&0\\
%			0&3&-1	&0&1&0\\
%			0&0&1	&0&0&1
%		\end{array}
%		\right) \stackrel{II+III, \frac{1}{3}II}{\sim>} 
%				\left(
%		\begin{array}{ccc|ccc}
%			2&-1&0 	&1&0&0\\
%			0&1&0	&0&\frac{1}{3}&\frac{1}{3}\\
%			0&0&1	&0&0&1
%		\end{array}
%		\right)\\
%		\stackrel{I+II, \frac{1}{2}I}{\sim>} 
%				&\left(
%		\begin{array}{ccc|ccc}
%			1&0&0 	&\frac{1}{2}&\frac{1}{6}&\frac{1}{6}\\
%			0&1&0	&0&\frac{1}{3}&\frac{1}{3}\\
%			0&0&1	&0&0&1
%		\end{array}
%		\right)
%\end{align*}
%Somit ist 
%$$
%J_f(x^{(0)})^{-1}=\begin{pmatrix} \frac{1}{2}&\frac{1}{6}&\frac{1}{6}\\
%			0&\frac{1}{3}&\frac{1}{3}\\
%			0&0&1 \end{pmatrix}
%$$
%Jetzt haben wir alles um die Iteration den Newton-Verfahrens durchzuführen:
%\begin{align*}
%x^{(1)}&=x^{(0)} - J_f(x^{(0)})^{-1}\cdot f(x^{(0)})\\
%&=\begin{pmatrix} 1\\1\\0 \end{pmatrix} - \begin{pmatrix} \frac{1}{2}&\frac{1}{6}&\frac{1}{6}\\0&\frac{1}{3}&\frac{1}{3}\\ 0&0&1 \end{pmatrix} \cdot \begin{pmatrix} 1\\3\\-1 \end{pmatrix}\\
%&=\begin{pmatrix} 1\\1\\0 \end{pmatrix} -\begin{pmatrix} \frac{1}{2}\cdot 1 + \frac{1}{6}\cdot 3 + \frac{1}{6}\cdot (-1) \\ \frac{1}{3}\cdot 3 + \frac{1}{3}\cdot (-1)\\ 1 \cdot (-1) \end{pmatrix}\\
%&=\begin{pmatrix} 1\\1\\0 \end{pmatrix} -\begin{pmatrix} \frac{5}{6}\\ \frac{2}{3}\\ -1 \end{pmatrix}\\
%&=\begin{pmatrix} \frac{1}{6}\\\frac{1}{3}\\1 \end{pmatrix}
%\end{align*}
%Setzen wir nun $x^{(1)}$ in $f$ ein, so erhalten wir:
%$$
%f(x^{(1)})=\begin{pmatrix} 0\\0\\-\frac{17}{18} \end{pmatrix}
%$$
%-------------------------------------------------------------------
%\newpage
\section{Integral über Quader}
%%%
Berechnen Sie das Integral der Funktion
$$
f:\R^4 \to \R, (x_1,x_2,x_3,x_4) \mapsto x_1x_2x_3x_4
$$
über dem 4-dimensionalen Quader
$$
Q=[0,1]\times[0,1]\times[0,1]\times[0,1]
$$


%---------------------------------
\subsection*{Lösung}
%%%
\begin{align*}
\int_Q f(x)dx&=\int_0^1 \left(\int_0^1 \left(\int_0^1 \left(\int_0^1 x_1x_2x_3x_4 \ d{x_4} \right)d{x_3} \right)d{x_2} \right)d{x_1} \\
&=\int_0^1 \left(\int_0^1 \left(\int_0^1 \left[ \frac{1}{2} x_1x_2x_3x_4^2\right]_{x_4=0}^{x_4=1}d{x_3} \right)d{x_2} \right)d{x_1} \\
&=\int_0^1 \left(\int_0^1 \left(\int_0^1\frac{1}{2} x_1x_2x_3 \ dx_3\right)d{x_2} \right)d{x_1} \\
&=\int_0^1 \left(\int_0^1 \left[ \frac{1}{2}\cdot\frac{1}{2} x_1x_2x_3^2\right]_{x_3=0}^{x_3=1} d{x_2} \right)d{x_1} \\
&=\int_0^1 \left(\int_0^1 \frac{1}{4} x_1x_2 \ d{x_2} \right)d{x_1} \\
&=\int_0^1 \left[ \frac{1}{2}\cdot\frac{1}{4} x_1x_2^2\right]_{x_2=0}^{x_2=1} d{x_1} \\
&=\int_0^1 \frac{1}{8} x_1 \ d{x_1} \\
&= \left[ \frac{1}{2}\cdot\frac{1}{8} x_1^2\right]_{x_1=0}^{x_1=1} \\
&= \frac{1}{16}
\end{align*}

%-------------------------------------------------------------------
\newpage
\section{Integral über Normalbereich}
%%%
Es sei $f: \R^2 \to \R, (x,y) \mapsto xy$ und $B=\{(x,y) \mid 0\le y\le 1 \text{ und } y^2 \le x \le y\}$. Berechnen Sie das Integral
$$
\int_B f(x,y)d(x,y)
$$
 
 %---------------------------------
\subsection*{Lösung}
%%%
Da $B$ ein Normalbereich bezüglich der y-Achse ist, berechnet sich das Integral wie folgt:
\begin{align*}
\int_B f(x,y)d(x,y)&=\int_0^1 \left( \int_{y^2}^y xy \ dx \right) dy\\
&= \int_0^1 \left[ \frac{1}{2}x^2y\right]_{x=y^2}^{x=y} dy\\
&=\frac{1}{2}\int_0^1  y^3-y^5 \ dy \\
&=\frac{1}{2} \left[ \frac{1}{4}y^4 - \frac{1}{6} y^6 \right]_{y=0}^{y=1}\\
&=\frac{1}{2}( \frac{1}{4} - \frac{1}{6})\\
&=\frac{1}{24}
\end{align*}


\end{document}