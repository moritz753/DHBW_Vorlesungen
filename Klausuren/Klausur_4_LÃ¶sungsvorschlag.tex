\documentclass[a4paper,11pt]{scrartcl}

\usepackage{graphicx}
\usepackage[T1]{fontenc}
\usepackage[utf8]{inputenc}
\usepackage{floatflt}
\usepackage{amsfonts}
\usepackage{amsmath}
\usepackage{amssymb}
\usepackage{amsthm}
\usepackage[ngerman]{babel}
%\usepackage[T1]{fontenc}


\usepackage{paralist}
\usepackage[usenames,dvipsnames]{color} 



\newcounter{auf}
\newcommand{\Aufgabe}%
        {\addtocounter{auf}{1} \subsubsection*{\rmfamily  Aufgabe \theauf \hspace{1em}} }









\pagestyle{empty}

\oddsidemargin0mm
\parindent0mm
\parskip2mm
\textheight24cm
\textwidth15.8cm
\unitlength1mm

\newcommand{\E}{\mathbb{E}}
\newcommand{\Hy}{\mathbb{H}}
\newcommand{\N}{\mathbb{N}}
\newcommand{\RR}{\mathbb{R}}
\newcommand{\Z}{\mathbb{Z}}
\newcommand{\Q}{\mathbb{Q}}
\newcommand{\C}{\mathbb{C}}
\newcommand{\K}{\mathbb{K}}
\newcommand*\e{\mathrm{e}}
\newcommand*\ii{\mathrm{i}}
\newcommand*\re{\mathrm{Re}}
\newcommand*\im{\mathrm{Im}}
\newcommand*\id{\mathrm{id}}
\newcommand*\glnr{\mathrm{\it GL}(n,\R)}
\newcommand*\slnr{\mathrm{\it SL}(n,\R)}
\newcommand*\on{\mathrm{\it O}(n)}
\newcommand*\son{\mathrm{\it SO}(n)}
\newcommand*\rang{\mathrm{Rang}}
\newcommand*\grad{\mathrm{grad~}}
\newcommand*\dive{\mathrm{div~}}
\newcommand*\sym{\mathrm{Sym}}
\newcommand*\spur{\mathrm{Spur}}
\newcommand*\isom{\mathrm{Isom}}
                  





\begin{document}





\Aufgabe

\begin{enumerate}[a)]

\item Seien $a$, $b$ und $c$  Aussagen. 
Bestimmen Sie die Wahrheitstafel der Aussage
$$
	c \land (\neg b \lor a).
$$
\item Negieren Sie die Aussage 
$$
\exists x \in \N\ \forall y \in \Z\ \exists z \in \N: (x\cdot y) > z^2
$$
und formen Sie das Ergebnis so um, dass kein Negationszeichen $\neg$ verwendet wird.
\item Es seien $A=\{1,2,3,4,5\}, B=\{1,3,5\} \text{ und } C=\{1,5,9\}.$\\
Bestimmen Sie $(A\setminus C)\cap B$.
\end{enumerate}
%
%
\subsubsection*{Lösungsvorschlag}
\begin{enumerate}[a)]

\item $$\begin{array}{c|c|c||c|c||c}
a&b&c&\neg b&\neg b\lor a&c\land(\neg b\lor a)\\ \hline
w&w&w&f&w&w\\
w&w&f&f&w&f\\
w&f&w&w&w&w\\
w&f&f&w&w&f\\
f&w&w&f&f&f\\
f&w&f&f&f&f\\
f&f&w&w&w&w\\
f&f&f&w&w&f
\end{array}$$
\item $\neg(\forall x \in \N\ \exists y \in \Z\ \forall z \in \N: (x\cdot y) > z^2) = \exists x \in \N \ \forall y \in \Z \ \exists z \in \N: (x \cdot y) \le z^2$ 
\item $(A\setminus C)\cap B=\{2,3,4\} \cap B=\{3\}$
\end{enumerate}
%
%
%%%%
%
\newpage
\Aufgabe
Es seien die Abbildungen $$f:\RR^2 \to \RR^4, (x,y) \mapsto (x, x^2, -y,-y^2)$$ und 
$$
g: \RR^4 \to \RR, (w,x,y,z) \mapsto w\cdot x \cdot y \cdot z
$$ 
gegeben. 
\begin{enumerate}[a)]
\item Sind $f$ und $g$ injektiv? Begründen Sie Ihre Antwort.
\item Sind $f$ und $g$ surjektiv? Begründen Sie Ihre Antwort.
\item Bestimmen Sie $g \circ f$. 
\end{enumerate}
%
%
\subsubsection*{Lösungsvorschlag}
\begin{enumerate}[a)]

\item $f$ ist injektiv, denn seien $(x_1,y_1),(x_2,y_2) \in \RR^2$ mit $f(x_1,y_1)=f(x_2,y_2)$, dann folgt
$$
(x_1,x_1^2,-y_1,y_1^2)=(x_2,x_2^2,-y_2,-y_2^2) \ \Rightarrow x_1=x_2 \text{ und } y_1=y_2 \Rightarrow (x_1,y_1)=(x_2,y_2)
$$
$g$ ist nicht injektiv, da $g((1,1,1,2))=2=g((2,1,1,1) \quad \text{aber} \quad (1,1,1,2) \ne (2,1,1,1)$.
\item $f$ ist nicht surjektiv, da $(0,-1,0,0) \notin f(\RR^2) \subseteq \{(x,x^2,-y,-y^2)\mid x,y \in \RR\} $.\\
$g$ ist surjektiv, denn für jedes $a \in \RR$ gilt
$$
(1,1,1,a) \in \RR^4 \quad \text{und} \quad g((1,1,1,a))=a
$$
\item
\begin{align*}
&(g\circ f)(x,y)=g(f(x,y))=g((x,x^2,-y,-y^2))=x\cdot x^2\cdot (-y)\cdot(-y^2)=x^3y^3\\
\Rightarrow&\quad g\circ f :\RR^2 \to \RR, (x,y) \mapsto x^3y^3
\end{align*}
\end{enumerate}



%%%%%
\newpage
\Aufgabe
Es sei 
$$K:=\{a+b\sqrt{3} \mid a,b \in  \Q \} \subset \RR$$
zusammen mit den Verknüpfungen $+$ und $\cdot$ wie in $\RR$ gegeben.
\begin{enumerate}[a)]
\item Geben Sie die Definition eines Körpers an.

\item Zeigen Sie, dass $(K,+,\cdot)$ ein Körper ist.

\item Bestimmen Sie das multiplikative Inverse zu $z=2+3\sqrt{3} \in K$. 

\end{enumerate}

\subsubsection*{Lösungsvorschlag}
\begin{enumerate}[a)]

\item Ein Körper ist eine Menge $K$ zusammen mit zwei Verknüpfungen $+$ und $\cdot$, sodass
\begin{enumerate}[i)]
\item $(K,+)$ eine abelsche Gruppe ist (mit Neutralelement $0$),
\item $(K\setminus\{0\},\cdot)$ eine abelsche Gruppe ist,
\item $+$ und $\cdot$ die Distributivgesetze erfüllen.
\end{enumerate}

\item $K \subseteq \RR$ und $0=0+0 \cdot \sqrt{3} \in K$, $1=1+0 \cdot \sqrt{3} \in K$ sowie für $a+b \sqrt{3} \in K$ auch $-(a+b \sqrt{3})=-a-b\sqrt{3} \in K$. Außerdem gilt für alle $z_1=a+b\sqrt{3}, z_2=x+y\sqrt{3} \in K$:
\begin{align*}
z_1+z_2&=(a+x)+(b+y)\sqrt{3}\in K\\
z_1\cdot z_2&=(ax+3by)+(ay+bx)\sqrt{3} \in K
\end{align*}
Damit ist $(K,+,\cdot)$ ein kommutativer Ring mit 1, da die Assoziativität und die Kommutativität der Verknüpfungen, sowie die Distributivgesetze auf ganz $(\RR,+,\cdot)$ gelten.\\
Es bleibt zu zeigen, dass jedes Element $\ne 0$ ein multiplikatives Inverses in $K$ besitzt. Dazu setzen wir $z_1\cdot z_2=1$. Dies ergibt das LGS
\begin{align*}
ax+3by&=1\\
bx+ay&=0
\end{align*}
mit den Unbekannten $x, y \in \Q$.\\
Für den Fall $b=0$ ergibt sich die Lösung $y=0$ und $x=\frac{1}{a}$, also $z_2=z_1^{-1}=\frac{1}{a} \in K$.\\
Für den Fall $b\ne 0$ erhalten wir
$$
\left( \begin{array}{cc|c} a& 3b&1\\b&a&0 \end{array}\right) \sim> ... \sim> \left( \begin{array}{cc|c}1&0&\frac{-a}{3b^2-a^2}\\  0&1&\frac{b}{3b^2-a^2}\end{array}\right) 
$$
Damit ergibt sich $z_2=z_1^{-1}=\frac{-a}{3b^2-a^2}+\frac{b}{3b^2-a^2}\sqrt{3} \in K$.
\item Mit der Formel aus Teil b):
$$
(2+3\sqrt{3})^{-1}=\frac{-2}{3\cdot 3^2 - 2^2}+\frac{3}{3\cdot 3^2 - 2^2}\sqrt{3}=\frac{-2}{23}-\frac{3}{23}\sqrt{3}
$$

\end{enumerate}
%%%%%
\newpage
\Aufgabe

Es seien $K$ ein Körper, $V$ ein $K$-Vektorraum und $U_1$ und $U_2$
Untervektorräume von $V.$

\begin{enumerate}[a)]
\item Geben Sie die Definition eines $K$-Vektorraums an.
\item Geben Sie die Definition eines Untervektorraums an.
\item Zeigen Sie: $V=U_1\cup U_2  \ \Longleftrightarrow \ V=U_1\ \text{ oder }\ V=U_2.$
\end{enumerate}
%
%
\subsubsection*{Lösungsvorschlag}
\begin{enumerate}[a)]
\item Ein $K$-Vektorraum ist eine Menge $V$ zusammen mit einem Körper $K$ und einer Verknüpfung $+$ auf $V$ und einer Abbildung $\cdot:K\times V \to V$, sodass
\begin{enumerate}[i)]
\item $(V,+)$ eine abelsche Gruppe ist,
\item $1_K\cdot v=v$ für alle $v\in V$,
\item $\forall a,b \in K \ \forall v \in V: a\cdot (b \cdot v)=(a\cdot_K b)\cdot v$
\item die Distributivgesetze $(a+b)\cdot v=av+bv$ und $a\cdot (v+w)=av+aw$ gelten.
\end{enumerate}
\item Ein Untervektorraum eines $K$-Vektorraumes $V$ ist eine Teilmenge $U \subseteq V$, sodass $(U,+)$ eine Untergruppe von $(V,+)$ ist und für alle $a\in K$ und alle $u\in U$ gilt $au \in U$.
\item 
\begin{itemize}
\item[$\Leftarrow$:] Falls $U_1=V$ oder $U_2=V$ gilt, dann ist auch $U_1\cup U_2=V$.
\item[$\Rightarrow$:] Annahme: $U_1\ne V \ne U_2$. Dann gibt es Vektoren $u_1 \in V\setminus U_2 \subseteq U_1$ und $u_2 \in V\setminus U_1 \subseteq U_2$. Da $U_1 \cup U_2=V$ ein Vektorraum ist, folgt $u_1+u_2 \in U_1\cup U_2$.\\
Fall 1: $u_1+u_2 \in U_1$. Dann folgt da $u_1 \in U_1$ auch $u_2=(u_1+u_2)-u_1 \in U_1$. Dies ist aber ein Widerspruch zur Definition von $u_2$.\\
Fall 2: $u_1+u_2 \in U_2$. Dann folgt da $u_2 \in U_2$ auch $u_1=(u_1+u_2)-u_2 \in U_2$. Dies ist aber ein Widerspruch zur Definition von $u_1$.\\
Somit muss die Annahme falsch sein und $V=U_1$ oder $V=U_2$ gelten. \hfill $\square$
\end{itemize}
\end{enumerate}



%%%%%
\newpage
\Aufgabe

Es seien die Vektoren $b_1=\begin{pmatrix} 2\\1\\0\end{pmatrix}$, $b_2=\begin{pmatrix} 0\\1\\2\end{pmatrix}$ und $b_3=\begin{pmatrix} 1\\-1\\-1\end{pmatrix} \in \RR^3$ sowie die lineare Abbildung 
$
\Phi: \RR^3 \to \RR^2, v \mapsto \Phi(v)
$
mit 
$$
\Phi(b_1)=\begin{pmatrix} 4\\0\end{pmatrix}, \ \Phi(b_2)=\begin{pmatrix} 0\\4\end{pmatrix} \text{ und } \Phi(b_3)=\begin{pmatrix} 4\\0\end{pmatrix}
$$
gegeben.

\begin{enumerate}[a)]

\item Zeigen Sie, dass die Vektoren $b_1,b_2$ und $b_3$ linear unabhängig sind.
\item Bestimmen Sie die Abbildungsmatrix $A$ von $\Phi$ bezüglich der Standardbasis.
\item Berechnen Sie $\Phi(\begin{pmatrix} 1\\2\\4 \end{pmatrix})$.

\end{enumerate}
%
%
\subsubsection*{Lösungsvorschlag}
\begin{enumerate}[a)]
\item Die Vektoren $b_1,b_2,b_3$ sind genau dann linear unabhängig, wenn das homogene LGS 
$$
\alpha b_1 +\beta b_2 + \gamma b_3 =0
$$
nur die triviale Lösung $\alpha=\beta=\gamma=0$ hat.
$$
\left(\begin{array}{ccc|c}2&0&1&0\\1&1&-1&0\\0&2&-1&0 \end{array}\right) \sim> ... \sim>\left(\begin{array}{ccc|c} 1&1&-1&0\\ 0&1&\frac{1}{2}&0 \\ 0&0&1&0\end{array}\right)
$$
Somit hat das LGS in jeder Spalte eine Stufe und daher nur die triviale Lösung.
\item Die Standardbasisvektoren lassen sich wie folgt durch $b_1,b_2,b_3$ darstellen:
\begin{align*}
e_1&=\frac{1}{4}(b_1+b_2+2b_3)\\
e_2&=\frac{1}{2}(b_1-b_2-2b_3)\\
e_3&=\frac{1}{2}(b_2-e_2)=\frac{1}{2}(b_2-\frac{1}{2}(b_1-b_2-2b_3))=\frac{1}{4}(-b_1+3b_2+2b_3)
\end{align*}
Damit ergibt sich die Abbildungsmatrix
$$
A=\begin{pmatrix} \Phi(e_1)|\Phi(e_2)|\Phi(e_3) \end{pmatrix}=\begin{pmatrix} 3 & -2 &1 \\ 1&-2&3 \end{pmatrix}
$$
\item 
$$
\Phi(\begin{pmatrix} 1\\2\\4 \end{pmatrix})=A\cdot \begin{pmatrix} 1\\2\\4 \end{pmatrix}=\begin{pmatrix} 3 \\ 9 \end{pmatrix}
$$

\end{enumerate}


%%%%%
\newpage
\Aufgabe
\begin{enumerate}[a)]
\item Bestimmen Sie die Lösungsmenge des folgenden linearen Gleichungssystems:
\[\begin{array}{rrrrrrrrr}
x_1&+&10x_2&+&6x_3&=&1\\
x_1&+&2x_2&+&x_3&=&-6\\
2x_1&+&12x_2&+&7x_3&=&-5
\end{array}\]
\item Berechnen Sie die zu 
$$
A=\begin{pmatrix} 1 & 0 &0 \\ 2 & 1 & 0 \\ 3 & 4 & 1 \end{pmatrix}
$$
inverse Matrix $A^{-1}$.
\end{enumerate}
%
%
\subsubsection*{Lösungsvorschlag}
\begin{enumerate}[a)]
\item ...
$$
\mathcal{L}=\left\{ \begin{pmatrix}-\frac{31}{4}\\\frac{7}{8}\\0 \end{pmatrix} +s \cdot \begin{pmatrix}-\frac{1}{4}\\\frac{5}{8}\\-1 \end{pmatrix} \mid s \in \RR \right\}
$$
\item $$
A^{-1}=\begin{pmatrix} 1 & 0 &0 \\ -2 & 1 & 0 \\ 5 & -4 & 1 \end{pmatrix}
$$
\end{enumerate}


%%%%%
\newpage
\Aufgabe
Berechnen Sie die Determinanten der folgenden Matrizen und geben Sie jeweils an, ob die Matrix invertierbar ist.\\
%
\hspace*{10mm} a) \ $A=\begin{pmatrix} -1 & 7 \\ 3 & 2 \end{pmatrix}$ \qquad
b) \ $B=\begin{pmatrix} 1 & \frac{4}{7} & 93 &1\\ 0 & 2 & 2 &2 \\1 & \frac{18}{7} & 95 &3 \\ 1& 2 &3 &4\end{pmatrix}$ \qquad
c) \ $C=\begin{pmatrix}0 & 0& 3 & -1 \\  1 & 1 &  2 & 1 \\ 0 & 2 & 0 & 1 \\ 1 & 1 & 5 & 4 \end{pmatrix}$
%
%
\subsubsection*{Lösungsvorschlag}
\begin{enumerate}[a)]
\item $\det(A)=-1\cdot 2 - 7\cdot 3=-23$ ist $\neq 0$ und somit ist $A$ invertierbar.
\item $\det(B)=\det(\begin{pmatrix} 1 & \frac{4}{7} & 93 &1\\ 0 & 2 & 2 &2 \\1 & \frac{18}{7} & 95 &3 \\ 1& 2 &3 &4\end{pmatrix})\stackrel{Z3-Z1-Z2}{=}\det(\begin{pmatrix} 1 & \frac{4}{7} & 93 &1 \\ 0 & 2 & 2 &2\\0&0&0&0 \\1&2&3&4 \end{pmatrix})=0$ und somit ist $B$ nicht invertierbar.
\item \begin{align*}\det(C)&=\det(\begin{pmatrix}0 & 0& 3 & -1 \\ 1 & 1 &  2 & 1 \\ 0 & 2 & 0 & 1 \\  1 & 1 & 5 & 4 \end{pmatrix})=\det(\begin{pmatrix} 1 & 1 &  2 & 1 \\ 0 & 2 & 0 & 1 \\ 0 & 0& 3 & -1 \\ 1 & 1 & 5 & 4 \end{pmatrix}) \\ &\stackrel{Z4-Z1-Z3}{=}\det(\begin{pmatrix} 1 & 1 &  2 & 1 \\ 0 & 2 & 0 & 1 \\ 0 & 0& 3 & -1 \\ 0 & 0 & 0 & 4 \end{pmatrix})=1\cdot2\cdot3\cdot4=24
\end{align*} ist $\neq 0$ und somit ist $C$ invertierbar.
\end{enumerate}

%%%%%
\newpage
\Aufgabe
Sei $A=\begin{pmatrix} -\frac{1}{4} & 0 & -\frac{1}{2} \\ 0 & 4 & 0 \\ \frac{1}{2} & 0 &1 \end{pmatrix}$. 
\begin{enumerate}[a)]
\item Geben Sie die Definition eines Eigenwerts einer Matrix $B \in \RR^{n\times n}$ an.
\item Bestimmen Sie das charakteristische Polynom $CP_A(X)$ und alle Eigenwerte von $A$.
\item Bestimmen Sie eine invertierbare Matrix $D \in \RR^{3 \times 3}$, sodass $D^{-1}AD$ eine Diagonalmatrix ist.
\end{enumerate}
%
%
\subsubsection*{Lösungsvorschlag}
\begin{enumerate}[a)]
\item Es sei $B\in\RR^{n\times n}$ eine Matrix. Ein Wert $\lambda \in \RR$ heißt Eigenwert von $B$, wenn es einen Eigenvektor $v$ von $B$ gibt mit $B\cdot v = \lambda\cdot v$.
\item $CP_A(X)=\det(A-X\cdot I_3)=...=X(4-X)(X-\frac{3}{4})$. \\
Somit sind die Eigenwerte von $A$ gegeben als $\lambda_1=0$, $\lambda_2=4$ und $\lambda_3=\frac{3}{4}$.
\item Für die invertierbare Matrix $D$ benötigen wir eine Basis aus Eigenvektoren von $A$. Dazu bestimmen wir die Eigenräume:
\begin{enumerate}[i)]
\item $E_0=kern(A-0I_3)=kern\begin{pmatrix} -\frac{1}{4} & 0 & -\frac{1}{2} \\ 0 & 4 & 0 \\ \frac{1}{2} & 0 &1 \end{pmatrix}=kern\begin{pmatrix} 1 & 0 & 2 \\ 0 & 1 & 0 \\ 0 & 0 &0 \end{pmatrix}=<\begin{pmatrix} 2\\0\\-1 \end{pmatrix}>$
\item $E_4=kern(A-4I_3)=kern\begin{pmatrix} -\frac{17}{4} & 0 & -\frac{1}{2} \\ 0 & 0 & 0 \\ \frac{1}{2} & 0 &-3 \end{pmatrix}=kern\begin{pmatrix} 1 & 0 & 0 \\ 0 & 0 & 0 \\ 0 & 0 &1 \end{pmatrix}=<\begin{pmatrix} 0\\1\\0 \end{pmatrix}>$
\item $E_\frac{3}{4}=kern(A-\frac{3}{4}I_3)=kern\begin{pmatrix} -1 & 0 & -\frac{1}{2} \\ 0 & 4-\frac{3}{4} & 0 \\ \frac{1}{2} & 0 &\frac{1}{4} \end{pmatrix}=kern\begin{pmatrix} 1 & 0 & \frac{1}{2} \\ 0 & 1 & 0 \\ 0 & 0 &0 \end{pmatrix}=<\begin{pmatrix} \frac{1}{2}\\0\\-1 \end{pmatrix}>$
\end{enumerate}
Insgesamt ergibt sich somit die Matrix
$$
D=\begin{pmatrix} 2 & 0 & \frac{1}{2} \\ 0 & 1 & 0 \\ -1 & 0 &-1 \end{pmatrix}
$$
\end{enumerate}

\quad\\

\vfill \hfill \textbf{Viel Erfolg!}





\end{document}