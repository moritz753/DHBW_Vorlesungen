\documentclass[a4paper,11pt]{scrartcl}

\usepackage{graphicx}
\usepackage[T1]{fontenc}
\usepackage[utf8]{inputenc}
\usepackage{floatflt}
\usepackage{amsfonts}
\usepackage{amsmath}
\usepackage{amssymb}
\usepackage{amsthm}
\usepackage[ngerman]{babel}
%\usepackage[T1]{fontenc}


\usepackage{paralist}
\usepackage[usenames,dvipsnames]{color} 



\newcounter{auf}
\newcommand{\Aufgabe}%
        {\addtocounter{auf}{1} \subsubsection*{\rmfamily  Aufgabe \theauf \hspace{1em}} }









\pagestyle{empty}

\oddsidemargin0mm
\parindent0mm
\parskip2mm
\textheight24cm
\textwidth15.8cm
\unitlength1mm

\newcommand{\E}{\mathbb{E}}
\newcommand{\Hy}{\mathbb{H}}
\newcommand{\N}{\mathbb{N}}
\newcommand{\RR}{\mathbb{R}}
\newcommand{\Z}{\mathbb{Z}}
\newcommand{\Q}{\mathbb{Q}}
\newcommand{\C}{\mathbb{C}}
\newcommand{\K}{\mathbb{K}}
\newcommand*\e{\mathrm{e}}
\newcommand*\ii{\mathrm{i}}
\newcommand*\re{\mathrm{Re}}
\newcommand*\im{\mathrm{Im}}
\newcommand*\id{\mathrm{id}}
\newcommand*\glnr{\mathrm{\it GL}(n,\R)}
\newcommand*\slnr{\mathrm{\it SL}(n,\R)}
\newcommand*\on{\mathrm{\it O}(n)}
\newcommand*\son{\mathrm{\it SO}(n)}
\newcommand*\rang{\mathrm{Rang}}
\newcommand*\grad{\mathrm{grad~}}
\newcommand*\dive{\mathrm{div~}}
\newcommand*\sym{\mathrm{Sym}}
\newcommand*\spur{\mathrm{Spur}}
\newcommand*\isom{\mathrm{Isom}}
                  





\begin{document}





\Aufgabe

\begin{enumerate}[a)]

\item Seien $a$, $b$ und $c$  Aussagen. 
Bestimmen Sie die Wahrheitstafel der Aussage
$$
	(\neg c) \land (b \lor a).
$$
\item Negieren Sie die Aussage \ $\exists x \in \N\ \forall y \in \Z\ \exists z \in \N: x+y > z$.\vspace{2mm}\\
(Hinweis: $\neg(\forall x \in \N\ \exists y \in \Z\ \forall z \in \N: x+y \le z)$ ist keine ausreichende Lösung!)\\
\item Es seien $f=\sqrt{12}X^3+4X^2+3 \in \RR[X]$ und $g=17X^6+\frac{6}{\sqrt{3}}X^4+5X^2+7 \in \RR[X]$ Polynome mit Koeffizienten in $\RR$. Bestimmen Sie den Grad des Polynoms $f\cdot g$.
\end{enumerate}
%
%
\subsubsection*{Lösungsvorschlag}
\begin{enumerate}[a)]

\item $$\begin{array}{c|c|c||c|c||c}
a&b&c&\neg c&b\lor a&(\neg c)\land(b\lor a)\\ \hline
w&w&w&f&w&f\\
w&w&f&w&w&w\\
w&f&w&f&w&f\\
w&f&f&w&w&w\\
f&w&w&f&w&f\\
f&w&f&w&w&w\\
f&f&w&f&f&f\\
f&f&f&w&f&f
\end{array}$$
\item $\neg(\forall x \in \N\ \exists y \in \Z\ \forall z \in \N: x+y \le z) = \exists x \in \N \ \forall y \in \Z \ \exists z \in \N: x+y> z$ 
\item Es gilt $deg(f)=3$ und $deg(g)=6$. Da $\RR$ eine Körper und somit nullteilerfrei ist, folgt: 
$$
deg(f\cdot g)=deg(f)+deg(g)=3+6=9
$$
\end{enumerate}
%%%%%
\newpage
\Aufgabe
Es seien die Abbildungen $$f:\RR^2 \to \RR^3, (x,y) \mapsto (x, 0, -y)$$ und $$g: \RR^3 \to \RR, (x,y,z) \mapsto x\cdot y+z$$ gegeben. 
\begin{enumerate}[a)]
\item Sind $f$ und $g$ injektiv? Begründen Sie Ihre Antwort.
\item Sind $f$ und $g$ surjektiv? Begründen Sie Ihre Antwort.
\item Bestimmen Sie $g \circ f$. 
\end{enumerate}
%
%
\subsubsection*{Lösungsvorschlag}
\begin{enumerate}[a)]

\item $f$ ist injektiv, denn seien $(x_1,y_1),(x_2,y_2) \in \RR^2$ mit $f(x_1,y_1)=f(x_2,y_2)$, dann folgt
$$
(x_1,0,-y_1)=(x_2,0,-y_2) \ \Rightarrow x_1=x_2 \text{ und } y_1=y_2 \Rightarrow (x_1,y_1)=(x_2,y_2)
$$
$g$ ist nicht injektiv, da $g((1,1,0))=1=g((0,0,1) \quad \text{aber} \quad (1,1,0) \ne (0,0,1)$.
\item $f$ ist nicht surjektiv, da $(0,1,0) \notin f(\RR^2) \subseteq \{(x,0,y)\mid x,y \in \RR\} $.\\
$g$ ist surjektiv, denn für jedes $a \in \RR$ gilt
$$
(0,0,a) \in \RR^3 \quad \text{und} \quad g((0,0,a))=a
$$
\item
\begin{align*}
&(g\circ f)(x,y)=g(f(x,y))=g((x,0,-y))=x\cdot 0 +(-y)=-y\\
\Rightarrow&\quad g\circ f :\RR^2 \to \RR, (x,y) \mapsto -y
\end{align*}
\end{enumerate}



%%%%%
\newpage
\Aufgabe
Es sei 
$$
H:=\left\{\begin{pmatrix} 1&x&z\\0&1&y\\0&0&1 \end{pmatrix} \mid x,y,z \in \RR \right\} \subset \RR^{3\times 3}.
$$
\begin{enumerate}[a)]
\item Geben Sie die Definition einer Gruppe an.

\item Zeigen Sie, dass $H$ mit der Matrizenmultiplikation ,,$\cdot$`` eine Gruppe ist.

\item Ist $(H,\cdot)$ eine abelsche Gruppe? Begründen Sie Ihre Antwort.

\end{enumerate}

%
%
\subsubsection*{Lösungsvorschlag}
\begin{enumerate}[a)]
\item Seien $G$ eine Menge und $\ast$ eine Verknüpfung auf $G$.
	Dann heißt das Paar $(G,\ast)$ eine  Gruppe, wenn folgende Bedingungen erfüllt sind:
	\begin{enumerate}[i)]
		\item $\ast$ ist assoziativ. 
		\item Es gibt ein Element $e\in G$, so dass
			$$
				\forall x\in G:\, x\ast e = e\ast x = x.
			$$

		\item Zu jedem Element $x\in G$ gibt es ein inverses Element:
			$$
				\forall x\in G\,\, \exists y\in G:\, x\ast y = y\ast x = e.
			$$
	\end{enumerate}
\item Die Menge $H$ ist unter der Matrizenmultiplikation abgeschlossen:
$$
 \forall \begin{pmatrix} 1&x&z\\0&1&y\\0&0&1 \end{pmatrix}, \begin{pmatrix} 1&a&c\\0&1&b\\0&0&1 \end{pmatrix} \in H: \quad \begin{pmatrix} 1&x&z\\0&1&y\\0&0&1 \end{pmatrix} \cdot \begin{pmatrix} 1&a&c\\0&1&b\\0&0&1 \end{pmatrix} =\begin{pmatrix} 1&a+x&c+z+xb\\0&1&b+y\\0&0&1 \end{pmatrix} \in H 
$$
Da $H \subseteq \RR^{3\times 3}$ und die Matrizenmultiplikation assoziativ ist, ist auch die Verknüpfung auf $H$ assoziativ.\\
Auch das neutrale Elemente, die Einheitsmatrix $I_3=diag(1,1,1)$ liegt in $H$.\\
Es bleibt also zu zeigen, dass jedes Element in auch ein Inverses in $H$ besitzt. Dies ist gegeben, da für 
$$
 \begin{pmatrix} 1&x&z\\0&1&y\\0&0&1 \end{pmatrix} \in H
$$
die inverse Matrix durch
$$
 \begin{pmatrix} 1&-x&xy-z\\0&1&-y\\0&0&1 \end{pmatrix} \in H
$$
geben ist.\\
Somit ist $(H,\cdot)$ eine Gruppe.
\item Die Gruppe $(H,\cdot)$ ist nicht abelsch, da für $xb\ne ay$ (also z.B. $x=0,b=1,a=1$ und $y=1$)
$$
\begin{pmatrix} 1&x&z\\0&1&y\\0&0&1 \end{pmatrix} \cdot \begin{pmatrix} 1&a&c\\0&1&b\\0&0&1 \end{pmatrix} =\begin{pmatrix} 1&a+x&c+z+xb\\0&1&b+y\\0&0&1 \end{pmatrix}\ne \begin{pmatrix} 1&a+x&c+z+ay\\0&1&b+y\\0&0&1 \end{pmatrix}=\begin{pmatrix} 1&a&c\\0&1&b\\0&0&1 \end{pmatrix}\cdot \begin{pmatrix} 1&x&z\\0&1&y\\0&0&1 \end{pmatrix} 
$$
\end{enumerate}

%%%%%
\newpage
\Aufgabe

Es seien $K$ ein Körper, $V$ ein $K$-Vektorraum und $U_1$ und $U_2$
Untervektorräume von $V.$

\begin{enumerate}[a)]
\item Geben Sie die Definition eines $K$-Vektorraums an.
\item Geben Sie die Definition eines Untervektorraums an.
\item Zeigen Sie: $V=U_1\cup U_2  \ \Longleftrightarrow \ V=U_1\ \text{ oder }\ V=U_2.$
\end{enumerate}
%
%
\subsubsection*{Lösungsvorschlag}
\begin{enumerate}[a)]
\item Ein $K$-Vektorraum ist eine Menge $V$ zusammen mit einem Körper $K$ und einer Verknüpfung $+$ auf $V$ und einer Abbildung $\cdot:K\times V \to V$, sodass
\begin{enumerate}[i)]
\item $(V,+)$ eine abelsche Gruppe ist,
\item $1_K\cdot v=v$ für alle $v\in V$,
\item $\forall a,b \in K \ \forall v \in V: a\cdot (b \cdot v)=(a\cdot_K b)\cdot v$
\item die Distributivgesetze $(a+b)\cdot v=av+bv$ und $a\cdot (v+w)=av+aw$ gelten.
\end{enumerate}
\item Ein Untervektorraum eines $K$-Vektorraumes $V$ ist eine Teilmenge $U \subseteq V$, sodass $(U,+)$ eine Untergruppe von $(V,+)$ ist und für alle $a\in K$ und alle $u\in U$ gilt $au \in U$.
\item 
\begin{itemize}
\item[$\Leftarrow$:] Falls $U_1=V$ oder $U_2=V$ gilt, dann ist auch $U_1\cup U_2=V$.
\item[$\Rightarrow$:] Annahme: $U_1\ne V \ne U_2$. Dann gibt es Vektoren $u_1 \in V\setminus U_2 \subseteq U_1$ und $u_2 \in V\setminus U_1 \subseteq U_2$. Da $U_1 \cup U_2=V$ ein Vektorraum ist, folgt $u_1+u_2 \in U_1\cup U_2$.\\
Fall 1: $u_1+u_2 \in U_1$. Dann folgt da $u_1 \in U_1$ auch $u_2=(u_1+u_2)-u_1 \in U_1$. Dies ist aber ein Widerspruch zur Definition von $u_2$.\\
Fall 2: $u_1+u_2 \in U_2$. Dann folgt da $u_2 \in U_2$ auch $u_1=(u_1+u_2)-u_2 \in U_2$. Dies ist aber ein Widerspruch zur Definition von $u_1$.\\
Somit muss die Annahme falsch sein und $V=U_1$ oder $V=U_2$ gelten. \hfill $\square$
\end{itemize}
\end{enumerate}



%%%%%
\newpage
\Aufgabe

Es seien die Vektoren $b_1=\begin{pmatrix} 2\\1\\0\end{pmatrix}$, $b_2=\begin{pmatrix} 0\\1\\2\end{pmatrix}$ und $b_3=\begin{pmatrix} 1\\-1\\-1\end{pmatrix} \in \RR^3$ sowie die lineare Abbildung 
$
\Phi: \RR^3 \to \RR^2, v \mapsto \Phi(v)
$
mit 
$$
\Phi(b_1)=\begin{pmatrix} 4\\0\end{pmatrix}, \ \Phi(b_2)=\begin{pmatrix} 0\\4\end{pmatrix} \text{ und } \Phi(b_3)=\begin{pmatrix} 4\\0\end{pmatrix}
$$
gegeben.

\begin{enumerate}[a)]

\item Zeigen Sie, dass die Vektoren $b_1,b_2$ und $b_3$ linear unabhängig sind.
\item Bestimmen Sie die Abbildungsmatrix $A$ von $\Phi$ bezüglich der Standardbasis.
\item Berechnen Sie $\Phi(\begin{pmatrix} 1\\2\\4 \end{pmatrix})$.

\end{enumerate}
%
%
\subsubsection*{Lösungsvorschlag}
\begin{enumerate}[a)]
\item Die Vektoren $b_1,b_2,b_3$ sind genau dann linear unabhängig, wenn das homogene LGS 
$$
\alpha b_1 +\beta b_2 + \gamma b_3 =0
$$
nur die triviale Lösung $\alpha=\beta=\gamma=0$ hat.
$$
\left(\begin{array}{ccc|c}2&0&1&0\\1&1&-1&0\\0&2&-1&0 \end{array}\right) \sim> ... \sim>\left(\begin{array}{ccc|c} 1&1&-1&0\\ 0&1&\frac{1}{2}&0 \\ 0&0&1&0\end{array}\right)
$$
Somit hat das LGS in jeder Spalte eine Stufe und daher nur die triviale Lösung.
\item Die Standardbasisvektoren lassen sich wie folgt durch $b_1,b_2,b_3$ darstellen:
\begin{align*}
e_1&=\frac{1}{4}(b_1+b_2+2b_3)\\
e_2&=\frac{1}{2}(b_1-b_2-2b_3)\\
e_3&=\frac{1}{2}(b_2-e_2)=\frac{1}{2}(b_2-\frac{1}{2}(b_1-b_2-2b_3))=\frac{1}{4}(-b_1+3b_2+2b_3)
\end{align*}
Damit ergibt sich die Abbildungsmatrix
$$
A=\begin{pmatrix} \Phi(e_1)|\Phi(e_2)|\Phi(e_3) \end{pmatrix}=\begin{pmatrix} 3 & -2 &1 \\ 1&-2&3 \end{pmatrix}
$$
\item 
$$
\Phi(\begin{pmatrix} 1\\2\\4 \end{pmatrix})=A\cdot \begin{pmatrix} 1\\2\\4 \end{pmatrix}=\begin{pmatrix} 3 \\ 9 \end{pmatrix}
$$

\end{enumerate}


%%%%%
\newpage
\Aufgabe
\begin{enumerate}[a)]
\item Bestimmen Sie alle $t\in\RR$, für die das folgende lineare Gleichungssystem lösbar ist:
\[\begin{array}{rrrrrrrrr}
x_1&+&x_2&-&4x_3&=&t-4\\
&&3x_2&-&x_3&=&3t+6\\
2x_1&+&5x_2&+&7x_3&=&4t
\end{array}\]
\item Geben Sie die Lösungsmenge des obigen linearen Gleichungssystems für $t=2$ an.
\end{enumerate}
%
%
\subsubsection*{Lösungsvorschlag}
\begin{enumerate}[a)]
\item Das LGS muss mit dem Gauß-Algorithmus auf Stufenform gebracht werden:
$$
\left( \begin{array}{ccc|c} 1&1&-4&t-4\\
0&3&-1&3t+6\\
2&5&7&4t \end{array}\right) \sim> ... \sim> \left( \begin{array}{ccc|c} 1&1&-4&t-4\\0&3&-1 & 3t+6 \\ 0&0&16&2-t \end{array} \right)
$$
Das LGS ist also für alle $t \in \RR$ lösbar.
\item Setzt man $t=2$ in die Stufenform aus a) ein, so erhält man
\begin{align*}
\left( \begin{array}{ccc|c} 1&1&-4&-2\\0&3&-1 & 12 \\ 0&0&16&0 \end{array}\right) &\stackrel{\sim>}{1/3\cdot Z2, Z1- Z2, 1/16Z3}\left( \begin{array}{ccc|c} 1&0&\frac{-11}{3}&-6\\0&1&\frac{-1}{3} & 4 \\ 0&0&1&0 \end{array} \right)\\
&\stackrel{\sim>}{Z1+11/3Z3, Z2+1/3Z3}\left( \begin{array}{ccc|c} 1&0&0&-6\\0&1&0 & 4 \\ 0&0&1&0 \end{array} \right)
\end{align*}
Die eindeutige Lösung des LGS ist somit
$$
\mathcal{L}=\left\{ \begin{pmatrix}-6\\4\\0 \end{pmatrix}  \right\}
$$
\end{enumerate}


%%%%%
\newpage
\Aufgabe
Berechnen Sie die Determinanten der folgenden Matrizen und geben Sie jeweils an, ob die Matrix invertierbar ist.\\
%
\hspace*{10mm} a) \ $A=\begin{pmatrix} 2 & 11 \\ -4 & -1 \end{pmatrix}$ \qquad
b) \ $B=\begin{pmatrix} 1 & \frac{4}{7} & 93 \\ 0 & 2 & 2 \\1 & \frac{18}{7} & 95 \end{pmatrix}$ \qquad
c) \ $C=\begin{pmatrix} 1 & 1 &  2 & 1 \\ 0 & 2 & 0 & 1 \\ 0 & 0& 3 & -1 \\ 1 & 1 & 5 & 4 \end{pmatrix}$
%
%
\subsubsection*{Lösungsvorschlag}
\begin{enumerate}[a)]
\item $\det(A)=2(-1)-11(-4)=-2+44=42$
\item $\det(B)=\det(\begin{pmatrix} 1 & \frac{4}{7} & 93 \\ 0 & 2 & 2 \\1 & \frac{18}{7} & 95 \end{pmatrix})\stackrel{Z3-Z1-Z2}{=}\det(\begin{pmatrix} 1 & \frac{4}{7} & 93 \\ 0 & 2 & 2 \\0&0&0 \end{pmatrix})=0$
\item $\det(C)=\det(\begin{pmatrix} 1 & 1 &  2 & 1 \\ 0 & 2 & 0 & 1 \\ 0 & 0& 3 & -1 \\ 1 & 1 & 5 & 4 \end{pmatrix})\stackrel{Z4-Z1-Z3}{=}\det(\begin{pmatrix} 1 & 1 &  2 & 1 \\ 0 & 2 & 0 & 1 \\ 0 & 0& 3 & -1 \\ 0 & 0 & 0 & 4 \end{pmatrix})=1\cdot2\cdot3\cdot4=24$
\end{enumerate}

%%%%%
\newpage
\Aufgabe
Sei $A=\begin{pmatrix} -\frac{1}{4} & 0 & -\frac{1}{2} \\ 0 & 4 & 0 \\ \frac{1}{2} & 0 &1 \end{pmatrix}$. 
\begin{enumerate}[a)]
\item Geben Sie die Definition eines Eigenwerts einer Matrix $B \in \RR^{n\times n}$ an.
\item Bestimmen Sie das charakteristische Polynom $CP_A(X)$ und alle Eigenwerte von $A$.
\item Bestimmen Sie eine invertierbare Matrix $D \in \RR^{3 \times 3}$, sodass $D^{-1}AD$ eine Diagonalmatrix ist.
\end{enumerate}
%
%
\subsubsection*{Lösungsvorschlag}
\begin{enumerate}[a)]
\item Es sei $B\in\RR^{n\times n}$ eine Matrix. Ein Wert $\lambda \in \RR$ heißt Eigenwert von $B$, wenn es einen Eigenvektor $v$ von $B$ gibt mit $B\cdot v = \lambda\cdot v$.
\item $CP_A(X)=\det(A-X\cdot I_3)=...=X(4-X)(X-\frac{3}{4})$. \\
Somit sind die Eigenwerte von $A$ gegeben als $\lambda_1=0$, $\lambda_2=4$ und $\lambda_3=\frac{3}{4}$.
\item Für die invertierbare Matrix $D$ benötigen wir eine Basis aus Eigenvektoren von $A$. Dazu bestimmen wir die Eigenräume:
\begin{enumerate}[i)]
\item $E_0=kern(A-0I_3)=kern\begin{pmatrix} -\frac{1}{4} & 0 & -\frac{1}{2} \\ 0 & 4 & 0 \\ \frac{1}{2} & 0 &1 \end{pmatrix}=kern\begin{pmatrix} 1 & 0 & 2 \\ 0 & 1 & 0 \\ 0 & 0 &0 \end{pmatrix}=<\begin{pmatrix} 2\\0\\-1 \end{pmatrix}>$
\item $E_4=kern(A-4I_3)=kern\begin{pmatrix} -\frac{17}{4} & 0 & -\frac{1}{2} \\ 0 & 0 & 0 \\ \frac{1}{2} & 0 &-3 \end{pmatrix}=kern\begin{pmatrix} 1 & 0 & 0 \\ 0 & 0 & 0 \\ 0 & 0 &1 \end{pmatrix}=<\begin{pmatrix} 0\\1\\0 \end{pmatrix}>$
\item $E_\frac{3}{4}=kern(A-\frac{3}{4}I_3)=kern\begin{pmatrix} -1 & 0 & -\frac{1}{2} \\ 0 & 4-\frac{3}{4} & 0 \\ \frac{1}{2} & 0 &\frac{1}{4} \end{pmatrix}=kern\begin{pmatrix} 1 & 0 & \frac{1}{2} \\ 0 & 1 & 0 \\ 0 & 0 &0 \end{pmatrix}=<\begin{pmatrix} \frac{1}{2}\\0\\-1 \end{pmatrix}>$
\end{enumerate}
Insgesamt ergibt sich somit die Matrix
$$
D=\begin{pmatrix} 2 & 0 & \frac{1}{2} \\ 0 & 1 & 0 \\ -1 & 0 &-1 \end{pmatrix}
$$
\end{enumerate}

\quad\\

\vfill \hfill \textbf{Viel Erfolg!}





\end{document}