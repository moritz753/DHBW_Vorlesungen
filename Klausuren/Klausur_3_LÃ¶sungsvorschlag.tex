\documentclass[a4paper,11pt]{scrartcl}

\usepackage{graphicx}
\usepackage[T1]{fontenc}
\usepackage[utf8]{inputenc}
\usepackage{floatflt}
\usepackage{amsfonts}
\usepackage{amsmath}
\usepackage{amssymb}
\usepackage{amsthm}
\usepackage[ngerman]{babel}
%\usepackage[T1]{fontenc}


\usepackage{paralist}
\usepackage[usenames,dvipsnames]{color} 



\newcounter{auf}
\newcommand{\Aufgabe}%
        {\addtocounter{auf}{1} \subsubsection*{\rmfamily  Aufgabe \theauf \hspace{1em}} }









\pagestyle{empty}

\oddsidemargin0mm
\parindent0mm
\parskip2mm
\textheight24cm
\textwidth15.8cm
\unitlength1mm

\newcommand{\E}{\mathbb{E}}
\newcommand{\Hy}{\mathbb{H}}
\newcommand{\N}{\mathbb{N}}
\newcommand{\RR}{\mathbb{R}}
\newcommand{\Z}{\mathbb{Z}}
\newcommand{\Q}{\mathbb{Q}}
\newcommand{\C}{\mathbb{C}}
\newcommand{\K}{\mathbb{K}}
\newcommand*\e{\mathrm{e}}
\newcommand*\ii{\mathrm{i}}
\newcommand*\re{\mathrm{Re}}
\newcommand*\im{\mathrm{Im}}
\newcommand*\id{\mathrm{id}}
\newcommand*\glnr{\mathrm{\it GL}(n,\R)}
\newcommand*\slnr{\mathrm{\it SL}(n,\R)}
\newcommand*\on{\mathrm{\it O}(n)}
\newcommand*\son{\mathrm{\it SO}(n)}
\newcommand*\rang{\mathrm{Rang}}
\newcommand*\grad{\mathrm{grad~}}
\newcommand*\dive{\mathrm{div~}}
\newcommand*\sym{\mathrm{Sym}}
\newcommand*\spur{\mathrm{Spur}}
\newcommand*\isom{\mathrm{Isom}}
                  





\begin{document}





\Aufgabe

\begin{enumerate}[a)]

\item Seien $a$ und $b$  Aussagen. 
Bestimmen Sie die Wahrheitstafel der Aussage
$$
	(\neg a) \lor (b \land a).
$$
\item Negieren Sie die Aussage 
$$
\forall x \in \N\ \exists y \in \Z\ \forall z \in \N: x+y \le z
$$
und formen Sie das Ergebnis so um, dass kein Negationszeichen $\neg$ verwendet wird.
\item Geben Sie die Definition der Potenzmenge ${\cal P}(M)$ einer Menge $M$ an und bestimmen Sie die Potenzmenge für $M = \{1,2,3,4\}.$
\end{enumerate}

\subsubsection*{Lösungsvorschlag}
\begin{enumerate}[a)]

\item $$\begin{array}{c|c||c|c||c}
a&b&\neg a&b\land a&(\neg a)\lor(b\land a)\\ \hline
w&w&f&w&w\\
w&f&f&f&f\\
f&w&w&f&w\\
f&f&w&f&w
\end{array}$$
\item $\neg(\forall x \in \N\ \exists y \in \Z\ \forall z \in \N: x+y \le z) \equiv \exists x \in \N \ \forall y \in \Z\ \exists z \in \N: x+y>z$ 
\item Die Potenzmenge einer Menge $M$ ist die Menge aller Teilmengen von $M$
$$
\mathcal{P}(M):=\{A \mid A \subseteq M\}
$$
Für $M=\{1,2,3,4\}$ ist die Potenzmenge
\begin{align*}
\mathcal{P}(M)=\Big\{&\emptyset,\\
&\{1\},\{2\},\{3\},\{4\},\\
&\{1,2\},\{1,3\},\{1,4\},\{2,3\},\{2,4\},\{3,4\},\\
&\{1,2,3\},\{1,2,4\},\{1,3,4\},\{2,3,4\},\\
&\{1,2,3,4\}\Big\}
\end{align*}
\end{enumerate}



%%%%%
\newpage
\Aufgabe
Es seien die Abbildungen $$f:\RR \to \RR^3, x \mapsto (x, 0, -2x)$$ und $$g: \RR^3 \to \RR, (x,y,z) \mapsto x+y+z$$ gegeben. 
\begin{enumerate}[a)]
\item Sind $f$ und $g$ injektiv? Begründen Sie Ihre Antwort.
\item Sind $f$ und $g$ surjektiv? Begründen Sie Ihre Antwort.
\item Bestimmen Sie $g \circ f$ und $f\circ g$.
\end{enumerate}

\subsubsection*{Lösungsvorschlag}
\begin{enumerate}[a)]

\item $f$ ist injektiv, denn seien $x,y \in \RR$ mit $f(x)=f(y)$, dann folgt
$$
(x,0,-2x)=(y,0,-2y) \ \Rightarrow x=y
$$
$g$ ist nicht injektiv, da $g((0,1,0))=1=g((0,0,1) \quad \text{aber} \quad (0,1,0) \ne (0,0,1)$.
\item $f$ ist nicht surjektiv, da $(0,1,0) \notin f(\RR) \subseteq \{(x,0,y)\mid x,y \in \RR\} $.\\
$g$ ist surjektiv, denn für jedes $a \in \RR$ gilt
$$
(0,0,a) \in \RR^3 \quad \text{und} \quad g((0,0,a))=a
$$
\item
\begin{align*}
&(g\circ f)(x)=g(f(x))=g((x,0,-2x))=x-2x=-x\\
\Rightarrow&\quad g\circ f :\RR \to \RR, x \mapsto -x\\
&\\
&(f \circ g)(x,y,z)=f(g(x,y,z))=f(x+y+z)=(x+y+z,0,-2(x+y+z))\\
\Rightarrow&\quad f\circ g : \RR^3 \to \RR^3, (x,y,z) \mapsto (x+y+z,0,-2x-2y-2z)
\end{align*}

\end{enumerate}

%%%%%
\newpage
\Aufgabe
Es sei 
$$
H:=\left\{\begin{pmatrix} 1&x&z\\0&1&y\\0&0&1 \end{pmatrix} \mid x,y,z \in \RR \right\} \subset \RR^{3\times 3}.
$$
\begin{enumerate}[a)]
\item Geben Sie die Definition einer Gruppe an.

\item Zeigen Sie, dass $H$ mit der Matrizenmultiplikation ,,$\cdot$`` eine Gruppe ist.

\item Ist $(H,\cdot)$ eine abelsche Gruppe? Begründen Sie Ihre Antwort.

\end{enumerate}

%
%
\subsubsection*{Lösungsvorschlag}
\begin{enumerate}[a)]
\item Seien $G$ eine Menge und $\ast$ eine Verknüpfung auf $G$.
	Dann heißt das Paar $(G,\ast)$ eine  Gruppe, wenn folgende Bedingungen erfüllt sind:
	\begin{enumerate}[i)]
		\item $\ast$ ist assoziativ. 
		\item Es gibt ein Element $e\in G$, so dass
			$$
				\forall x\in G:\, x\ast e = e\ast x = x.
			$$

		\item Zu jedem Element $x\in G$ gibt es ein inverses Element:
			$$
				\forall x\in G\,\, \exists y\in G:\, x\ast y = y\ast x = e.
			$$
	\end{enumerate}
\item Die Menge $H$ ist unter der Matrizenmultiplikation abgeschlossen:
$$
 \forall \begin{pmatrix} 1&x&z\\0&1&y\\0&0&1 \end{pmatrix}, \begin{pmatrix} 1&a&c\\0&1&b\\0&0&1 \end{pmatrix} \in H: \quad \begin{pmatrix} 1&x&z\\0&1&y\\0&0&1 \end{pmatrix} \cdot \begin{pmatrix} 1&a&c\\0&1&b\\0&0&1 \end{pmatrix} =\begin{pmatrix} 1&a+x&c+z+xb\\0&1&b+y\\0&0&1 \end{pmatrix} \in H 
$$
Da $H \subseteq \RR^{3\times 3}$ und die Matrizenmultiplikation assoziativ ist, ist auch die Verknüpfung auf $H$ assoziativ.\\
Auch das neutrale Elemente, die Einheitsmatrix $I_3=diag(1,1,1)$ liegt in $H$.\\
Es bleibt also zu zeigen, dass jedes Element in auch ein Inverses in $H$ besitzt. Dies ist gegeben, da für 
$$
 \begin{pmatrix} 1&x&z\\0&1&y\\0&0&1 \end{pmatrix} \in H
$$
die inverse Matrix durch
$$
 \begin{pmatrix} 1&-x&xy-z\\0&1&-y\\0&0&1 \end{pmatrix} \in H
$$
geben ist.\\
Somit ist $(H,\cdot)$ eine Gruppe.
\item Die Gruppe $(H,\cdot)$ ist nicht abelsch, da für $xb\ne ay$ (also z.B. $x=0,b=1,a=1$ und $y=1$)
$$
\begin{pmatrix} 1&x&z\\0&1&y\\0&0&1 \end{pmatrix} \cdot \begin{pmatrix} 1&a&c\\0&1&b\\0&0&1 \end{pmatrix} =\begin{pmatrix} 1&a+x&c+z+xb\\0&1&b+y\\0&0&1 \end{pmatrix}\ne \begin{pmatrix} 1&a+x&c+z+ay\\0&1&b+y\\0&0&1 \end{pmatrix}=\begin{pmatrix} 1&a&c\\0&1&b\\0&0&1 \end{pmatrix}\cdot \begin{pmatrix} 1&x&z\\0&1&y\\0&0&1 \end{pmatrix} 
$$
\end{enumerate}


%%%%%
\newpage
\Aufgabe

Es seien $K$ ein Körper, $V$ ein $K$-Vektorraum und $U_1$ und $U_2$
Untervektorräume von $V.$

\begin{enumerate}[a)]
\item Geben Sie die Definition eines $K$-Vektorraums an.
\item Geben Sie die Definition eines Untervektorraums an.
\item Zeigen Sie: $V=U_1\cup U_2  \ \Longleftrightarrow \ V=U_1\ \text{ oder }\ V=U_2.$
\end{enumerate}

\subsubsection*{Lösungsvorschlag}
\begin{enumerate}[a)]
\item Ein $K$-Vektorraum ist eine Menge $V$ zusammen mit einem Körper $K$ und einer Verknüpfung $+$ auf $V$ und einer Abbildung $\cdot:K\times V \to V$, sodass
\begin{enumerate}[i)]
\item $(V,+)$ eine abelsche Gruppe ist,
\item $1_K\cdot v=v$ für alle $v\in V$,
\item $\forall a,b \in K \ \forall v \in V: a\cdot (b \cdot v)=(a\cdot_K b)\cdot v$
\item die Distributivgesetze $(a+b)\cdot v=av+bv$ und $a\cdot (v+w)=av+aw$ gelten.
\end{enumerate}
\item Ein Untervektorraum eines $K$-Vektorraumes $V$ ist eine Teilmenge $U \subseteq V$, sodass $(U,+)$ eine Untergruppe von $(V,+)$ ist und für alle $a\in K$ und alle $u\in U$ gilt $au \in U$.
\item 
\begin{itemize}
\item[$\Leftarrow$:] Falls $U_1=V$ oder $U_2=V$ gilt, dann ist auch $U_1\cup U_2=V$.
\item[$\Rightarrow$:] Annahme: $U_1\ne V \ne U_2$. Dann gibt es Vektoren $u_1 \in V\setminus U_2 \subseteq U_1$ und $u_2 \in V\setminus U_1 \subseteq U_2$. Da $U_1 \cup U_2=V$ ein Vektorraum ist, folgt $u_1+u_2 \in U_1\cup U_2$.\\
Fall 1: $u_1+u_2 \in U_1$. Dann folgt da $u_1 \in U_1$ auch $u_2=(u_1+u_2)-u_1 \in U_1$. Dies ist aber ein Widerspruch zur Definition von $u_2$.\\
Fall 2: $u_1+u_2 \in U_2$. Dann folgt da $u_2 \in U_2$ auch $u_1=(u_1+u_2)-u_2 \in U_2$. Dies ist aber ein Widerspruch zur Definition von $u_1$.\\
Somit muss die Annahme falsch sein und $V=U_1$ oder $V=U_2$ gelten. \hfill $\square$
\end{itemize}
\end{enumerate}

%%%%%
\newpage
\Aufgabe

Es seien die Vektoren $b_1=\begin{pmatrix} 2\\1\\0\end{pmatrix}$, $b_2=\begin{pmatrix} 0\\1\\2\end{pmatrix}$ und $b_3=\begin{pmatrix} 1\\0\\1\end{pmatrix} \in \RR^3$ sowie die lineare Abbildung 
$
\Phi: \RR^3 \to \RR^2, v \mapsto \Phi(v)
$
mit 
$$
\Phi(b_1)=\begin{pmatrix} 4\\0\end{pmatrix}, \ \Phi(b_2)=\begin{pmatrix} 0\\4\end{pmatrix} \text{ und } \Phi(b_3)=\begin{pmatrix} 4\\4\end{pmatrix}
$$
gegeben.

\begin{enumerate}[a)]

\item Zeigen Sie, dass die Vektoren $b_1,b_2$ und $b_3$ linear unabhängig sind.
\item Bestimmen Sie die Abbildungsmatrix $A$ von $\Phi$ bezüglich der Standardbasis.
\item Berechnen Sie $\Phi(\begin{pmatrix} 3\\2\\1 \end{pmatrix})$.

\end{enumerate}

\subsubsection*{Lösungsvorschlag}
\begin{enumerate}[a)]
\item Die Vektoren $b_1,b_2,b_3$ sind genau dann linear unabhängig, wenn das homogene LGS 
$$
\alpha b_1 +\beta b_2 + \gamma b_3 =0
$$
nur die triviale Lösung $\alpha=\beta=\gamma=0$ hat.
$$
\left(\begin{array}{ccc|c}2&0&1&0\\1&1&0&0\\0&2&1&0 \end{array}\right) \sim> ... \sim>\left(\begin{array}{ccc|c} 1&0&\frac{1}{2}&0\\ 0&1&-\frac{1}{2}&0 \\ 0&0&1&0\end{array}\right)
$$
Somit hat das LGS in jeder Spalte eine Stufe und daher nur die triviale Lösung.
\item Die Standardbasisvektoren lassen sich wie folgt durch $b_1,b_2,b_3$ darstellen:
\begin{align*}
e_1&=\frac{1}{4}(b_1-b_2+2b_3)\\
e_2&=\frac{1}{2}(b_1+b_2-2b_3)\\
e_3&=b_3-e_1=b3-(\frac{1}{4}(b_1-b_2+2b_3))=-\frac{1}{4}b_1+\frac{1}{4}b_2+\frac{1}{2}b_3
\end{align*}
Damit ergibt sich die Abbildungsmatrix
$$
A=\begin{pmatrix} \Phi(e_1)|\Phi(e_2)|\Phi(e_3) \end{pmatrix}=\begin{pmatrix} 3 & -2 &1 \\ 1&-2&3 \end{pmatrix}
$$
\item 
$$
\Phi(\begin{pmatrix} 3\\2\\1 \end{pmatrix})=A\cdot \begin{pmatrix} 3\\2\\1 \end{pmatrix}=\begin{pmatrix} 6 \\ 2 \end{pmatrix}
$$

\end{enumerate}

%%%%%
\newpage
\Aufgabe
\begin{enumerate}[a)]
\item Bestimmen Sie alle $t\in\RR$, für die das folgende lineare Gleichungssystem lösbar ist:
\[\begin{array}{rrrrrrrrr}
2x_1&+&4x_2&+&2x_3&=&12t\\
2x_1&+&12x_2&+&7x_3&=&12t+7\\
x_1&+&10x_2&+&6x_3&=&7t+8
\end{array}\]
\item Geben Sie die Lösungsmenge des obigen linearen Gleichungssystems für $t=-1$ an.
\end{enumerate}


\subsubsection*{Lösungsvorschlag}
\begin{enumerate}[a)]
\item Das LGS muss mit dem Gauß-Algorithmus auf Stufenform gebracht werden:
$$
\left( \begin{array}{ccc|c} 2&4&2 &12t\\ 2 & 12 & 7 & 12t+7\\1&10&6&7t+8 \end{array}\right) \sim> ... \sim> \left( \begin{array}{ccc|c} 1&2&1&6t\\0&1&\frac{5}{8} & \frac{7}{8} \\ 0&0&0&t+1 \end{array} \right)
$$
Das LGS ist also genau dann lösbar, wenn $t+1=0$ gilt, also für $t=-1$.
\item Setzt man $t=-1$ in die Stufenform aus a) ein, so erhält man
$$
\left( \begin{array}{ccc|c} 1&2&1&-6\\0&1&\frac{5}{8} & \frac{7}{8} \\ 0&0&0&0 \end{array} \right) \stackrel{\sim>}{Z1-2\cdot Z2}\left( \begin{array}{ccc|c} 1&0&\frac{-1}{4}&\frac{-31}{4}\\0&1&\frac{5}{8} & \frac{7}{8} \\ 0&0&0&0 \end{array} \right)
$$
Mit dem $(-1)$-Trick folgt 
$$
\mathcal{L}=\left\{ \begin{pmatrix}-\frac{31}{4}\\\frac{7}{8}\\0 \end{pmatrix} +s \cdot \begin{pmatrix}-\frac{1}{4}\\\frac{5}{8}\\-1 \end{pmatrix} \mid s \in \RR \right\}
$$
\end{enumerate}


%%%%%
\newpage
\Aufgabe
Berechnen Sie die Determinanten der folgenden Matrizen und geben Sie jeweils an, ob die Matrix invertierbar ist.\\
%
\hspace*{10mm} a) \ $A=\begin{pmatrix} 2 & 11 \\ -4 & -1 \end{pmatrix}$ \qquad
b) \ $B=\begin{pmatrix} 1 & \frac{4}{7} & 93 \\ 0 & 2 & 2 \\1 & \frac{18}{7} & 95 \end{pmatrix}$ \qquad
c) \ $C=\begin{pmatrix} 1 & 1 &  2 & 1 \\ 0 & 2 & 0 & 1 \\ 0 & 0& 3 & -1 \\ 1 & 1 & 5 & 4 \end{pmatrix}$
%
%
\subsubsection*{Lösungsvorschlag}
\begin{enumerate}[a)]
\item $\det(A)=2(-1)-11(-4)=-2+44=42$ \\
$\Rightarrow$ $A$ ist invertierbar
\item $\det(B)=\det(\begin{pmatrix} 1 & \frac{4}{7} & 93 \\ 0 & 2 & 2 \\1 & \frac{18}{7} & 95 \end{pmatrix})\stackrel{Z3-Z1-Z2}{=}\det(\begin{pmatrix} 1 & \frac{4}{7} & 93 \\ 0 & 2 & 2 \\0&0&0 \end{pmatrix})=0$ \\
$\Rightarrow$ $B$ ist nicht invertierbar
\item $\det(C)=\det(\begin{pmatrix} 1 & 1 &  2 & 1 \\ 0 & 2 & 0 & 1 \\ 0 & 0& 3 & -1 \\ 1 & 1 & 5 & 4 \end{pmatrix})\stackrel{Z4-Z1-Z3}{=}\det(\begin{pmatrix} 1 & 1 &  2 & 1 \\ 0 & 2 & 0 & 1 \\ 0 & 0& 3 & -1 \\ 0 & 0 & 0 & 4 \end{pmatrix})=1\cdot2\cdot3\cdot4=24$ \\
$\Rightarrow$ $C$ ist invertierbar
\end{enumerate}

%%%%%
\newpage
\Aufgabe
Sei $A=\begin{pmatrix} -\frac{1}{3} & 0 & -\frac{2}{3} \\ 0 & 3 & 0 \\ \frac{2}{3} & 0 &\frac{4}{3} \end{pmatrix}$. 
\begin{enumerate}[a)]
\item Geben Sie die Definition eines Eigenvektors einer Matrix $B \in \RR^{n\times n}$ an.
\item Bestimmen Sie das charakteristische Polynom $CP_A(X)$ und alle Eigenwerte von $A$.
\item Bestimmen Sie eine invertierbare Matrix $D \in \RR^{3 \times 3}$, sodass $D^{-1}AD$ eine Diagonalmatrix ist.
\end{enumerate}

\subsubsection*{Lösungsvorschlag}
\begin{enumerate}[a)]
\item Ein Eigenvektor einer Matrix $B$ ist ein Vektor $v \in \RR^{n}\setminus \{0\}$, sodass es ein $\lambda \in \RR$ gibt mit $Bv=\lambda v$.
\item
Wir bestimmen das charakteristische Polynom $CP_A(X)=\det(A-X\cdot I_3)$:\\
\begin{align*}
det(A-X\cdot I_3)&=\det(\begin{pmatrix} -\frac{1}{3}-X & 0 & -\frac{2}{3} \\ 0 & 3-X & 0 \\ \frac{2}{3} & 0 &\frac{4}{3}-X \end{pmatrix})\\
&=(3-X)\cdot \det(\begin{pmatrix} -\frac{1}{3}-X &  -\frac{2}{3} \\ \frac{2}{3} &\frac{4}{3}-X \end{pmatrix})\\
&=(3-X)\cdot ((-\frac{1}{3}-X)(\frac{4}{3}-X)-(-\frac{2}{3})(\frac{2}{3}))\\
&=(3-X)\cdot(-\frac{4}{9}-X+X^2+\frac{4}{9})\\
&=(3-X)\cdot(X^2-X)\\
&=-(3-X)(1-X)X
\end{align*}
Die Eigenwerte von $A$ sind somit $\lambda_1=0$, $\lambda_2=1$ und $\lambda_3=3$.
\item Wir suchen eine Basis des $\RR^3$ aus Eigenvektoren von $A$. Dazu bestimmen wie die Eigenräume zu den Eigenwerten aus Teil b).
\begin{itemize}
\item[$\lambda_1$:] $\left(\begin{array}{ccc|c} -\frac{1}{3}&0& -\frac{2}{3}&0\\ 0&3&0&0\\ \frac{2}{3}&0&\frac{4}{3}&0\end{array} \right) \sim>...\sim>\left(\begin{array}{ccc|c} 1&0&2&0\\ 0&1&0&0\\ 0&0&0&0\end{array}\right) \quad \Rightarrow \  Eig(A,\lambda_1)=\langle \begin{pmatrix}2\\0\\-1 \end{pmatrix} \rangle$
\item[$\lambda_2$:]$\left(\begin{array}{ccc|c} -\frac{1}{3}-1&0& -\frac{2}{3}&0\\ 0&3-1&0&0\\ \frac{2}{3}&0&\frac{4}{3}-1&0\end{array} \right) \sim>...\sim>\left(\begin{array}{ccc|c} 1&0&\frac{1}{2}&0\\ 0&1&0&0\\ 0&0&0&0\end{array}\right) \quad \Rightarrow \ Eig(A,\lambda_1)=\langle \begin{pmatrix}\frac{1}{2}\\0\\-1 \end{pmatrix}\rangle $
\item[$\lambda_3$:]$\left(\begin{array}{ccc|c} -\frac{1}{3}-3&0& -\frac{2}{3}&0\\ 0&3-3&0&0\\ \frac{2}{3}&0&\frac{4}{3}-3&0\end{array} \right) \sim>...\sim>\left(\begin{array}{ccc|c}1&0&0&0\\ 0&0&0&0\\ 0&0&1&0\end{array}\right) \quad \Rightarrow \  Eig(A,\lambda_1)=\langle \begin{pmatrix}0\\-1\\0 \end{pmatrix} \rangle$
\end{itemize}\quad\\
Damit erfüllt die Matrix
$$
D= \begin{pmatrix}2&\frac{1}{2}&0\\0&0&-1\\-1&-1&0 \end{pmatrix} 
$$
das Gewünschte.
\end{enumerate}

\end{document}