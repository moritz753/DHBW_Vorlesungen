\documentclass[a4paper,11pt]{scrartcl}

\usepackage{graphicx}
\usepackage[T1]{fontenc}
\usepackage[utf8]{inputenc}
\usepackage{floatflt}
\usepackage{amsfonts}
\usepackage{amsmath}
\usepackage{amssymb}
\usepackage{amsthm}
\usepackage[ngerman]{babel}
%\usepackage[T1]{fontenc}


\usepackage{paralist}
\usepackage[usenames,dvipsnames]{color} 



\newcounter{auf}
\newcommand{\Aufgabe}%
        {\addtocounter{auf}{1} \subsubsection*{\rmfamily  Aufgabe \theauf \hspace{1em}} }









\pagestyle{empty}

\oddsidemargin0mm
\parindent0mm
\parskip2mm
\textheight24cm
\textwidth15.8cm
\unitlength1mm

\newcommand{\E}{\mathbb{E}}
\newcommand{\Hy}{\mathbb{H}}
\newcommand{\N}{\mathbb{N}}
\newcommand{\RR}{\mathbb{R}}
\newcommand{\Z}{\mathbb{Z}}
\newcommand{\Q}{\mathbb{Q}}
\newcommand{\C}{\mathbb{C}}
\newcommand{\K}{\mathbb{K}}
\newcommand*\e{\mathrm{e}}
\newcommand*\ii{\mathrm{i}}
\newcommand*\re{\mathrm{Re}}
\newcommand*\im{\mathrm{Im}}
\newcommand*\id{\mathrm{id}}
\newcommand*\glnr{\mathrm{\it GL}(n,\R)}
\newcommand*\slnr{\mathrm{\it SL}(n,\R)}
\newcommand*\on{\mathrm{\it O}(n)}
\newcommand*\son{\mathrm{\it SO}(n)}
\newcommand*\rang{\mathrm{Rang}}
\newcommand*\grad{\mathrm{grad~}}
\newcommand*\dive{\mathrm{div~}}
\newcommand*\sym{\mathrm{Sym}}
\newcommand*\spur{\mathrm{Spur}}
\newcommand*\isom{\mathrm{Isom}}
                  





\begin{document}





\Aufgabe

\begin{enumerate}[a)]

\item Geben Sie die Definition der Kondition eines Berechnungsproblems 
$$
	y=f(x)
$$
mit $x \in \RR$ und $f:\RR \to \RR$ differenzierbar, an.
\item  Berechnen Sie die Kondition von $y=x^2-4$ und geben Sie eine Stelle $x_g$ an, an der das Problem gut konditioniert ist, und eine Stelle $x_s$, an der das Problem schlecht konditioniert ist. Begründen Sie Ihre Wahl.
\item Bestimmen Sie die Zeilensummennorm der Matrix
$$
M=\begin{pmatrix} 1&1&1 \\ 2&-2&2\\ 3 & 2&0 \end{pmatrix}
$$
\end{enumerate}
%
%
%\subsubsection*{Lösungsvorschlag}
%\begin{enumerate}[a)]
%\item $kond_x(f)=\frac{|xf'(x)|}{|f(x)|}$
%\item $kond_x(f)=\frac{|xf'(x)|}{|f(x)|}=\frac{|x(2x)|}{|x^2-4|}=\frac{|2x^2|}{|x^2-4|}$\\
%$x_g=0$, da $kond_0(f)=\frac{0}{4}=0$\\
%$x_s=2$, da $kond_{2}(f)=\frac{8}{0}=\infty$
%\item $\|M\|_\infty = \max([1+1+1],[2+|-2|+2],[3+2+0])=\max(3,6,5)=6$
%\end{enumerate}
%%%%%
\newpage
\Aufgabe
Lösen Sie approximativ das Anfangswertproblem
$$
\begin{cases} f'(x)=2x\cdot f(x) & \text{ für } x \in [0,1] \\
f(0)=2
\end{cases}
$$
mit dem expliziten Eulerverfahren und Schrittweite $h=\frac{1}{2}$.


%%%%%%%
\newpage
\Aufgabe
\begin{enumerate}[a)]
\item Berechnen Sie das Integral
$$
\int_0^\pi \cos(x)^2 dx
$$
näherungsweise unter Verwendung
der Simpson-Regel.
\item Bestimmen Sie näherungsweise die Nullstelle von 
$$
f(x)=-2x^3+x^2+3
$$
indem Sie einen Schritt des Newton-Verfahrens mit Startwert $x_0=1$ durchführen.
\end{enumerate}
%
%
%\subsubsection*{Lösungsvorschlag}
%\begin{enumerate}[a)]
%\item 
%\begin{align*}
%S(f)&=\frac{b-a}{6}(f(a)+4f(\frac{a+b}{2})+f(b)\\
%&=\frac{\pi}{6}(\cos(0)^2+4\cos(\frac{\pi}{2})^2+\cos(\pi)^2)\\
%&=\frac{\pi}{6}(1^2+4\cdot 0^2+(-1)^2)\\
%&=\frac{\pi}{3}
%\end{align*}
%\item $x_1=x_0-\frac{f(x_0)}{f'(x_0)}=1-\frac{-2\cdot 1^3+1^2+3}{-6\cdot 1^2+2\cdot 1}=1-\frac{2}{-4}=\frac{3}{2}$
%\end{enumerate}


%%%%%
\newpage
\Aufgabe


\begin{enumerate}[a)]
\item Geben Sie die Definition eines Normalbereichs $A\subset \RR^2$ bezüglich der $y$-Achse an.
\item Berechnen Sie das Integral
$$
\int_A x\cdot \cos(y^3)\ d(x,y)
$$
mit $A=\{(x,y) \in \RR^2 \mid 0\le x \le 2y \text { und } 0 \le y \le \sqrt[3]{\pi} \}$.
\end{enumerate}
%
%
%\subsubsection*{Lösungsvorschlag}
%\begin{enumerate}[a)]
%\item $A=\{(x,y) \mid a\le y \le b, f(y) \le x \le g(y)\}$
%\item 
%\begin{align*}
%\int_A x\cdot \cos(y^3)\ d(x,y)&=\int_0^{2y} \int_0^{\sqrt[3]{\pi}} x\cdot \cos(y^3) dydx\\
%&=\int_0^{\sqrt[3]{\pi}} \int_0^{2y}x\cdot \cos(y^3) dxdy\\
%&=\int_0^{\sqrt[3]{\pi}}\cos(y^3)( \int_0^{2y}x dx)dy\\
%&=\int_0^{\sqrt[3]{\pi}}\cos(y^3)( [\frac{1}{2}x^2]_0^{2y} dy\\
%&=\int_0^{\sqrt[3]{\pi}}\cos(y^3)2y^2dy\\
%&=\frac{2}{3}\int_0^{\sqrt[3]{\pi}}\cos(y^3)3y^2dy\\
%&=\frac{2}{3}[\sin(y^3)]_0^{\sqrt[3]{\pi}}\\
%&=\frac{2}{3}(\sin(\pi)-\sin(0))\\
%&=0
%\end{align*}
%\end{enumerate}
%
%

%%%%%
%\newpage
%\Aufgabe
%
%Es seien 
%$$
%v_1=\begin{pmatrix}2\\0\\2 \end{pmatrix}, v_2=\begin{pmatrix}1\\0\\-1 \end{pmatrix} \text{ und }v_3=\begin{pmatrix}2\\2\\0 \end{pmatrix}
%$$
%
%\begin{enumerate}[a)]
%
%\item Berechnen Sie den Winkel zwischen $v_1$ und $v_2$.
%\item Bestimmen Sie aus den Vektoren $v_1,v_2$ und $v_3$ unter Verwendung des Gram-Schmidt-Verfahrens eine Orthonormalbasis von $\RR^3$.
%
%\end{enumerate}
%
%
%\subsubsection*{Lösungsvorschlag}
%\begin{enumerate}[a)]
%\item  $\cos(\alpha)=\frac{\langle v_1,v_2\rangle}{\|v_1\|\|v_2\|}=\frac{0}{\sqrt{8}\sqrt{2}}=0$ und somit $\alpha=\frac{\pi}{2}$.
%\item
%\begin{align*}
%w_1&=v_1\\
%w_2&=v_2-\frac{\langle v_2,w_1 \rangle}{\langle w_1,w_1\rangle}\cdot w_1=\begin{pmatrix} 1\\0\\-1\end{pmatrix}\\
%w_2&=v_3-\frac{\langle v_3,w_1 \rangle}{\langle w_1,w_1\rangle}\cdot w_1-\frac{\langle v_3,w_2 \rangle}{\langle w_2,w_2\rangle}\cdot w_2=\begin{pmatrix} 0\\2\\0\end{pmatrix}
%\end{align*}
%Damit ist $B=\{\frac{1}{2\sqrt{2}}w_1,\frac{1}{\sqrt{2}}w_2,\frac{1}{2}w_3\}$ die gesuchte ONB.
%\end{enumerate}


%%%%%
\newpage
\Aufgabe
\begin{enumerate}[a)]
\item Geben Sie die Definition der orthogonalen Gruppe $O(n)$ an.
\item Bestimmen Sie eine $QR$-Zerlegung von 
$$
A=\begin{pmatrix}2&3&1\\0&1&3\\0&-1&1 \end{pmatrix}
$$
unter Verwendung von Givens-Rotationen.
\end{enumerate}
%
%
%\subsubsection*{Lösungsvorschlag}
%\begin{enumerate}[a)]
%\item $O(n):=\{Q \in \RR^{n\times n} \mid Q^TQ=I_n\}$
%\item Da die erste Spalte schon die gewünschte Form hat, muss nur die zweite Spalte transformiert werden. Diese muss so gedreht werden, dass die dritte Koordinate zu 0 wird.\\
%Es ist $x=(3,1,-1)^t$ und der zu nullende Eintrag ist $x_{32}=-1$ und weiter $x_{22}=1$. Somit
%$$
%\cos(\alpha)=\frac{1}{+\sqrt{(-1)^2+1^2}}=\frac{1}{\sqrt{2}} \quad \text{und} \quad \sin(\alpha)=-\sqrt{1-\cos(\alpha)^2}=-\frac{1}{\sqrt{2}}
%$$
%Damit erhalten wir
%$$
%D(3,2,\alpha)=\begin{pmatrix}1&0&0\\0&\frac{1}{\sqrt{2}}&-\frac{1}{\sqrt{2}}\\0&\frac{1}{\sqrt{2}}&\frac{1}{\sqrt{2}} \end{pmatrix}
%$$
%und
%$$
%R=D(3,2,\alpha)\cdot A =\begin{pmatrix}1&0&0\\0&\frac{1}{\sqrt{2}}&-\frac{1}{\sqrt{2}}\\0&\frac{1}{\sqrt{2}}&\frac{1}{\sqrt{2}} \end{pmatrix}\begin{pmatrix}2&3&1\\0&1&3\\0&-1&1 \end{pmatrix}
%=\begin{pmatrix}2&3&1\\0&\sqrt{2}&\sqrt{2}\\0&0& 2\sqrt{2}\end{pmatrix}
%$$
%sowie
%$$
%Q=D(3,2,\alpha)^{-1}=D(3,2,\alpha)^T=\begin{pmatrix}1&0&0\\0&\frac{1}{\sqrt{2}}&\frac{1}{\sqrt{2}}\\0&-\frac{1}{\sqrt{2}}&\frac{1}{\sqrt{2}} \end{pmatrix}
%$$
%\end{enumerate}



%%%%%
\newpage
\Aufgabe
Interpolieren Sie die Punkte
$$
(x_1,y_1)=(1,1), (x_2,y_2)=(2,0) \text{ und } (x_3,y_3)=(3,2)
$$
durch ein quadratisches Polynom.
%
%
%\subsubsection*{Lösungsvorschlag}
%$p(x)=ax2+bx+c$ damit
%\begin{align*}
%1&=a+b+c\\
%0&=4x+2b+c\\
%2&=9a+3b+c
%\end{align*}
%Das ergibt das LGS
%$$
%\begin{pmatrix} 1&1&1 \\ 4 & 2 & 1\\ 9&3&1 \end{pmatrix}\begin{pmatrix}a\\b\\c\end{pmatrix}=\begin{pmatrix} 1\\0\\2 \end{pmatrix}
%$$
%äquivalent zu
%$$
%\begin{pmatrix} 1&1&1 \\ 1&2&4\\ 1&3&9 \end{pmatrix}\begin{pmatrix}c\\b\\a\end{pmatrix}=\begin{pmatrix} 1\\0\\2 \end{pmatrix}
%$$
%mit 
%\begin{align*}
%\left( \begin{array}{ccc|c} 1&1&1&1\\ 1&2&4&0\\1&3&9&2 \end{array}\right) \sim> \left( \begin{array}{ccc|c} 1&1&1&1\\ 0&1&3&-1\\0&2&8&1 \end{array} \right) \sim> \left( \begin{array}{ccc|c} 1&1&1&1\\ 0&1&3&-1\\0&0&2&3 \end{array} \right) \sim> \left( \begin{array}{ccc|c} 1&0&0&5\\ 0&1&0&-\frac{11}{2}\\0&0&1&\frac{3}{2} \end{array} \right)
%\end{align*}
%uns somit $a=\frac{3}{2},b=-\frac{11}{2}$ und $c=5$.\\
%Das ergibt
%$$
%p(x)=\frac{3}{2}x^2-\frac{11}{2}x+5
%$$
\quad\\

\vfill \hfill \textbf{Viel Erfolg!}





\end{document}