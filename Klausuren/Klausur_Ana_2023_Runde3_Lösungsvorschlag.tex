\documentclass[a4paper,11pt]{scrartcl}

\usepackage{graphicx}
\usepackage[T1]{fontenc}
\usepackage[utf8]{inputenc}
\usepackage{floatflt}
\usepackage{amsfonts}
\usepackage{amsmath}
\usepackage{amssymb}
\usepackage{amsthm}
\usepackage[ngerman]{babel}
%\usepackage[T1]{fontenc}


\usepackage{paralist}
\usepackage[usenames,dvipsnames]{color} 



\newcounter{auf}
\newcommand{\Aufgabe}%
        {\addtocounter{auf}{1} \subsubsection*{\rmfamily  Aufgabe \theauf \hspace{1em}} }









\pagestyle{empty}

\oddsidemargin0mm
\parindent0mm
\parskip2mm
\textheight24cm
\textwidth15.8cm
\unitlength1mm

\newcommand{\E}{\mathbb{E}}
\newcommand{\Hy}{\mathbb{H}}
\newcommand{\N}{\mathbb{N}}
\newcommand{\RR}{\mathbb{R}}
\newcommand{\Z}{\mathbb{Z}}
\newcommand{\Q}{\mathbb{Q}}
\newcommand{\C}{\mathbb{C}}
\newcommand{\K}{\mathbb{K}}
\newcommand*\e{\mathrm{e}}
\newcommand*\ii{\mathrm{i}}
\newcommand*\re{\mathrm{Re}}
\newcommand*\im{\mathrm{Im}}
\newcommand*\id{\mathrm{id}}
\newcommand*\glnr{\mathrm{\it GL}(n,\R)}
\newcommand*\slnr{\mathrm{\it SL}(n,\R)}
\newcommand*\on{\mathrm{\it O}(n)}
\newcommand*\son{\mathrm{\it SO}(n)}
\newcommand*\rang{\mathrm{Rang}}
\newcommand*\grad{\mathrm{grad~}}
\newcommand*\dive{\mathrm{div~}}
\newcommand*\sym{\mathrm{Sym}}
\newcommand*\spur{\mathrm{Spur}}
\newcommand*\isom{\mathrm{Isom}}
                  





\begin{document}





\Aufgabe

\begin{enumerate}[a)]

\item Es sei $(a_n)_{n \in \N}$ eine Folge. Geben Sie die Definition davon an, dass $(a_n)_{n \in \N}$ für $n \to \infty$ gegen $a \in \RR$ konvergiert.
\item Zeigen Sie, dass die Folge $(b_n)_{n \in \N}$ mit 
$$
b_n=\frac{5n^3+n^2-1}{3n^4+2n+4} \qquad \forall n \in \N
$$
konvergent ist und bestimmen Sie den Grenzwert $\lim \limits_{n\to \infty} b_n$.
\item Geben Sie eine Folge $(c_n)_{n \in \N}$ an, sodass \vspace{3mm}\\
\hspace*{5mm} i)$\ \lim \limits_{n \to \infty}c_n=0 \qquad$ und gleichzeitig \qquad ii)$\ \forall n \in \N: c_n < 0$ \vspace{3mm}\\
gilt.
\end{enumerate}
%
%
\subsubsection*{Lösungsvorschlag:}
	\begin{enumerate}[a)]
	\item $(a_n)_{n \in \N}$ konvergiert gegen $a \in \RR$ für $n \to \infty$ $:\Leftrightarrow$ $\forall \varepsilon >0 \exists n_0 \in \N \forall n \ge n_0: |a_n-a|<\varepsilon$.
	\item 
	$$
	b_n=\frac{5n^3+n^2-1}{3n^4+2n+4}=\frac{5n^3+n^2-1}{3n^4+2n+4}\cdot \frac{\frac{1}{n^4}}{\frac{1}{n^4}}=\frac{\frac{5}{n}+\frac{1}{n^2}-\frac{1}{n^4}}{3+\frac{2}{n^3}+\frac{4}{n^4}} \to \frac{0+0+0}{3+0+0}=0 \quad (n \to \infty)
	$$
	\item Die Folge $(c_n)_{n \in \N}$ mit $c_n=-\frac{1}{n}$ erfüllt das geforderte:
		\begin{enumerate}[i)]
		\item $c_n=\frac{1}{n} \to 0 \quad (n \to \infty)$
		\item $c_n=-\frac{1}{n} < 0 \quad \forall n \in \N, \text{ da } n-1 < n$
		\end{enumerate}
	\end{enumerate}


%%%%%
\newpage
\Aufgabe
\begin{enumerate}[a)]
\item Bestimmen Sie den Reihenwert der folgenden Reihen
	\begin{enumerate}[i)]
	\item $\sum \limits_{k=0}^\infty 6\cdot \frac{1}{3^{k-1}}$
	\item $\sum \limits_{k=1}^\infty\left( \frac{k}{k+2}-\frac{k-2}{k}\right)$
	\end{enumerate}
\item Bestimmen Sie den Konvergenzradius der Potenzreihe $\sum \limits_{k=0}^\infty 2^{4k} X^k$
\end{enumerate}
%
%
\subsubsection*{Lösungsvorschlag:}
\begin{enumerate}[a)]
\item
	\begin{enumerate}[i)] \item Geometrische Reihe:
	\begin{align*}
	\sum \limits_{k=0}^\infty 6\cdot \frac{1}{3^{k-1}}&=(6\cdot 3)  \cdot \sum \limits_{k=0}^\infty \frac{1}{3}\cdot \frac{1}{3^{k-1}}\\
	&=18 \cdot \sum \limits_{k=0}^\infty \left(\frac{1}{3^{k}}\right)^{k}\\
	&=18 \cdot \frac{1}{1-\frac{1}{3}}\\
	&=18 \cdot \frac{3}{2}\\
	&=27
	\end{align*}
	\item Teleskop-Reihe:
	 \begin{align*}
	S_n&=\sum \limits_{k=1}^n\left( \frac{k}{k+2}-\frac{k-2}{k}\right)\\
	&=(\frac{1}{3}-\frac{-1}{1})+(\frac{2}{4}-\frac{0}{2})+(\frac{3}{5}-\frac{1}{3})+...+(\frac{n-2}{n}-\frac{n-4}{n-2})+(\frac{n-1}{n+1}-\frac{n-3}{n-1})+(\frac{n}{n+2}-\frac{n-2}{n})\\
	&=1+0+\frac{n-1}{n+1}+\frac{n}{n+2}\\
	&\to 1+0+1+1 = 3 \quad (n \to \infty)
	\end{align*}
	\end{enumerate}
\item 
$$
\rho = \lim_{k\to \infty} \frac{1}{\sqrt[k]{2^{4k}}}=\frac{1}{2^4}=\frac{1}{16}
$$
\end{enumerate}

%%%%%
\newpage
\Aufgabe 
\begin{enumerate}[a)]
\item Zeigen Sie, dass 
$$
f(x):= \begin{cases} \sin(x) &, x<0\\ x &, 0\le x\le 1 \\ e^{x-1} &, x>1   \end{cases}
$$
in allen $x \in \RR$ stetig ist.
\item Geben Sie die Voraussetzungen und die Aussage des Zwischenwertsatzes an.

\end{enumerate}
%
%
\subsubsection*{Lösungsvorschlag:}
\begin{enumerate}[a)]
\item Für $x<0$, $0<x<1$ und $x>1$ ist $f$ stetig da $\sin(x)$ stetig ist, $x$ stetig ist und $e^{x-1}$ als Verkettung stetiger Funktionen stetig ist.
	\begin{itemize}
	\item[$x_0=0$:] $\lim \limits_{x\to 0} \sin(x)=0=\lim \limits_{x \to 0} x$ \ und somit ist $f$ stetig in $x_0=0$.
	\item[$x_0=1$:] $\lim \limits_{x\to 1} x=1=\lim \limits_{x \to 1} e^{x-1}$ \ und somit ist $f$ stetig in $x_0=1$.
	\end{itemize}
\item \underline{Zwischenwertsatz}\\ Es sei $f:[a,b] \to \RR$ stetig und $y \in \RR$ zwischen $f(a)$ und $f(b)$. Dann gibt es ein $c \in [a,b]$ mit $f(c)=y$.
\end{enumerate}

%%%%%
\newpage
\Aufgabe
Bestimmen Sie jeweils die erste Ableitung der folgenden Funktionen

\begin{enumerate}[a)]
\item $f:(0,\infty) \to \RR, x \mapsto \cos(x) \cdot \sin(x)$
\item $g: \RR \to \RR, x \mapsto\frac{e^x}{2x^2+1}$
\item $h:\RR \to \RR, x \mapsto \log(x^2+1)$
\end{enumerate}
%
%
\subsubsection*{Lösungsvorschlag:}
%
\begin{enumerate}[a)]
\item Produktregel: $f'(x)=-\sin(x)^2+\cos(x)^2$
\item Quotientenregel: $g'(x)=\frac{e^x(2x^2+1)-e^x(4x)}{(2x^2+1)^2}$
\item Kettenregel: $h'(x)=\frac{1}{x^2+1}\cdot (2x)=\frac{2x}{x^2+1}$
\end{enumerate}


%%%%%
\newpage
\Aufgabe

\begin{enumerate}[a)]

\item Geben Sie die Voraussetzungen und die Aussage des Mittelwertsatzes an.
\item Zeigen Sie, dass für alle $a,b \in [1,\infty)$ mit $a\le b$ gilt:
$$
\log(b)-\log(a) \le b-a
$$
\end{enumerate}
%
\subsubsection*{Lösungsvorschlag:}
\begin{enumerate}[a)]
\item \underline{Mittelwertsatz}\\ Es sei $f:[a,b] \to \RR$ stetig und auf $(a,b)$ differenzierbar. Dann gibt es ein $c \in (a,b)$ mit
$$
\frac{f(b)-f(a)}{b-a}=f'(c)
$$
\item  Seien $a,b \in [1,\infty)$ mit $a< b$ beliebig (für $a=b$ ist die Aussage klar). Dann:
	$$
	\log(b)-\log(a) \le b-a \qquad \Longleftrightarrow \qquad \frac{\log(b)-\log(a)}{b-a} \le 1
	$$
	Der Logarithmus $log(x)$ ist auf $[a,b]$ stetig (da $a,b \ge 1$) und auf $(a,b)$ differenzierbar mit Ableitung $\frac{d}{dx}\log(x)=\frac{1}{x}$. Somit folgt mit dem Mittelwert, dass es ein $c\in (a,b)$ gibt mit 
		$$
		 \frac{\log(b)-\log(a)}{b-a} = \frac{1}{c}
	$$
Insbesondere ist $c>1$ und somit $ \frac{1}{c} < 1$.
\end{enumerate}

%%%%%
\newpage
\Aufgabe
Bestimmen Sie den Wert der folgenden Integrale
\begin{enumerate}[a)]
\item $\displaystyle\int \limits_{0}^1 x^2+2\ dx$
\item $\displaystyle\int \limits_{0}^2 x\cdot e^{x^2} \ dx$
\item $\displaystyle\int \limits_{1}^\infty \frac{3}{x^2}\ dx$
\end{enumerate}
%
\subsubsection*{Lösungsvorschlag:}

\begin{enumerate}[a)]
\item $\displaystyle\int \limits_{0}^1 x^2+2\ dx=[\frac{1}{3}x^3+2x]_0^1=\frac{7}{3}$ (Hauptsatz)
\item $\displaystyle\int \limits_{0}^2 x\cdot e^{x^2} \ dx=[\frac{1}{2}e^{x^2}]_0^2=\frac{1}{2}(e^4-1)$ (Hauptsatz oder partielle Integration)
\item $\displaystyle\int \limits_{1}^\infty \frac{3}{x^2}\ dx=\lim_{n \to \infty}\displaystyle\int \limits_{1}^n \frac{3}{x^2}\ dx=\lim_{n \to \infty} (3[-\frac{1}{x}]_1^n)=3\cdot ( \lim_{n \to \infty}(-\frac{1}{n}-\frac{-1}{1})) =3$
\end{enumerate}
%%%%%
\newpage
\Aufgabe
Bestimmen Sie die allgemeine Lösung der Differentialgleichung
$$
f'(x)=\sin(x)f(x)+\sin(x)
$$
%%%%%

\subsubsection*{Lösungsvorschlag:}
\begin{enumerate}[1)]
\item Zuerst wird die allgemeine Lösung der zugehörigen homogenen DGL 
$$
f'(x)=\sin(x)f(x)
$$
bestimmt. Da $-\cos(x)$ eine Stammfunktion von $\sin(x)$ ist, ist diese gegeben durch
$$
f_h(x)=c\cdot e^{-\cos(x)} ,\qquad c \in \RR
$$
\item Als nächstes wird eine spezielle Lösung der inhomogenen DGL mit Hilfe der Variation der Konstanten ermittelt. Dafür wird die Funktion $c(x)$ als Stammfunktion von $\sin(x)e^{\cos(x)}$ gewählt. Eine solche ist $c(x)=-e^{\cos(x)}$. Damit
$$
f_s(x)=-e^{\cos(x)}e^{-\cos(x)}=-1
$$
\item Insgesamt ergibt sich somit als allgemeine Lösung der inhomogenen DGL
$$
f(x)=c\cdot e^{-\cos(x)}-1 ,\qquad c \in \RR
$$
\end{enumerate}
%%%%%
%%%%%
\newpage
\Aufgabe
Gegeben ist die Kurve $\gamma: [0,1] \to \RR^2, t \mapsto (3t,4t)$.
\begin{enumerate}
\item Skizzieren Sie den Bogen von $\gamma$.
\item Zeigen Sie, dass $\gamma$ \underline{nicht} nach Bogenlänge parametrisiert ist.
\item Bestimmen Sie die Länge von $\gamma$.
\end{enumerate}

%%%%%

\subsubsection*{Lösungsvorschlag:}
\begin{enumerate}
\item Geradenstück durch $(0,0)$ und $(3,4)$
\item $\|\dot \gamma(t)\|=\sqrt{3^2+4^2}=5 \ne 1$
\item $L(\gamma)=\int_0^1 \|\dot \gamma(t)\| dt = 5$
\end{enumerate}
%%%%%
\vfill \hfill Viel Erfolg!

\end{document}