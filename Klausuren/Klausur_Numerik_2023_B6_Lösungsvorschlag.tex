\documentclass[a4paper,11pt]{scrartcl}

\usepackage{graphicx}
\usepackage[T1]{fontenc}
\usepackage[utf8]{inputenc}
\usepackage{floatflt}
\usepackage{amsfonts}
\usepackage{amsmath}
\usepackage{amssymb}
\usepackage{amsthm}
\usepackage[ngerman]{babel}
%\usepackage[T1]{fontenc}


\usepackage{paralist}
\usepackage[usenames,dvipsnames]{color} 



\newcounter{auf}
\newcommand{\Aufgabe}%
        {\addtocounter{auf}{1} \subsubsection*{\rmfamily  Aufgabe \theauf \hspace{1em}} }









\pagestyle{empty}

\oddsidemargin0mm
\parindent0mm
\parskip2mm
\textheight24cm
\textwidth15.8cm
\unitlength1mm

\newcommand{\E}{\mathbb{E}}
\newcommand{\Hy}{\mathbb{H}}
\newcommand{\N}{\mathbb{N}}
\newcommand{\RR}{\mathbb{R}}
\newcommand{\Z}{\mathbb{Z}}
\newcommand{\Q}{\mathbb{Q}}
\newcommand{\C}{\mathbb{C}}
\newcommand{\K}{\mathbb{K}}
\newcommand*\e{\mathrm{e}}
\newcommand*\ii{\mathrm{i}}
\newcommand*\re{\mathrm{Re}}
\newcommand*\im{\mathrm{Im}}
\newcommand*\id{\mathrm{id}}
\newcommand*\glnr{\mathrm{\it GL}(n,\R)}
\newcommand*\slnr{\mathrm{\it SL}(n,\R)}
\newcommand*\on{\mathrm{\it O}(n)}
\newcommand*\son{\mathrm{\it SO}(n)}
\newcommand*\rang{\mathrm{Rang}}
\newcommand*\grad{\mathrm{grad~}}
\newcommand*\dive{\mathrm{div~}}
\newcommand*\sym{\mathrm{Sym}}
\newcommand*\spur{\mathrm{Spur}}
\newcommand*\isom{\mathrm{Isom}}
                  





\begin{document}





\Aufgabe

\begin{enumerate}[a)]

\item Geben Sie die Definition der Kondition eines Berechnungsproblems 
$$
	y=f(x)
$$
mit $x \in \RR$ und $f:\RR \to \RR$ differenzierbar, an.
\item  Berechnen Sie die Jacobi-Matrix von
$$
f: \RR^3 \to \RR^2, (x,y,z) \mapsto \begin{pmatrix} x+y \\ \sin(yz) \end{pmatrix}
$$
\item Bestimmen Sie die Zeilensummennorm der Matrix
$$
M=\begin{pmatrix} 1&2&3 \\ 2&2&-6\\ 0 & 2&4 \end{pmatrix}
$$
\end{enumerate}
%

\subsubsection*{Lösungsvorschlag}
\begin{enumerate}[a)]
\item $kond_x(f)=\frac{|xf'(x)|}{|f(x)|}$
\item 
$$
J_f(x,y,z)=\begin{pmatrix}1&1&0\\0&z\cdot\cos(yz)&y\cdot\cos(yz) \end{pmatrix}
$$
\item $\|M\|_\infty = \max([1+2+3],[2+2+|-6|],[0+2+4])=\max(6,10,6)=10$
\end{enumerate}
%%%%
\newpage
\Aufgabe
Berechnen Sie das Integral
$$
\int_0^\pi \sin(x)^2 dx
$$
näherungsweise unter Verwendung
der Simpson-Regel.

%
%
\subsubsection*{Lösungsvorschlag}

\begin{align*}
S(f)&=\frac{b-a}{6}(f(a)+4f(\frac{a+b}{2})+f(b)\\
&=\frac{\pi}{6}(\sin(0)^2+4\sin(\frac{\pi}{2})^2+\sin(\pi)^2)\\
&=\frac{\pi}{6}(0^2+4\cdot 1^2+(0)^2)\\
&=\frac{2\pi}{3}
\end{align*}

%%%%%
\newpage
\Aufgabe
\begin{enumerate}[a)]
\item
Geben Sie die Butcher-Tableaus für 
\begin{enumerate}[(i)]
\item das explizite Eulerverfahren und 
\item das implizite Eulerverfahren
\end{enumerate}
an.
\item
Lösen Sie approximativ das Anfangswertproblem
$$
\begin{cases} f'(x)=2x\cdot f(x) & \text{ für } x \in [0,1] \\
f(0)=2
\end{cases}
$$
mit dem expliziten Eulerverfahren und Schrittweite $h=\frac{1}{2}$.
\end{enumerate}
%
\subsubsection*{Lösungsvorschlag}
\begin{enumerate}[a)]
\item 
$$
\begin{array}{c|c}
\text{Explizites Euler  } &\quad \text{Implizites Euler}\\
&\\
\begin{array}{c|c}
0&0\\\hline
&1
\end{array}&
\begin{array}{c|c}
1&1\\\hline
&1
\end{array}
\end{array}
$$
\item 
\begin{align*}
f(0)&=2\\
&\\
\hat f(\frac{1}{2})&=f(0)+\frac{1}{2}\cdot (2\cdot 0 \cdot f(0))\\
&=f(0)\\
&=2\\
&\\
\hat f(1)&=f(\frac{1}{2})+\frac{1}{2} \cdot (2 \cdot \frac{1}{2} \cdot \hat f(\frac{1}{2}))\\
&=2+\frac{1}{2}\cdot 2\\
&=3
\end{align*}
\end{enumerate}


%%%%%
\newpage
\Aufgabe
Es sei $f: \RR \to \RR, x \mapsto -2x^3+2x^2-2$.
\begin{enumerate}[a)]
\item Bestimmen Sie näherungsweise die Nullstelle von $f$
indem Sie einen Schritt des Newton-Verfahrens mit Startwert $x_0=-1$ durchführen.
\item Bestimmen Sie näherungsweise die Nullstelle von $f$
indem Sie zwei Schritte des Bisektionsverfahrens mit geeigneten Startwerten $a_0$ und $b_0$ durchführen (d.h. $c_1$ bestimmen).\\
Zeigen Sie dabei, dass die gewählten Startwerte die Anforderungen des Bisektionsverfahren erfüllen.
\end{enumerate}
%
%
\subsubsection*{Lösungsvorschlag}
\begin{enumerate}[a)]
\item $x_1=x_0-\frac{f(x_0)}{f'(x_0)}=-1-\frac{-2\cdot (-1)^3+2\cdot(-1)^2-2}{-6\cdot (-1)^2+4\cdot (-1)}=-1-\frac{2}{-10}=-\frac{4}{5}$
\item $a_0=-1$, $b_0=0$ und $f(a_0)=2>0$ und $f(b_0)=-2<0$ und somit $sign(f(a_0)f(b_0)=-1$.\\
$c_0=(a_0+b_0)/2=\frac{-1}{2}$ mit $f(c_0)=\frac{1}{4}+\frac{1}{2}-2=-\frac{5}{4}<0$.\\
Damit $a_1=a_0=-1$ und $b_1=c_0=\frac{-1}{2}$.\\
Somit $c_1=(a_1+b_1)/2=-\frac{3}{4}$.
\end{enumerate}

%%%%%
\newpage
\Aufgabe


\begin{enumerate}[a)]
\item Berechnen Sie das Integral
$$
\int_0^{2\pi} \int_0^1 e^{\sin(x)x^2}\cos(y) dx\ dy
$$
\item Geben Sie die Definition eines Normalbereichs $A\subset \RR^2$ bezüglich der $x$-Achse an.
\item Berechnen Sie das Integral
$$
\int_A y\cdot \cos(x^3)\ d(x,y)
$$
mit $A=\{(x,y) \in \RR^2 \mid 0 \le x \le \sqrt[3]{\pi}  \text { und } 0\le y \le 2x \}$.
\end{enumerate}
%
%
\subsubsection*{Lösungsvorschlag}
\begin{enumerate}[a)]
\item 
$$
\int_0^{2\pi} \int_0^1 e^{\sin(x)x^2}\cos(y) dx\ dy \stackrel{Fubini}{=} \int_0^1\int_0^{2\pi} e^{\sin(x)x^2}\cos(y) dy\ dx =\int_0^1 \left[ e^{\sin(x)x^2}\sin(y)  \right]_{y=0}^{y=2\pi} dx=\int_0^1 0 dx =0
$$
\item $A=\{(x,y) \mid a\le x \le b, f(x) \le y \le g(x)\}$
\item 
\begin{align*}
\int_A y\cdot \cos(x^3)\ d(x,y)&=\int_0^{\sqrt[3]{\pi}} \int_0^{2x}y\cdot \cos(x^3) dydx\\
&=\int_0^{\sqrt[3]{\pi}}\cos(x^3)( \int_0^{2x}y dy)dx\\
&=\int_0^{\sqrt[3]{\pi}}\cos(x^3)( [\frac{1}{2}y^2]_0^{2x} dx\\
&=\int_0^{\sqrt[3]{\pi}}\cos(x^3)2x^2dx\\
&=\frac{2}{3}\int_0^{\sqrt[3]{\pi}}\cos(x^3)3x^2dx\\
&=\frac{2}{3}[\sin(x^3)]_0^{\sqrt[3]{\pi}}\\
&=\frac{2}{3}(\sin(\pi)-\sin(0))\\
&=0
\end{align*}
\end{enumerate}
%
%





%%%%%
\newpage
\Aufgabe
\begin{enumerate}[a)]
\item
Interpolieren Sie die Punkte
$$
(x_0,y_0)=(1,1), (x_1,y_1)=(2,1) \text{ und } (x_2,y_2)=(3,5)
$$
durch ein quadratisches Polynom.
\item Wieviele übrige Freiheitsgrade hätten Sie bei einer Spline-Interpolation mit kubischen \\Polynomen (k=3) ?
\end{enumerate}
%
%
\subsubsection*{Lösungsvorschlag}
\begin{enumerate}[a)]
\item
$p(x)=ax2+bx+c$ damit
\begin{align*}
1&=a+b+c\\
1&=4x+2b+c\\
5&=9a+3b+c
\end{align*}
Das ergibt das LGS
$$
\begin{pmatrix} 1&1&1 \\ 4 & 2 & 1\\ 9&3&1 \end{pmatrix}\begin{pmatrix}a\\b\\c\end{pmatrix}=\begin{pmatrix} 1\\1\\5 \end{pmatrix}
$$
äquivalent zu
$$
\begin{pmatrix} 1&1&1 \\ 1&2&4\\ 1&3&9 \end{pmatrix}\begin{pmatrix}c\\b\\a\end{pmatrix}=\begin{pmatrix} 1\\1\\5 \end{pmatrix}
$$
mit 
\begin{align*}
\left( \begin{array}{ccc|c} 1&1&1&1\\ 1&2&4&1\\1&3&9&5 \end{array}\right) \sim> \left( \begin{array}{ccc|c} 1&1&1&1\\ 0&1&3&0\\0&2&8&4 \end{array} \right) \sim> \left( \begin{array}{ccc|c} 1&1&1&1\\ 0&1&3&0\\0&0&2&4 \end{array} \right) \sim> \left( \begin{array}{ccc|c} 1&0&0&5\\ 0&1&0&-6\\0&0&1&2 \end{array} \right)
\end{align*}
uns somit $a=2,b=-6$ und $c=5$.\\
Das ergibt
$$
p(x)=2x^2-6x+5
$$
\item $k-1=2$
\end{enumerate}
\quad\\

\vfill \hfill \textbf{Viel Erfolg!}





\end{document}