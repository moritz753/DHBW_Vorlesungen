\documentclass[
				a4paper,
				10pt
			]
			{scrartcl}

\parindent0mm

\usepackage{amsfonts}
\usepackage{amsmath}
\usepackage{amssymb}
\usepackage{amsthm}
\usepackage[ngerman]{babel}
\usepackage{graphicx}
\usepackage{algorithm2e}

\usepackage[
			pdftex,
			colorlinks,
			breaklinks,
			linkcolor=blue,
			citecolor=blue,
			filecolor=black,
			menucolor=black,
			urlcolor=black,
			pdfauthor={Andreas Weber},
			pdftitle={Aufgaben zur Logik und Algebra},
			plainpages=false,
			pdfpagelabels,
			bookmarksnumbered=true
		   ]{hyperref}


 %%%%%%%%%%%%%%%Schriften%%%%%%%%%%%%%
\DeclareMathAlphabet{\lier}{U}{eur}{m}{n}  %% Gothisch/Fraktur - Roman



\newcommand{\M}{\mathbb{M}}
\newcommand{\E}{\mathbb{E}}
\newcommand{\Hy}{\mathbb{H}}
\newcommand{\N}{\mathbb{N}}
\newcommand{\Q}{\mathbb{Q}}
\newcommand{\R}{\mathbb{R}}
\newcommand{\Z}{\mathbb{Z}}
\newcommand{\C}{\mathbb{C}}
\newcommand{\K}{\mathbb{K}}
\newcommand*\ee{\mathrm{e}}
\newcommand*\e{\mathrm{e}}

\newcommand*\ii{\mathrm{i}}
\newcommand*\re{\mathrm{Re}}
\newcommand*\im{\mathrm{Im}}
\newcommand*\id{\mathrm{id}}
\newcommand*\rang{\mathrm{rang}}
\newcommand*\grad{\mathrm{grad}}
\newcommand*\dive{\mathrm{div}}
\newcommand*\sym{\mathrm{Sym}}
\newcommand*\spur{\mathrm{Spur}}
\newcommand*\isom{\mathrm{Isom}}
\newcommand*\vol{\mathrm{vol\,}}
\newcommand*\supp{\mathrm{supp}}
\newcommand*\inj{\mathrm{inj}}
\newcommand*\rank{\mathrm{rank}}
\newcommand*\qrank{\Q\mbox{-}\mathrm{rank}}
\newcommand*\rrank{\R\mbox{-}\mathrm{rank}}
\newcommand*\dom{\mathrm{dom}}
\newcommand*\tr{\mathrm{tr}}
\newcommand*\spa{\mathrm{span}}
\newcommand*\diam{\mathrm{diam}}

\newcommand*\ric{\mathrm{Ric}}

\newcommand*\con{\mathrm{con}}
\newcommand*\dis{\mathrm{dis}}

\newcommand\PX{{\cal P}(X)}
\newcommand\T{{\cal T}}
\newcommand\B{{\cal B}}



\newcommand\scp{\langle\cdot,\cdot\rangle}     %% Metric

%%%%%%%%%%%%Tilde%%%%%%%
\newcommand\tx{\tilde{x}}
\newcommand\ty{\tilde{y}}
\newcommand\tu{\tilde{u}}
\newcommand\tk{\tilde{k}}
\newcommand\td{\tilde{d}}
\newcommand\tD{\tilde{D}}
\newcommand\tX{\tilde{X}}
\newcommand\tY{\tilde{Y}}
\newcommand\tZ{\tilde{Z}}


%%%%%%Lie-Gruppen%%%%%%%%%
\newcommand\ad{\mathrm{ad}}
\newcommand\Ad{\mathrm{Ad}}
\newcommand{\kak}{K\exp\overline{\lier{a}^+}K}              %%%%Cartan-Zerlegung
\newcommand*\Rang{\mathrm{Rang}}
\newcommand*\glnr{\mathrm{\it GL}(n,\R)}
\newcommand*\glnc{\mathrm{\it GL}(n,\C)}
\newcommand*\slnr{\mathrm{\it SL}(n,\R)}
\newcommand*\on{\mathrm{\it O}(n)}
\newcommand*\son{\mathrm{\it SO}(n)}
\newcommand*\SLzr{\mathrm{\it SL}(2,\R)}
\newcommand*\SOzr{\mathrm{\it SO}(2,\R)}

%%%%%%%%%%%%%Algebraische Gruppen
\newcommand\bG{{\bf G}}
\newcommand\bT{{\bf T}}
\newcommand\bP{{\bf P}}
\newcommand\bN{{\bf N}}
\newcommand\bL{{\bf L}}
\newcommand\bS{{\bf S}}
\newcommand\bM{{\bf M}}

\newcommand\Mor{\mathrm{Mor}}



%%%%%%Geometry%%%%%%%%%%%
\newcommand{\Si}{\mathcal{S}}


%%%%%%Ableitungsoperatoren%%%%%%%%%%%
\newcommand*\pddt{\frac{\partial}{\partial t}}
\newcommand*\pddx{\frac{\partial}{\partial x}}
\newcommand*\pddxio{\frac{\partial}{\partial x^i}}
\newcommand*\pddxjo{\frac{\partial}{\partial x^j}}
\newcommand*\pddxko{\frac{\partial}{\partial x^k}}
\newcommand*\pddxlo{\frac{\partial}{\partial x^l}}

\newcommand*\pddy{\frac{\partial}{\partial y}}
\newcommand*\pddyio{\frac{\partial}{\partial y^i}}
\newcommand*\pddyjo{\frac{\partial}{\partial y^j}}
\newcommand*\pddyko{\frac{\partial}{\partial y^k}}
\newcommand*\pddylo{\frac{\partial}{\partial y^l}}

\newcommand*\pddyq{\frac{\partial^2}{\partial y^2}}
\newcommand*\pddyj{\frac{\partial}{\partial y_j}}
\newcommand*\pddyjq{\frac{\partial^2}{\partial y_j^2}}
\newcommand*\pddxiq{\frac{\partial^2}{\partial x_i^2}}
\newcommand*\pddxi{\frac{\partial}{\partial x_i}}
\newcommand*\ddt{\frac{d}{dt}}

\newcommand*\dx{\,dvol(x)}
\newcommand*\dy{\,dvol(y)}
\newcommand*\dty{\,dvol(\ty)}

\newcommand*\DMp{\Delta_{M,p}}                 %%%%Laplace-Operatoren
\newcommand*\DMq{\Delta_{M,q}}
\newcommand*\DM{\Delta_M}
\newcommand*\DX{\Delta_X}
\newcommand*\DXp{\Delta_{X,p}}
\newcommand*\DXq{\Delta_{X,q}}
\newcommand*\DAx{\Delta_{Ax_0}}
\newcommand*\Rad{\mathrm{Rad}}
\newcommand*\DXps{\Delta^{\#}_{X,p}}

\newcommand*\eDXps{\e^{-t(\Delta^{\#}_{X,p}-c)}} %%%%%% Semigroups
\newcommand*\LpsX{L^p_{\#}(X)}


%%%%%%%%%%%%%%%Komplexe Analysis
\newcommand*\Res{\mathrm{Res}}

%%%%%%%%%%%%%%%Definitionsmenge
\newcommand*\D{{\cal D}}







\author{Dr. Moritz Gruber\\ DHBW Karlsruhe}
\title{L\"osungen \"Ubungsaufgaben 5\\ 
	Ringhomomorphismen \& K\"orper
}
\date{}

%%%%%%%%%
\begin{document}
%%%%%%%%%
\maketitle

%%%
\section{Ringhomomorphismen}
Es seien $R,S$ zwei Ringe mit Eins und $R\times S$ das kartesische Produkt dieser Ringe. Zeigen Sie, dass die Abbildungen
		$$p_1 : R\times S \to R, (r,s) \mapsto r \ \text{ und } \ p_2 : R\times S \to S, (r,s) \mapsto s$$
Ring-Homomorphismen sind.
%%%
\paragraph{L\"osung:} \quad\\
%%%
F\"ur alle $(r,s),(x,y) \in R\times S$ gilt:\\
$$p_1((r,s)+(x,y))=p_1((r+x,s+y))=r+x=p_1((r,s))+p_1((x,y))$$
und
$$p_1((r,s)\cdot(x,y))=p_1((r\cdot x,s\cdot y))=r\cdot x=p_1((r,s))\cdot p_1((x,y))$$
Au\ss erdem gilt
$$
p_1((1_R,1_S)=1_R
$$
Damit ist $p_1$ ein Ring-Homomorphismus.\\
\quad\\
Analog zeigt man dies nun auch f\"ur $p_2$:\\
F\"ur alle $(r,s),(x,y) \in R\times S$ gilt:\\
$$p_2((r,s)+(x,y))=p_2((r+x,s+y))=s+y=p_2((r,s))+p_2((x,y))$$
und
$$p_2((r,s)\cdot(x,y))=p_2((r\cdot x,s\cdot y))=s\cdot y=p_2((r,s))\cdot p_2((x,y))$$
Au\ss erdem gilt
$$
p_2((1_R,1_S)=1_S
$$
Damit ist $p_2$ ein Ring-Homomorphismus.
%%%
\section{Schon bekannte Ringe}
%%%
\begin{itemize}
\item[a)] Sei
$
	R := \{ a + b\sqrt{2} ~|~a,b \in \Z \}. 
$
Ist $R$ ein K\"orper?
\item[b)]
Zeigen Sie, dass $(\Z_3,+_3,\cdot_3)$ ein K\"orper ist. Gilt das auch f\"ur $(\Z_4,+_4,\cdot_4)$?
\end{itemize}
%%%
%%%
\paragraph{L\"osung:}
%%%
\begin{itemize}
\item[a)]
$(R,+,\cdot)$ ist kein K\"orper, da z.B. das Element $\sqrt{2} = 0 + 1\cdot\sqrt{2}\in R\backslash\{0\}$ kein inverses Element in $R$ bzgl. $\cdot$ besitzt:
$$
	\sqrt{2}\cdot x = 1 \iff x = \frac{1}{\sqrt{2}}=0+\frac{1}{2}\sqrt{2} \notin R.
$$
\item[b)]
Ein K\"orper ist ein kommutativer Ring mit Eins, sodass alle Elemente au\ss er der Null ein multiplikatives Inverses besitzen.\\
F\"ur alle $n \in \N$ ist $(\Z_n,+_n,\cdot_n)$ ein kommutativer Ring mit Eins, da die Kommutativit\"at von $\cdot_n$ direkt aus der von $\cdot$ f\"ur $\Z$ folgt und $[1]_n$ das Einselement ist.\\
Es bleiben also die multiplikativen Inversen:\\
Daf\"ur betrachten wir $\Z_3=\{[0]_3,[1]_3,[2]_3\}$ und $\Z_3 \setminus \{[0]_3\}=\{[1]_3,[2]_3\}$. Da das Einselement aus seiner Definition heraus das eigene Inverse ist, m\"ussen wir nur ein multiplikatives Inverses f\"ur $[2]_3$ finden. Es gilt $[2]_3 \cdot_3 [2]_3=[4]_3=[1+3]_3=[1]_3$ und somit ist $[2]_3$ das multiplikative Inverse zu $[2]_3$. \\Damit $(\Z_3,+_3,\cdot_3)$ ist ein K\"orper.\\
\quad\\
F\"ur $\Z_4$ funktioniert das nicht, da $(\Z_4,+_4,\cdot_4)$ nicht nullteilerfrei ist: 
$$
[2]_4 \cdot_4 [2]_4 = [4]_4=[0+4]_4=[0]_4
$$
Nach der Vorlesung kann ein Nullteiler kein multiplikatives Inverses besitzen.
%
\paragraph{Bemerkung:} Man kann zeigen, dass $(\Z_n,+_n,\cdot_n)$ genau dann ein K\"orper ist, wenn $n$ eine Primzahl ist.
\end{itemize}



%%%
\section{K\"orper}
%%%

Sei
$$
	K := \{ a + b\sqrt{2} ~|~a,b \in \Q \}. 
$$
Zeigen Sie, dass $(K,+,\cdot)$ mit der \"ublichen Addition $+$ und Multiplikation $\cdot$ ein  K\"orper ist.
%%%
\paragraph{L\"osung:} 
%%%
Alle Eigenschaften eines kommutativen Ringes zeigt man wie in Aufgabe 1 auf  \"Ubungsblatt 4. Es bleibt also nur zu zeigen, dass jedes Element in $K \setminus \{0\}$ ein multiplikatives Inverses besitzt. Seien dazu $z=a_1 + b_1 \sqrt{2}, \tilde z=a_2+b_2\sqrt{2} \in K\setminus \{0\}$. Dann gilt:
$$
z \cdot \tilde z =(a_1 + b_1\sqrt{2}) \cdot (a_2 + b_2\sqrt{2}) = (a_1a_2 + 2b_1b_2) + (a_1b_2 + a_2b_1)\sqrt{2}
$$
Setzt man nun $z \cdot \tilde z=1$, so erh\"alt man zwei Gleichungen:
\begin{align*}
a_1a_2 + 2b_1b_2&=1\\
a_1b_2 + a_2b_1&=0
\end{align*}
\begin{itemize}
\item[$b_1=0$:]
Die beiden Gleichungen vereinfachen sich zu 
\begin{align*}
a_1a_2 &=1\\
a_1b_2&=0
\end{align*}
und es ergibt sich $a_2=\frac{1}{a_1}$ und $b_2=0$ und somit $\tilde z = \frac{1}{a_1}$.
\item[$b_1\ne 0$:]
L\"ost man die zweite Gleichung nach $a_2$ auf, so erh\"alt man $a_2=\frac{-a_1b_2}{b_1}$. Dies kann man wiederum in die erste Gleichung einsetzen und erh\"alt
$$
a_1\cdot \frac{-a_1b_2}{b_1} + 2b_1b_2=1
$$
L\"ost man diese Gleichung nun nach $b_2$, so ergibt das $b_2=\frac{b_1}{2b_1^2-a_1^2}$. Setzt man $b_2$ nun zur\"uck in die Formel von $a_2$ ein, erh\"alt man $a_2=\frac{-a_1}{2b_1^2-a_1^2}$.\\
Somit ist $\tilde z=\frac{-a_1}{2b_1^2-a_1^2} + \frac{b_1}{2b_1^2-a_1^2} \cdot \sqrt{2}$ das zu $z$ Inverse Element und es gilt auch offensichtlich $\tilde z \in K$. 
\end{itemize}
Damit ist $K$ ein K\"orper.



%%%
\section{komplexe Zahlen}

\begin{itemize}
\item[(a)] Bestimmen Sie additiven und multiplikativen Inversen f\"ur die folgenden komplexen Zahlen:
\begin{itemize}
\item[i)] $z_1=i$
\item[ii)] $z_2=1+i$
\item[iii)] $z_3=3-2i$
\end{itemize}

\item[(b)] L\"osen Sie die folgende Gleichung einmal in $\R$ und einmal in $\C$:
$$
2x^2+2x+1=0
$$
\end{itemize}
%%%
\paragraph{L\"osung:}
%%%
\begin{itemize}
\item[(a)]
\quad\\
$
\begin{array}{l|l|l}
&\text{additives Inverses} & \text{multiplikatives Inverses}\\\hline\hline
z_1&-z_1=-i & z_1^{-1}=-i \\\hline
z_2&-z_2=-1-i & z_2^{-1}=\frac{1}{2} - \frac{1}{2}i  \\\hline
z_3&-z_3=-3+2i & z_3^{-1}=\frac{3}{13}-\frac{-2}{13}i
\end{array}
$
\quad\\
\item[(b)]
Mit der Mitternachtsformel erhalten wir
$$
x_{1/2}=\frac{-b\pm \sqrt{b^2-4ac}}{2a}=\frac{-2\pm \sqrt{-4}}{4}
$$
In den reellen Zahlen gibt es keine Wurzel von $-4$ und somit auch keine L\"osung f\"ur die quadratische Gleichung.\\
In den komplexen Zahlen gilt dagegen $\sqrt{-4}=2i$. Damit gibt es in $\C$ folgende zwei L\"osungen f\"ur die quadratische Gleichung:
\begin{itemize}
\item[(i)] $x_1=\frac{-2+ \sqrt{-4}}{4}=-\frac{1}{2} + \frac{1}{2}i$
\item[(ii)] $x_2=\frac{-2- \sqrt{-4}}{4}=-\frac{1}{2} - \frac{1}{2}i$
\end{itemize}

\end{itemize}
\end{document}