\documentclass{beamer}

\usepackage{beamerthemesplit}

\usepackage{amsfonts}
\usepackage{amsmath}
\usepackage{amssymb}
\usepackage{amsthm}
\usepackage{amscd}

\usepackage{stmaryrd} 					%\lightning
\usepackage{algorithm2e}


\usepackage[ngerman]{babel}

\usepackage[utf8]{inputenc}
\usepackage[T1]{fontenc}
\usepackage{textcomp}


% Color Definitions
\definecolor{dhbwRed}{RGB}{226,0,26} 
\definecolor{dhbwGray}{RGB}{61,77,77}
\definecolor{lightBlue}{RGB}{28,134,230}

% Basic Theme
\usetheme{Malmoe}

% Color Re-Definitions
\usecolortheme[named=lightBlue]{structure}
\setbeamercolor*{alerted  text}{fg=dhbwRed, bg=white}
\setbeamercolor*{subsection in toc}{fg=dhbwGray, bg=white}

%\setbeamercolor*{palette primary}{fg=white,bg=lightBlue}
%\setbeamercolor*{palette secondary}{fg=white,bg=gray}
%\setbeamercolor*{palette tertiary}{fg=white,bg=gray}
%\setbeamercolor*{palette quaternary}{fg=white,bg=dhbwRed}

% no navigation symbols
\setbeamertemplate{navigation symbols}{}

% headline, footline
\setbeamertemplate{footline}{\color{dhbwGray} \hfill\insertframenumber\hspace{5mm}\vspace{2mm}}
\setbeamertemplate{headline}{}

% Title Page
\newcommand*{\makeTitlePage}{
	
	\begin{frame}[plain]
		
		\vfill
		\vfill
		\begin{center}
			{
				\usebeamerfont{title}
				\usebeamercolor[fg]{title}
				\Large
				\inserttitle
			}\\[3mm]
			{	
				\usebeamerfont{subtitle}
				\usebeamercolor[fg]{subtitle}
				\large
				\insertsubtitle
			}
		\end{center}
		%
		\vfill
		\vfill
		\vfill
		\vfill
		%
		\begin{columns}
			\begin{column}{0.5\textwidth}
				\begin{flushleft}
					{
						\usebeamerfont{normal text}
						\color{dhbwGray!80}
						\scriptsize
						Dr. Moritz Gruber\\
						DHBW Karlsruhe\\
						
					}
				\end{flushleft}
			\end{column}
			%
			\begin{column}{0.5\textwidth}
				\begin{flushright}
					\includegraphics[scale=0.06]{../DHBW.png}
				\end{flushright}
			\end{column}
		\end{columns}
		%
		\vspace{1mm}
		\begin{columns}
			\begin{column}{0.5\textwidth}
				\begin{flushleft}
					{
						\usebeamerfont{normal text}
						\color{dhbwGray!80}
						\scriptsize
						Version \today
					}
				\end{flushleft}
			\end{column}
			%
			\begin{column}{0.5\textwidth}
				% nothing (just a placeholder to be in line with the columns above
			\end{column}
		\end{columns}
	\end{frame}

}

% Section Divider Page
\newcommand*{\makeSectionDividerPage}{

	\begin{frame}[plain]
		\begin{center}
			\begin{flushleft}
				{				
					\usebeamercolor[fg]{frametitle}
					{\Large \insertsection} \\[3mm]
					{\large \insertsubsection}
				}
			\end{flushleft}
		\end{center}
        \end{frame}
	
}

% itemize
\setbeamertemplate{itemize items}[circle]
\setbeamertemplate{enumerate item}{(\theenumi)}




%--------------------------------------%
% Math ------------------------------%
%--------------------------------------%

% Mengen (Zahlen)
\newcommand{\N}{\mathbb{N}}
\newcommand{\Q}{\mathbb{Q}}
\newcommand{\R}{\mathbb{R}}
\newcommand{\Z}{\mathbb{Z}}
\newcommand{\C}{\mathbb{C}}

% Mengen (allgemein)
\newcommand{\K}{\mathbb{K}}
\newcommand\PX{{\cal P}(X)}

% Zahlentheorie
\newcommand{\ggT}{\mathrm{ggT}}


% Ableitungen
\newcommand{\ddx}{\frac{d}{dx}}
\newcommand{\pddx}{\frac{\partial}{\partial x}}
\newcommand{\pddy}{\frac{\partial}{\partial y}}
\newcommand{\grad}{\text{grad}}

%--------------------------------------%
% Layout Colors ------------------%
%--------------------------------------%
\newcommand*{\highlightDef}[1]{{\color{lightBlue}#1}}
\newcommand*{\highlight}[1]{{\color{lightBlue}#1}} % after theme for colours


%----------------------------------------------------------------------------------------------------
%--------- Document Title ---------------------------------------------------------------------
\title{Lineare Algebra\\[3mm] 
	\large Grad-Formel für Polynommultiplikation
}
\author{Dr. Moritz Gruber } 
\institute{DHBW Karlsruhe}
\date{2022}
%%%%%%%%%%%%%%
\begin{document}

\AtBeginSection[]{
	\begin{frame}				
		\usebeamercolor[fg]{frametitle}
		{\Large \insertsection} 
        \end{frame}
}

%
\begin{frame}[plain] 
 \titlepage
\end{frame}
%

\begin{frame}\frametitle{Satz}
Wenn $R$ ein nullteilerfreier Ring ist, dann 
%ist auch der Polynomring $R[X]$ nullteilerfrei und es 
gilt Gleichheit in der Grad-Formel:
$$
\deg(f\cdot g) = \deg(f) +\deg(g) \quad \forall f,g \in \R[X] \setminus \{0\}
$$\pause
\vfill
\highlightDef{Beweis:}\\
Es seien $f,g \in R[X]\setminus \{0\}$ und $n=deg(f), m=deg(g)$. Dann gibt es Koeffizienten $a_i,b_j \in R$ sodass
$$
f=\sum_{i=0}^n a_iX^i , \ g=\sum_{j=0}^m b_jX^j \quad \text{mit } a_n\ne0\ne b_m
$$\pause
Es gilt dann
	$$
	(f \cdot g)(X) :=	(a_0  b_0) + (a_0 b_1+ a_1 b_0)X +\ldots +a_n b_m X^{n+m}
	$$ \pause
Da $R$ nullteilerfrei ist gilt $a_nb_m \ne 0$ und damit $deg(f\cdot g)=n+m$.\\
\hfill $\square$
\end{frame}
%
\begin{frame}\frametitle{Der Grad des Nullpolynoms}
Im Satz wurde das Nullpolynom $0 \in R[X]$ ausgeschlossen, da unsere bisherige Definition für $f=0$ nicht funktioniert.\\\pause
\vfill
Was ist $deg(0)$?\pause\\
\vfill
\highlightDef{Definition}\\
Der Grad eines Polynoms $f=\sum_{k=0}^d a_kX^k \in R[X]$ ist definiert als
$$
deg(f):=\begin{cases} \max(\{k \in \N_0 \mid a_k\ne 0\}), \ & f\ne 0 \\\pause
						- \infty, & f=0 \end{cases}
$$\pause 
\vfill
Es gilt: $\forall m \in \N_0:  -\infty < m \text{ und somit } \max(m,-\infty)=m,$\\ \hspace{11mm} und $ -\infty+m=-\infty$ sowie $-\infty + (-\infty)=-\infty$.

\end{frame}
%
\begin{frame}\frametitle{Satz}
Wenn $R$ ein nullteilerfreier Ring ist, dann 
ist auch der Polynomring $R[X]$ nullteilerfrei und es 
gilt Gleichheit in der Grad-Formel:
\vspace{2mm}
$
\deg(f\cdot g) = \deg(f) +\deg(g) \quad \forall f,g \in \R[X] 
$\pause\\
\vspace{2mm}
\highlightDef{Beweis:}\\
Ist eines der Polynome $f,g$ das Nullpolynom (o.B.d.A. $f=0$), so gilt $f\cdot g=0$. \pause Damit: $deg(f\cdot g)=deg(0)=-\infty\pause=-\infty + deg(g)\pause=deg(f)+deg(g)$.\\\pause
Es seien $f,g \in R[X]\setminus \{0\}$ und $n=deg(f), m=deg(g)$. Dann gibt es Koeffizienten $a_i,b_j \in R$ sodass
$$
f=\sum_{i=0}^n a_iX^i , \ g=\sum_{j=0}^m b_jX^j \quad \text{mit } a_n\ne0\ne b_m
$$\pause
Es gilt dann
	$$
	(f \cdot g)(X) :=	(a_0  b_0) + (a_0 b_1+ a_1 b_0)X +\ldots +a_n b_m X^{n+m}
	$$ \pause
Da $R$ nullteilerfrei ist gilt $a_nb_m \ne 0$ und damit $deg(f\cdot g)=n+m$.
\end{frame}
%
\begin{frame}\frametitle{Satz}
Wenn $R$ ein nullteilerfreier Ring ist, dann ist auch der Polynomring $R[X]$ nullteilerfrei und es gilt Gleichheit in der Grad-Formel:
$$
\deg(f\cdot g) = \deg(f) +\deg(g) \quad \forall f,g \in \R[X]
$$
\vfill
\highlightDef{Beweis (Fortsetzung):}\\
Es seien nun $f,g \in R[X]$ und $f\cdot g=0$. \pause Es folgt dann mit der Grad-Formel:\\
\vspace{2mm}
$
-\infty=deg(0)\pause=deg(f\cdot g)\pause =deg(f) + deg(g)
$\\
\vspace{2mm}\pause
Da $\forall d \in \N_0: \ -\infty < d$ gilt, muss somit schon $deg(f)=-\infty$ oder $\deg(g)=-\infty$ gelten und damit $f=0$ oder $g=0$. Damit ist auch $R[X]$ nullteilerfrei. \hfill $\square$
\end{frame}
%
\begin{frame}\frametitle{Zusammenfassung}
Damit erhalten wir für den Grad von Polynomen:\\
\vfill
$\forall f,g \in R[X]:$
\begin{itemize}
\item $deg(f+g)\le \max(deg(f),deg(g))$ \pause
\item $deg(f\cdot g)\le deg(f)+deg(g)$ \pause
\end{itemize}
\vfill
und falls $R$ nullteilerfrei ist sogar
\begin{itemize}
\item $deg(f\cdot g)= deg(f)+deg(g)$
\end{itemize}
\end{frame}
%%%
%

\end{document}