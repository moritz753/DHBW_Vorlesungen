\documentclass[
				a4paper,
				10pt
			]
			{scrartcl}

\parindent0mm

\usepackage{amsfonts}
\usepackage{amsmath}
\usepackage{amssymb}
\usepackage{amsthm}
\usepackage[ngerman]{babel}

\usepackage[utf8]{inputenc}
\usepackage[T1]{fontenc}
\usepackage{textcomp}

\usepackage{graphicx}
\usepackage{xcolor}

\usepackage[
			pdftex,
			colorlinks,
			breaklinks,
			linkcolor=blue,
			citecolor=blue,
			filecolor=black,
			menucolor=black,
			urlcolor=black,
			pdfauthor={Andreas Weber},
			pdftitle={Aufgaben zu Analysis und Lineare Algebra},
			plainpages=false,
			pdfpagelabels,
			bookmarksnumbered=true
		   ]{hyperref}


%--------------------------------------%
% Math ------------------------------%
%--------------------------------------%

% Mengen (Zahlen)
\newcommand{\N}{\mathbb{N}}
\newcommand{\Q}{\mathbb{Q}}
\newcommand{\R}{\mathbb{R}}
\newcommand{\Z}{\mathbb{Z}}
\newcommand{\C}{\mathbb{C}}

% Mengen (allgemein)
\newcommand{\K}{\mathbb{K}}
\newcommand\PX{{\cal P}(X)}

% Zahlentheorie
\newcommand{\ggT}{\mathrm{ggT}}


% Ableitungen
\newcommand{\ddx}{\frac{d}{dx}}
\newcommand{\pddx}{\frac{\partial}{\partial x}}
\newcommand{\pddy}{\frac{\partial}{\partial y}}
\newcommand{\grad}{\text{grad}}

%--------------------------------------%
% Layout Colors ------------------%
%--------------------------------------%
\newcommand*{\highlightDef}[1]{{\color{lightBlue}#1}}
\newcommand*{\highlight}[1]{{\color{lightBlue}#1}}
% Color Definitions
\definecolor{dhbwRed}{RGB}{226,0,26} 
\definecolor{dhbwGray}{RGB}{61,77,77}
\definecolor{lightBlue}{RGB}{28,134,230}


%-------------------------------------------------------------------
\begin{document}

\vspace*{-20mm}
{
	%\usekomafont{title}
	\color{dhbwGray}
	Dr. Moritz Gruber	\hfill Version \today\\
	DHBW Karlsruhe\\
}

\vspace{10mm}
\begin{center}
	{
		\usekomafont{title}
		\color{lightBlue}
		{ \LARGE 	Übungsaufgaben 8}\\[3mm]
		{\Large Lineare Gleichungssysteme}
	}
\end{center}

\vspace{5mm}

%-------------------------------------------------------------------


%-------------------------
\section{LGS }
%%%

Berechnen Sie die L\"osungsmenge des LGS

\begin{align*}
	x_1 - x_2 + 6x_3 + 8x_4 &= 1,\\
	-x_1 +2x_2 - 7x_3-2x_4 &= 1.
\end{align*}

%-------------------------
\section{Noch ein LGS}
%%%

Berechnen Sie die L\"osungsmenge des LGS

$$
	\left(
	\begin{array}{cccccc|c}
		0	& 2	& 4	& -2 	& 1	& 7	& -1\\
		1	& 0	& 1	& 3	& 0	& -1	& 1\\
		1	& 1	& 3	& 2	& 0	& 1	& 1\\
		0	& 1	& 2	& -1	& -1	& -1	& 1\\
		3	& 2	& 7	& 7	& -1	& -2	& 4
	\end{array}
	\right).
$$

%-------------------------
\section{L\"osbarkeit *}
%%%

Seien $\alpha,\beta \in \R$ und
$$
	A =
	\begin{pmatrix}
		1	&5	&\alpha	\\
		0	&-2	&1	\\
		-1	&1	&3
	\end{pmatrix},
	\qquad
	b =
	\begin{pmatrix}
		-1	\\
		\beta	\\
		1
	\end{pmatrix}.
$$

F\"ur welche Werte von $\alpha$ und $\beta$ ist das LGS $A\cdot x = b$
\begin{itemize}
	\item eindeutig l\"osbar,
	\item nicht l\"osbar,
	\item l\"osbar, aber nicht eindeutig l\"osbar?
\end{itemize}


%-------------------------
\section{Spezielle Matrizen II}
%%%

Sei 
$$
	D = 
	\begin{pmatrix}
		d_{11}	&\cdots	&d_{15}\\
		d_{21}	&\cdots	&d_{25}\\		
		d_{31}	&\cdots	&d_{35}\\
		d_{41}	&\cdots	&d_{45}\\
	\end{pmatrix}
	\in \R^{4\times 5}.
$$

\begin{itemize}
	\item[(a)] Geben Sie eine invertierbare Matrix $A\in\R^{4\times 4}$ an, so dass
			$$
				A\cdot D = 
				\begin{pmatrix}
					d_{31}	&\cdots	&d_{35}\\
					d_{21}	&\cdots	&d_{25}\\		
					d_{11}	&\cdots	&d_{15}\\
					d_{41}	&\cdots	&d_{45}\\
				\end{pmatrix}.
			$$
			(Vertauschung von Zeilen 1 und 3.)
	\item[(b)] Geben Sie eine invertierbare Matrix $B\in\R^{4\times 4}$ an, so dass
			$$
				B\cdot D = 
				\begin{pmatrix}
					d_{11} 			&\cdots	&d_{15} \\
					d_{21}			&\cdots	&d_{25}\\		
					2d_{31}			&\cdots	&2d_{35}\\
					d_{41}			&\cdots	&d_{45}\\
				\end{pmatrix}.
			$$
			(Multiplikation von Zeile 3 mit 2.)
	\item[(c)] Geben Sie eine invertierbare Matrix $C\in\R^{4\times 4}$ an, so dass
			$$
				C\cdot D = 
				\begin{pmatrix}
					d_{11} 			&\cdots	&d_{15} \\
					d_{21}			&\cdots	&d_{25}\\		
					2d_{11} + d_{31}	&\cdots	&2d_{15} + d_{35}\\
					d_{41}			&\cdots	&d_{45}\\
				\end{pmatrix}.
			$$
			(Addition des 2-fachen von Zeile 1 zu Zeile 3.)
\end{itemize}
\quad

\section{Nochmal die Abbildungsmatrix}
%%%
Es sei $\Psi: \R^3 \to \R^3$ eine lineare Abbildung mit
$$
\Psi(\begin{pmatrix}1\\1\\1 \end{pmatrix}) = \begin{pmatrix}2 \\ 1 \\ 2 \end{pmatrix}, \quad \Psi(\begin{pmatrix} 2 \\ 0\\ 2 \end{pmatrix}) = \begin{pmatrix}0\\2\\4 \end{pmatrix} \quad \text{ und } \quad \Psi(\begin{pmatrix} 0\\1\\1\end{pmatrix} )= \begin{pmatrix} 2 \\ 0 \\ 2\end{pmatrix}.
$$
Bestimmen Sie eine Matrix $A \in \R^{3\times 3}$, sodass $\Psi=\Phi_A$ gilt, indem Sie das LGS
$$
A\cdot \begin{pmatrix}1\\1\\1 \end{pmatrix}= \begin{pmatrix}2 \\ 1 \\ 2 \end{pmatrix},\ 
A\cdot \begin{pmatrix}2\\0\\2 \end{pmatrix}= \begin{pmatrix}0 \\ 2 \\ 4 \end{pmatrix} \text{ und }
A\cdot \begin{pmatrix}0\\1\\1 \end{pmatrix}= \begin{pmatrix}2 \\ 0 \\ 2 \end{pmatrix} 
$$
lösen. 
%

\section{Inverse Matrix}
%%%
Bestimmen Sie mit Hilfe des Gauß-Verfahrens die inverse Matrix zu
$$
A= \begin{pmatrix}1&2 & 0&0 \\ 0 & 2 & 4 & 0 \\ 0& 2 & 0 & 1 \\ 5 & 0& 5 & 0  \end{pmatrix}
$$

\end{document}
