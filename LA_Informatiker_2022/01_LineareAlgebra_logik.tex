\documentclass{beamer}

\usepackage{beamerthemesplit}

\usepackage{amsfonts}
\usepackage{amsmath}
\usepackage{amssymb}
\usepackage{amsthm}
\usepackage{amscd}

\usepackage{stmaryrd} 					%\lightning
\usepackage{algorithm2e}


\usepackage[ngerman]{babel}

\usepackage[utf8]{inputenc}
\usepackage[T1]{fontenc}
\usepackage{textcomp}


% Color Definitions
\definecolor{dhbwRed}{RGB}{226,0,26} 
\definecolor{dhbwGray}{RGB}{61,77,77}
\definecolor{lightBlue}{RGB}{28,134,230}

% Basic Theme
\usetheme{Malmoe}

% Color Re-Definitions
\usecolortheme[named=lightBlue]{structure}
\setbeamercolor*{alerted  text}{fg=dhbwRed, bg=white}
\setbeamercolor*{subsection in toc}{fg=dhbwGray, bg=white}

%\setbeamercolor*{palette primary}{fg=white,bg=lightBlue}
%\setbeamercolor*{palette secondary}{fg=white,bg=gray}
%\setbeamercolor*{palette tertiary}{fg=white,bg=gray}
%\setbeamercolor*{palette quaternary}{fg=white,bg=dhbwRed}

% no navigation symbols
\setbeamertemplate{navigation symbols}{}

% headline, footline
\setbeamertemplate{footline}{\color{dhbwGray} \hfill\insertframenumber\hspace{5mm}\vspace{2mm}}
\setbeamertemplate{headline}{}

% Title Page
\newcommand*{\makeTitlePage}{
	
	\begin{frame}[plain]
		
		\vfill
		\vfill
		\begin{center}
			{
				\usebeamerfont{title}
				\usebeamercolor[fg]{title}
				\Large
				\inserttitle
			}\\[3mm]
			{	
				\usebeamerfont{subtitle}
				\usebeamercolor[fg]{subtitle}
				\large
				\insertsubtitle
			}
		\end{center}
		%
		\vfill
		\vfill
		\vfill
		\vfill
		%
		\begin{columns}
			\begin{column}{0.5\textwidth}
				\begin{flushleft}
					{
						\usebeamerfont{normal text}
						\color{dhbwGray!80}
						\scriptsize
						Dr. Moritz Gruber\\
						DHBW Karlsruhe\\
						
					}
				\end{flushleft}
			\end{column}
			%
			\begin{column}{0.5\textwidth}
				\begin{flushright}
					\includegraphics[scale=0.06]{../DHBW.png}
				\end{flushright}
			\end{column}
		\end{columns}
		%
		\vspace{1mm}
		\begin{columns}
			\begin{column}{0.5\textwidth}
				\begin{flushleft}
					{
						\usebeamerfont{normal text}
						\color{dhbwGray!80}
						\scriptsize
						Version \today
					}
				\end{flushleft}
			\end{column}
			%
			\begin{column}{0.5\textwidth}
				% nothing (just a placeholder to be in line with the columns above
			\end{column}
		\end{columns}
	\end{frame}

}

% Section Divider Page
\newcommand*{\makeSectionDividerPage}{

	\begin{frame}[plain]
		\begin{center}
			\begin{flushleft}
				{				
					\usebeamercolor[fg]{frametitle}
					{\Large \insertsection} \\[3mm]
					{\large \insertsubsection}
				}
			\end{flushleft}
		\end{center}
        \end{frame}
	
}

% itemize
\setbeamertemplate{itemize items}[circle]
\setbeamertemplate{enumerate item}{(\theenumi)}




%--------------------------------------%
% Math ------------------------------%
%--------------------------------------%

% Mengen (Zahlen)
\newcommand{\N}{\mathbb{N}}
\newcommand{\Q}{\mathbb{Q}}
\newcommand{\R}{\mathbb{R}}
\newcommand{\Z}{\mathbb{Z}}
\newcommand{\C}{\mathbb{C}}

% Mengen (allgemein)
\newcommand{\K}{\mathbb{K}}
\newcommand\PX{{\cal P}(X)}

% Zahlentheorie
\newcommand{\ggT}{\mathrm{ggT}}


% Ableitungen
\newcommand{\ddx}{\frac{d}{dx}}
\newcommand{\pddx}{\frac{\partial}{\partial x}}
\newcommand{\pddy}{\frac{\partial}{\partial y}}
\newcommand{\grad}{\text{grad}}

%--------------------------------------%
% Layout Colors ------------------%
%--------------------------------------%
\newcommand*{\highlightDef}[1]{{\color{lightBlue}#1}}
\newcommand*{\highlight}[1]{{\color{lightBlue}#1}} % after theme for colours


%----------------------------------------------------------------------------------------------------
%--------- Document Title ---------------------------------------------------------------------
\title{Lineare Algebra\\[3mm] 
	\large Aussagenlogik \& Prädikatenlogik
}
\author{
	Dr. Moritz Gruber \\[1mm]
	\small moritz.gruber@edu.dhbw-karlsruhe.de
}
\institute{DHBW Karlsruhe}
\date{2022}


%----------------------------------------------------------------------------------------------------
%--------- Document ----------------------------------------------------------------------------
\begin{document}

\AtBeginSubsection[]{
	\begin{frame}				
		\usebeamercolor[fg]{frametitle}
		{\Large \insertsection} \\[5mm] 
		{\large \insertsubsection}
        \end{frame}
}

%
\begin{frame}[plain] 
 \titlepage
\end{frame}
%
%
\begin{frame}\frametitle{Inhalt}
   \tableofcontents
\end{frame}
%

%%%
\section{Organisatorisches}
%%%

\begin{frame}
	
	\begin{itemize}
		\item Materialien in Moodle.
		\item Hausaufgaben zu jeder Vorlesung in Eigenverantwortung. 
		\item Besprechnung der Lösungen am Beginn des nächsten Termins. 
		\item Nutzen Sie das Begleitete Selbststudium.
% Termine:
%			\begin{itemize}
%				\item 20.10.2020 17:00-18:30 
%				\item 27.10.2020 17:00-18:30 
%				\item 04.11.2020 17:00-18:30 
%				\item 11.11.2020 17:00-18:30 
%				\item 17.11.2020 17:00-18:30 
%				\item 19.11.2020 17:00-18:30 
%				\item 24.11.2020 17:00-18:30 
%				\item 03.12.2020 17:00-18:30 
%			\end{itemize}
%			Der AlfaView-Link hierzu wird noch bekanntgegeben.
	\end{itemize}
	
\end{frame}


%%%
\section{Themen der Vorlesung}
%%%
%
\begin{frame}\frametitle{Themen der Vorlesung}
	
	\begin{itemize}
		\item Aussagenlogik \& Prädikatenlogik
		\item Mengenlehre und Abbildungen
		\item Wichtige algebraische Strukturen: Gruppen, Ringe und Körper
		\item Vektorräume
		\item Lineare Gleichungssysteme
		\item Lineare Abbildungen und Matrizen
		\item Skalarprodukte und Geometrie
		\item Graphentheorie
	\end{itemize}
	
\end{frame}
%
%
\begin{frame}\frametitle{Lehrbuch}
	
	\highlightDef{[Teschl]}\\ 
	G. Teschl, S. Teschl: Mathematik für Informatiker, 
	Band 1, 
	Diskrete Mathematik und Lineare Algebra, 
	2013.
	
\end{frame}
%
%%%
\section{Aussagenlogik}
%%%
%
%%%
\subsection{Aussagen}
%%%
%
\begin{frame}{Aussagen}
	
	Die Aussagenlogik beschäftigt sich mit Aussagen und Zusammenhängen zwischen Aussagen.\\[5mm]
	
	\begin{definition}
		Eine \highlightDef{Aussage} ist ein Satz, der entweder wahr (w) oder falsch (f) ist.
	\end{definition}
	
\end{frame}
%
%
\begin{frame}\frametitle{Beispiele}

	Die folgenden Sätze sind Aussagen:
	\begin{itemize}
		\item Berlin ist die Hauptstadt von Deutschland.
		\item $6\cdot 7 = 42$.
		\item $7<5$.
		\item $2\leq 2$.
		\item Wenn 2 ungerade ist, dann ist 1 = 0.
	\end{itemize}
	
	\pause
	\vfill
	Die folgenden Sätze sind {\em keine} Aussagen:
	\begin{itemize}
		\item Wie wird das Wetter morgen?
		\item Hallo!
		\item $x + 1 = 7$.
		\item $x < y$.
	\end{itemize}
	
\end{frame}
%
%%%
\subsection{Negation und Verknüpfungen von Aussagen}
%%%
%
\begin{frame}\frametitle{Verknüpfungen und Negationen von Aussagen}
	
	Die beiden Aussagen
	\begin{itemize}
		\item Ich wohne in Karlsruhe.
		\item Ich studiere an der DHBW.
	\end{itemize}
	können durch eine Verknüpfung wie \highlightDef{und} zu der neuen Aussage
	\begin{itemize}
		\item Ich wohne in Karlsruhe \highlightDef{und} ich studiere an der DHBW.
	\end{itemize}
	verknüpft werden.
	
	\vfill
	Man kann eine Aussage auch durch ein \highlightDef{nicht} negieren.
	\begin{itemize}
		\item Ich wohne \highlightDef{nicht} in Karlsruhe.
	\end{itemize}
	
\end{frame}
%
%
\begin{frame}\frametitle{Negation}
		Im Folgenden wollen wir Aussagen mit kleinen Buchstaben $a, b, c, \ldots$ bezeichnen.\\[1mm]
	Beispiel: 
	$a = $ ``Ich wohne in Karlsruhe'',\\ 
	\vfill
	\pause
	Sei $a$ eine Aussage. Dann nennt man die Aussage
	$$
		\neg a
	$$ 
	die \highlightDef{Negation} oder \highlightDef{Verneinung} der Aussage $a$.\\[1mm]
	$\neg a$ ist genau dann wahr, wenn $a$ falsch ist.\\
	
	\pause
	\vfill
	Wahrheitstafel:
	$$
		\begin{array}{c | c}
			a	& \neg a	\\ \hline
			w	& f		\\
			f	& w
		\end{array}
	$$
	
\end{frame}
%
%
\begin{frame}\frametitle{Negation: Beispiele}
	
	Wie lautet jeweils die Negation der folgenden Aussagen?
	\begin{itemize}
		\item Das Glas ist voll.
		\item Alle Studierenden sind anwesend.
		\item $2 > 5$.
	\end{itemize}
	%
	\vfill
	\pause
	Die Negationen lauten:
	\begin{itemize}
		\item<2-> Das Glas ist nicht voll. 
		\item<3-> Nicht alle Studierenden sind anwesend.
		\item<4-> $2\leq 5$.
	\end{itemize}
	
\end{frame}
%
%
\begin{frame}\frametitle{Konjunktion (Und-Verknüpfung)}
	
	Seien $a, b$ zwei Aussagen. Die Aussage
	$$
		a \land b
	$$
	heißt \highlightDef{Konjunktion} oder \highlightDef{Und-Verknüpfung}. \\[1mm]
	
	Die Aussage $a \land b$ ist genau dann wahr, wenn $a$ und $b$ wahr sind.

	\vfill
	Wahrheitstafel:
	$$
		\begin{array}{c | c || c}
			a	& b	& a\land b	\\ \hline
			w	& w	& w		\\
			w	& f	& f		\\
			f	& w	& f		\\
			f	& f	& f	
		\end{array}
	$$
	
\end{frame}
%
%
\begin{frame}\frametitle{Beispiel}
	
	Seien \\
	$a = $ ``Karlsruhe liegt in Deutschland'',\\
	$b = $ ``Die Erde ist eine Scheibe''\\
	 zwei Aussagen.\\[2mm]
	 
	 Wie lautet der Wahrheitswert von 
	 $$
	 	a\land b?
	 $$
	 \pause
	 (falsch)
	
\end{frame}
%
% DIESER FRAME IST KAPUTT ?!?!
\begin{frame}\frametitle{Disjunktion (Oder-Verknüpfung)}

	Seien $a, b$ zwei Aussagen. Die Aussage
	$$
		a \lor b
	$$
	heißt \highlightDef{Disjunktion} oder \highlightDef{Oder-Verknüpfung}. \\[1mm]
	
	Die Aussage $a \lor b$ ist genau dann wahr, wenn mindestens eine der beiden Aussagen $a, b$ wahr ist 
	(nicht-ausschließendes oder).

	\vfill
	Wahrheitstafel:
	$$
		\begin{array}{c | c || c}
			a	& b	& a\lor b	\\ \hline
			w	& w	& w		\\
			w	& f	& w		\\
			f	& w	& w		\\
			f	& f	& f	
		\end{array}
	$$
	
\end{frame}
%
%
\begin{frame}\frametitle{Beispiel}
	
	Seien \\
	$a = $ ``Karlsruhe liegt in Deutschland'',\\
	$b = $ ``Die Erde ist eine Scheibe''\\
	 zwei Aussagen.\\[2mm]
	 
	 Wie lautet der Wahrheitswert von 
	 $$
	 	a\lor b?
	 $$
	 \pause
	 (wahr)
	
\end{frame}
%
%
\begin{frame}\frametitle{Klammern}
	
	Was ist mit einer Aussage
	$$
		a\land b \lor c
	$$
	gemeint?\\ 
	\pause
	Es gibt die beiden Möglichkeiten
	$$
		(a\land b) \lor c\qquad\text{bzw.}\qquad a \land (b\lor c).
	$$
	Deshalb ist es wichtig, eine logische Verknüpfung von mehr als zwei Aussagen durch Klammern zu strukturieren.
	
\end{frame}
%
%
\begin{frame}\frametitle{Logische Äquivalenz}
	
	Seien $a_1,\ldots, a_n$ Aussagen.\\[1mm]
	
	Sind die Aussagen $a$ bzw. $b$ \highlightDef{logische Verknüpfungen} der Aussagen $a_1, \ldots, a_n$,
	dann heißen  $a$ und $b$  \highlightDef{logisch äquivalent}, wenn sie für jede Kombination von Wahrheitswerten 
	für $a_1, \ldots, a_n$ die gleichen Wahrheitswerte annehmen.\\[2mm]
	
	Notation: 
	$$
		a \equiv b.
	$$
	
	
\end{frame}
%
%
\begin{frame}\frametitle{De Morgan'sche Regeln}

	Seien $a,b$ Aussagen.
	Dann gilt:
	$$
		\neg( a\land b) \equiv (\neg a) \lor (\neg b)
	$$
	und 
	$$
		\neg( a\lor b) \equiv (\neg a) \land (\neg b).
	$$
	
\end{frame}
%
%
\begin{frame}\frametitle{De Morgan'sche Regeln: Beweis}

	 \highlightDef{$\neg( a\land b) \equiv (\neg a) \lor (\neg b)$:}\\[3mm]
	
	Wahrheitstafel:
	$$
		\begin{array}{c | c || c || c}
			a	& b	& \neg(a\land b)	& (\neg a) \lor (\neg b)	\\ \hline
			w	& w	& f				& f	\\
			w	& f	& w				& w	\\
			f	& w	& w				& w	\\
			f	& f	& w				& w
		\end{array}
	$$
	
	\vspace{10mm}
	Der Nachweis der zweiten Regel $\neg( a\lor b) \equiv (\neg a) \land (\neg b)$ ist eine Ubungsaufgabe.
	
\end{frame}
%
%
\begin{frame}\frametitle{Distributivgesetze}
	
	Seien $a,b,c$ Aussagen.
	Dann gilt:
	\begin{eqnarray*}
		a\land (b \lor c)	&\equiv& (a\land b) \lor (a\land c)\\
		a\lor (b\land c) 	&\equiv& (a\lor b) \land (a\lor c)
	\end{eqnarray*}
	
\end{frame}
%
%
\begin{frame}\frametitle{Distributivgesetze: Beweis}
	
	\highlightDef{$a\land (b \lor c) \equiv (a\land b) \lor (a\land c)$:}
	
	$$
		\begin{array}{c | c | c || c || c}
			a	& b	&c				&a\land (b \lor c)	& (a\land b) \lor (a\land c)	\\ \hline
			w	& w	& w				&w				& w					\\
			w	& w	&f				&w				& w					\\
			w	& f	&w				&w				& w				 	\\
			f	&w	&w				&f 				&f					\\
			f	&f	&w				&f				&f					\\
			f	&w	&f				&f				&f					\\
			w	&f	&f				&f				&f					\\
			f	&f	&f				&f				&f	
		\end{array}
	$$
	\pause
	\vfill
	$a\lor (b\land c) \equiv (a\lor b) \land (a\lor c)$ zeigt man analog.
	
\end{frame}
%
%
\begin{frame}\frametitle{Exclusive OR: XOR}

	Seien $a, b$ zwei Aussagen. Die Aussage
	$$
		a \oplus b
	$$
	heißt \highlightDef{XOR} oder \highlightDef{Exklusive Oder Verknüpfung}. \\[1mm]
	
	Die Aussage $a \oplus b$ ist genau dann wahr, wenn genau eine der beiden Aussagen $a, b$ wahr ist 
	(ausschließendes oder).

	\vfill
	Wahrheitstafel:
	$$
		\begin{array}{c | c || c}
			a	& b	& a\oplus b	\\ \hline
			w	& w	& f		\\
			w	& f	& w		\\
			f	& w	& w		\\
			f	& f	& f	
		\end{array}
	$$
	
\end{frame}
%
%
\begin{frame}\frametitle{Beispiel}
	
	Seien \\
	$a = $ ``Karlsruhe liegt in Deutschland'',\\
	$b = $ ``Die Erde ist eine Kugel''\\
	 zwei Aussagen.\\[2mm]
	 
	 Wie lautet der Wahrheitswert von 
	 $$
	 	a\oplus b?
	 $$
	 \pause
	 (falsch, da die Wahrheitswerte von $a$ und $b$ wahr sind)
	
\end{frame}
%
%
\begin{frame}\frametitle{Subjunktion}

	Seien $a, b$ zwei Aussagen. Die Aussage
	$$
		a \rightarrow b
	$$
	heißt \highlightDef{Subjunktion} oder \highlightDef{Wenn-Dann-Verknüpfung}. \\

	\vfill
	Wahrheitstafel:
	$$
		\begin{array}{c | c || c}
			a	& b	& a\rightarrow b	\\ \hline
			w	& w	& w		\\
			w	& f	& f		\\
			f	& w	& w		\\
			f	& f	& w	
		\end{array}
	$$
	
\end{frame}
%
%
\begin{frame}\frametitle{Beispiel}
	
	Wenn $2+2=5$, dann ist die Erde eine Scheibe.\\[2mm]
	
	\pause
	Seien \\
	$a = $ ``2 + 2 = 5'',\\
	$b = $ ``Die Erde ist eine Scheibe''\\
	 zwei Aussagen.\\[2mm]
	 
	 Wie lautet der Wahrheitswert von 
	 $$
	 	a\rightarrow  b?
	 $$
	 \pause
	 (wahr, da der Wahrheitswert von $a$ falsch ist)
	 
\end{frame}
%
%
\begin{frame}\frametitle{Subjunktion}
	
	Eine sehr wichtige Tatsache ist
	$$
		a \rightarrow b \equiv (\neg b) \rightarrow (\neg a).
	$$
	(Dies ist die Grundlage für Beweise durch Kontraposition.)\\[3mm]
	\pause
	Beweis:\\
	$$
		\begin{array}{c | c || c || c}
			a	& b	& a\rightarrow b	&(\neg b) \rightarrow (\neg a)	\\ \hline
			w	& w	& w				& w						\\
			w	& f	& f				& f						\\
			f	& w	& w				& w						\\
			f	& f	& w				& w						
		\end{array}
	$$
	
\end{frame}
%
%
\begin{frame}\frametitle{Bijunktion}
	
	Seien $a, b$ zwei Aussagen. Die Aussage
	$$
		a \leftrightarrow b
	$$
	heißt \highlightDef{Bijunktion} oder \highlightDef{Genau-Dann-Verknüpfung}. \\

	\vfill
	Wahrheitstafel:
	$$
		\begin{array}{c | c || c}
			a	& b	& a\leftrightarrow b	\\ \hline
			w	& w	& w		\\
			w	& f	& f		\\
			f	& w	& f		\\
			f	& f	& w	
		\end{array}
	$$
	
\end{frame}
%
%
\begin{frame}\frametitle{Bijunktion}
	
	Seien $a, b$ zwei Aussagen. Dann gilt
	$$
		a \leftrightarrow b \equiv (a\rightarrow b) \land (b\rightarrow a).
	$$
	(Beweis: Übungsaufgabe)
	
\end{frame}
%
%
\begin{frame}\frametitle{Implikation / Logischer Schluss}
	
	Man schreibt in Schlussfolgerungen
	$$
		a \Rightarrow b,
	$$
	wenn die Aussage $a\rightarrow b$ wahr ist.\\[2mm]
	
	Sprechweise: Aus {\em $a$ folgt $b$} oder {\em $a$ impliziert $b$}.
	
\end{frame}
%
%
\begin{frame}\frametitle{Äquivalenz}
	
	Man schreibt in Schlussfolgerungen
	$$
		a \Leftrightarrow b,
	$$
	wenn die Aussage $a\leftrightarrow b$ wahr ist.\\[2mm]
	
	Sprechweise: {\em $a$ genau dann, wenn $b$}.
	
\end{frame}
%
%%%
\section{Prädikatenlogik}
%%%
%%%
\subsection{Aussageform \& Prädikat}
%%%
%
\begin{frame}{Motivation}
	
	Es gibt Sätze, deren Wahrheitswert von einer oder mehreren Variablen abhängt:
	$$
		x < 5.
	$$
	Solche Sätze sind keine Aussagen, können aber in der Prädikatenlogik behandelt werden.
	
\end{frame}
%
%
\begin{frame}{Aussageform \& Prädikat}
	
	Eine \highlightDef{Aussageform $a(x)$} entsteht, wenn man in einer Aussage eine Konstante durch eine Variable ersetzt.\\[3mm]
	
	\pause
	Beispiel: Aussageform
	$$
		a(x): x < 5.
	$$
	Zwei Bestandteile:
	\begin{itemize}
		\item Variable $x$ und
		\item \highlightDef{Prädikat} ``Ist kleiner $5$'': $a(x): \text{ist\_kleiner\_5}(x)$.\\[1mm]
		
			In einer (objektorientierten) Programmiersprache entspricht ein Prädikat $\text{ist\_kleiner\_5}(\cdot)$
			einer Methode mit boolschem Rückgabewert (wahr/falsch). 
	\end{itemize}
	
	\vfill
	\pause
	Eine Aussageform $a(x)$ wird zu einer Aussage, wenn man für $x$ ein konkretes Objekt einsetzt:
	$a(1)$ ist wahr und $a(7)$ ist falsch.
	
\end{frame}
%
\begin{frame}\frametitle{Verknüpfungen von Aussageformen}
	
	Aussageformen $a(x), b(x)$ können wie Aussagen negiert werden ($\neg$) oder durch 
	$\land, \lor, \oplus, \rightarrow, \leftrightarrow$ verknüpft werden.\\
	Hierdurch entstehen neue Aussageformen:
	$$
		\neg a(x), \quad a(x) \land b(x), \quad a(x) \lor b(x), 
	$$
	$$
		a(x) \oplus b(x), \quad a(x) \rightarrow b(x),\quad a(x) \leftrightarrow b(x).
	$$
\end{frame}
%
%%%
\subsection{Quantoren}
%%%
%
\begin{frame}\frametitle{Quantoren}
	
	Quantoren erzeugen Aussagen aus Aussageformen.
	
\end{frame}
%
%
\begin{frame}\frametitle{Allquantor $\forall$}
	
	Sei $a(x)$ eine Aussageform. Dann ist die Aussage 
	$$
		\forall x: a(x)
	$$
	genau dann wahr, wenn $a(x)$ \highlightDef{für alle} $x$ wahr ist.
	
\end{frame}
%
%
\begin{frame}\frametitle{Allquantor $\forall$}
	
	Sei $M$ eine Menge.\\[2mm]
	Die Aussage 
	$$
		\forall x\in M: a(x)
	$$
	ist eine andere (lesbarere) Schreibweise für die Aussage
	$$
		\forall x: \big(x\in M \rightarrow a(x) \big) .
	$$
	(``$x\in M$'' ist ebenfalls eine Aussageform.)
\end{frame}
%
%
\begin{frame}\frametitle{Beispiele}
	
	Welche der folgenden Aussagen ist wahr?
	\begin{itemize}
		\item $\forall x\in \N: x^2 \geq x$. 
		\item $\forall x\in \R: x^2 \geq x$.
	\end{itemize}
	
\end{frame}
%
%
\begin{frame}\frametitle{Existenzquantor $\exists$}

	Sei $a(x)$ eine Aussageform. Dann ist die Aussage 
	$$
		\exists x: a(x)
	$$
	genau dann wahr, wenn ein $x$ \highlightDef{existiert}, sodass $a(x)$ für dieses $x$ wahr ist.

\end{frame}
%
%
\begin{frame}\frametitle{Existenzquantor $\exists$}

	Die Schreibweise
	$$
		\exists x \in M: a(x)
	$$
	wird für die Aussage
	$$
		\exists x: \big( (x\in M) \land a(x) \big)
	$$ 
	verwendet.
	
\end{frame}
%
%
\begin{frame}\frametitle{Beispiele}
	
	Welche der folgenden Aussagen ist wahr?
	\begin{itemize}
		\item $\exists x\in \R: x^2 \geq x$.
		\item $\exists x\in\N: x^2 = 2$.
		\item $\exists x\in\Q: x^2 = 2$.
	\end{itemize}
	
\end{frame}
%
%
\begin{frame}\frametitle{Negation von Aussagen mit Quantoren}
	
	Sei $a(x)$ eine Aussageform. Dann gilt:
	\begin{eqnarray*}
		\neg( \forall x: a(x) ) 	&\equiv	& \exists x: \neg a(x)\\
		\neg( \exists x: a(x) )	& \equiv	& \forall x: \neg a(x)
	\end{eqnarray*}
	
\end{frame}
%
%
\begin{frame}\frametitle{Beispiel}
	
	\begin{itemize}
		\item Jeder spricht Englisch.
		\pause
		\item $\forall x: x$ spricht Englisch.
		\pause
		\item Negation: 
			$$
				\neg( \forall x: x \text{\;spricht Englisch}) \equiv \exists x : x \text{\; spricht nicht Englisch}
			$$
		\item ``Nicht jeder spricht Englisch.'' $\equiv$ ``Es gibt jemanden, der nicht Englisch spricht.'' 
	\end{itemize}
\end{frame}
%
%
\begin{frame}\frametitle{Reihenfolge von Quantoren}
	
	Es ist sehr wichtig, auf die Reihenfolge der Quantoren zu achten.
	\begin{itemize}
		\item $\forall x\in\N\;\; \exists y\in \N: x + y \text{\,ungerade}$
		\item $\exists y\in \N\;\; \forall x\in \N: x +y \text{\,ungerade}$ 
	\end{itemize}
	Welche der Aussagen ist wahr?
	
\end{frame}
%
%%%




\end{document}