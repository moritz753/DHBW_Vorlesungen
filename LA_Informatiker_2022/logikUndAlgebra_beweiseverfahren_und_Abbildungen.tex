\documentclass[
				hyperref={pdftex,
						breaklinks,
						pdfauthor={Andreas Weber},
						pdftitle={Logik und Algebra},
						colorlinks=true,
						urlcolor=blue,
						linkcolor=		%Also color of header links		
				},
				xcolor=dvipsnames
			]
			{beamer}


\usepackage{beamerthemesplit}

\usepackage{amsfonts}
\usepackage{amsmath}
\usepackage{amssymb}
\usepackage{amsthm}
\usepackage{amscd}
\usepackage{stmaryrd} %\lightning

\usepackage[ngerman]{babel}

%----------------------------------------------------------------------------------------------------
%--------- Theme --------------------------------------------------------------------------------
\usetheme{Malmoe}

%---------- Colors 
%\setbeamercolor*{palette primary}{fg=white,bg=black!95}
%\setbeamercolor*{palette secondary}{fg=white,bg=black}
%\setbeamercolor*{palette tertiary}{use=structure,fg=white,bg=black}
%\setbeamercolor*{palette quaternary}{fg=white,bg=black!95}
%
%\usecolortheme[named=white]{structure}%RGB={072,118,255 }]{structure}%named=lightgray]{structure}%RGB={0,90,156}]{structure}
%%\DefineNamedColor{named}{lightgrey}{RGB}{245,245,250}
%\setbeamercolor{background canvas}{bg=gray}%lightgray!30}
%\setbeamercolor{normal text}{fg=white}
%\setbeamercolor{itemize item}{fg=blue!60}
%---------- /Colors 


\setbeamertemplate{navigation symbols}{}

% page number at bottom right
\setbeamertemplate{footline}{\color{gray} \hfill\insertframenumber\hspace{5mm}\vspace{2mm}}

\setbeamertemplate{enumerate item}{(\theenumi)}
\setbeamertemplate{itemize items}[circle]
% No indention in itemize
\settowidth{\leftmargini}{\usebeamertemplate{itemize item}} \addtolength{\leftmargini}{\labelsep}

% Captions (Figures, Tables,...)
\setbeamertemplate{caption}{\insertcaption}

% blocks
%\setbeamercolor{block body}{bg=structure!10}
%\setbeamercolor{block title}{bg=structure!100,fg=structure!10}
\setbeamercolor{block body alerted}{bg=alerted text.fg!10}
\setbeamercolor{block title alerted}{bg=alerted text.fg!20}
\setbeamercolor{block body example}{bg=green!15}
\setbeamercolor{block title example}{bg=green!20}
%\setbeamertemplate{blocks}[rounded][shadow]

%----------------------------------------------------------------------------------------------------
%--------- /Theme --------------------------------------------------------------------------------

%--------- Color Definitions ------------------------------------------------------------
\newcommand{\itemColor}{\usebeamercolor[fg]{itemize item}}
%--------- /Color Definitions ------------------------------------------------------------


%--------- Own definitions
 %%%%%%%%%%%%%%%Schriften%%%%%%%%%%%%%
\DeclareMathAlphabet{\lier}{U}{eur}{m}{n}  %% Gothisch/Fraktur - Roman



\newcommand{\M}{\mathbb{M}}
\newcommand{\E}{\mathbb{E}}
\newcommand{\Hy}{\mathbb{H}}
\newcommand{\N}{\mathbb{N}}
\newcommand{\Q}{\mathbb{Q}}
\newcommand{\R}{\mathbb{R}}
\newcommand{\Z}{\mathbb{Z}}
\newcommand{\C}{\mathbb{C}}
\newcommand{\K}{\mathbb{K}}
\newcommand*\ee{\mathrm{e}}
\newcommand*\e{\mathrm{e}}

\newcommand*\ii{\mathrm{i}}
\newcommand*\re{\mathrm{Re}}
\newcommand*\im{\mathrm{Im}}
\newcommand*\id{\mathrm{id}}
\newcommand*\rang{\mathrm{rang}}
\newcommand*\grad{\mathrm{grad}}
\newcommand*\dive{\mathrm{div}}
\newcommand*\sym{\mathrm{Sym}}
\newcommand*\spur{\mathrm{Spur}}
\newcommand*\isom{\mathrm{Isom}}
\newcommand*\vol{\mathrm{vol\,}}
\newcommand*\supp{\mathrm{supp}}
\newcommand*\inj{\mathrm{inj}}
\newcommand*\rank{\mathrm{rank}}
\newcommand*\qrank{\Q\mbox{-}\mathrm{rank}}
\newcommand*\rrank{\R\mbox{-}\mathrm{rank}}
\newcommand*\dom{\mathrm{dom}}
\newcommand*\tr{\mathrm{tr}}
\newcommand*\spa{\mathrm{span}}
\newcommand*\diam{\mathrm{diam}}

\newcommand*\ric{\mathrm{Ric}}

\newcommand*\con{\mathrm{con}}
\newcommand*\dis{\mathrm{dis}}

\newcommand\PX{{\cal P}(X)}
\newcommand\T{{\cal T}}
\newcommand\B{{\cal B}}



\newcommand\scp{\langle\cdot,\cdot\rangle}     %% Metric

%%%%%%%%%%%%Tilde%%%%%%%
\newcommand\tx{\tilde{x}}
\newcommand\ty{\tilde{y}}
\newcommand\tu{\tilde{u}}
\newcommand\tk{\tilde{k}}
\newcommand\td{\tilde{d}}
\newcommand\tD{\tilde{D}}
\newcommand\tX{\tilde{X}}
\newcommand\tY{\tilde{Y}}
\newcommand\tZ{\tilde{Z}}


%%%%%%Lie-Gruppen%%%%%%%%%
\newcommand\ad{\mathrm{ad}}
\newcommand\Ad{\mathrm{Ad}}
\newcommand{\kak}{K\exp\overline{\lier{a}^+}K}              %%%%Cartan-Zerlegung
\newcommand*\Rang{\mathrm{Rang}}
\newcommand*\glnr{\mathrm{\it GL}(n,\R)}
\newcommand*\glnc{\mathrm{\it GL}(n,\C)}
\newcommand*\slnr{\mathrm{\it SL}(n,\R)}
\newcommand*\on{\mathrm{\it O}(n)}
\newcommand*\son{\mathrm{\it SO}(n)}
\newcommand*\SLzr{\mathrm{\it SL}(2,\R)}
\newcommand*\SOzr{\mathrm{\it SO}(2,\R)}

%%%%%%%%%%%%%Algebraische Gruppen
\newcommand\bG{{\bf G}}
\newcommand\bT{{\bf T}}
\newcommand\bP{{\bf P}}
\newcommand\bN{{\bf N}}
\newcommand\bL{{\bf L}}
\newcommand\bS{{\bf S}}
\newcommand\bM{{\bf M}}

\newcommand\Mor{\mathrm{Mor}}



%%%%%%Geometry%%%%%%%%%%%
\newcommand{\Si}{\mathcal{S}}


%%%%%%Ableitungsoperatoren%%%%%%%%%%%
\newcommand*\pddt{\frac{\partial}{\partial t}}
\newcommand*\pddx{\frac{\partial}{\partial x}}
\newcommand*\pddxio{\frac{\partial}{\partial x^i}}
\newcommand*\pddxjo{\frac{\partial}{\partial x^j}}
\newcommand*\pddxko{\frac{\partial}{\partial x^k}}
\newcommand*\pddxlo{\frac{\partial}{\partial x^l}}

\newcommand*\pddy{\frac{\partial}{\partial y}}
\newcommand*\pddyio{\frac{\partial}{\partial y^i}}
\newcommand*\pddyjo{\frac{\partial}{\partial y^j}}
\newcommand*\pddyko{\frac{\partial}{\partial y^k}}
\newcommand*\pddylo{\frac{\partial}{\partial y^l}}

\newcommand*\pddyq{\frac{\partial^2}{\partial y^2}}
\newcommand*\pddyj{\frac{\partial}{\partial y_j}}
\newcommand*\pddyjq{\frac{\partial^2}{\partial y_j^2}}
\newcommand*\pddxiq{\frac{\partial^2}{\partial x_i^2}}
\newcommand*\pddxi{\frac{\partial}{\partial x_i}}
\newcommand*\ddt{\frac{d}{dt}}

\newcommand*\dx{\,dvol(x)}
\newcommand*\dy{\,dvol(y)}
\newcommand*\dty{\,dvol(\ty)}

\newcommand*\DMp{\Delta_{M,p}}                 %%%%Laplace-Operatoren
\newcommand*\DMq{\Delta_{M,q}}
\newcommand*\DM{\Delta_M}
\newcommand*\DX{\Delta_X}
\newcommand*\DXp{\Delta_{X,p}}
\newcommand*\DXq{\Delta_{X,q}}
\newcommand*\DAx{\Delta_{Ax_0}}
\newcommand*\Rad{\mathrm{Rad}}
\newcommand*\DXps{\Delta^{\#}_{X,p}}

\newcommand*\eDXps{\e^{-t(\Delta^{\#}_{X,p}-c)}} %%%%%% Semigroups
\newcommand*\LpsX{L^p_{\#}(X)}


%%%%%%%%%%%%%%%Komplexe Analysis
\newcommand*\Res{\mathrm{Res}}

%%%%%%%%%%%%%%%Definitionsmenge
\newcommand*\D{{\cal D}}






%--------- /Own definitions

\newcommand{\comp}{\mathsf{c}}	%Complement

%----------------------------------------------------------------------------------------------------
%--------- Document Title ---------------------------------------------------------------------

%%%%%%%%%%%%%%
\title{Logik \& Algebra\\[3mm] 
	\large Beweisverfahren \& Abbildungen
}
\author{Dr. Moritz Gruber} 
\institute{DHBW Karlsruhe}
\date{2021}
%%%%%%%%%%%%%%
\begin{document}

\AtBeginSection[]{
	\begin{frame}				
		\usebeamercolor[fg]{frametitle}
		{\Large \insertsection} 
        \end{frame}
}

\AtBeginSubsection[]{
	\begin{frame}				
		\usebeamercolor[fg]{frametitle}
		{\Large \insertsection \\[2mm]
		 \large \insertsubsection } 
        \end{frame}
}

%
\begin{frame}[plain] 
 \titlepage
\end{frame}
%
%
\begin{frame}\frametitle{Inhalt}
   \tableofcontents
\end{frame}
%
%%%
\section{Beweise}
%%%
%
\begin{frame}

	Sie fragen sich eventuell, wieso Beweise f\"ur Nicht-Mathematiker relevant sein k\"onnten.\\
	\pause
	
	\begin{itemize}
		\item Fragestellungen in der Informatik oder den Wirtschafts\-wissenschaften lassen sich oft in der Sprache der Mathematik ausdr\"ucken.
		 	Deshalb sollte man nicht nur die mathematische Notation kennen, sondern auch mit logischen Schlussfolgerungen vertraut sein.
		\item Algorithmus korrekt? (Ergebnis in allen F\"allen  korrekt?)
		\item Bricht ein gegebener  Algorithmus immer in endlicher Zeit ab?
		\item Grunds\"atzlich ist es eine gute Idee, Behauptungen nicht einfach zu glauben, sondern diese nachzuvollziehen. 
		\item Der Beweis hilft oft die Aussage (besser) zu verstehen.
	\end{itemize}

\end{frame}
%
%
\begin{frame}\frametitle{Beweise}
	
	\begin{itemize}
		\item Mathematische S\"atze bzw. Schlussfolgerungen sind Wenn-Dann-Aussagen.
		\item Aus einer gegebenen Aussage $v$ (die Voraussetzung) wird mittels logischer Schlussfolgerungen eine andere Aussage 
			$b$ (die Behauptung) abgeleitet:
			$$
				v \Rightarrow b.
			$$
		\item Der Schluss $v \Rightarrow b$ ist zul\"assig, wenn die Aussage $v\rightarrow b$ wahr ist.
		\item Ein Beweis hat also das Ziel, mit Mitteln der Logik die Wahrheit der Aussage $v\rightarrow b$ zu zeigen.
		\item Hierf\"ur gibt es verschiedene Methoden.
	\end{itemize}
	
\end{frame}
%
%%%
\section{Direkter Beweis}
%%%
%
\begin{frame}\frametitle{Direkter Beweis}
	
	Aus einer Voraussetzung wird die Behauptung direkt bewiesen. \\[1mm]
	Wir haben diese Methode schon beim Beweis der de Morgan'schen Regeln in der Elementaren Mengenlehre verwendet.
	
	
	$$
		v \Rightarrow a_1 \Rightarrow a_2 \Rightarrow\ldots\Rightarrow b
	$$
	
\end{frame}
%
%%%
\section{Indirekter Beweis}
%%%
%
%%%
\subsection{Beweis durch Kontraposition}
%%%
\begin{frame}\frametitle{Beweis durch Kontraposition}
	
	\begin{itemize}
		\item Seien $v$ und $b$ Aussagen (Voraussetzung \& Behauptung). 
		\item Wir erinnern uns an die Tatsache
			$$
				v\rightarrow b \equiv (\neg b)\rightarrow (\neg v).
			$$
		\item<2-> Die Implikation $v \Rightarrow b$ ist somit genau dann zul\"assig, wenn die Implikation 
			$ (\neg  b) \Rightarrow (\neg v)$ zul\"assig ist. 
		\item<2-> In der Logik nennt man die Aussage $(\neg b)\rightarrow (\neg v)$ die {\itemColor Kontraposition} 
		der Aussage $v\rightarrow b$. 
	\end{itemize}
	
\end{frame}
%
%
\begin{frame}\frametitle{Beispiel}

	\begin{Satz}
		Sind $m,n \in \N$, $n$ gerade mit $m = \sqrt{n}$, dann gilt: $m$ ist gerade.
	\end{Satz}
	\pause
	\vfill
	\begin{itemize}
		\item {\itemColor Voraussetzung $v$}:  $(m,n \in \N)$ {\em und} ($n$ gerade) {\em und} ($m = \sqrt{n}$).
		\item {\itemColor Behauptung $b$:} $m$ ist gerade.
	\end{itemize}
	
\end{frame}
%
%
\begin{frame}\frametitle{Beispiel}

	\begin{itemize}
		\item {\itemColor Voraussetzung $v$}:  $(m,n \in \N)$ {\em und} ($n$ gerade) {\em und} ($m = \sqrt{n}$).
		\item {\itemColor Behauptung $b$:} $m$ ist gerade.
	\end{itemize}
	
	\vspace{3mm}
	Annahme: $m$ ist ungerade ($\neg b$ ist wahr).\\[1mm]
	
	Dann folgt:
	\begin{itemize}
		\item $m = 2k + 1$.
		\pause
		\item $n = m^2 = (2k+1)^2 = 4k^2 + 4k + 1= 2(2k^2+2k)+1$.\\
			Damit ist $n$ eine ungerade Zahl, d.h. $\neg v$ ist wahr.\\
			Man sagt auch: Dies ist ein Widerspruch zur Voraussetzung. (Symbolisch: $\lightning$).
	\end{itemize}
	\pause
	\vfill
	Wir haben also gezeigt: $(\neg b) \Rightarrow (\neg v)$. Somit gilt auch: $v \Rightarrow b$. \qed
	
\end{frame}
%
%%%
\subsection{Beweis durch Widerspruch (Reductio ad absurdum)}
%%%
%
\begin{frame}\frametitle{Beweis durch Widerspruch (Reductio ad absurdum)}
	
	\begin{itemize}
		\item Man zeigt, dass sich aus der {\itemColor Annahme}, die Aussage
			$$
				v \land (\neg b)
			$$
			sei wahr, ein {\itemColor Widerspruch} ergibt.
		\pause
		\item In dem Fall muss dann die Aussage $v \land (\neg b)$ falsch sein und die Negation dieser Aussage, d.h.
			$$
				\neg \big( v \land (\neg b) \big) \equiv (\neg v) \lor b \equiv v \rightarrow b
			$$
			ist wahr.\footnote<2->{vgl. Aussagenlogik: de Morgan'sche Regel bzw. Aufgabe zur Subjunktion}
		\pause
		\item Genauer: Man zeigt, dass die Aussage
			$$
				\big( v \land (\neg b) \big) \rightarrow \big(c \land (\neg c) \big)
			$$
			f\"ur irgendeine Aussage $c$ wahr ist. Da  $c \land (\neg c)$ immer falsch ist, muss dann auch $v \land (\neg b)$ falsch sein.
	\end{itemize}
	\vspace{2mm}
	
\end{frame}
%
\begin{frame}\frametitle{Beispiel Primzahlen}
	
	\begin{Satz}
	Es gibt unendlich viele
	Primzahlen.\footnote{Eine Zahl $p\in \mathbb{N} \backslash\{1\}$ hei{\ss}t Primzahl, wenn sie nur durch $1$ und $p$ teilbar ist.}
	\end{Satz}
	
	\vfill
	\pause
	Was sind Voraussetzung $v$ und Behauptung $b$?
	\pause
	\begin{itemize}
		\item {\itemColor Behauptung $b$:} Es gibt unendlich viele Primzahlen.
		\pause
		\item {\itemColor Voraussetzung $v$:} Die Peano\footnote<4->{Giuseppe Peano, 1858 - 1932}-Axiome, die die nat\"urlichen Zahlen charakterisieren.\\
		Solche Voraussetzungen werden in mathematischen S\"atzen nicht aufgef\"uhrt sondern implizit angenommen, 
		da die ganze Mathematik auf Axiomensystemen basiert. 
	\end{itemize}
	
\end{frame}
%
%
\begin{frame}\frametitle{Beispiel Primzahlen: Widerspruchsbeweis nach Euklid\footnote{Euklid von Alexandria, ca. 300 v. Chr.} }
	
	{\itemColor Annahme ($\neg b$):}\\ 
	Es gibt nur endlich viele Primzahlen $p_1, p_2, \ldots, p_n$.\\[3mm]
	
	\pause
	Betrachte
	$$	
		p = p_1\cdot p_2 \cdot \ldots \cdot p_n + 1.
	$$
	\begin{itemize}
		\item Da $p$ gr\"o{\ss}er als jede der Zahlen $p_1, \ldots, p_n$ ist, gilt insbesondere
			$$
				p \neq p_1, \ldots, p\neq p_n.
			$$
			$p$ ist also (aufgrund unserer Annahme) keine Primzahl.
	\end{itemize}
	
\end{frame}
%
%
\begin{frame}\frametitle{Beispiel Primzahlen: Widerspruchsbeweis nach Euklid}
	
	Betrachte
	$$	
		p = p_1\cdot p_2 \cdot \ldots \cdot p_n + 1.
	$$
	\begin{itemize}
		\item Wenn $p$ keine Primzahl ist, muss sie durch eine Primzahl teilbar sein. (Primfaktorzerlegung)
		\pause
		\item Für alle $j$ l\"asst $p$ bei Teilung durch die Primzahlen $p_j$ den Rest $1$, 
		d.h. $p$ ist durch keine der Zahlen $p_1,\ldots, p_n$ teilbar.
		\pause
		\item Wir haben somit einen Widerspruch:\\ 
		\begin{center}
			``$p$ ist durch eine der Zahlen $p_1,\ldots p_n$ teilbar'' \\
			und \\
			``$p$ ist durch keine der Zahlen $p_1,\ldots, p_n$ teilbar''.
		\end{center}
		\pause
		\item Die Aussage $v \rightarrow b$ ist also wahr. \qed
	\end{itemize}
	
\end{frame}
%
%
%%%
\section{Vollst\"andige Induktion}
%%%
%
\begin{frame}\frametitle{Vollst\"andige Induktion}

	Die Vollst\"andige Induktion ist ein Beweisverfahren, das verwendet werden kann, um nachzuweisen, dass Aussagen der Form
	$$
		\forall n\in\N : a(n)	
	$$	
	wahr sind.
	
	\pause
	\vspace{2mm}
	{\itemColor Beispiel:} \\
	Zeigen Sie, dass gilt
	$$
		\forall n\in \N:\quad  \underbrace{\sum_{k=0}^n k = \frac{1}{2}n(n+1)}_{a(n)}.
	$$	
	
	\vfill
	{\scriptsize Bemerkung: $\displaystyle\sum_{k=0}^n a_k = a_0 + a_1 + a_2 + \ldots + a_n$} 
\end{frame}
%
%
\begin{frame}\frametitle{Beispiel}
	Die Aussage
	$$
		\forall n\in \N:\quad  \underbrace{\sum_{k=0}^n k = \frac{1}{2}n(n+1)}_{a(n)}
	$$	
	kann f\"ur einzelne nat\"urliche Zahlen $n$ \"uberpr\"uft werden: 
	$$
		\begin{array}{lcll}
			n=0 	&:& 0=0\quad \checkmark		&(a(0)\text{~ist wahr})		\\
			n=1	&:& 0+1=1\quad \checkmark	&(a(1)\text{~ist wahr})		\\
			n=2	&:& 0+1+2=3\quad \checkmark	&(a(2)\text{~ist wahr})	
		\end{array}
	$$
	\pause
	Damit kann man nat\"urlich {\em nicht} zeigen, 
	dass die Aussageform $a(n)$ {\em f\"ur alle} $n\in\N$ wahr ist - es gibt schlie{\ss}lich unendlich viele nat\"urliche Zahlen.
\end{frame}
%
%
%\begin{frame}\frametitle{Prinzip der vollst\"andigen Induktion}
%
%	\begin{itemize}
%		\item[(1)] Sei $S\subset \N$ mit $S\neq \emptyset$.
%				\begin{itemize}
%					\item $S$ enth\"alt mindestens ein Element $n_0$, da  $S\neq \emptyset$.
%					\item Da es nur endlich viele nat\"urlichen Zahlen gibt, die kleiner als $n_0$ sind, gibt es ein kleinstes Element $s_0\in S$.
%				\end{itemize}
%		\pause
%		\item[(2)] Wir nehmen zus\"atzlich an, dass f\"ur jede Zahl $n\in S$ auch gilt: $n+1 \in S$.\\ 
%				Dann folgt:
%				$$
%					\{s_0, s_0+1, s_0 +2, \ldots \} \subset S.
%				$$
%				Da es keine weiteren nat\"urlichen Zahlen gibt, die gr\"o{\ss}er als $s_0$ sind, gilt sogar:
%				$$
%					\{s_0, s_0+1, s_0 +2, \ldots \} = S.
%				$$
%	\end{itemize}
%
%\end{frame}
%
%
\begin{frame}\frametitle{Prinzip der vollst\"andigen Induktion}
	
	Um nachzuweisen, dass eine Aussageform $a(n)$ f\"ur alle $n\geq N$ wahr ist, setzt man
	$$
		S := \{ n\in \N~|~ n\geq N \text{~und~} a(n)\text{~wahr} \}.
	$$
	\vfill
	Es gen\"ugt nun zu zeigen, dass
	\begin{itemize}
		\item $N\in S$ und
		\item $\forall n\in S: n+1 \in S$.
	\end{itemize}
	Denn dann folgt
	$$
		S = \{N, N+1, N+2,\ldots\}.
	$$
	
\end{frame}
%
%
\begin{frame}\frametitle{Prinzip der vollst\"andigen Induktion}
	
	Das Prinzip der vollst\"andigen Induktion wird oft auch folgenderma{\ss}en formuliert:\\
	\vfill
	Sei f\"ur alle $n\geq N$ die Aussageform $a(n)$ gegeben.\\[1mm]
	
	Falls\\[1mm]
	
	{\itemColor (1) Induktionsanfang:} $a(N)$ ist wahr.\\[1mm]  
	und\\[1mm]
	{\itemColor (2) Induktionsvoraussetzung:} $a(n)$ ist wahr.\\[1mm]  
	und\\[1mm]
	{\itemColor (3) Induktionsschluss:} Dann ist auch $a(n+1)$ wahr.\\[1mm]
	
	gilt, ist die Aussage 
	$$
		\forall n\in \N: n\geq N \rightarrow a(n).
	$$	
	wahr.
	
\end{frame}
%
%
\begin{frame}\frametitle{Beispiel}
	
	Zeigen Sie, dass gilt
	$$
		\forall n\in \N:\quad  \underbrace{\sum_{k=0}^n k = \frac{1}{2}n(n+1)}_{a(n)}.
	$$	
	
\end{frame}
%
%
\begin{frame}
	
	Zeigen Sie, dass gilt
	$$
		\forall n\in \N:\quad  \underbrace{\sum_{k=0}^n k = \frac{1}{2}n(n+1)}_{a(n)}.
	$$	
	
	{\itemColor (1) Induktionsanfang:} $a(0)$ ist wahr, da $0 = 0$.
	
\end{frame}
%
%
\begin{frame}

	Zeigen Sie, dass gilt
	$$
		\forall n\in \N:\quad  \sum_{k=0}^n k = \frac{1}{2}n(n+1).
	$$	
	
	{\itemColor (2) Induktionsvoraussetzung:} Es sei $a(n)$ wahr. \\
	{\itemColor (3) Induktionsschluss:}
	\begin{align*}
		\sum_{k=0}^{n+1} k	&= \sum_{k=0}^{n}k + (n+1)\\
						&=  \frac{1}{2}n(n+1) + (n+1)\\
						&=  \frac{1}{2}(n+1)(n+2),
	\end{align*}
	wobei im zweiten Schritt f\"ur  $\sum_{k=0}^{n} =  \frac{1}{2}n(n+1)$ die {\itemColor Induktionsvoraussetzung} 
	``$a(n)$ ist wahr'' verwendet wurde.

	
\end{frame}
%
%
\begin{frame}

Es gilt also, unter der Annahme, dass $a(n)$ wahr ist, dass 
	$$
		\sum_{k=0}^{n+1} k =  \frac{1}{2}(n+1)(n+2).
	$$
	Somit ist in diesem Fall auch $a(n+1)$ wahr.\\[5mm]

	
	Damit ist die Aussage durch vollst\"andige Induktion bewiesen. \qed
	
\end{frame}
%
%
%%%
\section{Abbildungen}
%%%
%
%%%
\subsection{Definition}
%%%
%
\begin{frame}\frametitle{Versuch einer Definition}

	Eine {\itemColor Abbildung} 
	$$
		f:M\to N
	$$ 
	zwischen zwei Mengen $M$ und $N$ ist eine \textit{Vorschrift}, die jedem $m\in M$ genau ein $n \in N$ \textit{zuordnet}.
	
	\vfill
	\pause
	{\itemColor Problem}: Die Begriffe \textit{Vorschrift} und \textit{zuordnet} sind nicht definiert und somit ist deren Bedeutung unklar.\\
	Deshalb: Definition \"uber die Mengenlehre.
	
\end{frame}
%
%
\begin{frame}
	
	\begin{block}{Definition}
		Eine {\itemColor Abbildung} 
		$$
			f:M\to N
		$$ 
		zwischen zwei Mengen $M$ und $N$ ist eine Teilmenge 
		$$
			f \subset M\times N
		$$ 
		mit der Eigenschaft dass f\"ur alle $m\in M$ genau ein $n\in N$ existiert, so dass $(m,n)\in f$.\\
		F\"ur dieses $n$ schreibt man $n = f(m)$.\\[2mm]
		
		\pause
		$M$ hei{\ss}t {\itemColor Definitionsmenge}, $N$ {\itemColor Wertebereich} von $f$ und 
		$f(M) := \{ f(m)~|~  m\in M \}$ {\itemColor Bildmenge}.\\[2mm]
		
		\pause
		Schreibweise f\"ur Abbildungen:
		\vspace{-2mm}
		$$ 
			f: M\to N, \quad m\mapsto f(m).
		$$
	\end{block}
	
\end{frame}
%
%
\begin{frame}\frametitle{Beispiele}

	\begin{itemize}
		\item $M=N=\{0,1\}, f = \{(0,1),(1,1)\}$:
			$$
				f(0) = 1,\quad f(1)=1.
			$$
		\item<2-> $f: \R\to \R, x\mapsto x^2$:
			$$
				f = \{(x,x^2)~|~ x\in \R\}.
			$$
		\item<3-> $f: \N\to \R, x\mapsto x^2$.
			$$
				f = \{(x,x^2)~|~ x\in \N\}.
			$$
		\item<4-> Die Abbildung 
			$$
				\text{Id}_M: M\to M, \quad m\mapsto m
			$$
			hei{\ss}t {\itemColor Identit\"at} auf $M$.
	\end{itemize}

\end{frame}
%
%
%%%
\subsection{Eigenschaften}
%%%
%
\begin{frame}
	
	\begin{block}{Definition}
		Eine Abbildung $f: M\to N$ hei{\ss}t 
		\begin{itemize}
			\item {\itemColor injektiv}, wenn f\"ur alle $m_1, m_2 \in M$ gilt:
				$$	f(m_1) = f(m_2)	
					\quad
					\Rightarrow
					\quad
					m_1 = m_2.
				$$
			\item {\itemColor surjektiv}, wenn $f(M)=N$, d.h.:
				$$
					\forall n\in N\; \exists m\in M: f(m)=n.
				$$
			\item {\itemColor bijektiv}, wenn $f$ injektiv und surjektiv ist.
		\end{itemize}
	\end{block}
	
\end{frame}
%
%
\begin{frame}
	
	{\itemColor Beispiele:}
	\begin{itemize}
		\item $f: \Z \to \Z,\, x\mapsto x+1$ 
			\pause 
			ist bijektiv.
		\pause
		\item $f: \N \to \N,\, x\mapsto x^2$ 
			\pause
			ist injektiv aber nicht surjektiv. 
		\pause
		\item $f:\Z \to \N,\, x\mapsto x^2$ 
			\pause
			ist weder injektiv noch surjektiv.
	\end{itemize}
	
\end{frame}
%
%%%
\subsection{Verkettung}
%%%
%
\begin{frame}
	
	\begin{block}{Definition}
		Seien $f: A\to B$ und $g:B\to C$ Abbildungen. 
		Die {\itemColor Verkettung} oder {\itemColor Komposition} von $f$ und $g$ ist die Abbildung
		$$
			g\circ f: A\to C,\; x\mapsto g\big( f(x) \big).
		$$
	\end{block}
	
	\pause
	\vfill
	{\itemColor Beispiel:} $f:\R\to\R,\, x\mapsto x^2$ und $g:\R\to\R,\, x\mapsto e^x$: 
	\begin{align*}
		g\circ f(x) 	&= g\big( f(x) \big) = g\big( x^2 \big) = e^{x^2},\\
		f\circ g(x)	&= f\big( g(x) \big) = f\big( e^x\big) = (e^x)^2 = e^{2x}.
	\end{align*}
	
\end{frame}
%
% 
\begin{frame}
	
	\begin{block}{Umkehrabbildung}
		Eine Abbildung $f: M\to N$ ist {\itemColor genau dann} bijektiv, {\itemColor wenn} 
		eine Abbildung $g: N \to M$ existiert mit
		$$
			g\circ f = \text{id}_M
			\qquad
			\text{und}
			\qquad
			f\circ g = \text{id}_N.
		$$  
		
		\pause
		\vspace{2mm}
		Die Abbildung $g$ hei{\ss}t {\itemColor Umkehrabbildung} von $f$. Statt $g$ schreibt man meist $f^{-1}$.
	\end{block}
	
\end{frame}
%
% 
\begin{frame}
	
	Eine Abbildung $f: M\to N$ ist {\itemColor genau dann} bijektiv, {\itemColor wenn} 
	eine Abbildung $g: N \to M$ existiert mit
	$$
		g\circ f = \text{id}_M
		\qquad
		\text{und}
		\qquad
		f\circ g = \text{id}_N.
	$$  
	
	{\itemColor Beweis $\Rightarrow$:}\\ 
	Voraussetzung: $f$ ist bijektiv.\\[1mm]
	$f$ surjektiv: $\forall n\in N\;  \exists m\in M: f(m) = n$.\\
	\pause
	Da $f$ injektiv, ist dieses $m$ eindeutig bestimmt.\\
	\pause
	Definition: $g:N\to M, g(n) = m$ f\"ur das eindeutige $m\in M$ mit $f(m) = n$.\\
	\pause
	Es folgt:
	$$
		g\circ f(m) = g\big( f(m) \big) = m, \qquad \text{d.h. } g\circ f = \text{id}_M
	$$
	und
	$$
		f\circ g(n) = f\big( g(n) \big) = n, \qquad \text{d.h. } f\circ g = \text{id}_N.
	$$	
	
\end{frame}
%
% 
\begin{frame}
	
	Eine Abbildung $f: M\to N$ ist {\itemColor genau dann} bijektiv, {\itemColor wenn} 
	eine Abbildung $g: N \to M$ existiert mit
	$$
		g\circ f = \text{id}_M
		\qquad
		\text{und}
		\qquad
		f\circ g = \text{id}_N.
	$$  
	
	{\itemColor Beweis $\Leftarrow$:} \\
	Voraussetzung: Es gibt eine Abbildung $g: N \to M$ mit $g\circ f = \text{id}_M$ und $f\circ g = \text{id}_N$.\\[1mm]
	\pause
	$f$ ist surjektiv, denn: Sei $n\in N$. F\"ur $m=g(n) \in M$ folgt 
	$$
		f(m) = f\big(g(n)\big) = f\circ g(n) = n.
	$$
	\pause
	$f$ ist injektiv, denn: Seien $m_1, m_2\in M$ mit $f(m_1)=f(m_2) = n \in N$. \\
	\pause
	Dann folgt:
	$$
		m_1 = g\circ f(m_1) =  g\big( f(m_1)\big) =g(n)= g\big( f(m_2)\big) = g\circ f(m_2) = m_2.
	$$
	\qed
	
\end{frame}
%
%%%
\section{Literatur}
%%%
%
\begin{frame}\frametitle{Literatur}

	{\itemColor [Teschl]:}\\[2mm]
	{\itemColor 2.3} Vollst\"andige Induktion\\
	{\itemColor 5.2} Funktionen\\
		
\end{frame}
%
%
\end{document}