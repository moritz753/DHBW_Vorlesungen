\documentclass{beamer}

\usepackage{beamerthemesplit}

\usepackage{amsfonts}
\usepackage{amsmath}
\usepackage{amssymb}
\usepackage{amsthm}
\usepackage{amscd}

\usepackage{stmaryrd} 					%\lightning
\usepackage{algorithm2e}


\usepackage[ngerman]{babel}

\usepackage[utf8]{inputenc}
\usepackage[T1]{fontenc}
\usepackage{textcomp}


% Color Definitions
\definecolor{dhbwRed}{RGB}{226,0,26} 
\definecolor{dhbwGray}{RGB}{61,77,77}
\definecolor{lightBlue}{RGB}{28,134,230}

% Basic Theme
\usetheme{Malmoe}

% Color Re-Definitions
\usecolortheme[named=lightBlue]{structure}
\setbeamercolor*{alerted  text}{fg=dhbwRed, bg=white}
\setbeamercolor*{subsection in toc}{fg=dhbwGray, bg=white}

%\setbeamercolor*{palette primary}{fg=white,bg=lightBlue}
%\setbeamercolor*{palette secondary}{fg=white,bg=gray}
%\setbeamercolor*{palette tertiary}{fg=white,bg=gray}
%\setbeamercolor*{palette quaternary}{fg=white,bg=dhbwRed}

% no navigation symbols
\setbeamertemplate{navigation symbols}{}

% headline, footline
\setbeamertemplate{footline}{\color{dhbwGray} \hfill\insertframenumber\hspace{5mm}\vspace{2mm}}
\setbeamertemplate{headline}{}

% Title Page
\newcommand*{\makeTitlePage}{
	
	\begin{frame}[plain]
		
		\vfill
		\vfill
		\begin{center}
			{
				\usebeamerfont{title}
				\usebeamercolor[fg]{title}
				\Large
				\inserttitle
			}\\[3mm]
			{	
				\usebeamerfont{subtitle}
				\usebeamercolor[fg]{subtitle}
				\large
				\insertsubtitle
			}
		\end{center}
		%
		\vfill
		\vfill
		\vfill
		\vfill
		%
		\begin{columns}
			\begin{column}{0.5\textwidth}
				\begin{flushleft}
					{
						\usebeamerfont{normal text}
						\color{dhbwGray!80}
						\scriptsize
						Dr. Moritz Gruber\\
						DHBW Karlsruhe\\
						
					}
				\end{flushleft}
			\end{column}
			%
			\begin{column}{0.5\textwidth}
				\begin{flushright}
					\includegraphics[scale=0.06]{../DHBW.png}
				\end{flushright}
			\end{column}
		\end{columns}
		%
		\vspace{1mm}
		\begin{columns}
			\begin{column}{0.5\textwidth}
				\begin{flushleft}
					{
						\usebeamerfont{normal text}
						\color{dhbwGray!80}
						\scriptsize
						Version \today
					}
				\end{flushleft}
			\end{column}
			%
			\begin{column}{0.5\textwidth}
				% nothing (just a placeholder to be in line with the columns above
			\end{column}
		\end{columns}
	\end{frame}

}

% Section Divider Page
\newcommand*{\makeSectionDividerPage}{

	\begin{frame}[plain]
		\begin{center}
			\begin{flushleft}
				{				
					\usebeamercolor[fg]{frametitle}
					{\Large \insertsection} \\[3mm]
					{\large \insertsubsection}
				}
			\end{flushleft}
		\end{center}
        \end{frame}
	
}

% itemize
\setbeamertemplate{itemize items}[circle]
\setbeamertemplate{enumerate item}{(\theenumi)}




%--------------------------------------%
% Math ------------------------------%
%--------------------------------------%

% Mengen (Zahlen)
\newcommand{\N}{\mathbb{N}}
\newcommand{\Q}{\mathbb{Q}}
\newcommand{\R}{\mathbb{R}}
\newcommand{\Z}{\mathbb{Z}}
\newcommand{\C}{\mathbb{C}}

% Mengen (allgemein)
\newcommand{\K}{\mathbb{K}}
\newcommand\PX{{\cal P}(X)}

% Zahlentheorie
\newcommand{\ggT}{\mathrm{ggT}}


% Ableitungen
\newcommand{\ddx}{\frac{d}{dx}}
\newcommand{\pddx}{\frac{\partial}{\partial x}}
\newcommand{\pddy}{\frac{\partial}{\partial y}}
\newcommand{\grad}{\text{grad}}

%--------------------------------------%
% Layout Colors ------------------%
%--------------------------------------%
\newcommand*{\highlightDef}[1]{{\color{lightBlue}#1}}
\newcommand*{\highlight}[1]{{\color{lightBlue}#1}} % after theme for colours

%----------------------------------------------------------------------------------------------------
%--------- Document Title ---------------------------------------------------------------------
\title{Lineare Algebra\\[3mm] 
	\large Vektorraum-Aufgabe: Polynome von Grad $\le n$
}
\author{Dr. Moritz Gruber} 
\institute{DHBW Karlsruhe}
\date{2022}
%%%%%%%%%%%%%%
\begin{document}

\AtBeginSection[]{
	\begin{frame}				
		\usebeamercolor[fg]{frametitle}
		{\Large \insertsection} 
        \end{frame}
}

%
\begin{frame}[plain] 
 \titlepage
\end{frame}
%
\begin{frame}\frametitle{Aufgabe: Polynome von Grad $\le n$}
Sei
$$
	V := \{a_0 +a_1X+\ldots +a_nX^n~|~ a_0,\ldots, a_n\in \R \} \subset \R[X]
$$
die Menge der Polynome mit Koeffezienten in $\R$ und maximalem Grad $n$.\\

Seien weiter die Verkn\"upfung $+:V \times V \to V$ und die Abbildung $\cdot: \R\times V \to V$ definiert:
$$
	(a_0 +\ldots +a_nX^n) + (b_0 +\ldots +b_nX^n) := a_0 +b_0+\ldots +(a_n+b_n)X^n
$$
$$
	a\cdot(a_0 +\ldots +a_nX^n) := aa_0 +\ldots +aa_nX^n.
$$

\begin{itemize}
	\item[a)] Zeigen Sie, dass $(V,+,\cdot)$  ein $\R$-Vektorraum ist.
	\item[b)] Zeigen Sie, dass die Vektoren $1, X, \ldots, X^n \in V$ linear unabh\"angig sind.
\end{itemize}
%
\end{frame}
%
%
\begin{frame}\frametitle{Lösung: Teil a)}
%
$(V,+)$ ist eine abelsche Gruppe: \pause
\vfill
\begin{itemize}
\item $+$ ist eine Verknüpfung auf $V$, da für alle $f,g \in V$ gilt: $deg(f+g)\le \max(deg(f),deg(g))\le n$ \pause
\item Die Assoziativität gilt für die Polynom-Addition, also auch insbesondere für die Polynome mit Grad $\le n$.\pause
\item Das neutrale Element ist das Nullpolynom. Und da $deg(0)=-\infty \le n$, gilt $0 \in V$.\pause
\item Das additive Inverse zu $f \in V$ ist $-f$ und es gilt $deg(-f)=deg(f) \le n$ und somit auch $-f \in V$.
\end{itemize}

\end{frame}
%
%
\begin{frame}\frametitle{Lösung: Teil a) Fortsetzung}


		Desweiteren gilt für $a,b \in \R$:
		%
		\begin{itemize}
			\item[(1)] $1\cdot (a_0 + a_1X+\ldots + a_nX^n) = a_0 + a_1X+\ldots + a_nX^n$.\pause
			\item[(2)] Assoziativgesetz:
				\begin{align*}
					a\cdot(b\cdot (a_0 + a_1X+\ldots + a_nX^n) ) 
						&= aba_0 + aba_1X+\ldots + aba_nX^n 		\\
						&= (a\cdot b)\cdot (a_0 + a_1X+\ldots + a_nX^n) 
				\end{align*}\pause
			\item[(3)] Distributivgesetze:
				\begin{align*}
						&(a+b)\cdot (a_0 + a_1X+\ldots + a_nX^n)  								\\		
						&= aa_0 + aa_1X+\ldots + aa_nX^n + ba_0 + ba_1X+\ldots + ba_nX^n 			\\
						&= a\cdot(a_0 + a_1X+\ldots + a_nX^n) +  b\cdot(a_0 + a_1X+\ldots + a_nX^n) 
				\end{align*}	\pause
				(analog zweites Distributivgesetz)\pause
		\end{itemize}
		Damit sind alle definierenden Eigenschaften eines Vektorraums erfüllt.
\end{frame}
%
%
\begin{frame} \frametitle{Lösung: Teil b)}
		Die Vektoren $1, X, \ldots, X^n \in V$ sind linear unabh\"angig:
		$$
			\sum_{j=0}^n a_j X^j=\pause a_0 + a_1X+\ldots + a_nX^n = 0 $$\pause
			$$\Rightarrow \quad a_0=a_1=\ldots=a_n=0.
		$$\pause \vfill
		Tats\"achlich bilden diese Vektoren eine Basis, denn für alle $a_0,...,a_n \in \R$ sind die Vektoren (Polynome)
		$$
			1,X, \ldots, X^n,\, a_0 + a_1X + \ldots + a_nX^n
		$$
		linear abh\"angig: Das letzte Polynom l\"asst sich als Linearkombination der ersten Vektoren schreiben.\\

\end{frame}
%
%


\end{document}