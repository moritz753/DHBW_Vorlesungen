\documentclass{beamer}

\usepackage{beamerthemesplit}

\usepackage{amsfonts}
\usepackage{amsmath}
\usepackage{amssymb}
\usepackage{amsthm}
\usepackage{amscd}

\usepackage{stmaryrd} 					%\lightning
\usepackage{algorithm2e}


\usepackage[ngerman]{babel}

\usepackage[utf8]{inputenc}
\usepackage[T1]{fontenc}
\usepackage{textcomp}


% Color Definitions
\definecolor{dhbwRed}{RGB}{226,0,26} 
\definecolor{dhbwGray}{RGB}{61,77,77}
\definecolor{lightBlue}{RGB}{28,134,230}

% Basic Theme
\usetheme{Malmoe}

% Color Re-Definitions
\usecolortheme[named=lightBlue]{structure}
\setbeamercolor*{alerted  text}{fg=dhbwRed, bg=white}
\setbeamercolor*{subsection in toc}{fg=dhbwGray, bg=white}

%\setbeamercolor*{palette primary}{fg=white,bg=lightBlue}
%\setbeamercolor*{palette secondary}{fg=white,bg=gray}
%\setbeamercolor*{palette tertiary}{fg=white,bg=gray}
%\setbeamercolor*{palette quaternary}{fg=white,bg=dhbwRed}

% no navigation symbols
\setbeamertemplate{navigation symbols}{}

% headline, footline
\setbeamertemplate{footline}{\color{dhbwGray} \hfill\insertframenumber\hspace{5mm}\vspace{2mm}}
\setbeamertemplate{headline}{}

% Title Page
\newcommand*{\makeTitlePage}{
	
	\begin{frame}[plain]
		
		\vfill
		\vfill
		\begin{center}
			{
				\usebeamerfont{title}
				\usebeamercolor[fg]{title}
				\Large
				\inserttitle
			}\\[3mm]
			{	
				\usebeamerfont{subtitle}
				\usebeamercolor[fg]{subtitle}
				\large
				\insertsubtitle
			}
		\end{center}
		%
		\vfill
		\vfill
		\vfill
		\vfill
		%
		\begin{columns}
			\begin{column}{0.5\textwidth}
				\begin{flushleft}
					{
						\usebeamerfont{normal text}
						\color{dhbwGray!80}
						\scriptsize
						Dr. Moritz Gruber\\
						DHBW Karlsruhe\\
						
					}
				\end{flushleft}
			\end{column}
			%
			\begin{column}{0.5\textwidth}
				\begin{flushright}
					\includegraphics[scale=0.06]{../DHBW.png}
				\end{flushright}
			\end{column}
		\end{columns}
		%
		\vspace{1mm}
		\begin{columns}
			\begin{column}{0.5\textwidth}
				\begin{flushleft}
					{
						\usebeamerfont{normal text}
						\color{dhbwGray!80}
						\scriptsize
						Version \today
					}
				\end{flushleft}
			\end{column}
			%
			\begin{column}{0.5\textwidth}
				% nothing (just a placeholder to be in line with the columns above
			\end{column}
		\end{columns}
	\end{frame}

}

% Section Divider Page
\newcommand*{\makeSectionDividerPage}{

	\begin{frame}[plain]
		\begin{center}
			\begin{flushleft}
				{				
					\usebeamercolor[fg]{frametitle}
					{\Large \insertsection} \\[3mm]
					{\large \insertsubsection}
				}
			\end{flushleft}
		\end{center}
        \end{frame}
	
}

% itemize
\setbeamertemplate{itemize items}[circle]
\setbeamertemplate{enumerate item}{(\theenumi)}




%--------------------------------------%
% Math ------------------------------%
%--------------------------------------%

% Mengen (Zahlen)
\newcommand{\N}{\mathbb{N}}
\newcommand{\Q}{\mathbb{Q}}
\newcommand{\R}{\mathbb{R}}
\newcommand{\Z}{\mathbb{Z}}
\newcommand{\C}{\mathbb{C}}

% Mengen (allgemein)
\newcommand{\K}{\mathbb{K}}
\newcommand\PX{{\cal P}(X)}

% Zahlentheorie
\newcommand{\ggT}{\mathrm{ggT}}


% Ableitungen
\newcommand{\ddx}{\frac{d}{dx}}
\newcommand{\pddx}{\frac{\partial}{\partial x}}
\newcommand{\pddy}{\frac{\partial}{\partial y}}
\newcommand{\grad}{\text{grad}}

%--------------------------------------%
% Layout Colors ------------------%
%--------------------------------------%
\newcommand*{\highlightDef}[1]{{\color{lightBlue}#1}}
\newcommand*{\highlight}[1]{{\color{lightBlue}#1}} % after theme for colours

%----------------------------------------------------------------------------------------------------
%--------- Document Title ---------------------------------------------------------------------
\title{Lineare Algebra\\[3mm] 
	\large Gruppen
	}
\author{Dr. Moritz Gruber} 
\institute{DHBW Karlsruhe}
\date{2022}
%%%%%%%%%%%%%%
\begin{document}

\AtBeginSection[]{
	\begin{frame}				
		\usebeamercolor[fg]{frametitle}
		{\Large \insertsection} 
        \end{frame}
}

%
\begin{frame}[plain] 
 \titlepage
\end{frame}
%
%
\begin{frame}\frametitle{Inhalt}
   \tableofcontents
\end{frame}
%
%%%
\section{Verknüpfungen}
%%%
%
\begin{frame}\frametitle{Verknüpfungen}
	
	Bisher:
	\begin{itemize}
		\item Mengen \& Mengenoperationen.
		\item Relationen zwischen Mengen.
	\end{itemize}
	
	\vfill
	In diesem Teil der Vorlesung:
	\begin{itemize}
		\item Verknüpfungen zwischen Elementen einer Menge.
		\item Einführung von \highlightDef{ Gruppen} als wichtige algebraische Objekte.
	\end{itemize} 
	
\end{frame}
%
%
\begin{frame}\frametitle{Definition}

	Sei $M$ eine Menge. Eine \highlightDef{ Verknüpfung} auf $M$ ist eine Abbildung
	$$
		\ast: M \times M \to M.
	$$
	Statt $\ast(m_1,m_2)$ schreibt man $m_1\ast m_2$ (\highlightDef{ Infixnotation}\footnote{statt Präfixnotation}).
	
\end{frame}
%
%
\begin{frame}\frametitle{Beispiele}

	\begin{enumerate}
		\item $M = \R$ und $\ast=$ Multiplikation, d.h. $m_1\ast m_2 := m_1 \cdot m_2$.
		\pause
		\item $M=\R$ und $\ast=$ Addition, d.h. $m_1\ast m_2 := m_1 + m_2$.
		\pause
		\item $M=\N$ und $m_1\ast m_2 := m_1 + m_1\cdot m_2 + m_2$.
		\pause
		\item Sei $M$ eine beliebige Menge. Dann definiert 
			$$
				M_1\ast M_2 := M_1\cap M_2
			$$ 
			eine Verknüpfung auf der Potenzmenge ${\cal P}(M)$.
		\pause
		\item Sei $M$ die Menge aller Abbildungen auf $\R$. 
		Dann ist die Verkettung $\circ$ eine Verknüpfung auf $M$, da mit $f$ und $g$ auch $f\circ g$  eine Abbildung auf $\R$ ist.
	\end{enumerate}

\end{frame}
%
%
\begin{frame}\frametitle{Matrixmultiplikation}
	
	Sei 
	$$
		\R^{2\times 2} 
		:=
		\left\{
			\begin{pmatrix}
				a_{11}	&	a_{12}	\\
				a_{21}	&	a_{22}
			\end{pmatrix}
			~ \Big| ~
			a_{11}, a_{12}, a_{21}, a_{22} \in \R
		\right\}
	$$
	die Menge der \highlightDef{ reellen $(2\times2)$-Matrizen}.\\
	
	\pause
	\vspace{2mm}
	Dann definiert
	$$
		\begin{pmatrix}
			a_{11}	&	a_{12}	\\
			a_{21}	&	a_{22}
		\end{pmatrix}
		\ast
		\begin{pmatrix}
			b_{11}	&	b_{12}	\\
			b_{21}	&	b_{22}
		\end{pmatrix}
		:=
	$$
	$$
		\begin{pmatrix}
			a_{11}\cdot b_{11} + a_{12}\cdot b_{21}	&	a_{11}\cdot b_{12} + a_{12}\cdot b_{22}	\\
			a_{21}\cdot b_{11} + a_{22}\cdot b_{21}	&	a_{21}\cdot b_{12} + a_{22}\cdot b_{22}	\\
		\end{pmatrix}
	$$
	eine Verknüpfung auf $\R^{2\times 2}$. \\[1mm]
	Schreibweise $A\cdot B = AB = A\ast B$ für $A,B \in \R^{2\times 2}$.
	
\end{frame}
%
%
\begin{frame}\frametitle{Verknüpfungen: Eigenschaften}

	Sei $\ast$ eine Verknüpfung auf einer Menge $M$.
	\begin{enumerate}
		\item $\ast$ hei{\ss}t \highlightDef{ assoziativ}, wenn
			$$
				\forall m_1, m_2, m_3 \in M: (m_1\ast m_2) \ast m_3 = m_1 \ast(m_2\ast m_3).
			$$
			Da die Auswertungsreihenfolge in diesem Fall keine Rolle spielt,  ist die kürzere Schreibweise $m_1\ast m_2 \ast m_3$ hier legitim.
		\pause
		\item $\ast$ hei{\ss}t \highlightDef{ kommutativ}, wenn 
			$$
				\forall m_1, m_2 \in M: m_1 \ast m_2 = m_2 \ast m_1.
			$$
	\end{enumerate}

\end{frame}
%
%
\begin{frame}\frametitle{Beispiele}
	
	\begin{enumerate}
		\item Auf den Mengen $\N, \Z, \Q, \R$ sind $+$ und $\cdot$ jeweils assoziative und kommutative Verknüpfungen.
		\pause
		\item Die Matrizenmultiplikation von reellen $2\times 2$-Matrizen ist assoziativ aber nicht kommutativ. (\"Ubungsaufgabe)
		\pause
		\item Sei $M$ die Menge aller Abbildungen auf $\R$. 
		Dann ist die Hintereinanderausführung $\circ$ eine assoziative Verknüpfung auf $M$:
		$$
			(f\circ g) \circ h = f\circ (g\circ h)
		$$
		Diese Verknüpfung ist aber nicht kommutativ:\\
		Für $f: \R\to \R, x\mapsto 1$ und $g: \R\to \R, x\mapsto 2$ folgt:
		$$
			\forall x\in\R: f\circ g(x) = 1 \quad\text{und}\quad g\circ f(x) = 2.
		$$
		
	\end{enumerate}
	
\end{frame}
%
%%%
\section{Gruppen}
%%%
%
%%%
\subsection{Definition}
%%%
%
\begin{frame}\frametitle{Definition}
	
	Seien $G$ eine Menge und $\ast$ eine Verknüpfung auf $G$.
	Dann hei{\ss}t das Paar $(G,\ast)$ eine \highlightDef{ Gruppe}, wenn folgende Bedingungen erfüllt sind:
	\begin{enumerate}
		\item $\ast$ ist assoziativ. \pause
		\item Es gibt ein Element $e\in G$, so dass
			$$
				\forall x\in G:\, x\ast e = e\ast x = x.
			$$
			Das Element $e$ nennt man \highlightDef{ neutrales Element} von $(G,\ast)$. Oft schreibt man $e_G$. \pause 
		\item Zu jedem Element $x\in G$ gibt es ein \highlightDef{ inverses Element}:
			$$
				\forall x\in G\,\, \exists y\in G:\, x\ast y = y\ast x = e.
			$$
	\end{enumerate}

\end{frame}
%
%
\begin{frame}\frametitle{Beispiele}
	
	\begin{enumerate}
		\item $(\Z,+)$ ist eine Gruppe:\\
			$+$ ist assoziativ, das neutrale Element ist $0$ und das zu $x\in\Z$ inverse Element ist $-x$.\\
			NB: ,,$-$'' sollte hier nicht als Verknüpfung betrachtet werden, $x-y$ ist eine Kurzform für $x+(-y)$.
		\pause
		\item $(\Q,+), (\R,+)$ sind ebenfalls Gruppen. $(\N,+)$ nicht. Warum? 
		\pause
		\item Bezüglich Multiplikation  haben wir die Gruppen
			$
				(\Q\backslash\{0\}, \cdot)$ und $ (\R\backslash\{0\}, \cdot).
			$
			Das neutrale Element ist jeweils $e=1$ und das zu $x$ inverse Element ist $\frac{1}{x}$.
		\pause
		\item $(\Z\backslash\{0\},\cdot)$ ist keine Gruppe. Warum?
	\end{enumerate}
	
\end{frame}
%
\begin{frame}\frametitle{Gleichungen in Gruppen}
Die Existenz von Inversen erlaubt es uns beim Lösen von Gleichungen Umformungen vorzunehmen ohne dabei den Wahrheitswert zu verändern.\\\pause
Beispiel in $(\R\setminus \{0\}, \cdot)$:
\begin{align*}
&1/3 \cdot x &=&\ 4/3 \qquad & | \ 3 \cdot \ \text{ (von links)} \\
\Leftrightarrow \quad& 3\cdot 1/3\cdot x &=&\ 3 \cdot 4/3 & \\
\Leftrightarrow \quad& x &=&\ 4 &  \\
\end{align*}
(Da die Multiplikation mit $1/3$, dem Inversen zu $3$, die Umformung rückgängig macht, ist das eine \highlightDef{Äquivalenzumformung}.)
\end{frame}
%
\begin{frame}\frametitle{Das neutrale Element ist eindeutig}
	
	Sei $(G,\ast)$ eine Gruppe und $e$ ein neutrales Element, d.h.
	
	\begin{equation}\label{l_neutralesElement_1}
		\forall x\in G:\, x\ast e = e\ast x = x.
	\end{equation}
	
	\pause
	Angenommen, es gilt für $e_1\in G$ ebenfalls
	
	\begin{equation}\label{l_neutralesElement_2}
		\forall x\in G:\, x\ast e_1 = e_1\ast x = x.
	\end{equation}
	
	\pause
	Dann folgt mit $x=e_1$ aus (1)
	$$
		e_1\ast e = e_1
	$$\pause
	und mit $x=e$ aus (2)
	$$
		e_1\ast e = e.
	$$\pause
	Somit:
	$$
		e_1 = e.
	$$
	\hfill$\qed$
		
\end{frame}
%
%%%
\subsection{Die symmetrische Gruppe}
%%%
%
\begin{frame}\frametitle{Die symmetrische Gruppe}

	Sei $D$ eine endliche Menge und $Sym(D)$ die Menge aller bijektiven Abbildungen $f:D \to D$. Dann ist
	$$
		(Sym(D),\circ)
	$$
	eine Gruppe, die \highlightDef{ symmetrische Gruppe} von $D$. (Übungsaufgabe!)\\
	\quad \pause\\
	Sei $n$ die Kardinalität von $D$, dann können wir die Elemente von $D$ mit $1,2,3,...,n$ bezeichnen und man schreibt statt $Sym(D)$ auch kurz $S_n$. Jedes Element von $S_n$ ist eine \highlightDef{Permutation} der Elemente von $D$. Zum Beispiel gibt es für $n=3$ die folgenden Permutationen:
		
\end{frame}
%
%%
\begin{frame}\frametitle{$S_3$}
	Permutationen von $D=\{1,2,3\}$:
	$$
		\begin{array}{c | c|c|c}
				& 1 & 2 & 3	\\ \hline
			Id	& 1 & 2 & 3 \\
			\zeta_1	& 2 & 3 & 1 \\
			\zeta_2 & 3 & 1 & 2 \\
			\tau_1 & 1 & 3 & 2 \\
			\tau_2 & 3 & 2 & 1 \\
			\tau_3 & 2 & 1 & 3
		\end{array}
	$$
	\pause
Jede Permutation läßt sich als ein Zyklus von Vertauschungen schreiben, z.B. \\$\zeta_1: \zeta_1(1)=2, \ \zeta_1(2)=3, \ \zeta_1(3)=1$ \quad oder\\
$\zeta_2: \zeta_2(1)=3, \ \zeta_2(3)=2, \ \zeta_2(2)=1$. 	
\end{frame}
%
%
\begin{frame}\frametitle{Zykel und Transpositionen}
Allgemeiner:
\begin{itemize}
	\item Es sei $D$ eine endliche Menge und $x_1,x_2,...,x_k \in D$ paarweise verschiedene Elemente. Dann wird durch
	 $$
	\zeta(x) :=\begin{cases}	
							x_{i+1} & \text{wenn } x=x_i,\ i<k\\
							x_1 & \text{wenn } x=x_k\\
							x & \text{sonst}
	\end{cases}
	$$
	wird eine bijektive Abbildung $\zeta:D\to D$ definiert. Man nennt sie einen \highlightDef{$k$-Zykel} und schreibt sie kurz als $\zeta=(x_1\ x_2\ ... x_k)$. \pause
	\item Einen 2-Zykel $\tau=(x_1\ x_2)$ nennt man eine \highlightDef{Transposition}.
\end{itemize}
%
	
\end{frame}
%
\begin{frame}\frametitle{Satz}
	
Es sei $\sigma \in S_n$. Dann gibt es ein $k \in \N$ und Transpositionen $\tau_1,\tau_2,...,\tau_k \in S_n$, sodass
$$ \sigma=\tau_1 \circ \tau_2 \circ ... \circ \tau_k. $$
\vfill \pause
Beispiel
\begin{itemize}
\item $(2\ 3\ 1) = (2\ 3) \circ (3\ 1)$
\end{itemize}
\end{frame}
%
%%%%%%%%%%%%%
%
\begin{frame}\frametitle{Exkurs: Vollständige Induktion}

	Die Vollständige Induktion ist ein Beweisverfahren, das verwendet werden kann, um nachzuweisen, dass Aussagen der Form
	$$
		\forall n\in\N : a(n)	
	$$	
	oder allgemeiner
	$$
		\forall n \in \N, n\ge N: a(n)
	$$
	wahr sind.
	
	\pause
	\vspace{2mm}
	\highlightDef{ Beispiel:} \\
	Zeigen Sie, dass gilt
	$$
		\forall n\in \N:\quad  \underbrace{\sum_{k=0}^n k = \frac{1}{2}n(n+1)}_{a(n)}.
	$$	
	
	\vfill
	{\scriptsize Bemerkung: $\displaystyle\sum_{k=0}^n a_k = a_0 + a_1 + a_2 + \ldots + a_n$} 
\end{frame}
%
%
\begin{frame}\frametitle{Beispiel}
	Die Aussage
	$$
		\forall n\in \N:\quad  \underbrace{\sum_{k=0}^n k = \frac{1}{2}n(n+1)}_{a(n)}
	$$	
	kann für einzelne natürliche Zahlen $n$ überprüft werden: 
	$$
		\begin{array}{lcll}
			n=1	&:& 0+1=1\quad \checkmark	&(a(1)\text{~ist wahr})		\\
			n=2	&:& 0+1+2=3\quad \checkmark	&(a(2)\text{~ist wahr})	\\
			...
		\end{array}
	$$
	\pause
	Damit kann man natürlich {\em nicht} zeigen, 
	dass die Aussageform $a(n)$ {\em für alle} $n\in\N$ wahr ist - es gibt schlie{\ss}lich unendlich viele natürliche Zahlen.
\end{frame}
%
%
\begin{frame}\frametitle{Prinzip der vollständigen Induktion}
	
	Um nachzuweisen, dass eine Aussageform $a(n)$ für alle $n\geq N$ wahr ist, setzt man
	$$
		S := \{ n\in \N~|~ n\geq N \text{~und~} a(n)\text{~wahr} \}.
	$$\pause
	\vfill
	Es genügt nun zu zeigen, dass
	\begin{itemize}
		\item $N\in S$ und
		\item $\forall n\in S: n+1 \in S$.
	\end{itemize}
	Denn dann folgt
	$$
		S = \{N, N+1, N+2,\ldots\}.
	$$
	
\end{frame}
%
%
\begin{frame}\frametitle{Prinzip der vollständigen Induktion}
	
	Das Prinzip der vollständigen Induktion wird oft auch folgenderma{\ss}en formuliert:\\
	\vfill
	Sei für alle $n\geq N$ die Aussageform $a(n)$ gegeben.\\[1mm]
	
	Falls\\[1mm]
	
	\highlightDef{ (1) Induktionsanfang:} $a(N)$ ist wahr.\\[1mm]  
	und\\[1mm]
	\highlightDef{ (2) Induktionsvoraussetzung:} $a(n)$ ist wahr.\\[1mm]  
	und\\[1mm]
	\highlightDef{ (3) Induktionsschluss:} Dann ist auch $a(n+1)$ wahr.\\[1mm]
	
	gilt, ist die Aussage 
	$$
		\forall n\in \N: n\geq N \rightarrow a(n).
	$$	
	wahr.
	
\end{frame}
%
%
\begin{frame}\frametitle{Beispiel}
	
	Zeigen Sie, dass gilt
	$$
		\forall n\in \N:\quad  \underbrace{\sum_{k=0}^n k = \frac{1}{2}n(n+1)}_{a(n)}.
	$$	
	
\end{frame}
%
%
\begin{frame}
	
	Zeigen Sie, dass gilt
	$$
		\forall n\in \N:\quad  \underbrace{\sum_{k=0}^n k = \frac{1}{2}n(n+1)}_{a(n)}.
	$$	
	
	\highlightDef{ (1) Induktionsanfang:} $a(1)$ ist wahr, da $1 = \frac{1}{2}\cdot 1 \cdot (1+1)$.
	
\end{frame}
%
%
\begin{frame}

	Zeigen Sie, dass gilt
	$$
		\forall n\in \N:\quad  \sum_{k=0}^n k = \frac{1}{2}n(n+1).
	$$	
	
	\highlightDef{ (2) Induktionsvoraussetzung:} Es sei $a(n)$ wahr. \\\pause
	\highlightDef{ (3) Induktionsschluss:}
	\begin{align*}
		\sum_{k=0}^{n+1} k	&= \sum_{k=0}^{n}k + (n+1)\\
						&=  \frac{1}{2}n(n+1) + (n+1)\\
						&=  \frac{1}{2}(n+1)(n+2),
	\end{align*}
	wobei im zweiten Schritt für  $\sum_{k=0}^{n} =  \frac{1}{2}n(n+1)$ die \highlightDef{ Induktionsvoraussetzung} 
	``$a(n)$ ist wahr'' verwendet wurde.

	
\end{frame}
%
%
\begin{frame}

Es gilt also, unter der Annahme, dass $a(n)$ wahr ist, dass 
	$$
		\sum_{k=0}^{n+1} k =  \frac{1}{2}(n+1)(n+2).
	$$
	Somit ist in diesem Fall auch $a(n+1)$ wahr.\\[5mm]

	
	Damit ist die Aussage durch vollständige Induktion bewiesen. \qed
	
\end{frame}
%

%
%
\begin{frame}\frametitle{Satz}
	
Es sei $\sigma \in S_n$. Dann gibt es ein $k \in \N$ und Transpositionen $\tau_1,\tau_2,...,\tau_k \in S_n$, sodass
$$ \sigma=\tau_1 \circ \tau_2 \circ ... \circ \tau_k. $$
\vfill \pause
Dies ist eigentlich eine Aussage der Form ``$\forall n \in \N \text{ gilt:}$'', denn für ein festes $n$ kann man alle Elemente von $S_n$ und ihre Transpositionszerlegung konkret aufschreiben.\pause
\vfill 
$\forall n \in \N \ \forall \sigma \in S_n \ \exists k \in \N \land \tau_1,\tau_2,...,\tau_k \in S_n:$
$$ \sigma=\tau_1 \circ \tau_2 \circ ... \circ \tau_k. $$
\end{frame}
%
%
\begin{frame}\frametitle{Beweis}
Wir machen eine vollständige Induktion nach $n$.\\\pause
Wir beginnen mit $n=2$ (warum?) und $S_2$ hat nur zwei Elemente $Id$ und $(1\ 2)$.	\\\pause
Sei also $n\ge 2$ und für alle kleineren Werte von $n$ sei die Behauptung bereits bewiesen.\\ \pause
Wenn $\sigma(n)=n$ gilt, so ist $\sigma$ auch schon ein Element von $S_{n-1}$ und somit gibt es ein $l \in \N$ und  Transpositionen $\tau_1,\tau_2,...,\tau_l$ sodass $\sigma=\tau_1 \circ \tau_2 \circ ... \circ \tau_l$.\pause\\
Wenn $a=\sigma(n) \ne n$ gilt, dann gilt mit der Transposition $\tau=(a\ n)$:
$$(\tau \circ \sigma)(n)=\tau(a)=n$$
und folglich ist $\tau \circ \sigma$ in $S_{n-1}$ und damit ein Produkt von Transpositionen.\pause Da nun auch $\tau^{-1} = \tau$ eine Transposition ist, läßt sich auch $\sigma=\tau^{-1} \circ (\tau \circ \sigma)$ also durch Transpositionen darstellen. \hfill $\square$

	
\end{frame}
%
%


%
%%%
\subsection{Abel'sche Gruppen}
%%%
%
%
\begin{frame}\frametitle{Abel'sche Gruppen}

	Eine Gruppe $(G,\ast)$ hei{\ss}t \highlightDef{ kommutativ} oder \highlightDef{ abelsch}\footnote{Niels Henrik Abel, 1802 - 1829},
	wenn die Verknüpfung $\ast$ kommutativ ist, d.h., wenn für alle $a,b \in G$ gilt:
	$$
		a\ast b = b \ast a.
	$$

\end{frame}
%

\begin{frame}\frametitle{Abel'sche Gruppen: Beispiele}
	
	Die folgenden Gruppen sind abelsch:
	\begin{enumerate}
		\item $(\Z,+), (\Q,+), (\R,+),$\pause
		\item $(\Q\backslash\{0\}, \cdot), (\R\backslash\{0\}, \cdot),$
	\end{enumerate}
	\vfill \pause
	Sei
	$$
		G:= \{ A \in \R^{2\times 2}~|~ A \text{ invertierbar} \}.
	$$
	Dann ist $(G,\cdot)$ eine Gruppe, die \underline{nicht} abelsch ist.\\[1mm]
	(\"Ubungsaufgabe)
	
	
\end{frame}
%
\subsection{Untergruppen}
%
\begin{frame}\frametitle{Untergruppen}
Es sei $(G,\ast)$ eine Gruppe und $H$ eine Teilmenge von $G$. Dann heißt $(H, \ast)$ eine \highlightDef{Untergruppe} von $(G,\ast)$, wenn
\begin{itemize}
\item[(i)] $e_G \in H$
\item[(ii)] $\forall h_1,h_2 \in H: h_1\ast h_2 \in H$
\item[(iii)] $\forall h \in H: h^{-1} \in H$.
\end{itemize}
\pause
\vfill
\highlightDef{Beispiele:}
\begin{itemize}
\item $(\Z,+)$ ist eine Untergruppe von $(\Q,+)$. \pause
\item Für alle $n \in \N$ ist $n\Z:=\{nk \mid k \in \Z\}$ mit $+$ ein Untergruppe von $(\Z,+)$. \pause
\item Für $E \subseteq D$ ist $Sym(E)$ eine Untergruppe von $Sym(D)$.
\end{itemize}
\end{frame}
%
\begin{frame}\frametitle{Untergruppenkriterium}
Es sei $(G,\ast)$ eine Gruppe und $H \subset G$. $H$ ist genau dann eine Untergruppe von $G$, wenn gilt:
$$
H\ne \emptyset \text{  und  } \forall h_1,h_2 \in H: h_1 \ast h_2^{-1} \in H
$$
\vfill \pause
Dieses Kriterium ist sehr praktisch um Untergruppen zu identifizieren.
\end{frame}
%%%
\subsection{Homomorphismen von Gruppen}
%%%
%
\begin{frame}\frametitle{Definition: Homomorphismus}
	
	Seien $(G_1,\ast)$ und $(G_2, \bullet)$ zwei Gruppen. Eine Abbildung 
	$$
		f: G_1 \to G_2
	$$ 
	hei{\ss}t \highlightDef{ (Gruppen-)Homomorphismus}, wenn
	$$
		\forall x,y \in G_1:\, f(x\ast y) = f(x)\bullet f(y).
	$$
	
	\pause
	\vfill
	Homomorphismus: 
	\begin{itemize}
		\item Abbildung, die {\em verträglich} mit der algebraischen Strukur der jeweiligen Gruppen ist.
		\item Strukturerhaltende Abbildung.
	\end{itemize}

\end{frame}
%
%
\begin{frame}\frametitle{Beispiele}
	
	\begin{enumerate}
		\item $(\R,+)$ und $( (0,\infty),\cdot)$ sind Gruppen und die Abbildung
			$$
				\exp: \R \to (0,\infty),\, x\mapsto \exp(x)
			$$
			ist ein Gruppenhomomorphismus, da
			$$
				\forall x,y \in\R:\, \exp(x+y) = \exp(x)\cdot\exp(y).
			$$
		\pause
		\item Die Umkehrabbildung
			$$
				\log: (0,\infty) \to \R,\, x\mapsto \log(x)
			$$
			ist ebenfalls ein Gruppenhomomorphismus, da
			$$
				\forall x,y\in (0,\infty):\, \log(x \cdot y) = \log(x) + \log(y).
			$$
	\end{enumerate}
	
	\vfill
	\pause
	Bemerkung:\\
	Einen bijektiven Homomorphismus nennt man \highlightDef{ Isomorphismus}.
	
\end{frame}
%
%
\begin{frame}\frametitle{Eigenschaften}

	Seien $(G_1,\ast)$ und $(G_2,\bullet)$ Gruppen sowie $f: G_1\to G_2$ ein Homomorphismus.
	Dann gilt:
	\begin{itemize}
		\item[(a)] $f(e_{G_1}) = e_{G_2}$. \pause
		\item[(b)] $\forall x\in G_1:\, f(x^{-1}) = f(x)^{-1}$.\\
				Hierbei bezeichnet $x^{-1}$ das zu $x$ inverse Element in $G_1$ und  $f(x)^{-1}$ das zu $f(x)$ inverse Element in $G_2$.\pause
		\item[(c)] $f$ ist genau dann injektiv, wenn $f^{-1}(e_{G_2}) = \{e_{G_1}\}$.\\
			Hierbei bezeichnet $f^{-1}(e_{G_2})$ die \highlightDef{ Urbildmenge} von $e_{G_2}$:
			$$
				f^{-1}(e_{G_2}) := \{x\in G_1~|~ f(x) = e_{G_2} \}.
			$$
	\end{itemize} 
	
\end{frame}
%
%
\begin{frame}\frametitle{Eigenschaften: Beweis}

	\begin{itemize}
		\item[(a)] $f(e_{G_1}) = e_{G_2}$:\\
				Es gilt:
				$$
					f(e_{G_1}) = f(e_{G_1}\ast e_{G_1}) = f(e_{G_1})\bullet f(e_{G_1}).
				$$\pause
				Wenn wir diese Gleichung mit dem inversen Element  $f(e_{G_1})^{-1}$ von $f(e_{G_1})$ ``multiplizieren'', erhalten wir:
				$$
					f(e_{G_1})\bullet f(e_{G_1})^{-1} = e_{G_2}
				$$\pause
				und
				$$
					\Big(f(e_{G_1})\bullet f(e_{G_1})\Big) \bullet  f(e_{G_1})^{-1}  \pause= 
						f(e_{G_1})\bullet \Big( f(e_{G_1}) \bullet  f(e_{G_1})^{-1} \Big) \pause= f(e_{G_1}).
				$$

	\end{itemize} 
	
\end{frame}
%
%
\begin{frame}\frametitle{Eigenschaften: Beweis}

	\begin{itemize}
		\item[(b)] $\forall x\in G_1:\, f(x^{-1}) = f(x)^{-1}$:\\
				Es gilt:
				$$
					f(x)\bullet f(x^{-1}) \pause= f(x\ast x^{-1}) \pause= f(e_{G_1}) = e_{G_2}.
				$$
				D.h., $f(x^{-1})$ ist das zu $f(x)$ inverse Element in $G_2$.
	\end{itemize} 
	
\end{frame}
%
%
\begin{frame}\frametitle{Eigenschaften: Beweis}
	\begin{itemize}
		\item[(c)] $f$ ist genau dann injektiv, wenn $f^{-1}(e_{G_2}) = \{e_{G_1}\}$:\\
				\pause \vfill
				\begin{itemize}
					\item[(1)] Sei $f$ injektiv. 
							\begin{itemize}
								\item Aus (a) folgt: $e_{G_1} \in f^{-1}(e_{G_2})$.
								\pause
								\item Da $f$ injektiv ist, kann höchstens ein Element aus $G_1$ auf $e_{G_2}$ abgebildet werden.
								\pause
								\item Somit: $f^{-1}(e_{G_2}) = \{e_{G_1}\}$.
							\end{itemize}
					\pause		 \vfill
					\item[(2)] Es gelte $f^{-1}(e_{G_2}) = \{e_{G_1}\}$.
							\begin{itemize}
								\item Seien $x,y \in G_1$ mit $f(x)=f(y)$. 
								\pause
								\item $e_{G_2} = f(y)\bullet f(y)^{-1} = f(x)\bullet f(y^{-1}) = f(x\ast y^{-1})$.
								\pause
								\item Somit: $x\ast y^{-1} = e_{G_1}$ bzw. $x=y$.
							\end{itemize}
							\hfill\qed
				\end{itemize} 
	\end{itemize} 
\end{frame}
%
%
%%%
\section{Rechnen ohne Rechner: Logarithmentafeln}
%%%
%
\begin{frame}\frametitle{Ein Anwendungsfall: Logarithmentafeln}

	Tabellarische Darstellung von Logarithmen: 
	$$
		\log_{10}(6.006) \sim 0.7786.
	$$

	\includegraphics[scale=0.23]{Logarithmentafel.png}

\end{frame}
%
%
\begin{frame}\frametitle{Logarithmentafeln}

	\begin{itemize}
		\item John Napier (1550 - 1617) gilt als Erfinder der Logarithmentafeln.
		\item Erleichterten die Multiplikation von Zahlen.
		\item Wichtige Rechenhilfe über mehrere Jahrhunderte bis zur Einführung von Rechenmaschinen.
		\item Im Schulunterricht verwendet bis in die 1970er Jahre.
	\end{itemize}

\end{frame}
%
%
\begin{frame}\frametitle{Multiplikation}
	
	$$
		\begin{array}{ccccccccccc}
 			1	&2	&3	&4	&5	&\cdot	&5	&4	&3	&2	&1	\\\hline
 			6	&1	&7	&2	&5								\\
   				& 4	&9	&3	&8	&0							\\
   				&    	&3	&3	&0	&3	&5						\\
   				&    	&   	&2	&4	&6	&9	&0					\\
   				&    	&   	&  	&1	&2	&3	&4	&5				\\\hline
   			6	&7	&0	&5	&9	&2	&7	&4	&5
		\end{array}
	$$
	
	\vfill
	\begin{itemize}
		\item Die schriftliche Multiplikation (Schulmethode) zweier Zahlen mit $n$ Ziffern benötigt $\sim n^2$ Rechenschritte 
		(Additionen und Multiplikationen). 
		\item Jeder Rechenschritt kostet Zeit und ist eine mögliche Fehlerursache.
	\end{itemize}
\end{frame}
%
%
\begin{frame}\frametitle{Logarithmentafeln: Addition statt Multiplikation}

	$$
		\begin{CD}
			(0,\infty) @> x \cdot y>> (0,\infty)\\
			@V{\log}VV 	@AA{\exp}A\\
			\R @> \log(x) + \log(y) >> \R
		\end{CD}
	$$
	\pause
	\begin{itemize}
		\item Schlage die Logarithmen der beiden Faktoren nach:
			$$
				\log(12345) \sim 9.421006, \qquad \log(54321) \sim 10.902666
			$$
			(hier: Logarithmus zur Basis $e$) \pause
		\item Addiere die Logarithmen:
			$$
				\log(12345) + \log(54321) \sim 20.3236. 
			$$\pause
		\item Schlage nach, welche Zahl den Logarithmus $20.323672$ hat:
			$$
				12345\cdot 54321 \sim \exp(20.323672) \sim 670592360.57.
			$$
	\end{itemize}

\end{frame}
%
%
\begin{frame}\frametitle{Logarithmentafeln: Addition statt Multiplikation}

	\begin{itemize}
		\item Die Addition zweier Zahlen mit $n$ Ziffern benötigt nur $\sim n$ Rechenschritte.
		\item Das Nachschlagen der Logarithmen entspricht einem Suchalgorithmus und benötigt $\sim \log(n)$ Schritte.
	\end{itemize}
	\pause
	\vfill
	Der Rechenaufwand wurde somit erheblich reduziert:
	$$
		\text{von~} \sim n^2 \text{~auf~} \sim n.
	$$
	
\end{frame}
%
%
\begin{frame}\frametitle{Zusammenfassung}
	
		$$
		\begin{CD}
			(0,\infty) @> x \cdot y>> (0,\infty)\\
			@V{\log}VV 	@AA{\exp}A\\
			\R @> \log(x) + \log(y) >> \R
		\end{CD}
	$$
	
	\begin{itemize}
		\item Rechenaufgabe in der Gruppe $\Big( (0,\infty),\cdot\Big)$.
		\item \"Ubersetzung mit dem Isomorphismus $\log$ in eine Rechenaufgabe in der Gruppe $(\R,+)$.
		\item Lösen der Rechenaufgabe in $(\R,+)$.
		\item \"Ubersetzung der Lösung mit dem Isomorphismus $\exp$ in eine Lösung in der Gruppe $\Big( (0,\infty),\cdot\Big)$.
	\end{itemize}
	
\end{frame}
%

%
\end{document}