\documentclass{beamer}

\usepackage{beamerthemesplit}

\usepackage{amsfonts}
\usepackage{amsmath}
\usepackage{amssymb}
\usepackage{amsthm}
\usepackage{amscd}

\usepackage{stmaryrd} 					%\lightning
\usepackage{algorithm2e}


\usepackage[ngerman]{babel}

\usepackage[utf8]{inputenc}
\usepackage[T1]{fontenc}
\usepackage{textcomp}


% Color Definitions
\definecolor{dhbwRed}{RGB}{226,0,26} 
\definecolor{dhbwGray}{RGB}{61,77,77}
\definecolor{lightBlue}{RGB}{28,134,230}

% Basic Theme
\usetheme{Malmoe}

% Color Re-Definitions
\usecolortheme[named=lightBlue]{structure}
\setbeamercolor*{alerted  text}{fg=dhbwRed, bg=white}
\setbeamercolor*{subsection in toc}{fg=dhbwGray, bg=white}

%\setbeamercolor*{palette primary}{fg=white,bg=lightBlue}
%\setbeamercolor*{palette secondary}{fg=white,bg=gray}
%\setbeamercolor*{palette tertiary}{fg=white,bg=gray}
%\setbeamercolor*{palette quaternary}{fg=white,bg=dhbwRed}

% no navigation symbols
\setbeamertemplate{navigation symbols}{}

% headline, footline
\setbeamertemplate{footline}{\color{dhbwGray} \hfill\insertframenumber\hspace{5mm}\vspace{2mm}}
\setbeamertemplate{headline}{}

% Title Page
\newcommand*{\makeTitlePage}{
	
	\begin{frame}[plain]
		
		\vfill
		\vfill
		\begin{center}
			{
				\usebeamerfont{title}
				\usebeamercolor[fg]{title}
				\Large
				\inserttitle
			}\\[3mm]
			{	
				\usebeamerfont{subtitle}
				\usebeamercolor[fg]{subtitle}
				\large
				\insertsubtitle
			}
		\end{center}
		%
		\vfill
		\vfill
		\vfill
		\vfill
		%
		\begin{columns}
			\begin{column}{0.5\textwidth}
				\begin{flushleft}
					{
						\usebeamerfont{normal text}
						\color{dhbwGray!80}
						\scriptsize
						Dr. Moritz Gruber\\
						DHBW Karlsruhe\\
						
					}
				\end{flushleft}
			\end{column}
			%
			\begin{column}{0.5\textwidth}
				\begin{flushright}
					\includegraphics[scale=0.06]{../DHBW.png}
				\end{flushright}
			\end{column}
		\end{columns}
		%
		\vspace{1mm}
		\begin{columns}
			\begin{column}{0.5\textwidth}
				\begin{flushleft}
					{
						\usebeamerfont{normal text}
						\color{dhbwGray!80}
						\scriptsize
						Version \today
					}
				\end{flushleft}
			\end{column}
			%
			\begin{column}{0.5\textwidth}
				% nothing (just a placeholder to be in line with the columns above
			\end{column}
		\end{columns}
	\end{frame}

}

% Section Divider Page
\newcommand*{\makeSectionDividerPage}{

	\begin{frame}[plain]
		\begin{center}
			\begin{flushleft}
				{				
					\usebeamercolor[fg]{frametitle}
					{\Large \insertsection} \\[3mm]
					{\large \insertsubsection}
				}
			\end{flushleft}
		\end{center}
        \end{frame}
	
}

% itemize
\setbeamertemplate{itemize items}[circle]
\setbeamertemplate{enumerate item}{(\theenumi)}




%--------------------------------------%
% Math ------------------------------%
%--------------------------------------%

% Mengen (Zahlen)
\newcommand{\N}{\mathbb{N}}
\newcommand{\Q}{\mathbb{Q}}
\newcommand{\R}{\mathbb{R}}
\newcommand{\Z}{\mathbb{Z}}
\newcommand{\C}{\mathbb{C}}

% Mengen (allgemein)
\newcommand{\K}{\mathbb{K}}
\newcommand\PX{{\cal P}(X)}

% Zahlentheorie
\newcommand{\ggT}{\mathrm{ggT}}


% Ableitungen
\newcommand{\ddx}{\frac{d}{dx}}
\newcommand{\pddx}{\frac{\partial}{\partial x}}
\newcommand{\pddy}{\frac{\partial}{\partial y}}
\newcommand{\grad}{\text{grad}}

%--------------------------------------%
% Layout Colors ------------------%
%--------------------------------------%
\newcommand*{\highlightDef}[1]{{\color{lightBlue}#1}}
\newcommand*{\highlight}[1]{{\color{lightBlue}#1}} % after theme for colours

%----------------------------------------------------------------------------------------------------
%--------- Document Title ---------------------------------------------------------------------
\title{Lineare Algebra\\[3mm] 
	\large LGS-Aufgabe: Lösbarkeit
}
\author{Dr. Moritz Gruber} 
\institute{DHBW Karlsruhe}
\date{2022}
%%%%%%%%%%%%%%
\begin{document}

\AtBeginSection[]{
	\begin{frame}				
		\usebeamercolor[fg]{frametitle}
		{\Large \insertsection} 
        \end{frame}
}

%
\begin{frame}[plain] 
 \titlepage
\end{frame}
%
\begin{frame}\frametitle{Aufgabe: Lösbarkeit LGS}
%
Seien $\alpha,\beta \in \R$ und
$$
	A =
	\begin{pmatrix}
		1	&5	&\alpha	\\
		0	&-2	&1	\\
		-1	&1	&3
	\end{pmatrix},
	\qquad
	b =
	\begin{pmatrix}
		-1	\\
		\beta	\\
		1
	\end{pmatrix}.
$$

F\"ur welche Werte von $\alpha$ und $\beta$ ist das LGS $A\cdot x = b$
\begin{itemize}
	\item eindeutig l\"osbar,
	\item nicht l\"osbar,
	\item l\"osbar, aber nicht eindeutig l\"osbar?
\end{itemize}


%
\end{frame}
%
%
\begin{frame}\frametitle{Lösung:}
%
%-------------------------
\subsection*{L\"osung}
%%%
Wir schreiben das LGS in der Kurzform und verwenden das Gauß-Verfahren:
$$
	\left(
		\begin{array}{rrr | r}
			1	&5	&\alpha	&-1\\
			0	&-2	&1	&\beta\\
			-1	&1	&3	&1
		\end{array}
	\right)
$$\pause

Addition von Zeile 1 zu Zeile 3:
$$
	\sim>
	\left(
		\begin{array}{rrr | r}
			1	&5	&\alpha		&-1\\
			0	&-2	&1	&\beta\\
			0	&6	&\alpha+3	&0
		\end{array}
	\right)
$$\pause
Multiplikation von Zeile 2 mit $-\frac{1}{2}$:
$$
	\sim>
	\left(
		\begin{array}{rrr | r}
			1	&5	&\alpha		&-1\\
			0	&1	&-1/2	&-\beta/2\\
			0	&6	&\alpha+3	&0
		\end{array}
	\right)
$$
\end{frame}
%
\begin{frame}\frametitle{Lösung (Fortsetzung):}
$$
	\left(
		\begin{array}{rrr | r}
			1	&5	&\alpha		&-1\\
			0	&1	&-1/2	&-\beta/2\\
			0	&6	&\alpha+3	&0
		\end{array}
	\right)
$$\pause
Addition des $(-6)$-fachen von Zeile 2 zu Zeile 3:
$$
	\sim>
	\left(
		\begin{array}{rrr | r}
			1	&5	&\alpha		&-1\\
			0	&1	&-1/2	&-\beta/2\\
			0	&0	&\alpha+6	&3\beta
		\end{array}
	\right)
$$\pause

Das LGS ist somit
\begin{itemize}
	\item eindeutig l\"osbar, \pause wenn $\alpha\neq -6$,\pause
	\item nicht l\"osbar, \pause wenn $\alpha=-6$ \underline{und} $\beta\neq 0$\pause
	\item l\"osbar, aber nicht eindeutig l\"osbar, \pause wenn $\alpha=-6$ \underline{und} $\beta=0$.
\end{itemize}
\end{frame}


\end{document}