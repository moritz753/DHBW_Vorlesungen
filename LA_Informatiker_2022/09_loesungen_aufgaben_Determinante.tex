\documentclass[
				a4paper,
				10pt
			]
			{scrartcl}

\parindent0mm

\usepackage{amsfonts}
\usepackage{amsmath}
\usepackage{amssymb}
\usepackage{amsthm}
\usepackage[ngerman]{babel}

\usepackage[utf8]{inputenc}
\usepackage[T1]{fontenc}
\usepackage{textcomp}

\usepackage{graphicx}
\usepackage{xcolor}

\usepackage[
			pdftex,
			colorlinks,
			breaklinks,
			linkcolor=blue,
			citecolor=blue,
			filecolor=black,
			menucolor=black,
			urlcolor=black,
			pdfauthor={Andreas Weber},
			pdftitle={Aufgaben zu Analysis und Lineare Algebra},
			plainpages=false,
			pdfpagelabels,
			bookmarksnumbered=true
		   ]{hyperref}


%--------------------------------------%
% Math ------------------------------%
%--------------------------------------%

% Mengen (Zahlen)
\newcommand{\N}{\mathbb{N}}
\newcommand{\Q}{\mathbb{Q}}
\newcommand{\R}{\mathbb{R}}
\newcommand{\Z}{\mathbb{Z}}
\newcommand{\C}{\mathbb{C}}

% Mengen (allgemein)
\newcommand{\K}{\mathbb{K}}
\newcommand\PX{{\cal P}(X)}

% Zahlentheorie
\newcommand{\ggT}{\mathrm{ggT}}


% Ableitungen
\newcommand{\ddx}{\frac{d}{dx}}
\newcommand{\pddx}{\frac{\partial}{\partial x}}
\newcommand{\pddy}{\frac{\partial}{\partial y}}
\newcommand{\grad}{\text{grad}}

%--------------------------------------%
% Layout Colors ------------------%
%--------------------------------------%
\newcommand*{\highlightDef}[1]{{\color{lightBlue}#1}}
\newcommand*{\highlight}[1]{{\color{lightBlue}#1}}
% Color Definitions
\definecolor{dhbwRed}{RGB}{226,0,26} 
\definecolor{dhbwGray}{RGB}{61,77,77}
\definecolor{lightBlue}{RGB}{28,134,230}


%-------------------------------------------------------------------
\begin{document}

\vspace*{-20mm}
{
	%\usekomafont{title}
	\color{dhbwGray}
	Dr. Moritz Gruber	\hfill Version \today\\
	DHBW Karlsruhe\\
}

\vspace{10mm}
\begin{center}
	{
		\usekomafont{title}
		\color{lightBlue}
		{ \LARGE L\"osungen Übungsaufgaben 9}\\[3mm]
		{\Large Determinante}
	}
\end{center}

\vspace{5mm}

%-------------------------------------------------------------------


%-------------------------
%-------------------------
\section{Invertierbar?}
%%%

Prüfen Sie mit Hilfe der Determinanten, ob die folgenden Matrizen invertierbar sind.

$$
A=\begin{pmatrix} 1 & 1 & 1 \\ 1 & 1 &2 \\ 0&0&1\end{pmatrix}, \quad 
B=\begin{pmatrix} 1 & 2 & 3 \\ 1 & 0 &2 \\ 0&0&1\end{pmatrix} \ \text{ und } \
C=\begin{pmatrix} 0 & 1 & 1 \\ 1 & 0 &1 \\ 1&1&0\end{pmatrix}
$$
%-------------------------
\subsection*{L\"osung}
%%%	
Eine quadratische Matrix ist genau dann invertierbar, wenn ihre Determinante unngleich 0 ist. \\
Addieren des Vielfachen einer Zeile zu einer anderen Zeile ändert die Determinante nicht, dies der Multiplikation mit einer Additionsmatrix entspricht und diese Determinante 1 haben. 
$$
\det(A)=\det(\begin{pmatrix} 1 & 1 & 1 \\ 1 & 1 &2 \\ 0&0&1\end{pmatrix})=\det(\begin{pmatrix} 1 & 1 & 1 \\ 1 & 1 &1 \\ 0&0&1\end{pmatrix})=\det(\begin{pmatrix} 0 & 0 & 0 \\ 1 & 1 &1 \\ 0&0&1\end{pmatrix})=0
$$
Die letzte Gleichheit folgt, da die Matrix eine Null-Zeile besitzt. Somit ist die Matrix $A$ nicht invertierbar.
$$
\det(B)=\det(\begin{pmatrix} 1 & 2 & 3 \\ 1 & 0 &2 \\ 0&0&1\end{pmatrix})=\det(\begin{pmatrix} 1 & 2 & 3 \\ 0 & -2 &-1 \\ 0&0&1\end{pmatrix})=-2 \ne 0
$$
Wir nutzen hier die Formel für die Determinte einer oberen Dreiecksmatrix. Somit ist die Matrix $B$ invertierbar.
$$
\det(C)=\det(\begin{pmatrix} 0 & 1 & 1 \\ 1 & 0 &1 \\ 1&1&0\end{pmatrix})=-1\cdot \det(\begin{pmatrix} 1  &1 \\ 1&0\end{pmatrix})+1\cdot \det(\begin{pmatrix} 1 &0 \\ 1&1\end{pmatrix})=-1\cdot(-1) + 1 \cdot 1=2 \ne 0
$$
Somit ist auch die Matrix $B$ invertierbar.

\newpage
%-------------------------
\section{Determinante}
%%%

Berechnen Sie die Determinante der Matrix
$$
	M=\begin{pmatrix}
		0	& 2	& 4	& -2 & 1	\\
		1	& 0	& 1	& 3	& 0	\\
		1	& 1	& 3	& 2	& 0\\
		0	& 1	& 2	& -1	& -1	\\
		3	& 2	& 7	& 7	& -1	
	\end{pmatrix}.
$$
%-------------------------
\subsection*{L\"osung}
%%%	
Wir benutzen, dass für Additionsmatrizen $\det(A_{ij}(\alpha))=1$ gilt (wenn $i\ne j$).
Damit:
$$
	A_{3,2}(-1)A_{5,2}(-3)M=\begin{pmatrix}
		0	& 2	& 4	& -2 & 1	\\
		1	& 0	& 1	& 3	& 0	\\
		0	& 1	& 2	& -1	& 0\\
		0	& 1	& 2	& -1	& -1	\\
		0	& 2	& 4	& -5	& -1	
	\end{pmatrix}.
$$
Wir entwickeln die Determinante nun nach der ersten Spalte:
\begin{align*}
\det(M)&=\det(A_{3,2}(-1))\cdot\det(A_{5,2}(-3))\cdot\det(M)=\det(A_{3,2}(-1)A_{5,2}(-3)M)\\
&=(-1)^{2+1}\cdot \det(\begin{pmatrix}
			 2	& 4	& -2 & 1	\\
			 1	& 2	& -1	& 0\\
			 1	& 2	& -1	& -1	\\
			 2	& 4	& -5	& -1	
	\end{pmatrix})
\end{align*}
Setzte $\tilde M^{21}:=\begin{pmatrix}
			 2	& 4	& -2 & 1	\\
			 1	& 2	& -1	& 0\\
			 1	& 2	& -1	& -1	\\
			 2	& 4	& -5	& -1	
	\end{pmatrix}$.
Wieder nutzen wir Additionsmatrizen:
$$
A_{3,1}(1)A_{4,1}(1)\tilde M^{21}=\begin{pmatrix}
			 2	& 4	& -2 & 1	\\
			 1	& 2	& -1	& 0\\
			 3	& 6	& -3& 0	\\
			 4	& 8	& -7	& 0	
	\end{pmatrix}
$$
Damit erhalten wir durch Entwickeln nach der 4. Spalte:
\begin{align*}
\det(M)&=-\det(\tilde M^{21})=-\det(A_{3,1}(-1))\cdot\det(A_{4,1}(-3))\cdot\det(\tilde M^{21})\\
&=-\det(A_{3,1}(-1)A_{4,1}(-3)\tilde M^{21})\\
&=(-1)\cdot(-1)^{1+4}\det(\begin{pmatrix}
			 1	& 2	& -1	\\
			 3	& 6	& -3	\\
			 4	& 8	& -7	
	\end{pmatrix})
\end{align*}
Setze $\tilde M^{14}:=\begin{pmatrix}
			 1	& 2	& -1	\\
			 3	& 6	& -3	\\
			 4	& 8	& -7	
	\end{pmatrix}$.
Erneut benutzen wir Additionsmatrizen
$$
A_{3,1}(-4)A_{2,1}(-3)\tilde M^{14}=\begin{pmatrix}
			 1	& 2	& -1	\\
			 0	& 0	& 0	\\
			 0	& 0	& -3	
	\end{pmatrix}
$$
Damit erhalten wir 
\begin{align*}
\det(M)&=\det(\tilde M^{14})=\det(A_{3,1}(-4))\cdot\det(A_{2,1}(-3))\cdot\det(\tilde M^{14})\\
&=\det(A_{3,1}(-4)A_{2,1}(-3)\tilde M^{14})\\
&=0
\end{align*}
Für die letzte Gleichheit nutzen wir, dass sich die Determinante einer oberen Dreiecksmatrix als Produkt ihrer Diagonaleinträge berechnet.

\newpage
%-------------------------
\section{Invertierbarkeit}
%%%

Sei $\alpha \in \R$ und
$$
	A =
	\begin{pmatrix}
		1	&5	&\alpha	\\
		0	&-2	&1	\\
		-1	&1	&3
	\end{pmatrix}
$$

F\"ur welche Werte von $\alpha$ ist $A$ invertierbar?
%-------------------------
\subsection*{L\"osung}
%%%	
$A$ ist genau dann invertierbar, wenn $\det(A)\ne 0$. Daher berechnen wir die Determinante von $A$:
\begin{align*}
\det(A)&=\det(A_{3,1}(1)A)=\det(\begin{pmatrix}
		1	&5	&\alpha	\\
		0	&-2	&1	\\
		0	&6	&3+\alpha
	\end{pmatrix})=1\cdot(-1)^{1+1}\cdot\det(\begin{pmatrix}
		-2	&1	\\
		6	&3+\alpha
	\end{pmatrix})\\
	&=\det(\begin{pmatrix}
		-2	&1	\\
		6	&3+\alpha
	\end{pmatrix})=(-2)\cdot(3+\alpha)-1\cdot 6\\
	&=-6-2\alpha-6\\
	&=-2\alpha-12
\end{align*}
Damit gilt:
\begin{align*}
&&\det(A)&=0\\
\Leftrightarrow\quad&& -2\alpha-12&=0\\
\Leftrightarrow\quad&& \alpha&=-6
\end{align*}
Somit ist $A$ invertierbar für alle $\alpha \in \R\setminus\{-6\}$.

\newpage
%-------------------------
\section{Inverse bestimmen mittels Determinanten-Formel}
%%%

Sei 
$$
	D = 
	\begin{pmatrix}
		1	& x	&z\\
		0	&1&y\\		
		0	&0	&1
	\end{pmatrix}
	\in \R^{3\times 3}.
$$
Bestimmen Sie mit Hilfe der komplementären Matrix $\bar D$ die zu $D$ inverse Matrix.
%-------------------------
\subsection*{L\"osung}
%%%	
Um die komplementäre Matrix $\bar D$ aufzustellen, benötigen wir die Determinanten von den Matrizen $D^{ij}$ für alle $i,j\in \{1,2,3\}$.
\begin{align*}
&D^{11}=\begin{pmatrix}
		1	&y	\\
		0	&1
	\end{pmatrix}
&D^{12}=\begin{pmatrix}
		0	&y	\\
		0	&1
	\end{pmatrix}
&&D^{13}=\begin{pmatrix}
		0	&1	\\
		0	&0
	\end{pmatrix}\\
%
&D^{21}=\begin{pmatrix}
		x	&z	\\
		0	&1
	\end{pmatrix}
&D^{22}=\begin{pmatrix}
		1	&z	\\
		0	&1
	\end{pmatrix}
&&D^{23}=\begin{pmatrix}
		1	&x	\\
		0	&0
	\end{pmatrix}\\
%
&D^{31}=\begin{pmatrix}
		x	&z	\\
		1	&y
	\end{pmatrix}
&D^{32}=\begin{pmatrix}
		1	&z	\\
		0	&y
	\end{pmatrix}
&&D^{33}=\begin{pmatrix}
		1	&x	\\
		0	&1
	\end{pmatrix}
\end{align*}
Damit können wir die komplementäre Matrix aufstellen:
$$
\bar D=\begin{pmatrix}
		(-1)^{1+1}\det(D^{11})	&(-1)^{1+2}\det(D^{21})	&(-1)^{1+3}\det(D^{31}) \\
		(-1)^{2+1}\det(D^{12})	&(-1)^{2+2}\det(D^{22})	&(-1)^{2+3}\det(D^{32}) \\
		(-1)^{3+1}\det(D^{13})	&(-1)^{3+2}\det(D^{23})	&(-1)^{3+3}\det(D^{33}) 
	\end{pmatrix}
	= 
	\begin{pmatrix}
	1 & -x & xy-z \\
	0 & 1 & -y \\
	0 & 0& 1
	\end{pmatrix}
$$
Da $\det(D)=1$ gilt, folgt somit 
$$
D^{-1}=\frac{1}{\det(D)}\cdot \bar D = \bar D= \begin{pmatrix}
	1 & -x & xy-z \\
	0 & 1 & -y \\
	0 & 0& 1
	\end{pmatrix}.
$$
\newpage
%-------------------------
\section{Determinante einer dünn-besetzten Matrix *}
%%%
Berechnen Sie die Determinante der Matrix
$$
	M=\begin{pmatrix}
		0	& 2	&  0& 1	\\
		1	& 0	&  3	& 0	\\
		1	& 0	&  2	& 0\\
		0	& 1	&  -1	& -1	
	\end{pmatrix}.
$$


%-------------------------
\subsection*{L\"osung}
%%%	
Wir beginnen mit einigen Umformungen durch Additionsmatrizen ($\det(A_{i,j}(\alpha))=1$) und Vertauschungsmatrizen ($\det(V_{i,j})=-1$). 
\begin{align*}
V_{2,4}V_{1,2}A_{1,4}(-2)A_{4,3}(-1)A_{3,2}(-1)M&=V_{2,4}V_{1,2}A_{1,4}(-2)A_{4,3}(-1)\begin{pmatrix}
		0	& 2	&  0& 1	\\
		1	& 0	&  3	& 0	\\
		0	& 0	&  -1	& 0\\
		0	& 1	&  -1	& -1	
	\end{pmatrix}\\
	&=V_{2,4}V_{1,2}A_{1,4}(-2)\begin{pmatrix}
		0	& 2	&  0& 1	\\
		1	& 0	&  3	& 0	\\
		0	& 0	&  -1	& 0\\
		0	& 1	&  0	& -1	
	\end{pmatrix}\\
	&=V_{2,4}V_{1,2}\begin{pmatrix}
		0	& 0	&  0& 3	\\
		1	& 0	&  3	& 0	\\
		0	& 0	&  -1	& 0\\
		0	& 1	&  0	& -1	
	\end{pmatrix}\\
	&=V_{2,4}\begin{pmatrix}
		1	& 0	&  3	& 0	\\
		0	& 0	&  0& 3	\\
		0	& 0	&  -1	& 0\\
		0	& 1	&  0	& -1	
	\end{pmatrix}\\
	&=\begin{pmatrix}
		1	& 0	&  3	& 0	\\
		0	& 1	&  0	& -1	\\
		0	& 0	&  -1	& 0\\
		0	& 0	&  0& 3
	\end{pmatrix}
\end{align*}
Insgesamt erhalten wir somit
\begin{align*}
\det(M)&=\frac{1}{-1}\cdot\frac{1}{-1}\cdot\frac{1}{1}\cdot\frac{1}{1}\cdot\frac{1}{1}\cdot\det(V_{2,4}V_{1,2}A_{1,4}(-2)A_{4,3}(-1)A_{3,2}(-1)M)\\
&=\det(\begin{pmatrix}
		1	& 0	&  3	& 0	\\
		0	& 1	&  0	& -1	\\
		0	& 0	&  -1	& 0\\
		0	& 0	&  0& 3
	\end{pmatrix}\\
	&=1\cdot1\cdot(-1)\cdot3\\
	&=-3
\end{align*}
%
%
%
%
%
%
\end{document}
