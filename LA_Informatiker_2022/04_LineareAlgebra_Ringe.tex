\documentclass{beamer}

\usepackage{beamerthemesplit}

\usepackage{amsfonts}
\usepackage{amsmath}
\usepackage{amssymb}
\usepackage{amsthm}
\usepackage{amscd}

\usepackage{stmaryrd} 					%\lightning
\usepackage{algorithm2e}


\usepackage[ngerman]{babel}

\usepackage[utf8]{inputenc}
\usepackage[T1]{fontenc}
\usepackage{textcomp}


% Color Definitions
\definecolor{dhbwRed}{RGB}{226,0,26} 
\definecolor{dhbwGray}{RGB}{61,77,77}
\definecolor{lightBlue}{RGB}{28,134,230}

% Basic Theme
\usetheme{Malmoe}

% Color Re-Definitions
\usecolortheme[named=lightBlue]{structure}
\setbeamercolor*{alerted  text}{fg=dhbwRed, bg=white}
\setbeamercolor*{subsection in toc}{fg=dhbwGray, bg=white}

%\setbeamercolor*{palette primary}{fg=white,bg=lightBlue}
%\setbeamercolor*{palette secondary}{fg=white,bg=gray}
%\setbeamercolor*{palette tertiary}{fg=white,bg=gray}
%\setbeamercolor*{palette quaternary}{fg=white,bg=dhbwRed}

% no navigation symbols
\setbeamertemplate{navigation symbols}{}

% headline, footline
\setbeamertemplate{footline}{\color{dhbwGray} \hfill\insertframenumber\hspace{5mm}\vspace{2mm}}
\setbeamertemplate{headline}{}

% Title Page
\newcommand*{\makeTitlePage}{
	
	\begin{frame}[plain]
		
		\vfill
		\vfill
		\begin{center}
			{
				\usebeamerfont{title}
				\usebeamercolor[fg]{title}
				\Large
				\inserttitle
			}\\[3mm]
			{	
				\usebeamerfont{subtitle}
				\usebeamercolor[fg]{subtitle}
				\large
				\insertsubtitle
			}
		\end{center}
		%
		\vfill
		\vfill
		\vfill
		\vfill
		%
		\begin{columns}
			\begin{column}{0.5\textwidth}
				\begin{flushleft}
					{
						\usebeamerfont{normal text}
						\color{dhbwGray!80}
						\scriptsize
						Dr. Moritz Gruber\\
						DHBW Karlsruhe\\
						
					}
				\end{flushleft}
			\end{column}
			%
			\begin{column}{0.5\textwidth}
				\begin{flushright}
					\includegraphics[scale=0.06]{../DHBW.png}
				\end{flushright}
			\end{column}
		\end{columns}
		%
		\vspace{1mm}
		\begin{columns}
			\begin{column}{0.5\textwidth}
				\begin{flushleft}
					{
						\usebeamerfont{normal text}
						\color{dhbwGray!80}
						\scriptsize
						Version \today
					}
				\end{flushleft}
			\end{column}
			%
			\begin{column}{0.5\textwidth}
				% nothing (just a placeholder to be in line with the columns above
			\end{column}
		\end{columns}
	\end{frame}

}

% Section Divider Page
\newcommand*{\makeSectionDividerPage}{

	\begin{frame}[plain]
		\begin{center}
			\begin{flushleft}
				{				
					\usebeamercolor[fg]{frametitle}
					{\Large \insertsection} \\[3mm]
					{\large \insertsubsection}
				}
			\end{flushleft}
		\end{center}
        \end{frame}
	
}

% itemize
\setbeamertemplate{itemize items}[circle]
\setbeamertemplate{enumerate item}{(\theenumi)}




%--------------------------------------%
% Math ------------------------------%
%--------------------------------------%

% Mengen (Zahlen)
\newcommand{\N}{\mathbb{N}}
\newcommand{\Q}{\mathbb{Q}}
\newcommand{\R}{\mathbb{R}}
\newcommand{\Z}{\mathbb{Z}}
\newcommand{\C}{\mathbb{C}}

% Mengen (allgemein)
\newcommand{\K}{\mathbb{K}}
\newcommand\PX{{\cal P}(X)}

% Zahlentheorie
\newcommand{\ggT}{\mathrm{ggT}}


% Ableitungen
\newcommand{\ddx}{\frac{d}{dx}}
\newcommand{\pddx}{\frac{\partial}{\partial x}}
\newcommand{\pddy}{\frac{\partial}{\partial y}}
\newcommand{\grad}{\text{grad}}

%--------------------------------------%
% Layout Colors ------------------%
%--------------------------------------%
\newcommand*{\highlightDef}[1]{{\color{lightBlue}#1}}
\newcommand*{\highlight}[1]{{\color{lightBlue}#1}} % after theme for colours


%----------------------------------------------------------------------------------------------------
%--------- Document Title ---------------------------------------------------------------------
\title{Lineare Algebra\\[3mm] 
	\large Ringe
}
\author{Dr. Moritz Gruber } 
\institute{DHBW Karlsruhe}
\date{2022}
%%%%%%%%%%%%%%
\begin{document}

\AtBeginSection[]{
	\begin{frame}				
		\usebeamercolor[fg]{frametitle}
		{\Large \insertsection} 
        \end{frame}
}

%
\begin{frame}[plain] 
 \titlepage
\end{frame}
%
%
\begin{frame}\frametitle{Inhalt}
   \tableofcontents
\end{frame}
%
%%%
\section{Motivation}
%%%
%
\begin{frame}\frametitle{Motivation}
	
	Bisher:
	\begin{itemize}
		\item Verkn\"upfungen auf Mengen
		\item Gruppe $(G,\ast)$: Menge mit einer Verkn\"upfung mit gewissen Eigenschaften 
	\end{itemize}
	\vfill
	Beispiele f\"ur (abelsche) Gruppen:
	\begin{itemize}
		\item $(\R, +)$, $(\Q,+)$, $(\Z,+)$
		\item $(\R\backslash\{0\},\cdot)$, $(\Q\backslash\{0\},\cdot)$ 
	\end{itemize}
	
	
\end{frame}
%
%
\begin{frame}\frametitle{Motivation}
	
	Auf den Mengen $\R, \Q, \Z$ sind zwei Verkn\"upfungen $+$ und $\cdot$ definiert, f\"ur die \highlightDef{ Distributivgesetze} gelten:\\\vfill
	$\forall a,b,c: $
	$$
		a\cdot (b+c) = a\cdot b + a\cdot c
	$$
	und
	$$
		 (a+b)\cdot c = a\cdot c + b\cdot c.
	$$
	
\end{frame}
%
%%%
\section{Ringe}
%%%
%
\begin{frame}\frametitle{Definition Ring}
	
	Seien $R$ eine Menge und $+, \cdot$ zwei Verkn\"upfungen auf $R$. 
	Dann hei{\ss}t $(R,+,\cdot)$ ein \highlightDef{ Ring}, wenn gilt:
	\begin{enumerate}
		\item $(R,+)$ ist eine abelsche Gruppe.
		\item Die Verkn\"upfung $\cdot$ ist assoziativ.
		\item Die Distributivgesetze gelten, d.h. $\forall a,b,c \in R$:
			$$
				a\cdot (b+c) = a\cdot b + a\cdot c
			$$
			und
			$$
		 		(a+b)\cdot c = a\cdot c + b\cdot c.
			$$
	\end{enumerate}
	
\end{frame}
%
%
\begin{frame}\frametitle{Definition Ring: Beispiele}
	
	\begin{itemize}
		\item $(\R,+,\cdot),\, (\Q,+,\cdot),\, (\Z,+,\cdot),$ \pause
		\item $(\R^{2\times 2}, +, \cdot)$ bzgl. Matrixaddition 
			$$
				\begin{pmatrix}
					a_{11}	&	a_{12}	\\
					a_{21}	&	a_{22}
				\end{pmatrix}
				+
				\begin{pmatrix}
					b_{11}	&	b_{12}	\\
					b_{21}	&	b_{22}
				\end{pmatrix}
				:=
				\begin{pmatrix}
					a_{11} + b_{11}	&	a_{12} + b_{12}	\\
					a_{21} +b_{21}	&	a_{22} + b_{22}
				\end{pmatrix}
			$$
			und Matrixmultiplikation.
	\end{itemize}
	
\end{frame}
%
\subsection{Definition}
%
\begin{frame}\frametitle{Definition Ring}
	
	\begin{itemize}
		\item Das neutrale Element bzgl. $+$ bezeichnet man mit $0$ (\highlightDef{ Nullelement}). \pause
		\item Ist die Verkn\"upfung $\cdot$ kommutativ, so nennt man den Ring \highlightDef{ kommutativ}. \pause
		\item Gibt es bzgl. der Verkn\"upfung $\cdot$ ein neutrales Element, so schreibt man hierf\"ur $1$ und nennt es \highlightDef{ Einselement} oder kurz \highlightDef{Eins}.\\
			$(R,+,\cdot)$ hei{\ss}t dann \highlightDef{ Ring mit Eins}.\pause
		\item Ist $a\in R$, so bezeichnen wir das zu $a$ inverse Element bzgl. der Verkn\"upfung $+$ mit $-a$. Es gilt also:
			$$
				(-a) + a = a + (-a) = 0.
			$$
		Bemerkung: F\"ur $a + (-b)$ schreibt man oft kurz $a-b$. 
	\end{itemize}
	
\end{frame}
%
%
\begin{frame}\frametitle{Eigenschaften von Ringen} 
		
	Seien $(R,+,\cdot)$ ein Ring, $a,b \in R$ und $0\in R$ das Nullelement.
	\begin{itemize}
		\item Was ergibt $0\cdot a$ ?
		\item Stimmen $(-a)\cdot b$ und $-(a\cdot b)$ \"uberein?\\
			(Ist das bzgl. $+$ inverse Element von $a\cdot b$ das Element $(-a)\cdot b$?)
	\end{itemize} \pause
	
	\vspace{2mm}
	Die Antworten hierzu stehen nicht {\em direkt} in der Definition.
		
\end{frame}
%
%
\begin{frame}\frametitle{Eigenschaften von Ringen} 
	
	Seien $R$ ein Ring, $a,b \in R$ und $0$ das Nullelement von $R$. 
	Dann gilt:
	\begin{itemize}
		\item[(1)] $0\cdot a = a\cdot 0 = 0$.
		\item[(2)] $(-a)\cdot b = -(a\cdot b)$.
	\end{itemize}
		
\end{frame}
%
%
\begin{frame}\frametitle{Eigenschaften von Ringen} 
	
	\begin{itemize}
		\item[(1)] \highlightDef{ Beweis  $0\cdot a = a\cdot 0 = 0$:}
			$$
			\begin{array}{llll}
				0\cdot a	&=\pause	&(0+0)\cdot a		&\text{($0$ ist das neutrale Element bzgl. $+$)}	\\ \pause
						&=	&0\cdot a + 0\cdot a	&\text{(Distributivgesetz)} 			\pause		
			\end{array}
			$$
			Wenn wir auf beiden Seiten $-(0\cdot a)$ addieren ($(R,+)$ ist eine Gruppe), erhalten wir
			$$
				0 = 0\cdot a.
			$$\pause
			Analog zeigt man $a\cdot 0 = 0$.
	\end{itemize}	
		
\end{frame}
%
%
\begin{frame}\frametitle{Eigenschaften von Ringen} 
	
	\begin{itemize}
		\item[(2)]  \highlightDef{ Beweis $(-a)\cdot b = -(a\cdot b)$:}\\
		Aus dem Distributivgesetz folgt:
			$$
				(-a)\cdot b + a\cdot b = ( -a + a)\cdot b
			$$
			\pause
			Da $-a$ das zu $a$ inverse Element in der Gruppe $(R,+)$ ist, folgt $-a + a = 0$ und somit
			$$
				(-a)\cdot b + a\cdot b= ( -a + a)\cdot b = 0\cdot b = 0.
			$$
			\pause
			$(-a)\cdot b$ ist somit das zu $a\cdot b$ inverse Element in der abelschen Gruppe $(R,+)$, d.h.
			$(-a)\cdot b = -(a\cdot b)$.
	\end{itemize}	
	\hfill$\qed$
	
\end{frame}
%
\subsection{Der Polynomring}
%
\begin{frame}\frametitle{Beispiel Polynome}
	
	Sei $(R,+,\cdot)$ ein Ring. Ein \highlightDef{ Polynom} $f$ in der Variable $X$ mit Koeffizienten in $R$ ist ein Ausdruck der Form
	$$
		f = f(X) = a_nX^n + a_{n-1}X^{n-1} + \ldots + a_1X + a_0 
	$$
	mit $a_n,\ldots, a_0 \in R$ und $a_n\neq 0$. \\\pause
	Dabei hei{\ss}t $\highlightDef{\deg(f)} := n$ der \highlightDef{Grad}\footnote{In der nächsten Vorlesung werden wir eine mathematisch genauere Definition kennenlernen.} von $f$.
	Die Menge aller Polynome mit Koeffizienten in $R$ wird mit \highlightDef{$R[X]$} bezeichnet.
	\pause
	\vfill
	\highlightDef{ Beispiel:}
	$$
		f(X) = 3X^4 + \frac{1}{2}X^3 - 12X + 17 
	$$
	ist ein Polynom in $\R[X]$ mit Grad $4$. Es ist aber auch ein Polynom in $\Q[X]$. 
	
\end{frame}
%
%
\begin{frame}\frametitle{Beispiel Polynome: Addition}
	
	Seien $(R,+,\cdot)$ ein Ring und  
	$$
		f = f(X) = a_nX^n + a_{n-1}X^{n-1} + \ldots + a_1X + a_0,
	$$
	$$
		g = g(X) = b_mX^m + b_{m-1}X^{m-1} + \ldots + b_1X + b_0
	$$
	zwei Polynome mit $m\leq n$.\\[1mm] 
	Dann definieren wir eine Addition $f+g$:\pause
	\begin{align*}
		f + g &= (f+g)(X) :=\sum_{j=m+1}^n a_j X^j + \sum_{k=0}^m (a_k+b_k)X^k\\
		&= a_nX^n +\ldots + a_{m+1}X^{m+1} \\
		& \hspace{20mm}+ (a_m+b_m) X^m +\ldots+ (a_1+b_1)X +(a_0+b_0)  
	\end{align*} 
	\vfill \pause
	$f+g$ ist somit ein Polynom mit $\deg(f+g) \le \max( \deg(f), \deg(g))$.
\end{frame}
%
%
\begin{frame}\frametitle{Beispiel Polynome: Multiplikation}
	
	Seien $(R,+,\cdot)$ ein Ring und  
	$$
		f = f(X) = a_nX^n + a_{n-1}X^{n-1} + \ldots + a_1X + a_0,
	$$
	$$
		g = g(X) = b_mX^m + b_{m-1}X^{m-1} + \ldots + b_1X + b_0
	$$
	zwei Polynome mit $m\leq n$.\\[1mm]
	Dann definieren wir eine Multiplikation $f \cdot g$: \pause
	$$
		\begin{array}{lccl}
			(f \cdot g)(X) 	&:=	&& 	(a_0 \cdot b_0) + (a_0\cdot b_1+ a_1\cdot b_0)X +\ldots \\
						&	&&+	 (a_n\cdot b_0 + a_{n-1}\cdot b_1 + \ldots + a_{n-m}\cdot b_m)X^n  + \ldots\\
						&	&&+	a_n\cdot b_m X^{n+m}\\ \pause
						&&&\\
						&=&& \sum_{k=0}^{m+n} \left( \sum_{j=0}^k a_j b_{k-j}  \right) X^k
		\end{array}
	$$ 
\end{frame}
%
%
\begin{frame}\frametitle{Beispiel Polynome: Multiplikation}
	
\highlightDef{Beispiel in $\Z[X]$}:\\
	$
		f = X^2 + 2X + 5, g = 2X^3 + X 
	$\\\pause \vfill
Dann gilt\\
\vspace{2mm}
$
f+g=\pause (X^2 + 2X + 5) + (2X^3 + X)\pause = 2X^3+X^2+3X+5
$\\
\vspace{2mm}
und\\
\vspace{2mm}
$
f\cdot g=\pause (X^2 + 2X + 5) \cdot (2X^3 + X) \pause $\\
\hspace{8mm}$=2 X^3X^2 + 2\cdot2 X^3X+(5\cdot2X^3 + X^2X) + (2X^2+2XX)+5X$\\\pause
$\hspace{8mm}=2X^5+4X^4+11X^3+4X^2+5X
$
\end{frame}
%
%
\begin{frame}\frametitle{Grad-Formel}
	$f\cdot g$ ist somit ein Polynom mit 
	$$
		\deg(f\cdot g) \le n + m = \deg(f) + \deg(g).
	$$
	\vfill \pause
	Warum $=$ statt $\le$ im Allgemeinen nicht stimmen muss, werden wir uns später noch ansehen.
\end{frame}
%
%
\begin{frame}\frametitle{Beispiel Polynome}
	
	Seien $(R,+,\cdot)$ ein Ring und $R[X]$ die Menge aller Polynome mit Koeffizienten in $R$. Dann ist
	$$
		(R[X],+,\cdot)
	$$
	ein Ring.\\[2mm]
	\pause
	Ist $R$ ein kommutativer Ring mit Einselement, dann ist auch $(R[X],+,\cdot)$ ein kommutativer Ring mit Einselement.\\[2mm]

	Beweis: \"Ubungsaufgabe.
\end{frame}
%
%%
%\begin{frame}\frametitle{Polynome: Teilbarkeit}
%	
%	\begin{itemize}
%		\item \highlightDef{ Teilbarkeitsrelation} auf der Menge der Polynome $R[X]$:
%			$$
%				p \teXt{~teilt~} f	:\iff	\eXists q \in R[X]: f(X) = p(X) \cdot q(X). 
%			$$
%			\pause
%			Beispiel in $\R[X]$:
%			$$
%				f(X) = X^2 - 4 = (X+2)\cdot(X-2)
%			$$
%			Aber auch
%			$$
%				X^2 - 4 = \left(\frac{1}{2}X + 1\right) \cdot (2X-4).
%			$$
%			D.h., 
%			$$
%				p_1(X) = X+2,\, 
%				p_2(X) = X-2,\, 
%				p_3(X) =  \frac{1}{2}X + 1,\, 
%				p_4(X) = 2X-4
%			$$ 
%			sind Teiler von $f$.
%	\end{itemize}
%	
%\end{frame}
%%
%
%\begin{frame}\frametitle{Polynome: Teilbarkeit}
%	
%	\begin{itemize}
%		\item \highlightDef{ Division mit Rest (Polynomdivision) / Modulare Arithmetik:}\\
%			Seien $p(X), q(X) \in R[X]$. Dann gibt es $s(X), r(X) \in R[X]$ mit $\deg(r) < \deg(q)$ und
%			$$
%				p(X) = s(X)\cdot q(X) + r(X).
%			$$
%			(Beweis durch vollst\"andige Induktion nach $\deg(p)$).
%		 \pause
%		 \item $\ggT(p,q)$ zweier Polynome $p,q \in R[X]$:\\ 
%		 	Teiler mit gr\"o{\ss}tem Grad und h\"ochster Koeffizient $a_n = 1$.
%		\pause
%		\item Berechnung des  $\ggT(p,q)$ mit dem euklidischen Algorithmus.
%	\end{itemize}
%	
%\end{frame}
%%
%
\begin{frame}\frametitle{Polynomringe: Anwendungen}
	
	\begin{itemize}
		\item Polynomringe werden in der \highlightDef{ Codierungstheorie} verwendet - die Theorie der fehlererkennenden und -korrigierenden Codes. 
		\item Ziel der Codierungstheorie:\\ 
			Daten fehlerfrei \"ubertragen/speichern (Kommunikation, Speichern von Daten auf einer CD).
		\item \"Ubertragungsfehler erkennen und korrigieren.
		\item Begr\"under der Codierungstheorie:\\ 
			Marcel Golay  (1902 - 1989,``Notes on digital coding'' (1949)), \\
			Richard Hamming (1915 - 1998, ``Error detecting and error correcting codes'' (1950)).
	\end{itemize}
	
\end{frame}
%
\subsection{Einheiten}
%
\begin{frame}\frametitle{Multiplikative Inverse}
Die Definition eines Rings mit Eins fordert \underline{nicht} die Existenz von multiplikativen Inversen. Trotzdem oder auch gerade deshalb sind die Elemente eines Rings für die multiplikative Inverse existieren von besonderem Interesse.
\end{frame}
%
%
\begin{frame}\frametitle{Definition}
\begin{itemize}
\item Es sei $R$ ein Ring mit Eins. Ein Element $x \in R$ heißt eine \highlightDef{Einheit in $R$}, wenn es ein $y \in R$ gibt mit $x\cdot y = 1_R$. \pause
\item Die \highlightDef{Einheitengruppe} $R^\times$ ist die Menge aller Einheiten von $R$. Zusammen mit der Multiplikation auf $R$ bildet $(R^\times, \cdot)$ eine Gruppe: \pause
	\begin{itemize}
	\item $1_R$ ist eine Einheit und bildet das neutrale Element\pause
	\item Für alle $x_1,x_2 \in R^\times$ gilt: 
	$$(x_1x_2)\cdot(x_2^{-1}x_1^{-1})=x_1\cdot(x_2x_2^{-1})\cdot x_1^{-1}=x_1\cdot 1_R \cdot x_1^{-1}=1_R$$
	Und somit auch $x_1x_2 \in R^\times$.\pause
	\item Außerdem ist die Multiplikation assoziativ (Definition Ring) und mit $x\in R^\times$ ist auch $x^{-1} \in R^\times$.
	\end{itemize}
\end{itemize}


\end{frame}
%
%
\begin{frame}\frametitle{Beispiele}
\begin{itemize}
\item Für den Ring $(\Z,+,\cdot)$ ist $\Z^\times = \{1,-1\}$.\pause \vfill
\item Für den Ring $(\R,+,\cdot)$ ist $\R^\times = \R \setminus \{0\}$.\pause \vfill
\item Für den Polynomring $\R[X]$ ist $\R[X]^\times = \R^\times$.\\ (Mehr dazu später)
\end{itemize}
\end{frame}
%
%%
\subsection{Nullteiler}
%
\begin{frame}\frametitle{Definition}
Es sei $R$ ein Ring. Ein Element $x\in R\setminus \{0\}$ heißt \highlightDef{Nullteiler}, wenn es ein Element $y \in R\setminus \{0\}$ gibt, sodass $x \cdot y = 0$ oder $y \cdot x = 0$.\\
Ein Ring in dem es keine Nullteiler gibt, heißt \highlightDef{nullteilerfrei}.\\
\vfill \pause
\textbf{Bemerkung:} Ein Nullteiler $x \in R\setminus \{0\}$ kann natürlich niemals eine Einheit sein, da sich sonst durch Multiplikation mit $x^{-1}$ folgern lässt, dass $y=0$ gilt. Dies wäre aber ein Widerspruch!
\end{frame}
%
\begin{frame}\frametitle{Beispiele}
\begin{itemize}
\item $\Z,\Q$ und $\R$ mit $+$ und $\cdot$ sind alle nullteilerfrei. \pause \vfill
\item Der Ring der reellen $2\times 2$-Matrizen ist \underline{nicht} nullteilerfrei: \pause
$$
\begin{pmatrix} 0 & 1 \\ 0 &0\end{pmatrix} \cdot \begin{pmatrix} 0 & 1 \\ 0 &0\end{pmatrix} = \begin{pmatrix} 0 & 0 \\ 0 &0\end{pmatrix}
$$
Somit ist die Matrix $A=\begin{pmatrix} 0&1\\0&0 \end{pmatrix}$ ein Nullteiler.
\end{itemize}
\end{frame}
%
\begin{frame}\frametitle{Nochmal Grad-Formel}
Es seien $f,g \in R[X]$. Dann gilt $\deg(f\cdot g) \le \deg(f) +\deg(g)$.\\ \vfill \pause
Beispiel für $<$:\\
$R=\R^{2\times 2}$ und \pause
$$f=g=\begin{pmatrix} 0 & 1 \\ 0 &0\end{pmatrix} \cdot X + \begin{pmatrix} 1 & 0 \\ 0 &1\end{pmatrix}$$ \pause
Dann gilt $\deg(f)=\deg(g)=1$ aber \pause
\begin{align*}
f \cdot g &= (\begin{pmatrix} 0 & 1 \\ 0 &0\end{pmatrix} \cdot \begin{pmatrix} 0 & 1 \\ 0 &0\end{pmatrix}) \cdot X^2 +2 \cdot \begin{pmatrix} 0 & 1 \\ 0 &0\end{pmatrix} X + \begin{pmatrix} 1 & 0 \\ 0 &1\end{pmatrix} \\ 
&=0_{\R^{2\times2}} \cdot X^2 + \begin{pmatrix} 0 & 2 \\ 0 &0\end{pmatrix} X + \begin{pmatrix} 1 & 0 \\ 0 &1\end{pmatrix}\\
&=\begin{pmatrix} 0 & 2 \\ 0 &0\end{pmatrix} X + \begin{pmatrix} 1 & 0 \\ 0 &1\end{pmatrix}
\end{align*}\pause
und somit $\deg(f\cdot g)=1 < 2 = \deg(f) +\deg(g)$.
\end{frame}
%
\begin{frame}\frametitle{Satz}
Wenn $R$ ein nullteilerfreier Ring ist, dann 
%ist auch der Polynomring $R[X]$ nullteilerfrei und es 
gilt Gleichheit in der Grad-Formel:
$$
\deg(f\cdot g) = \deg(f) +\deg(g) \quad \forall f,g \in \R[X] \setminus \{0\}
$$\pause
\vfill
\highlightDef{Beweis:}\\
Es seien $f,g \in R[X]\setminus \{0\}$ und $n=deg(f), m=deg(g)$. Dann gibt es Koeffizienten $a_i,b_j \in R$ sodass
$$
f=\sum_{i=0}^n a_iX^i , \ g=\sum_{j=0}^m b_jX^j \quad \text{mit } a_n\ne0\ne b_m
$$\pause
Es gilt dann
	$$
	(f \cdot g)(X) :=	(a_0  b_0) + (a_0 b_1+ a_1 b_0)X +\ldots +a_n b_m X^{n+m}
	$$ \pause
Da $R$ nullteilerfrei ist gilt $a_nb_m \ne 0$ und damit $deg(f\cdot g)=n+m$.\\
\hfill $\square$
\end{frame}
%
%\begin{frame}\frametitle{Satz}
%Wenn $R$ ein nullteilerfreier Ring ist, dann ist auch der Polynomring $R[X]$ nullteilerfrei und es gilt Gleichheit in der Grad-Formel:
%$$
%\deg(f\cdot g) = \deg(f) +\deg(g) \quad \forall f,g \in \R[X]
%$$
%\vfill
%\highlightDef{Beweis (Fortsetzung):}\\
%Es seien nun $f,g \in R[X]$ und $f\cdot g=0$. \pause Es folgt dann mit der Grad-Formel:\\
%\vspace{2mm}
%$
%0=deg(0)\pause=deg(f\cdot g)\pause =deg(f) + deg(g)
%$\\
%\vspace{2mm}\pause
%Da nur Elemente von $R$ Grad null besitzen folgt $f,g \in R$. Und da $R$ nach Voraussetzung nullteilerfrei ist, muss entweder $f$ oder $g$ schon $0$ sein. Damit ist auch $R[X]$ nullteilerfrei. \hfill $\square$
%\end{frame}
%%%
%

\end{document}