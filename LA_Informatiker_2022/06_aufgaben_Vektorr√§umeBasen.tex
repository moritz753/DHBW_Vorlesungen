\documentclass[
				a4paper,
				10pt
			]
			{scrartcl}

\parindent0mm

\usepackage{amsfonts}
\usepackage{amsmath}
\usepackage{amssymb}
\usepackage{amsthm}
\usepackage[ngerman]{babel}

\usepackage[utf8]{inputenc}
\usepackage[T1]{fontenc}
\usepackage{textcomp}

\usepackage{graphicx}
\usepackage{xcolor}

\usepackage[
			pdftex,
			colorlinks,
			breaklinks,
			linkcolor=blue,
			citecolor=blue,
			filecolor=black,
			menucolor=black,
			urlcolor=black,
			pdfauthor={Andreas Weber},
			pdftitle={Aufgaben zu Analysis und Lineare Algebra},
			plainpages=false,
			pdfpagelabels,
			bookmarksnumbered=true
		   ]{hyperref}


%--------------------------------------%
% Math ------------------------------%
%--------------------------------------%

% Mengen (Zahlen)
\newcommand{\N}{\mathbb{N}}
\newcommand{\Q}{\mathbb{Q}}
\newcommand{\R}{\mathbb{R}}
\newcommand{\Z}{\mathbb{Z}}
\newcommand{\C}{\mathbb{C}}

% Mengen (allgemein)
\newcommand{\K}{\mathbb{K}}
\newcommand\PX{{\cal P}(X)}

% Zahlentheorie
\newcommand{\ggT}{\mathrm{ggT}}


% Ableitungen
\newcommand{\ddx}{\frac{d}{dx}}
\newcommand{\pddx}{\frac{\partial}{\partial x}}
\newcommand{\pddy}{\frac{\partial}{\partial y}}
\newcommand{\grad}{\text{grad}}

%--------------------------------------%
% Layout Colors ------------------%
%--------------------------------------%
\newcommand*{\highlightDef}[1]{{\color{lightBlue}#1}}
\newcommand*{\highlight}[1]{{\color{lightBlue}#1}}
% Color Definitions
\definecolor{dhbwRed}{RGB}{226,0,26} 
\definecolor{dhbwGray}{RGB}{61,77,77}
\definecolor{lightBlue}{RGB}{28,134,230}


%-------------------------------------------------------------------
\begin{document}

\vspace*{-20mm}
{
	%\usekomafont{title}
	\color{dhbwGray}
	Dr. Moritz Gruber	\hfill Version \today\\
	DHBW Karlsruhe\\

}

\vspace{10mm}
\begin{center}
	{
		\usekomafont{title}
		\color{lightBlue}
		{ \LARGE 	\"Ubungsaufgaben 6}\\[3mm]
		{\Large Vektorr\"aume}
	}
\end{center}

\vspace{5mm}

%-------------------------------------------------------------------
\section{Untervektorräume von $\R^n$}
Es sei $\R^n=\{\begin{pmatrix} v_1 \\ v_2 \\ ... \\ v_n \end{pmatrix} \mid v_1,...,v_n \in \R \}$ mit der Addition
$$
+: \R^n \times \R^n \to \R^n, (v,w) \mapsto v+w \text{ definiert durch } \begin{pmatrix} v_1 \\ v_2 \\ ... \\ v_n \end{pmatrix}+\begin{pmatrix} w_1 \\ w_2 \\ ... \\ w_n \end{pmatrix}:=\begin{pmatrix} v_1+w_1 \\ v_2+w_2 \\ ... \\ v_n+w_n \end{pmatrix}
$$
und der Skalarmultiplikation
$$
\cdot : \R \times \R^n \to \R^n, (a,v) \mapsto a\cdot v \text{ definiert durch } a \cdot \begin{pmatrix} v_1 \\ v_2 \\ ... \\ v_n \end{pmatrix}:=\begin{pmatrix} a v_1 \\ a v_2 \\ ... \\ a v_n \end{pmatrix}
$$
wie in der Vorlesung. Dann ist $\R^n$ ein $n$-dimensionaler reeller Vektorraum.
\quad\\
\begin{itemize}
\item[a)] Zeigen Sie, dass für jedes $m \in \N, m\le n$ die Mengen
$$
U(m):= \{\begin{pmatrix} v_1 \\ v_2 \\ ... \\ v_n \end{pmatrix} \in \R^n \mid v_1,...,v_m =0 \} \ \text{ und } \ W(m) := \{\begin{pmatrix} v_1 \\ v_2 \\ ... \\ v_n \end{pmatrix} \in \R^n \mid v_{m+1},...,v_n =0 \}
$$ 
Untervektorräume von $\R^n$ sind.

\item[b)] Zeigen Sie, dass für $\forall m \in \N, m\le n$ gilt: $\langle U(m)\cup W(m) \rangle = \R^n$.

\item[d)] Finden Sie einen Untervektorraum $U \le \R^n$, sodass gilt
$$
\forall u=\begin{pmatrix} u_1 \\ u_2 \\ ... \\ u_n \end{pmatrix} \in U: ( \exists j \in \{1,...,n\}: u_j=0 \ \Rightarrow \ u=\begin{pmatrix} 0 \\ 0 \\ ... \\ 0 \end{pmatrix} )
$$
\end{itemize}


%-------------------------------------------------------------------
\section{Polynome als Vektorraum *}
%%%

Sei
$$
	V := \{a_0 +a_1X+\ldots +a_nX^n~|~ a_0,\ldots, a_n\in \R \} \subseteq \R[X]
$$
die Menge der Polynome mit Koeffezienten in $\R$ und maximalem Grad $n$.\\

Seien weiter die Verkn\"upfung $+$ auf $V$ und die Abbildung $\cdot: \R\times V \to V$ definiert:
$$
	(a_0 +\ldots +a_nX^n) + (b_0 +\ldots +b_nX^n) := a_0 +b_0+\ldots +(a_n+b_n)X^n
$$
$$
	a\cdot(a_0 +\ldots +a_nX^n) := aa_0 +\ldots +aa_nX^n.
$$

\begin{itemize}
	\item[(a)] Zeigen Sie, dass $(V,+,\cdot)$  ein $\R$-Vektorraum ist.
	\item[(b)] Zeigen Sie, dass die Vektoren $1, X, \ldots, X^n \in V$ linear unabh\"angig sind.
\end{itemize}

%-------------------------------------------------------------------
\section{Lineare Unabh\"angigkeit}
%%%

Seien $V = \R^3$ und
$$
	e_1 :=
	\begin{pmatrix}
		1\\
		0\\
		0	
	\end{pmatrix},
	\quad
	e_2 :=
	\begin{pmatrix}
		0\\
		1\\
		0	
	\end{pmatrix},
	\quad
	e_3 :=
	\begin{pmatrix}
		0\\
		0\\
		1	
	\end{pmatrix}
	\quad
	\in V.
$$
\begin{itemize}
	\item[a)] Zeigen Sie, dass die Vektoren $e_1, e_2, e_3 \in V$ linear unabh\"angig sind.
	\item[b)] Zeigen Sie, dass f\"ur jeden Vektor $v\in V$, die vier Vektoren $e_1, e_2, e_3, v \in V$ linear abh\"angig sind.
	\item[c)] Finden Sie drei Vektoren $b_1,b_2,b_3 \in V$ mit $\{b_1,b_2,b_3\}\cap\{e_1,e_2,e_3\} = \emptyset$, für die die Aussagen aus a) und b) auch wahr sind.
\end{itemize}

%
%-----------------
\section{Die komplexen Zahlen als $\R$-Vektorraum - Teil 1}
Es sei $\C=\{a+bi \mid a,b \in \R\}$ die Menge der komplexen Zahlen. Es sei weiter \\ $+: \C \times \C \to \C$ die Addition von komplexen Zahlen wie in der Vorlesung und \\$\cdot: \R \times \C \to \C, (x,a+bi) \mapsto xa+xbi$.
\begin{itemize}
\item[a)] Zeigen Sie, dass $\C$ mit der Addition und $\cdot$ wie oben als Skalarmultiplikation ein reeller Vektorraum ist.
\item[b)] Finden Sie eine Basis des $\R$-Vektorraums $\C$ und bestimmen Sie dessen \\Dimension.
\item[c)] Betrachten Sie die komplexen Zahlen $z_1=5, z_2=i$ und $z_3=10+2i$ als Vektoren und geben Sie deren Koordinaten bezüglich der Basis aus b) an.

\end{itemize}
\end{document}
