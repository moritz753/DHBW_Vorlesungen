\documentclass{beamer}

\usepackage{beamerthemesplit}

\usepackage{amsfonts}
\usepackage{amsmath}
\usepackage{amssymb}
\usepackage{amsthm}
\usepackage{amscd}

\usepackage{stmaryrd} 					%\lightning
\usepackage{algorithm2e}


\usepackage[ngerman]{babel}

\usepackage[utf8]{inputenc}
\usepackage[T1]{fontenc}
\usepackage{textcomp}


% Color Definitions
\definecolor{dhbwRed}{RGB}{226,0,26} 
\definecolor{dhbwGray}{RGB}{61,77,77}
\definecolor{lightBlue}{RGB}{28,134,230}

% Basic Theme
\usetheme{Malmoe}

% Color Re-Definitions
\usecolortheme[named=lightBlue]{structure}
\setbeamercolor*{alerted  text}{fg=dhbwRed, bg=white}
\setbeamercolor*{subsection in toc}{fg=dhbwGray, bg=white}

%\setbeamercolor*{palette primary}{fg=white,bg=lightBlue}
%\setbeamercolor*{palette secondary}{fg=white,bg=gray}
%\setbeamercolor*{palette tertiary}{fg=white,bg=gray}
%\setbeamercolor*{palette quaternary}{fg=white,bg=dhbwRed}

% no navigation symbols
\setbeamertemplate{navigation symbols}{}

% headline, footline
\setbeamertemplate{footline}{\color{dhbwGray} \hfill\insertframenumber\hspace{5mm}\vspace{2mm}}
\setbeamertemplate{headline}{}

% Title Page
\newcommand*{\makeTitlePage}{
	
	\begin{frame}[plain]
		
		\vfill
		\vfill
		\begin{center}
			{
				\usebeamerfont{title}
				\usebeamercolor[fg]{title}
				\Large
				\inserttitle
			}\\[3mm]
			{	
				\usebeamerfont{subtitle}
				\usebeamercolor[fg]{subtitle}
				\large
				\insertsubtitle
			}
		\end{center}
		%
		\vfill
		\vfill
		\vfill
		\vfill
		%
		\begin{columns}
			\begin{column}{0.5\textwidth}
				\begin{flushleft}
					{
						\usebeamerfont{normal text}
						\color{dhbwGray!80}
						\scriptsize
						Dr. Moritz Gruber\\
						DHBW Karlsruhe\\
						
					}
				\end{flushleft}
			\end{column}
			%
			\begin{column}{0.5\textwidth}
				\begin{flushright}
					\includegraphics[scale=0.06]{../DHBW.png}
				\end{flushright}
			\end{column}
		\end{columns}
		%
		\vspace{1mm}
		\begin{columns}
			\begin{column}{0.5\textwidth}
				\begin{flushleft}
					{
						\usebeamerfont{normal text}
						\color{dhbwGray!80}
						\scriptsize
						Version \today
					}
				\end{flushleft}
			\end{column}
			%
			\begin{column}{0.5\textwidth}
				% nothing (just a placeholder to be in line with the columns above
			\end{column}
		\end{columns}
	\end{frame}

}

% Section Divider Page
\newcommand*{\makeSectionDividerPage}{

	\begin{frame}[plain]
		\begin{center}
			\begin{flushleft}
				{				
					\usebeamercolor[fg]{frametitle}
					{\Large \insertsection} \\[3mm]
					{\large \insertsubsection}
				}
			\end{flushleft}
		\end{center}
        \end{frame}
	
}

% itemize
\setbeamertemplate{itemize items}[circle]
\setbeamertemplate{enumerate item}{(\theenumi)}




%--------------------------------------%
% Math ------------------------------%
%--------------------------------------%

% Mengen (Zahlen)
\newcommand{\N}{\mathbb{N}}
\newcommand{\Q}{\mathbb{Q}}
\newcommand{\R}{\mathbb{R}}
\newcommand{\Z}{\mathbb{Z}}
\newcommand{\C}{\mathbb{C}}

% Mengen (allgemein)
\newcommand{\K}{\mathbb{K}}
\newcommand\PX{{\cal P}(X)}

% Zahlentheorie
\newcommand{\ggT}{\mathrm{ggT}}


% Ableitungen
\newcommand{\ddx}{\frac{d}{dx}}
\newcommand{\pddx}{\frac{\partial}{\partial x}}
\newcommand{\pddy}{\frac{\partial}{\partial y}}
\newcommand{\grad}{\text{grad}}

%--------------------------------------%
% Layout Colors ------------------%
%--------------------------------------%
\newcommand*{\highlightDef}[1]{{\color{lightBlue}#1}}
\newcommand*{\highlight}[1]{{\color{lightBlue}#1}} % after theme for colours

%----------------------------------------------------------------------------------------------------
%--------- Document Title ---------------------------------------------------------------------
\title{Lineare Algebra\\[3mm] 
	\large Verknüpfungsaufgabe: Matrizenmultiplikation
}
\author{Dr. Moritz Gruber} 
\institute{DHBW Karlsruhe}
\date{2022}
%%%%%%%%%%%%%%
\begin{document}

\AtBeginSection[]{
	\begin{frame}				
		\usebeamercolor[fg]{frametitle}
		{\Large \insertsection} 
        \end{frame}
}

%
\begin{frame}[plain] 
 \titlepage
\end{frame}
%
%
%%%
\begin{frame} \frametitle{Aufgabe: Matrixmultiplikation von reellen $(2\times 2)$-Matrizen}

\begin{itemize}
	\item[(a)] Zeigen Sie, dass die Matrixmultiplikation assoziativ ist.
	\item[(b)] Zeigen Sie, dass die Matrixmultiplikation nicht kommutativ ist: 
			Finden Sie hierzu $A,B \in \R^{2\times 2}$ mit
			$$
				AB \neq BA.
			$$
	\item[(c)] Zeigen Sie, dass  
			$$
				E
				=
				\begin{pmatrix}
					1	&	0\\
					0	&	1
				\end{pmatrix}
			$$ 
			das neutrale Element der Matrixmultiplikation ist. 

\end{itemize}
%%%
\end{frame}
%

%
\begin{frame}\frametitle{Lösung: Teil a)}
%%
Seien
		$$
			A 
			=
			\begin{pmatrix}
				a_{11}	&a_{12}	\\
				a_{21}	&a_{22}
			\end{pmatrix}, \quad
			B 
			=
			\begin{pmatrix}
				b_{11}	&b_{12}	\\
				b_{21}	&b_{22}
			\end{pmatrix}, \quad		
			C 
			=
			\begin{pmatrix}
				c_{11}	&c_{12}	\\
				c_{21}	&c_{22}
			\end{pmatrix}
			\in \R^{2\times 2}.		
		$$
		Es folgt:\pause
		$$
			(A\cdot B) \cdot C	
			=
			\begin{pmatrix}
				a_{11}b_{11} + a_{12}b_{21}	&	a_{11}b_{12} + a_{12}b_{22}	\\
				a_{21}b_{11} + a_{22}b_{21}	&	a_{21}b_{12} + a_{22}b_{22}	\\
			\end{pmatrix}
			\cdot C
			= \pause
		$$
		\tiny$$													
			\begin{pmatrix}
				a_{11}b_{11}c_{11} + a_{12}b_{21}c_{11} +  a_{11}b_{12}c_{21} + a_{12}b_{22}c_{21}	
					&	a_{11}b_{11}c_{12} + a_{12}b_{21}c_{12} +  a_{11}b_{12}c_{22} + a_{12}b_{22}c_{22}	\\
				a_{21}b_{11}c_{11} + a_{22}b_{21}c_{11} +  a_{21}b_{12}c_{21} + a_{22}b_{22}c_{21}	
					&	a_{21}b_{11}c_{12} + a_{22}b_{21}c_{12} +  a_{21}b_{12}c_{22} + a_{22}b_{22}c_{22}
			\end{pmatrix}			
		$$\pause
		\normalsize$$
			= \ldots
			= A\cdot(B\cdot C).
		$$
\end{frame}
%
%
\begin{frame}\frametitle{Lösung: Teil b)}
%
F\"ur 
		$$
			A
			=
			\begin{pmatrix}
				1	&	0	\\
				0	&	0
			\end{pmatrix}, \quad			
			B
			=
			\begin{pmatrix}
				0	&	1	\\
				0	&	0
			\end{pmatrix}			
		$$
		folgt:\pause
		$$
			A \cdot B = 	\begin{pmatrix}
				1\cdot0+0\cdot0	&	1\cdot1+0\cdot0	\\
				0\cdot0+0\cdot0	&	0\cdot1+0\cdot0
			\end{pmatrix}=
			\begin{pmatrix}
				0	&	1	\\
				0	&	0
			\end{pmatrix}
		$$\pause
		und
		$$
			B \cdot A = 			\begin{pmatrix}
				0\cdot1+1\cdot0	&	0\cdot0+1\cdot0	\\
				0\cdot1+0\cdot0	&	0\cdot0+0\cdot0
			\end{pmatrix}=
			\begin{pmatrix}
				0	&	0	\\
				0	&	0
			\end{pmatrix}.
		$$		
\end{frame}
%
%
%%
\begin{frame}\frametitle{Lösung: Teil c)}
%%
		F\"ur alle
		$$
			A 
			=
			\begin{pmatrix}
				a_{11}	&a_{12}	\\
				a_{21}	&a_{22}
			\end{pmatrix} 
			\in \R^{2\times 2}
		$$
		folgt:
		$$
			A\cdot E 
			=
			\begin{pmatrix}
				a_{11}	&a_{12}	\\
				a_{21}	&a_{22}
			\end{pmatrix} 
			\cdot
			\begin{pmatrix}
				1	&	0\\
				0	&	1
			\end{pmatrix}			
			=\pause
			\begin{pmatrix}
				a_{11}\cdot1+a_{12}\cdot0	&a_{11}\cdot0+a_{12}\cdot1	\\
				a_{21}\cdot1+a_{22}\cdot0	&a_{21}\cdot0+a_{22}\cdot1
			\end{pmatrix} 
			=\pause
			A
		$$
		und\pause
		$$
			E\cdot A 			=
			\begin{pmatrix}
				1	&	0\\
				0	&	1
			\end{pmatrix}
						\cdot	
						\begin{pmatrix}
				a_{11}	&a_{12}	\\
				a_{21}	&a_{22}
			\end{pmatrix} 		
			=\pause
			\begin{pmatrix}
				1\cdot a_{11}+0\cdot a_{21}	&1\cdot a_{12}+0\cdot a_{22}	\\
				0\cdot a_{11}+1\cdot a_{21} &0\cdot a_{12}+1\cdot a_{22}
			\end{pmatrix} 
			=\pause
			A
		$$
\end{frame}
%

\end{document}