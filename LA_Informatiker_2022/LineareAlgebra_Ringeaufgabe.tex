\documentclass{beamer}

\usepackage{beamerthemesplit}

\usepackage{amsfonts}
\usepackage{amsmath}
\usepackage{amssymb}
\usepackage{amsthm}
\usepackage{amscd}

\usepackage{stmaryrd} 					%\lightning
\usepackage{algorithm2e}


\usepackage[ngerman]{babel}

\usepackage[utf8]{inputenc}
\usepackage[T1]{fontenc}
\usepackage{textcomp}


% Color Definitions
\definecolor{dhbwRed}{RGB}{226,0,26} 
\definecolor{dhbwGray}{RGB}{61,77,77}
\definecolor{lightBlue}{RGB}{28,134,230}

% Basic Theme
\usetheme{Malmoe}

% Color Re-Definitions
\usecolortheme[named=lightBlue]{structure}
\setbeamercolor*{alerted  text}{fg=dhbwRed, bg=white}
\setbeamercolor*{subsection in toc}{fg=dhbwGray, bg=white}

%\setbeamercolor*{palette primary}{fg=white,bg=lightBlue}
%\setbeamercolor*{palette secondary}{fg=white,bg=gray}
%\setbeamercolor*{palette tertiary}{fg=white,bg=gray}
%\setbeamercolor*{palette quaternary}{fg=white,bg=dhbwRed}

% no navigation symbols
\setbeamertemplate{navigation symbols}{}

% headline, footline
\setbeamertemplate{footline}{\color{dhbwGray} \hfill\insertframenumber\hspace{5mm}\vspace{2mm}}
\setbeamertemplate{headline}{}

% Title Page
\newcommand*{\makeTitlePage}{
	
	\begin{frame}[plain]
		
		\vfill
		\vfill
		\begin{center}
			{
				\usebeamerfont{title}
				\usebeamercolor[fg]{title}
				\Large
				\inserttitle
			}\\[3mm]
			{	
				\usebeamerfont{subtitle}
				\usebeamercolor[fg]{subtitle}
				\large
				\insertsubtitle
			}
		\end{center}
		%
		\vfill
		\vfill
		\vfill
		\vfill
		%
		\begin{columns}
			\begin{column}{0.5\textwidth}
				\begin{flushleft}
					{
						\usebeamerfont{normal text}
						\color{dhbwGray!80}
						\scriptsize
						Dr. Moritz Gruber\\
						DHBW Karlsruhe\\
						
					}
				\end{flushleft}
			\end{column}
			%
			\begin{column}{0.5\textwidth}
				\begin{flushright}
					\includegraphics[scale=0.06]{../DHBW.png}
				\end{flushright}
			\end{column}
		\end{columns}
		%
		\vspace{1mm}
		\begin{columns}
			\begin{column}{0.5\textwidth}
				\begin{flushleft}
					{
						\usebeamerfont{normal text}
						\color{dhbwGray!80}
						\scriptsize
						Version \today
					}
				\end{flushleft}
			\end{column}
			%
			\begin{column}{0.5\textwidth}
				% nothing (just a placeholder to be in line with the columns above
			\end{column}
		\end{columns}
	\end{frame}

}

% Section Divider Page
\newcommand*{\makeSectionDividerPage}{

	\begin{frame}[plain]
		\begin{center}
			\begin{flushleft}
				{				
					\usebeamercolor[fg]{frametitle}
					{\Large \insertsection} \\[3mm]
					{\large \insertsubsection}
				}
			\end{flushleft}
		\end{center}
        \end{frame}
	
}

% itemize
\setbeamertemplate{itemize items}[circle]
\setbeamertemplate{enumerate item}{(\theenumi)}




%--------------------------------------%
% Math ------------------------------%
%--------------------------------------%

% Mengen (Zahlen)
\newcommand{\N}{\mathbb{N}}
\newcommand{\Q}{\mathbb{Q}}
\newcommand{\R}{\mathbb{R}}
\newcommand{\Z}{\mathbb{Z}}
\newcommand{\C}{\mathbb{C}}

% Mengen (allgemein)
\newcommand{\K}{\mathbb{K}}
\newcommand\PX{{\cal P}(X)}

% Zahlentheorie
\newcommand{\ggT}{\mathrm{ggT}}


% Ableitungen
\newcommand{\ddx}{\frac{d}{dx}}
\newcommand{\pddx}{\frac{\partial}{\partial x}}
\newcommand{\pddy}{\frac{\partial}{\partial y}}
\newcommand{\grad}{\text{grad}}

%--------------------------------------%
% Layout Colors ------------------%
%--------------------------------------%
\newcommand*{\highlightDef}[1]{{\color{lightBlue}#1}}
\newcommand*{\highlight}[1]{{\color{lightBlue}#1}} % after theme for colours

%----------------------------------------------------------------------------------------------------
%--------- Document Title ---------------------------------------------------------------------
\title{Lineare Algebra\\[3mm] 
	\large Ringe-Aufgabe: Restklassenringe $\Z_n$
}
\author{Dr. Moritz Gruber} 
\institute{DHBW Karlsruhe}
\date{2022}
%%%%%%%%%%%%%%
\begin{document}

\AtBeginSection[]{
	\begin{frame}				
		\usebeamercolor[fg]{frametitle}
		{\Large \insertsection} 
        \end{frame}
}

%
\begin{frame}[plain] 
 \titlepage
\end{frame}
%
%
%%%
\begin{frame} \frametitle{Aufgabe: Restklassenringe $\Z_n$}

Es sei $n \in \N$. Wir notieren f\"ur $a \in \Z$ mit
$$
[a]_n:=\{m \in \Z \mid \exists k\in \Z: m=a+kn\}
$$
die Menge aller Zahlen, die beim Teilen durch $n$ den gleichen Rest wie $a$ haben.

\begin{itemize}
\item[(a)] Zeigen Sie, dass aus $b \in [a]_n$ folgt, dass $[a]_n=[b]_n$.
\end{itemize}
Wir definieren die Menge
$
\Z_n := \{[a]_n \mid a \in \Z\}
$
\begin{itemize}
\item[(b)] Zeigen Sie, dass $\Z_n=\{[0]_n,[1]_n,[2]_n,...,[n-1]_n\}$ gilt.
\end{itemize}
Wir definieren weiter die Verkn\"upfungen $[a]_n+_n[b]_n=[a+b]_n$ und $[a]_n\cdot_n [b]_n:=[a\cdot b]_n$.
\begin{itemize}
\item[(c)] Zeigen Sie, dass $(\Z_n,+_n,\cdot_n)$ ein Ring ist.
\end{itemize}
%%%
\end{frame}
%

%
\begin{frame}\frametitle{Lösung: Teil a)}
%
\textcolor{blue}{Zeigen Sie, dass aus $b \in [a]_n$ folgt, dass $[a]_n=[b]_n$.}
%
\vfill
Da $b \in [a]_n$ gilt, gibt es ein $k_b\in \Z$ sodass $b=a+k_b n$.\\ \pause
Sei nun $m \in [a]_n$ beliebig. Dann gibt es ein $k_m \in \Z$ sodass $m=a+k_m n$. Damit folgt: \pause
\vfill
$
m=a+k_m n \pause= a+k_m n + (b- (a+k_b n))\pause=b + (a+k_m n -a- k_b n)$\\$ \pause \hspace{3mm}=b + (k_m-k_b)n
$\pause
\vfill
Mit $k:=k_m -k_b$ folgt also $m=b+kn$ und damit $m \in [b]_n$. \\Somit gilt $[a]_n \subseteq [b]_n$.\\\pause
Da insbesondere $a \in [a]_n$ und somit $a\in [b]_n$ gilt, folgert man mit vertauschten Rollen von $a$ und $b$ nun $[b]_n \subseteq [a]_n$ und damit $[a]_n=[b]_n$.

\end{frame}
%
%
\begin{frame}\frametitle{Lösung: Teil b)}
%
\textcolor{blue}{Zeigen Sie, dass $\Z_n=\{[0]_n,[1]_n,[2]_n,...,[n-1]_n\}$ gilt.}
%
\vfill
Es sei $m \in \Z$ beliebig. \\ \pause
Dann gibt es ein $k \in \Z$ sodass $0\le m-kn \le n-1$.\\ \pause
Wir definieren $a:=m-kn$. Dann gilt $m=a+kn$ und damit $m \in [a]_n$.\\ \pause
Mit Aufgabenteil (a) folgt $[m]_n=[a]_n$.\\ \pause
Und da $a \in \{0,1,2,...,n-1\}$ folgt somit die Behauptung.
	
\end{frame}
%
%
%%
\begin{frame}\frametitle{Lösung: Teil c)}
%
\textcolor{blue}{Zeigen Sie, dass $(\Z_n,+_n,\cdot_n)$ ein Ring ist.} \pause
%
\vfill
Wir zeigen zuerst 
(i) $(\Z_n,+_n)$ ist eine abelsche Gruppe\\ und dann 
(ii) die Verkn\"upfungen $+_n$ und $\cdot_n$ erf\"ullen die Distributivit\"at. \pause
\vfill
\begin{itemize}
\item[zu (i)]
Das neutrale Element ist $[0]_n$, da 
$$[0]_n+_n[a]_n=[0+a]_n=[a]_n=[a+0]_n=[a]_n+_n[0]_n$$
f\"ur alle $[a]_n \in \Z_n$.\\ \pause
F\"ur jedes Element $[a]_n \in \Z_n$ ist $[n-a]_n$ das Inverse, denn \\
\vspace{1mm}
$[a]_n+_n[n-a]_n=[a+n-a]_n=[n]_n=[0]_n$\\
$[n-a]_n+_n[a]_n=[n-a+a]_n=[n]_n=[0]_n$\\\pause
\vspace{1mm}
Die Assoziativ\"at und die Kommutativit\"at folgt direkt aus den entsprechenden Eigenschaften von $+$ in $\Z$.

\end{itemize}

\end{frame}
%
%
%%
\begin{frame}\frametitle{Lösung: Teil c)}
%
\textcolor{blue}{Zeigen Sie, dass $(\Z_n,+_n,\cdot_n)$ ein Ring ist.}
%
\vfill
Wir zeigen zuerst \\
(i) $(\Z_n,+_n)$ ist eine abelsche Gruppe und dann \\
(ii) die Verkn\"upfungen $+_n$ und $\cdot_n$ erf\"ullen die Distributivit\"at.
\vfill
\begin{itemize}
\item[zu (ii)]
Es seien $[a]_n,[b]_n,[c]_n \in \Z_n$, dann gilt mit der Distributivit\"at von $(\Z,+,\cdot)$:\pause
\vfill
$
[a]_n \cdot_n ([b]_n +_n [c]_n)\pause=[a]_n\cdot_n [b+c]_n\pause=[a\cdot (b+c)]_n$\\\pause
$\hspace{33mm}=[ab+ac]_n\pause=[ab]_n+_n[ac]_n$\\\pause
$\hspace{33mm}=([a]_n\cdot_n[b]_n) +_n ([a]_n\cdot_n[c]_n)
$
\vfill
\end{itemize}\pause
\vfill
Somit ist $(\Z_n,+_n,\cdot_n)$ ein Ring.

\end{frame}

\end{document}