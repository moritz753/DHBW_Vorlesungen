\documentclass[
				a4paper,
				10pt
			]
			{scrartcl}

\parindent0mm

\usepackage{amsfonts}
\usepackage{amsmath}
\usepackage{amssymb}
\usepackage{amsthm}
\usepackage[ngerman]{babel}

\usepackage[
			pdftex,
			colorlinks,
			breaklinks,
			linkcolor=blue,
			citecolor=blue,
			filecolor=black,
			menucolor=black,
			urlcolor=black,
			pdfauthor={Andreas Weber},
			pdftitle={Aufgaben zur Logik und Algebra},
			plainpages=false,
			pdfpagelabels,
			bookmarksnumbered=true
		   ]{hyperref}

 %%%%%%%%%%%%%%%Schriften%%%%%%%%%%%%%
\DeclareMathAlphabet{\lier}{U}{eur}{m}{n}  %% Gothisch/Fraktur - Roman



\newcommand{\M}{\mathbb{M}}
\newcommand{\E}{\mathbb{E}}
\newcommand{\Hy}{\mathbb{H}}
\newcommand{\N}{\mathbb{N}}
\newcommand{\Q}{\mathbb{Q}}
\newcommand{\R}{\mathbb{R}}
\newcommand{\Z}{\mathbb{Z}}
\newcommand{\C}{\mathbb{C}}
\newcommand{\K}{\mathbb{K}}
\newcommand*\ee{\mathrm{e}}
\newcommand*\e{\mathrm{e}}

\newcommand*\ii{\mathrm{i}}
\newcommand*\re{\mathrm{Re}}
\newcommand*\im{\mathrm{Im}}
\newcommand*\id{\mathrm{id}}
\newcommand*\rang{\mathrm{rang}}
\newcommand*\grad{\mathrm{grad}}
\newcommand*\dive{\mathrm{div}}
\newcommand*\sym{\mathrm{Sym}}
\newcommand*\spur{\mathrm{Spur}}
\newcommand*\isom{\mathrm{Isom}}
\newcommand*\vol{\mathrm{vol\,}}
\newcommand*\supp{\mathrm{supp}}
\newcommand*\inj{\mathrm{inj}}
\newcommand*\rank{\mathrm{rank}}
\newcommand*\qrank{\Q\mbox{-}\mathrm{rank}}
\newcommand*\rrank{\R\mbox{-}\mathrm{rank}}
\newcommand*\dom{\mathrm{dom}}
\newcommand*\tr{\mathrm{tr}}
\newcommand*\spa{\mathrm{span}}
\newcommand*\diam{\mathrm{diam}}

\newcommand*\ric{\mathrm{Ric}}

\newcommand*\con{\mathrm{con}}
\newcommand*\dis{\mathrm{dis}}

\newcommand\PX{{\cal P}(X)}
\newcommand\T{{\cal T}}
\newcommand\B{{\cal B}}



\newcommand\scp{\langle\cdot,\cdot\rangle}     %% Metric

%%%%%%%%%%%%Tilde%%%%%%%
\newcommand\tx{\tilde{x}}
\newcommand\ty{\tilde{y}}
\newcommand\tu{\tilde{u}}
\newcommand\tk{\tilde{k}}
\newcommand\td{\tilde{d}}
\newcommand\tD{\tilde{D}}
\newcommand\tX{\tilde{X}}
\newcommand\tY{\tilde{Y}}
\newcommand\tZ{\tilde{Z}}


%%%%%%Lie-Gruppen%%%%%%%%%
\newcommand\ad{\mathrm{ad}}
\newcommand\Ad{\mathrm{Ad}}
\newcommand{\kak}{K\exp\overline{\lier{a}^+}K}              %%%%Cartan-Zerlegung
\newcommand*\Rang{\mathrm{Rang}}
\newcommand*\glnr{\mathrm{\it GL}(n,\R)}
\newcommand*\glnc{\mathrm{\it GL}(n,\C)}
\newcommand*\slnr{\mathrm{\it SL}(n,\R)}
\newcommand*\on{\mathrm{\it O}(n)}
\newcommand*\son{\mathrm{\it SO}(n)}
\newcommand*\SLzr{\mathrm{\it SL}(2,\R)}
\newcommand*\SOzr{\mathrm{\it SO}(2,\R)}

%%%%%%%%%%%%%Algebraische Gruppen
\newcommand\bG{{\bf G}}
\newcommand\bT{{\bf T}}
\newcommand\bP{{\bf P}}
\newcommand\bN{{\bf N}}
\newcommand\bL{{\bf L}}
\newcommand\bS{{\bf S}}
\newcommand\bM{{\bf M}}

\newcommand\Mor{\mathrm{Mor}}



%%%%%%Geometry%%%%%%%%%%%
\newcommand{\Si}{\mathcal{S}}


%%%%%%Ableitungsoperatoren%%%%%%%%%%%
\newcommand*\pddt{\frac{\partial}{\partial t}}
\newcommand*\pddx{\frac{\partial}{\partial x}}
\newcommand*\pddxio{\frac{\partial}{\partial x^i}}
\newcommand*\pddxjo{\frac{\partial}{\partial x^j}}
\newcommand*\pddxko{\frac{\partial}{\partial x^k}}
\newcommand*\pddxlo{\frac{\partial}{\partial x^l}}

\newcommand*\pddy{\frac{\partial}{\partial y}}
\newcommand*\pddyio{\frac{\partial}{\partial y^i}}
\newcommand*\pddyjo{\frac{\partial}{\partial y^j}}
\newcommand*\pddyko{\frac{\partial}{\partial y^k}}
\newcommand*\pddylo{\frac{\partial}{\partial y^l}}

\newcommand*\pddyq{\frac{\partial^2}{\partial y^2}}
\newcommand*\pddyj{\frac{\partial}{\partial y_j}}
\newcommand*\pddyjq{\frac{\partial^2}{\partial y_j^2}}
\newcommand*\pddxiq{\frac{\partial^2}{\partial x_i^2}}
\newcommand*\pddxi{\frac{\partial}{\partial x_i}}
\newcommand*\ddt{\frac{d}{dt}}

\newcommand*\dx{\,dvol(x)}
\newcommand*\dy{\,dvol(y)}
\newcommand*\dty{\,dvol(\ty)}

\newcommand*\DMp{\Delta_{M,p}}                 %%%%Laplace-Operatoren
\newcommand*\DMq{\Delta_{M,q}}
\newcommand*\DM{\Delta_M}
\newcommand*\DX{\Delta_X}
\newcommand*\DXp{\Delta_{X,p}}
\newcommand*\DXq{\Delta_{X,q}}
\newcommand*\DAx{\Delta_{Ax_0}}
\newcommand*\Rad{\mathrm{Rad}}
\newcommand*\DXps{\Delta^{\#}_{X,p}}

\newcommand*\eDXps{\e^{-t(\Delta^{\#}_{X,p}-c)}} %%%%%% Semigroups
\newcommand*\LpsX{L^p_{\#}(X)}


%%%%%%%%%%%%%%%Komplexe Analysis
\newcommand*\Res{\mathrm{Res}}

%%%%%%%%%%%%%%%Definitionsmenge
\newcommand*\D{{\cal D}}







%-------------------------------------------------------------------
\author{Dr. Moritz Gruber\\ DHBW Karlsruhe}
\title{L\"osungen \"Ubungsaufgaben 1\\
	Logik
}
\date{}
%-------------------------------------------------------------------


%-------------------------------------------------------------------
%-------------------------------------------------------------------
\begin{document}

\maketitle
%-------------------------------------------------------------------
%-------------------------------------------------------------------


%%%
\section{Subjunktion}
%%%
Seien $a,b$  Aussagen. 
Bestimmen Sie die Wahrheitstafel der Aussage
$$
	(\neg a) \lor b.
$$
Vergleichen Sie die Wahrheitstafel mit der Wahrheitstafel der Subjunktion und schlie{\ss}en Sie
$$
	a \rightarrow b \equiv (\neg a) \lor b.
$$

%%
\paragraph{L\"osung:}
%%

$$
	\begin{array}{c | c || c}
		a	&b	& (\neg a) \lor b	\\ \hline
		w	&w	& w			\\
		w	&f	& f			\\
		f	&w	& w			\\
		f	&f	& w	
	\end{array}
$$

Die Wahrheitstafel stimmt mit der Wahrheitstafel der Subjunktion \"uberein. Somit: $a \rightarrow b \equiv (\neg a) \lor b$.

%%%
\newpage
\section{Klammern}
%%%
Seien $a,b,c$  Aussagen.
Erstellen Sie Wahrheitstafeln f\"ur 
$$
	a\land (b \lor c)
$$
und
$$
	(a\land b) \lor c.
$$

%%
\paragraph{L\"osung:}
%%

$$
	\begin{array}{c | c | c || c || c}
		a	&b	&c	& a\land (b \lor c)	& (a\land b) \lor c	\\ \hline
		w	&w	&w	& w				& w				\\
		w	&w	&f	& w				& w				\\
		w	&f	&w	& w				& w				\\
		f	&w	&w	& f				& w				\\
		f	&f	&w	& f				& w				\\
		f	&w	&f	& f				& f				\\
		w	&f	&f	& f				& f				\\
		f	&f	&f	& f				& f				\\
	\end{array}
$$

Somit: $a\land (b \lor c) \not\equiv (a\land b) \lor c$.

%%
\section{De Morgan'sche Regeln}
%%
Seien $a,b$ Aussagen.
Erstellen Sie die Wahrheitstafeln f\"ur
$$
	\neg (a \lor b)
$$
und
$$
	(\neg a) \land (\neg b). 
$$
Folgern Sie, dass
$$
	\neg (a \lor b) \equiv (\neg a) \land (\neg b). 
$$

%%
\paragraph{L\"osung:}
%%
$$
	\begin{array}{c | c || c || c}
		a	&b	& \neg (a \lor b)	& (\neg a) \land (\neg b)	\\ \hline
		w	&w	& f			& f					\\
		w	&f	& f			& f					\\
		f	&w	& f			& f					\\
		f	&f	& w			& w
	\end{array}
$$

Da die Wahrheitstafeln die selben Wahrheitswerte aufweisen folgt: $(\neg a) \land (\neg b).$

%%%
\newpage
\section{Wahrheitstafel}
%%%
Seien $a,b,c$ Aussagen. 
Bestimmen Sie die Wahrheitstafel der Aussage
$$
	(a \rightarrow b) \lor (c \land \neg a).
$$

%%
\paragraph{L\"osung:}
%%

$$
	\begin{array}{c | c | c || c || c || c }
		a	&b	&c	& (a \rightarrow b)	& (c \land \neg a)	& (a \rightarrow b) \lor (c \land \neg a)	\\ \hline
		w	&w	&w	& w				& f				& w								\\
		w	&w	&f	& w				& f				& w								\\
		w	&f	&w	& f				& f				& f								\\
		f	&w	&w	& w				& w				& w								\\
		f	&f	&w	& w				& w				& w								\\
		f	&w	&f	& w				& f				& w								\\
		w	&f	&f	& f				& f				& f								\\
		f	&f	&f	& w				& f				& w
	\end{array}
$$

%%%
\section{Quantoren}
%%%
Sind die folgenden Aussagen wahr?
\begin{itemize}
	\item[(a)] $\forall x\in \N: \exists y\in \N \text{~mit~} y = x+1.$
	\item[(b)] $\exists y\in \N: \forall x\in \N \text{~mit~} y = x+1.$
\end{itemize}

%%
\paragraph{L\"osung:}
%%
\begin{itemize}
	\item[(a)] Ja: Jede nat\"urliche Zahl hat einen Nachfolger. 
	\item[(b)] Nein: Es gibt keine Zahl, die um 1 gr\"o{\ss}er ist als {\em alle} nat\"urlichen Zahlen.
\end{itemize}


%%%
\newpage
\section{Negation}
%%%
 Negieren Sie die folgenden Aussagen:
 \begin{itemize}
	\item[(a)] Alle Karlsruher fahren mit der S-Bahn oder mit dem Rad.
 	\item[(b)] Jede Entscheidung schafft Unzufriedene.
 \end{itemize}
 
 %%
\paragraph{L\"osung:}
%%
 \begin{itemize}
	\item[(a)] Mit Quantoren lautet die Aussage
			$$
				\forall p \in K: (p\in S) \lor (p\in R). 
			$$	
			($p$: Personen, $K$: Menge aller Karlsruher, $S$: Menge aller S-Bahn-Fahrer, $R$: Menge aller Radfahrer)\\
			Negation:
			$$
				\exists p \in K:  (p\notin S) \land (p\notin R). 
			$$
			Es gibt eine Person in Karlsruhe, die weder S-Bahn noch Rad f\"ahrt.
			
 	\item[(b)] Mit Quatoren formuliert:\\
			F\"ur alle Entscheidungen gibt es eine Person, die mit der Entscheidung unzufrieden ist:
			$$
				\forall e \exists p : p \text{~ist unzufrieden mit~} e.
			$$
			Negation:\\
			$$
				\exists e \forall p : \neg(p \text{~ist unzufrieden mit~} e)
			$$
			bzw.
			$$
				\exists e \forall p : p \text{~ist zufrieden mit~} e.
			$$			
			Es gibt eine Entscheidung, mit der alle zufrieden sind.	
 \end{itemize}

\end{document}