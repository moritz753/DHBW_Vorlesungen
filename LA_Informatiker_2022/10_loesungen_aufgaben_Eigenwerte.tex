\documentclass[
				a4paper,
				10pt
			]
			{scrartcl}

\parindent0mm

\usepackage{amsfonts}
\usepackage{amsmath}
\usepackage{amssymb}
\usepackage{amsthm}
\usepackage[ngerman]{babel}
\usepackage{graphicx}
\usepackage{xcolor}

\usepackage[
			pdftex,
			colorlinks,
			breaklinks,
			linkcolor=blue,
			citecolor=blue,
			filecolor=black,
			menucolor=black,
			urlcolor=black,
			pdfauthor={Andreas Weber},
			pdftitle={Aufgaben zu Analysis und Lineare Algebra},
			plainpages=false,
			pdfpagelabels,
			bookmarksnumbered=true
		   ]{hyperref}


%--------------------------------------%
% Math ------------------------------%
%--------------------------------------%

% Mengen (Zahlen)
\newcommand{\N}{\mathbb{N}}
\newcommand{\Q}{\mathbb{Q}}
\newcommand{\R}{\mathbb{R}}
\newcommand{\Z}{\mathbb{Z}}
\newcommand{\C}{\mathbb{C}}

% Mengen (allgemein)
\newcommand{\K}{\mathbb{K}}
\newcommand\PX{{\cal P}(X)}

% Zahlentheorie
\newcommand{\ggT}{\mathrm{ggT}}


% Ableitungen
\newcommand{\ddx}{\frac{d}{dx}}
\newcommand{\pddx}{\frac{\partial}{\partial x}}
\newcommand{\pddy}{\frac{\partial}{\partial y}}
\newcommand{\grad}{\text{grad}}

%--------------------------------------%
% Layout Colors ------------------%
%--------------------------------------%
\newcommand*{\highlightDef}[1]{{\color{lightBlue}#1}}
\newcommand*{\highlight}[1]{{\color{lightBlue}#1}}
% Color Definitions
\definecolor{dhbwRed}{RGB}{226,0,26} 
\definecolor{dhbwGray}{RGB}{61,77,77}
\definecolor{lightBlue}{RGB}{28,134,230}

%
\addtokomafont{section}{\color{dhbwGray}}
\addtokomafont{subsection}{\color{dhbwGray}}


%-------------------------------------------------------------------
\begin{document}

\vspace*{-20mm}
{
	%\usekomafont{title}
	\color{dhbwGray}
	Dr. Moritz Gruber	\hfill Version \today\\
	DHBW Karlsruhe\\
}

\vspace{10mm}
\begin{center}
	{
		\usekomafont{title}
		\color{lightBlue}
		{ \LARGE L\"osungen Übungsaufgaben 10}\\[3mm]
		{\Large Diagonalisierbarkeit und Eigenwerte}
	}
\end{center}

\vspace{5mm}

%-------------------------------------------------------------------


%-------------------------
%-------------------------------------------------------------------
\section{Charakteristisches Polynom}
%%%
Sei $A=	\begin{pmatrix}
		1	&5	&8	\\
		0	&-2	&1	\\
		-1	&1	&3
	\end{pmatrix}$. 
\begin{itemize}
\item[a)] Berechnen Sie das charakteristische Polynom der Matrix $A$:
\item[b)] Bestimmen Sie das Spektrum $Spec(A)$ von $A$.
\end{itemize}


%-------------------------
\subsection*{L\"osung}
%%%	
\begin{itemize}
\item[a)] Für das charakteristische Polynom gilt 
\begin{align*}
CP_A(X)&=\det(A-X\cdot I_3)\\
&=\det(\begin{pmatrix}
		1-X	&5&8	\\
		0	&-2-X	&1	\\
		-1	&1	&3-X
	\end{pmatrix})\\
&\stackrel{(1)}{=} (1-X)\cdot(-1)^{1+1}\cdot \det(\begin{pmatrix}
		-2-X	&1	\\
		1	&3-X
	\end{pmatrix})+(-1)\cdot(-1)^{3+1}\cdot \det(\begin{pmatrix}
		5	&8	\\
		-2-X	&1
	\end{pmatrix})\\
&= (1-X)\cdot \det(\begin{pmatrix}
		-2-X	&1	\\
		1	&3-X
	\end{pmatrix})-\det(\begin{pmatrix}
		5	&8	\\
		-2-X	&1
	\end{pmatrix})\\
&\stackrel{(2)}{=}(1-X)((-2-X)(3-X)-1)-(5-8(-2-X))\\
&=
-X^3 + 2 X^2 - 2 X - 28
\end{align*}
wobei man bei
\begin{itemize}
\item[(1)] nach der 1. Spalte entwickelt
\item[(2)] die Formel für die Determinante einer $2\times2$ -Matrix bei beiden Matrizen verwendet.
\end{itemize}
\item[b)] Das Spektrum einer Matrix ist die Menge aller Eigenwerte der Matrix. Diese sind wiederum die Nullstellen des charakteristischen Polynom. Daher
$$
Spec(A)=\{\lambda \in \R \mid CP_A(\lambda)=0\}
$$
Die Zahlen in der Matrix $A$ sind aber so unangenehm gewählt, dass da charakteristische Polynom nur eine reelle Nullstelle hat:
$$
\lambda=\frac{1}{3} \left(2 - \frac{2^{\frac{2}{3}}}{(-194 + 3 \sqrt{4182})^{\frac{1}{3}}} + \left(2 (-194 + 3 \sqrt{4182})\right)^{\frac{1}{3}}\right)
$$
\end{itemize}



\newpage

%-------------------------
\section{Basis aus Eigenvektoren}
%%%
Sei $A=\begin{pmatrix} -\frac{1}{4} & 0 & -\frac{1}{2} \\ 0 & \frac{9}{4} & 0 \\ \frac{1}{2} & 0 &1 \end{pmatrix}$. 
\begin{itemize}
\item[a)] Bestimmen Sie das charakteristische Polynom $CP_A(X)$ und alle Eigenwerte der Matrix $A$.
\item[b)] Bestimmen Sie eine invertierbare Matrix $D \in \R^{3 \times 3}$, sodass $D^{-1}AD$ eine Diagonalmatrix ist.
\end{itemize}

%-------------------------
\subsection*{L\"osung}
%%%	
\begin{itemize}
\item[a)] 
\begin{align*}
CP_A(X)&=\det(A-X\cdot I_3)\\
&=\det(\begin{pmatrix} -\frac{1}{4}-X & 0 & -\frac{1}{2} \\ 0 & \frac{9}{4}-X & 0 \\ \frac{1}{2} & 0 &1-X \end{pmatrix})\\
&\stackrel{(1)}{=}(\frac{9}{4}-X )\cdot(-1)^{2+2}\cdot \det(\begin{pmatrix} -\frac{1}{4}-X & -\frac{1}{2} \\ \frac{1}{2} & 1-X \end{pmatrix})\\
&\stackrel{(2)}{=}(\frac{9}{4}-X )\cdot((-\frac{1}{4}-X)(1-X )-(-\frac{1}{2})(\frac{1}{2}))\\
&=(\frac{9}{4}-X )\cdot(X^2-\frac{3}{4}X)\\
&=X(\frac{9}{4}-X )(X-\frac{3}{4})
\end{align*}
wobei man bei
\begin{itemize}
\item[(1)] nach der 2. Spalte entwickelt
\item[(2)] die Formel für die Determinante einer $2\times2$ -Matrix verwendet.
\end{itemize}
Damit sind die Eigenwerte von $A$ gegeben durch $\lambda_1=0$, $\lambda_2=\frac{3}{4}$ und $\lambda_3=\frac{9}{4}$.
\item[b)] Die gesuchte Matrix $D$ hat als Spalten eine Basis aus Eigenvektoren von $A$. Um eine solche Basis zu bestimmen, lösen wir die LGSe
$$
(A-\lambda_1I_3)x=0, \quad (A-\lambda_2I_3)x=0 \ \text{ und } \  (A-\lambda_3I_3)x=0
$$
\begin{itemize}
\item[$\lambda_1=0$]
\begin{align*}
&\left(\begin{array}{ccc|c}
-\frac{1}{4} & 0 & -\frac{1}{2} &0\\ 0 & \frac{9}{4} & 0 &0\\ \frac{1}{2} & 0 &1 &0
\end{array}\right)\stackrel{-4\cdot Z1}{\sim >}
\left(\begin{array}{ccc|c}
1 & 0 &2 &0\\ 0 & \frac{9}{4} & 0 &0\\ \frac{1}{2} & 0 &1 &0
\end{array}\right)\stackrel{Z3-\frac{1}{2}\cdot Z1}{\sim >}
\left(\begin{array}{ccc|c}
1 & 0 & 2 &0\\ 0 & \frac{9}{4} & 0 &0\\ 0 & 0 &0 &0
\end{array}\right)\\
&\stackrel{\frac{4}{9}\cdot Z2}{\sim >}
\left(\begin{array}{ccc|c}
1 & 0 & 2 &0\\ 0 & 1 & 0 &0\\ 0 & 0 &0 &0
\end{array}\right)
\end{align*}
Mit dem $(-1)$-Trick erhalten wir $Eig(A,\lambda_1)=\langle \begin{pmatrix}2 \\ 0\\ -1 \end{pmatrix}\rangle$.
%
\item[$\lambda_2=\frac{3}{4}$]
\begin{align*}
&\left(\begin{array}{ccc|c}
-\frac{1}{4}-\frac{3}{4} & 0 & -\frac{1}{2} &0\\ 0 & \frac{9}{4}-\frac{3}{4} & 0 &0\\ \frac{1}{2} & 0 &1-\frac{3}{4} &0
\end{array}\right){\sim >}
\left(\begin{array}{ccc|c}
-1 & 0 & -\frac{1}{2} &0\\ 0 & \frac{3}{2} & 0 &0\\ \frac{1}{2} & 0 &\frac{1}{4} &0
\end{array}\right)\stackrel{-1\cdot Z1}{\sim >}
\left(\begin{array}{ccc|c}
1 & 0 & \frac{1}{2} &0\\ 0 & \frac{3}{2} & 0 &0\\ \frac{1}{2} & 0 &\frac{1}{4} &0
\end{array}\right)\\
&\stackrel{Z3-\frac{1}{4}\cdot Z1}{\sim >}
\left(\begin{array}{ccc|c}
1 & 0 & \frac{1}{2} &0\\ 0 & \frac{3}{2} & 0 &0\\ 0 & 0 &0 &0
\end{array}\right)\stackrel{\frac{2}{3}\cdot Z2}{\sim >}
\left(\begin{array}{ccc|c}
1 & 0 & \frac{1}{2} &0\\ 0 & 1 & 0 &0\\ 0 & 0 &0 &0
\end{array}\right)
\end{align*}
Mit dem $(-1)$-Trick erhalten wir $Eig(A,\lambda_2)=\langle \begin{pmatrix}\frac{1}{2} \\ 0\\ -1 \end{pmatrix}\rangle$.
%
\item[$\lambda_3=\frac{9}{4}$]
\begin{align*}
&\left(\begin{array}{ccc|c}
-\frac{1}{4} -\frac{9}{4}& 0 & -\frac{1}{2} &0\\ 0 & \frac{9}{4} -\frac{9}{4}& 0 &0\\ \frac{1}{2} & 0 &1-\frac{9}{4} &0
\end{array}\right){\sim >}
\left(\begin{array}{ccc|c}
-\frac{5}{2}& 0 & -\frac{1}{2} &0\\ 0 & 0& 0 &0\\ \frac{1}{2} & 0 &\frac{5}{4} &0
\end{array}\right)\stackrel{-\frac{2}{5}\cdot Z1}{\sim >}
\left(\begin{array}{ccc|c}
1& 0 & \frac{1}{5} &0\\ 0 & 0& 0 &0\\ \frac{1}{2} & 0 &\frac{5}{4} &0
\end{array}\right)\\
&\stackrel{Z3-\frac{1}{2}\cdot Z1}{\sim >}
\left(\begin{array}{ccc|c}
1& 0 & \frac{1}{5} &0\\ 0 & 0& 0 &0\\ 0 & 0 &\frac{23}{10} &0
\end{array}\right)\stackrel{\frac{10}{23}\cdot Z3}{\sim >}
\left(\begin{array}{ccc|c}
1& 0 & \frac{1}{5} &0\\ 0 & 0& 0 &0\\ 0 & 0 &1 &0
\end{array}\right)
\end{align*}
Mit dem $(-1)$-Trick erhalten wir $Eig(A,\lambda_3)=\langle \begin{pmatrix}0 \\ -1\\ 0 \end{pmatrix}\rangle$.
\end{itemize}
Somit ergibt sich die Matrix $D$ als
$$
D=\begin{pmatrix}2&\frac{1}{2}&0 \\0& 0&-1\\-1& -1&0 \end{pmatrix}
$$\quad\\
\textbf{Bemerkung:} Die Reihenfolge der Spalten ist hierbei nicht eindeutig festgelegt.
\end{itemize}

\newpage
%-------------------------

\section{Rückwärts}
%%%
\begin{itemize}
\item[a)] Finden Sie eine Matrix $A$ mit charakteristischem Polynom 
$$CP_A(X)=(X-1)(X-2)^2(X-3).$$
\item[b)] Finden Sie eine Matrix $B \in \R^{3\times 3}$ mit Eigenwerten $\lambda_1= 1$, $\lambda_2=-1$ und $\lambda_3=4$ sowie den Eigenräumen
$$
Eig(B,\lambda_1)=\langle \begin{pmatrix} 1 \\ 1 \\ 0 \end{pmatrix} \rangle, \ Eig(B,\lambda_2)=\langle \begin{pmatrix} 0 \\ 1 \\ 2 \end{pmatrix} \rangle \text{ und } Eig(B,\lambda_3)=\langle \begin{pmatrix} 0 \\ 2 \\ 1 \end{pmatrix} \rangle
$$
\end{itemize}


%-------------------------
\subsection*{L\"osung}
%%%	
\begin{itemize}
\item[a)] Die Matrix 
$$
A:=\begin{pmatrix}1 & 0&0&0 \\ 0& 2 & 0&0 \\ 0& 0&2&0 \\ 0&0&0&3 \end{pmatrix}
$$
hat das charakteristische Polynom 
\begin{align*}
CP_A(X)&=\det(A-X\cdot I_4)\\
&=\det(\begin{pmatrix}1 -X& 0&0&0 \\ 0& 2-X & 0&0 \\ 0& 0&2-X&0 \\ 0&0&0&3-X \end{pmatrix})\\
&=(1-X)(2-X)(2-X)(3-X)\\
&=(X-1)(X-2)^2(X-3)
\end{align*}
\item[b)] Eine Matrix $B \in \R^{3\times 3}$ mit den Eigenwerten $\lambda_1= 1$, $\lambda_2=-1$ und $\lambda_3=4$ wird durch eine invertierbare Matrix $D$ mit einer (passend geordneten) Basis aus Eigenvektoren in die Diagonalgestalt
$$
D^{-1}BD=diag(1,-1,4)
$$
transformiert.\\
Durch die in der Aufgabenstellung angegebenen Eigenräume ist 
$$
D:=\begin{pmatrix} 1 & 0 & 0 \\ 1 & 1 & 2 \\ 0 & 2 & 1\end{pmatrix}
$$
solch eine geeignete Matrix.\\
Um $B$ berechnen zu können, muss man nun noch $D^-1$ bestimmen:
\begin{align*}
&\left(\begin{array}{ccc|ccc}
1 & 0 & 0 &1&0&0\\ 1 & 1 & 2 &0&1&0\\ 0 & 2 & 1&0&0&1
\end{array}\right)\stackrel{Z2-Z1}{\sim >}
\left(\begin{array}{ccc|ccc}
1 & 0 & 0 &1&0&0\\ 0 & 1 & 2 &-1&1&0\\ 0 & 2 & 1&0&0&1
\end{array}\right)\stackrel{Z3-2\cdot Z2}{\sim >}
\left(\begin{array}{ccc|ccc}
1 & 0 & 0 &1&0&0\\ 0 & 1 & 2 &-1&1&0\\ 0 & 0 & -3&2&-2&1
\end{array}\right)\\
&\stackrel{-\frac{1}{3}\cdot Z3}{\sim >}
\left(\begin{array}{ccc|ccc}
1 & 0 & 0 &1&0&0\\ 0 & 1 & 2 &-1&1&0\\ 0 & 0 & 1&-\frac{2}{3}&\frac{2}{3}&-\frac{1}{3}
\end{array}\right)\stackrel{Z2-2\cdot Z3}{\sim >}
\left(\begin{array}{ccc|ccc}
1 & 0 & 0 &1&0&0\\ 0 & 1 & 0 &\frac{1}{3}&-\frac{1}{3}&-\frac{2}{3}\\ 0 & 0 & 1&\frac{2}{3}&\frac{2}{3}&-\frac{1}{3}
\end{array}\right)
\end{align*}
Damit können wir $B$ berechnen:
\begin{align*}
B&=D\cdot diag(1,-1,4)\cdot D^{-1}\\
&=\begin{pmatrix} 1 & 0 & 0 \\ 1 & 1 & 2 \\ 0 & 2 & 1\end{pmatrix}\cdot \begin{pmatrix} 1 & 0& 0 \\ 0 & -1 & 0\\ 0&0&4\end{pmatrix}\cdot  \begin{pmatrix}1&0&0\\\frac{1}{3}&-\frac{1}{3}&\frac{2}{3}\\ -\frac{2}{3}&\frac{2}{3}&-\frac{1}{3} \end{pmatrix}\\
&=\begin{pmatrix} 1 & 0 & 0 \\ 1 & 1 & 2 \\ 0 & 2 & 1\end{pmatrix}\cdot \begin{pmatrix} 1 & 0& 0 \\ -\frac{1}{3} & \frac{1}{3} & -\frac{2}{3}\\-\frac{8}{3}&\frac{8}{3}&-\frac{4}{3}\end{pmatrix}\\
&=\begin{pmatrix} 1 & 0 & 0 \\ -\frac{14}{3} & \frac{17}{3} & -\frac{10}{3} \\ -\frac{10}{3}& \frac{10}{3} & -\frac{8}{3}\end{pmatrix}
\end{align*}
\end{itemize}
%
%
%
%
%
%
\end{document}
