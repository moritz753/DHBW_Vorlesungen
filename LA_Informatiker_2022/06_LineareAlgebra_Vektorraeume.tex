\documentclass{beamer}

\usepackage{beamerthemesplit}

\usepackage{amsfonts}
\usepackage{amsmath}
\usepackage{amssymb}
\usepackage{amsthm}
\usepackage{amscd}

\usepackage{stmaryrd} 					%\lightning
\usepackage{algorithm2e}


\usepackage[ngerman]{babel}

\usepackage[utf8]{inputenc}
\usepackage[T1]{fontenc}
\usepackage{textcomp}


% Color Definitions
\definecolor{dhbwRed}{RGB}{226,0,26} 
\definecolor{dhbwGray}{RGB}{61,77,77}
\definecolor{lightBlue}{RGB}{28,134,230}

% Basic Theme
\usetheme{Malmoe}

% Color Re-Definitions
\usecolortheme[named=lightBlue]{structure}
\setbeamercolor*{alerted  text}{fg=dhbwRed, bg=white}
\setbeamercolor*{subsection in toc}{fg=dhbwGray, bg=white}

%\setbeamercolor*{palette primary}{fg=white,bg=lightBlue}
%\setbeamercolor*{palette secondary}{fg=white,bg=gray}
%\setbeamercolor*{palette tertiary}{fg=white,bg=gray}
%\setbeamercolor*{palette quaternary}{fg=white,bg=dhbwRed}

% no navigation symbols
\setbeamertemplate{navigation symbols}{}

% headline, footline
\setbeamertemplate{footline}{\color{dhbwGray} \hfill\insertframenumber\hspace{5mm}\vspace{2mm}}
\setbeamertemplate{headline}{}

% Title Page
\newcommand*{\makeTitlePage}{
	
	\begin{frame}[plain]
		
		\vfill
		\vfill
		\begin{center}
			{
				\usebeamerfont{title}
				\usebeamercolor[fg]{title}
				\Large
				\inserttitle
			}\\[3mm]
			{	
				\usebeamerfont{subtitle}
				\usebeamercolor[fg]{subtitle}
				\large
				\insertsubtitle
			}
		\end{center}
		%
		\vfill
		\vfill
		\vfill
		\vfill
		%
		\begin{columns}
			\begin{column}{0.5\textwidth}
				\begin{flushleft}
					{
						\usebeamerfont{normal text}
						\color{dhbwGray!80}
						\scriptsize
						Dr. Moritz Gruber\\
						DHBW Karlsruhe\\
						
					}
				\end{flushleft}
			\end{column}
			%
			\begin{column}{0.5\textwidth}
				\begin{flushright}
					\includegraphics[scale=0.06]{../DHBW.png}
				\end{flushright}
			\end{column}
		\end{columns}
		%
		\vspace{1mm}
		\begin{columns}
			\begin{column}{0.5\textwidth}
				\begin{flushleft}
					{
						\usebeamerfont{normal text}
						\color{dhbwGray!80}
						\scriptsize
						Version \today
					}
				\end{flushleft}
			\end{column}
			%
			\begin{column}{0.5\textwidth}
				% nothing (just a placeholder to be in line with the columns above
			\end{column}
		\end{columns}
	\end{frame}

}

% Section Divider Page
\newcommand*{\makeSectionDividerPage}{

	\begin{frame}[plain]
		\begin{center}
			\begin{flushleft}
				{				
					\usebeamercolor[fg]{frametitle}
					{\Large \insertsection} \\[3mm]
					{\large \insertsubsection}
				}
			\end{flushleft}
		\end{center}
        \end{frame}
	
}

% itemize
\setbeamertemplate{itemize items}[circle]
\setbeamertemplate{enumerate item}{(\theenumi)}




%--------------------------------------%
% Math ------------------------------%
%--------------------------------------%

% Mengen (Zahlen)
\newcommand{\N}{\mathbb{N}}
\newcommand{\Q}{\mathbb{Q}}
\newcommand{\R}{\mathbb{R}}
\newcommand{\Z}{\mathbb{Z}}
\newcommand{\C}{\mathbb{C}}

% Mengen (allgemein)
\newcommand{\K}{\mathbb{K}}
\newcommand\PX{{\cal P}(X)}

% Zahlentheorie
\newcommand{\ggT}{\mathrm{ggT}}


% Ableitungen
\newcommand{\ddx}{\frac{d}{dx}}
\newcommand{\pddx}{\frac{\partial}{\partial x}}
\newcommand{\pddy}{\frac{\partial}{\partial y}}
\newcommand{\grad}{\text{grad}}

%--------------------------------------%
% Layout Colors ------------------%
%--------------------------------------%
\newcommand*{\highlightDef}[1]{{\color{lightBlue}#1}}
\newcommand*{\highlight}[1]{{\color{lightBlue}#1}} % after theme for colours

%---------------------------------------------%
\title{Analysis \& Lineare Algebra}
\subtitle{Vektorräume}

%---------------------------------------------%
\begin{document}

%---------------------------------------------%
\makeTitlePage

%---------------------------------------------%
\begin{frame}\frametitle{Inhalt}
   \tableofcontents
\end{frame}
%

%---------------------------------------------%
% Folien -----------------------------------%
%---------------------------------------------%
%

%--------------------------------------------
\section{Vektorräume}
\makeSectionDividerPage
%%%
\subsection{Definition}
%
%
\begin{frame}\frametitle{Definition}

	Sei $K$ ein Körper\footnote{genauer: $(K, +_{K}, \cdot_{K})$ mit Einselement $1_{K}$}. 
	Ein \highlightDef{Vektorraum über dem Körper $K$} (oder \highlightDef{$K$-Vektorraum}) 
	ist eine abelsche Gruppe $(V,+)$, für die eine Abbildung
	$$
		\cdot: K\times V \to V,\,
		(a,v) \mapsto a\cdot v 
	$$ 
	mit folgenden Eigenschaften definiert ist:\\[1mm]
	
	$\forall a,b \in K, \forall u,v\in V$ gilt:
	\begin{itemize}
		\item[(1)] 
			$1_{K}\cdot v = v$.
		\item[(2)]
			$a\cdot(b\cdot v) = (a\cdot_{K} b)\cdot v$.
			(Assoziativgesetz)
		\item[(3)]
			$ (a+_{K}b)\cdot v = a\cdot v + b\cdot v$ \quad und\\
			$ a\cdot (u+v) = a\cdot u + a\cdot v.$  
			(Distributivgesetze)
	\end{itemize}
	
	\vfill 
	\pause
	Die Elemente $v\in V$ hei{\ss}en \highlightDef{Vektoren}, die Abbildung `` $\cdot$ '' heißt \highlightDef{Skalarmultiplikation}.
	
		\vfill \pause
	Mit $(u-v)$ ist immer der Vektor $u + (-v)$ gemeint.
	
\end{frame}
%
\subsection{Eigenschaften und Beispiele}
%
\begin{frame}\frametitle{Eigenschaften}
	
	Sei $(V,+,\cdot)$ ein $K$-Vektorraum.
	Dann gilt für alle $a\in K, v\in V$:
	\begin{itemize}
		\item[(1)] 
			$0_{K}\cdot v = 0_V$,\\[1mm]
			
			Hierbei:\\
			$0_{K}$: Neutrales Element bzgl. $+_{K}$ in $K$,\\
			$0_V$: Neutrales Element bzgl. $+$ in $V$.  
			\pause
		\item[(2)] 
			$a\cdot 0_V = 0_V$,
			\pause
		\item[(3)] 
			$(-1_{K})\cdot v = -v$.\\[1mm]
			
			Hierbei:\\
			$-1_{K}$: Inverses Element von $1_{K}$ bzgl. $+_{K}$, \\
			$-v$: Inverses Element von $v$ in Gruppe $(V,+)$.\\[2mm]						
	\end{itemize}
	
\end{frame}
%
%
\begin{frame}\frametitle{Beweis}
	
	Sei $(V,+,\cdot)$ ein $K$-Vektorraum.
	Dann gilt für alle $a\in K, v\in V$: \pause
	\begin{itemize}
		\item[(1)]
			\highlight{$0_{K}\cdot v = 0_V$:}
			$$
				0_{K}\cdot v =\pause (0_{K} + 0_{K})\cdot v =\pause 0_{K}\cdot v + 0_{K}\cdot v.
			$$
			Es folgt:
			$0_V = 0_{K}\cdot v.$		\pause
		\item[(2)] 
			\highlight{$a\cdot 0_V = 0_V$:}
			$$
				a\cdot 0_V =\pause a \cdot (0_V + 0_V) = \pause a\cdot 0_V + a\cdot 0_V.
			$$\pause
		\item[(3)] 
			\highlight{$(-1_{K})\cdot v = -v$:}
			$$
				(-1_{K})\cdot v + 1_{K}\cdot v =\pause \big((-1_{K}) + 1_{K}\big)\cdot v =\pause 0_{K}\cdot v = 0_V.
			$$
			Somit ist $v= 1_{K}\cdot v$ das Inverse von $(-1_{K})\cdot v$ bzgl. $+$ in $V$.
	\end{itemize}
	\qed	
	
\end{frame}
%
%
\begin{frame}\frametitle{Beispiel $\R^n$}
	
	Die Menge
	$$
		\R^n :=
		\left\{
			\begin{pmatrix}
				x_1\\ 
				\vdots\\
				x_n
			\end{pmatrix}
			~| ~
			x_1,\ldots, x_n \in \R
		\right\}
	$$ 
	ist zusammen mit der Verknüpfung
	$$
		\begin{pmatrix}
			x_1\\ 
			\vdots\\
			x_n
		\end{pmatrix}	
		+
		\begin{pmatrix}
			y_1\\ 
			\vdots\\
			y_n
		\end{pmatrix}
		:=
		\begin{pmatrix}
			x_1 + y_1\\ 
			\vdots\\
			x_n + y_n
		\end{pmatrix}							
	$$
	eine abelsche Gruppe.

\end{frame}
%
%
\begin{frame}\frametitle{Beispiel $\R^n$}
	
	$(\R^n,+)$ wird zusammen mit der Abbildung
	$$
		\cdot: \R\times\R^n \to \R^n,\quad
			a
			\cdot
			\begin{pmatrix}
				x_1\\ 
				\vdots\\
				x_n
			\end{pmatrix}
			:=
			\begin{pmatrix}
				ax_1\\ 
				\vdots\\
				ax_n
			\end{pmatrix}			
	$$ 
	zu einem Vektorraum $(\R^n,+,\cdot)$ über $\R$ (reeller Vektorraum).
	
	\pause
	\vspace{10mm}
	Analog ist für jeden Körper $K$ die Menge $K^n$ ein $K$-Vektorraum.\\[2mm]
	\pause
	Kryptographie, Codierungstheorie: $\Z_p^n$.\\ 
	($p$: Primzahl, $\Z_p$: Ring der Restklassen modulo $p$ ist ein Körper.)
	
\end{frame}
%
%
\begin{frame}\frametitle{Anschauliches Beispiel $\R^2$}

	\begin{center}
		\includegraphics[]{Grafiken/Vektoren/vektoren.pdf}
	\end{center}
	
	$$
		u = 
		\begin{pmatrix}
			1 \\
			2
		\end{pmatrix},
		\quad
		v = 
		\begin{pmatrix}
			3 \\
			1
		\end{pmatrix},	
		\quad
		u+v = 
		\begin{pmatrix}
			1+3 \\
			2+1
		\end{pmatrix}= 
		\begin{pmatrix}
			4 \\
			3
		\end{pmatrix}.				
	$$
	
\end{frame}
%--------------------------------------------
\section{Untervektorräume}
\makeSectionDividerPage
%%%
\subsection{Definition und Beispiele}
%
\begin{frame}\frametitle{Definition}
Es sei $K$ ein Körper und $V$ eine $K$-Vektorraum. Ein \highlightDef{Untervektorraum} von $V$ ist eine Teilmenge $U \subseteq V$, die bezüglich der Addition eine Untergruppe von $V$ ist und für die gilt:
$$
\forall a \in K, u\in U: a \cdot u \in U.
$$
\pause
$U$ ist dann selbst ein $K$-Vektorraum. Um einen Untervektorraum von beliebigen Teilmengen zu unterscheiden schreibt man oft $U\le V$.
\end{frame}
%
%
\begin{frame}\frametitle{Beispiele}
\begin{itemize}
\item Jeder Vektorraum $V$ ist ein Untervektorraum von sich selbst.\pause
\item Für jeden Vektorraum $V$ ist der \highlightDef{Nullraum} $\{0_V\}$ ein Untervektorraum von $V$.\pause
\item Für den 3-dimensionalen reellen Vektorraum $\R^3$ ist 
$$
U:=\{\begin{pmatrix} x \\ y \\ 0 \end{pmatrix} \mid x,y \in \R\}
$$
ein 2-dimensionaler Untervektorraum.\pause
\item Die Teilmenge
$$
M:=\{\begin{pmatrix} x \\ y \\ 1 \end{pmatrix} \mid x,y \in \R\}
$$\pause
ist \underline{kein} Untervektorraum von $\R^3$.
\end{itemize}
\end{frame}
%
%
\begin{frame}\frametitle{Untervektorraumkriterium}
Es sei $K$ ein Körper und $V$ eine $K$-Vektorraum, und $U \subseteq V$. Dann ist $U$ genau dann ein Untervektorraum, wenn folgendes gilt:
\begin{itemize}
\item[i)] $U \ne \emptyset$,
\item[ii)] $\forall u_1,u_2 \in U: u_1+u_2 \in U$,
\item[iii)] $\forall a \in K\ \forall u \in U: a\cdot u \in U$.
\end{itemize} \pause
\highlightDef{Beweis:}\\
Wenn $U\le V$, dann sind offensichtlich alle drei Eigenschaften erfüllt.\pause Es gilt also zu zeigen, dass aus den drei Eigenschaften folgt, dass $U$ ein Untervektorraum von $V$ ist.\\\pause
Mit $a=-1\in K$ und iii) folgt für alle $u\in U$, dass auch $-u \in U$. Und mit zusätzlich ii) erhalten wir $\forall u_1,u_2 \in U: u_1+(-u_2)\in U$.\pause \\Da $U\ne \emptyset$ erhalten wir mit dem \highlightDef{Untergruppenkriterium}, dass $U$ bzgl. der Addition eine Untergruppe von $V$ ist. Zusammen mit iii) ist das die Definition eines Untervektorraums. \hfill $\square$
\end{frame}
%
%
\begin{frame}\frametitle{Beispiel}
Wir nutzen das Untervektorraumkriterium um zu zeigen, dass der Schnitt zweier Untervektorräume wieder ein Untervektorraum ist:
$$
U,W \le V \ \Longrightarrow \ U \cap W \le V
$$
\highlightDef{Beweis mit Hilfe des UVR-Kriteriums:}\\ \pause
\begin{itemize}
\item[i):] $U \cap W \ne \emptyset$, da $0_V \in U$ und $0_V \in W$. \pause
\item[ii):] Für alle $v_1,v_2 \in U \cap W$ gilt insbesondere $v_1,v_2 \in U$ und $v_1,v_2 \in W$ und da beide UVRe von $V$ sind, gilt auch $v_1+v_2 \in U$ und $v_1+v_2 \in W$. Damit folgt $v_1+v_2 \in U \cap W$. \pause
\item[iii):] Für alle $v \in U\cap W$ gilt insbesondere $v\in U$ und $v \in W$. Da $U$ und $W$ UVRe von $V$ sind, gilt für alle $a \in K$ auch $a\cdot v \in U$ und $a \cdot v \in W$ und somit $a\cdot v \in U \cap W$.
\end{itemize}\pause
Mit dem Untervektorraumkriterium folgt nun die Behauptung. \hfill $\square$
\end{frame}
%
%
%--------------------------------------------
\section{Lineare Unabhängigkeit \& Basis}
\makeSectionDividerPage
%%%
\subsection{Definitionen}
%
\begin{frame}\frametitle{Linearkombination}
	
	\vspace{2mm}
	Seien $V$ ein $K$-Vektorraum, $v_1,\ldots, v_k \in V$ und $a_1,\ldots, a_k \in K$. \\[1mm]
	
	Die Summe
	$$
		\sum_{j=1}^k a_j\cdot v_j = a_1\cdot v_1 + \ldots + a_k\cdot v_k
	$$ 
	nennt man eine \highlightDef{Linearkombination} der Vektoren $v_1, \ldots, v_k$.
	
	\pause
	\vspace{3mm}
	\highlight{Beispiele ($V=\R^2$)}\\[1mm]
	
	Für 
	$$
		v = \begin{pmatrix}
				3\\
				-1
			\end{pmatrix},
		\,
		v_1 = \begin{pmatrix}
				1\\
				1
			\end{pmatrix},
		\,
		v_2 = \begin{pmatrix}
				-1\\
				1
			\end{pmatrix},
		\,	
		v_3 = \begin{pmatrix}
				-1\\
				-1
			\end{pmatrix},
		\,
		0 = \begin{pmatrix}
				0\\
				0
			\end{pmatrix}	
	$$
	gilt:
	\begin{align*}
		v	&= v_1 -2\cdot v_2,\\
		0	&= v_1 + v_3	= 2\cdot v_1 + 2\cdot v_3 = 0\cdot v_1 + 0\cdot v_3.	
	\end{align*}
	
\end{frame}
%
%
\subsection{Lineare Hülle}
%
%
\begin{frame}\frametitle{Definition: Lineare Hülle}
Es sei $V$ ein $K$-Vektorraum und $M \subseteq V$ eine Teilmenge. Die Menge
$$
\langle M \rangle := \{v \mid v \text{ ist eine Linearkombination von Vektoren aus } M\}
$$
nennt man die \highlightDef{lineare Hülle von M}. \\\pause
$\langle M \rangle$ ist der kleinste Untervektorraum von $V$ der $M$ als Teilmenge enthält.
\vfill\pause
\highlightDef{Beispiel}\\
Für $V=\R^3$ und  $M=\{\begin{pmatrix} 1 \\ 0 \\ 0 \end{pmatrix},\begin{pmatrix} 0 \\ 1 \\ 0 \end{pmatrix} \}$ ist \pause
$$
\langle M \rangle= \pause\{x\cdot \begin{pmatrix} 1 \\0\\ 0 \end{pmatrix} +y \cdot \begin{pmatrix} 0 \\1\\ 0 \end{pmatrix}\mid x,y \in \R \}=\pause \{\begin{pmatrix} x \\ y \\ 0 \end{pmatrix}\mid x,y \in \R \}
$$
\end{frame}
%
%
\begin{frame}\frametitle{Lineare Unabhängigkeit}
	
	Sei $V$ ein $K$-Vektorraum. 
	Die Vektoren $v_1,\ldots, v_k \in V$ hei{\ss}en \highlightDef{linear unabhängig}, wenn die einzige Möglichkeit den Nullvektor als Linearkombination darzustellen, die triviale ist,\pause d.h.
	für $a_1,\ldots, a_k\in K$  gilt
	$$
		\left(\sum_{j=1}^k a_j\cdot v_j = a_1\cdot v_1 + \ldots + a_k\cdot v_k = 0 \right) \ \Leftrightarrow \ \left( a_1=\ldots = a_k = 0 \right)
	$$
	
	\pause
	\highlight{Beispiel}\\[1mm]
	
	Die Vektoren
	$$
		v_1 = \begin{pmatrix}
				1\\
				1
			\end{pmatrix},
		\,
		v_2 = \begin{pmatrix}
				-1\\
				1
			\end{pmatrix}
		\in \R^2
	$$
	sind linear unabhängig.
	
	\pause
	\vspace{3mm}
	Vektoren $v_1,\ldots, v_k \in V$, die nicht linear unabhängig sind, nennt man \highlightDef{linear abhängig}.
		
\end{frame}
%
%
\begin{frame}\frametitle{Basis}
	
	Sei $V$ ein $K$-Vektorraum.
	Eine \textit{maximale} Menge linear unabhängiger Vektoren $B=\{b_1,\ldots, b_k\}$ hei{\ss}t
	\highlightDef{Basis} von $V$.\\
	
	\vspace{10mm}
	\pause
	Mit \textit{maximal} ist gemeint, dass nach Hinzunahme eines beliebigen weiteren Vektors $v\in V$, die Vektoren
	$$
		b_1,\ldots, b_k, v
	$$
	linear abhängig sind.
	
\end{frame}
%
%
\begin{frame}\frametitle{Basis}
	
	Seien $V$ ein $K$-Vektorraum und $B=\{b_1,\ldots, b_k\}$ eine Basis von $V$.\\[1mm]
	Dann lässt sich jeder Vektor $v\in V$ als Linearkombination 
	$$
		v = a_1\cdot b_1 + \ldots + a_k\cdot b_k
	$$
	der Basisvektoren schreiben und die Koeffizienten $a_1,\ldots, a_k\in K$ sind \highlight{eindeutig} bestimmt.
	Insbesondere gilt $\langle B \rangle = V$.

	\pause
	\vspace{3mm}
	\highlight{Beweis}
	
	\begin{itemize}
		\item
			Angenommen, $b$ lässt sich nicht als Linearkombination von $b_1,\ldots, b_k$ schreiben. \\[1mm]
			\pause
			Dann wären $b_1,\ldots, b_k, b$ linear unabhängige Vektoren.\\[1mm] 
			\pause
			Ein Widerspruch zur Voraussetzung, dass $\{b_1,\ldots, b_k\}$ eine Basis 
			(eine maximale Menge linear unabhängiger Vektoren) ist.
	\end{itemize}
	
\end{frame}
%
%
\begin{frame}\frametitle{Basis}
	
	Seien $V$ ein $K$-Vektorraum und $\{b_1,\ldots, b_k\}$ eine Basis von $V$.\\[1mm]
	Dann lässt sich jeder Vektor $v\in V$ als Linearkombination 
	$$
		v = a_1\cdot b_1 + \ldots + a_k\cdot b_k
	$$
	der Basisvektoren schreiben und die Koeffizienten $a_1,\ldots, a_k\in K$ sind \highlight{eindeutig} bestimmt.

	\vspace{3mm}
	\highlight{Beweis (Fortsetzung)}
	
	\begin{itemize}
		\item \highlight{Eindeutigkeit:} 
			Seien $a_1,\ldots, a_k, c_1,\ldots, c_k \in K$ mit
			$$
				v = a_1\cdot b_1 + \ldots + a_k\cdot b_k = c_1\cdot b_1 + \ldots + c_k\cdot b_k.
			$$
			\pause
			Es folgt:
			$$
				0_V =v-v= (a_1-c_1)\cdot b_1 + \ldots + (a_k-c_k)\cdot b_k
			$$
			und somit 
			$$
				a_1=c_1, \ldots, a_k=c_k,
			$$
			da die Vektoren $b_1,\ldots, b_k$ linear unabhängig sind.
			\qed
	\end{itemize}
	
\end{frame}
%
\subsection{Koordinaten und Dimension}
%
\begin{frame}\frametitle{Koordinaten}

	Seien $V$ ein $K$-Vektorraum und $\{b_1,\ldots, b_k\}$ eine Basis von $V$.
	Die eindeutigen Koeffizienten $a_1,\ldots, a_k\in K$ in der Linearkombination
	$$
		v = a_1\cdot b_1 + \ldots + a_k\cdot b_k
	$$
	für $v\in V$ hei{\ss}en \highlightDef{Koordinaten} von $v$ bzgl. der Basis $\{b_1,\ldots, b_k\}$.
	
\end{frame}
%
%
\begin{frame}\frametitle{Beispiel}
	
	\vspace{1mm}
	Die Vektoren
	$$
		b_1 = \begin{pmatrix}
				1\\
				1
			\end{pmatrix},
		\quad
		b_2 = \begin{pmatrix}
				-1\\
				1
			\end{pmatrix}
		\quad
		\in \R^2
	$$
	bilden eine Basis des $\R^2$. \pause
	Für 
	$
		v := 
		\begin{pmatrix}
			-1\\
			3
		\end{pmatrix}
	$
	gilt
	$$
		v = 
		1\cdot b_1 + 2\cdot b_2.
	$$
	\pause
	Bzgl. der Basis $\{b_1, b_2\}$ hat $v$ somit die Koordinaten $1$ und $2$.\\[3mm]
	
	\begin{center}
		\includegraphics[scale=0.7]{Grafiken/Koordinaten/koordinaten.pdf}
	\end{center}
	
	
\end{frame}
%
\subsection{Dimension}
%
\begin{frame}\frametitle{Dimension}
	
	Sei $V$ ein Vektorraum. 
	Die maximale Anzahl von linear unabhängigen Vektoren in $V$ wird als \highlightDef{Dimension $\dim(V)$} von $V$ bezeichnet.\\
	\vfill
	\pause
	Man kann zeigen, dass alle maximalen Mengen von linear unabhängigen Vektoren eines Vektorraums (= Basen) die gleiche Kardinalität haben, nämlich die Dimension von $V$.\pause
	\vfill
	Außerdem ist die Dimension eines Vektorraums $V$ auch die minimale Kardinalität einer Teilmenge $M \subseteq V$ mit $\langle M \rangle=V$.
	\vfill
	\pause
	Es gibt \highlight{endlichdimensionale} und \highlight{unendlichdimensionale} Vektorräume. \\[1mm]
	Wir beschränken uns (in der Regel) auf endlichdimensionale Vektorräume.
	
\end{frame}
%
%
\begin{frame}\frametitle{Standardbasis in $V=\R^n$}
	
	Die Vektoren
	$$
		b_1 = \begin{pmatrix} 1\\ 0 \\ 0\\ \vdots \\ 0 \end{pmatrix},
		b_2 = \begin{pmatrix} 0\\ 1 \\ 0\\ \vdots \\ 0 \end{pmatrix},
		\ldots,
		b_n = \begin{pmatrix} 0\\ 0 \\ 0\\ \vdots \\ 1 \end{pmatrix} \in \R^n
	$$	
	bilden eine Basis des $\R^n$ (\"Ubungsaufgabe). Diese Basis hei{\ss}t \highlightDef{Standardbasis}.\\[1mm]
	
	Somit: $\dim(\R^n) = n$.
	
\end{frame}
%

%

%

\end{document}