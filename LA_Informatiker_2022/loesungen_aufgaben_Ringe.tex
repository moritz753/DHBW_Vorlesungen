\documentclass[
				a4paper,
				10pt
			]
			{scrartcl}

\parindent0mm

\usepackage{amsfonts}
\usepackage{amsmath}
\usepackage{amssymb}
\usepackage{amsthm}
\usepackage[ngerman]{babel}
\usepackage{graphicx}
\usepackage{algorithm2e}

\usepackage[
			pdftex,
			colorlinks,
			breaklinks,
			linkcolor=blue,
			citecolor=blue,
			filecolor=black,
			menucolor=black,
			urlcolor=black,
			pdfauthor={Andreas Weber},
			pdftitle={Aufgaben zur Logik und Algebra},
			plainpages=false,
			pdfpagelabels,
			bookmarksnumbered=true
		   ]{hyperref}


 %%%%%%%%%%%%%%%Schriften%%%%%%%%%%%%%
\DeclareMathAlphabet{\lier}{U}{eur}{m}{n}  %% Gothisch/Fraktur - Roman



\newcommand{\M}{\mathbb{M}}
\newcommand{\E}{\mathbb{E}}
\newcommand{\Hy}{\mathbb{H}}
\newcommand{\N}{\mathbb{N}}
\newcommand{\Q}{\mathbb{Q}}
\newcommand{\R}{\mathbb{R}}
\newcommand{\Z}{\mathbb{Z}}
\newcommand{\C}{\mathbb{C}}
\newcommand{\K}{\mathbb{K}}
\newcommand*\ee{\mathrm{e}}
\newcommand*\e{\mathrm{e}}

\newcommand*\ii{\mathrm{i}}
\newcommand*\re{\mathrm{Re}}
\newcommand*\im{\mathrm{Im}}
\newcommand*\id{\mathrm{id}}
\newcommand*\rang{\mathrm{rang}}
\newcommand*\grad{\mathrm{grad}}
\newcommand*\dive{\mathrm{div}}
\newcommand*\sym{\mathrm{Sym}}
\newcommand*\spur{\mathrm{Spur}}
\newcommand*\isom{\mathrm{Isom}}
\newcommand*\vol{\mathrm{vol\,}}
\newcommand*\supp{\mathrm{supp}}
\newcommand*\inj{\mathrm{inj}}
\newcommand*\rank{\mathrm{rank}}
\newcommand*\qrank{\Q\mbox{-}\mathrm{rank}}
\newcommand*\rrank{\R\mbox{-}\mathrm{rank}}
\newcommand*\dom{\mathrm{dom}}
\newcommand*\tr{\mathrm{tr}}
\newcommand*\spa{\mathrm{span}}
\newcommand*\diam{\mathrm{diam}}

\newcommand*\ric{\mathrm{Ric}}

\newcommand*\con{\mathrm{con}}
\newcommand*\dis{\mathrm{dis}}

\newcommand\PX{{\cal P}(X)}
\newcommand\T{{\cal T}}
\newcommand\B{{\cal B}}



\newcommand\scp{\langle\cdot,\cdot\rangle}     %% Metric

%%%%%%%%%%%%Tilde%%%%%%%
\newcommand\tx{\tilde{x}}
\newcommand\ty{\tilde{y}}
\newcommand\tu{\tilde{u}}
\newcommand\tk{\tilde{k}}
\newcommand\td{\tilde{d}}
\newcommand\tD{\tilde{D}}
\newcommand\tX{\tilde{X}}
\newcommand\tY{\tilde{Y}}
\newcommand\tZ{\tilde{Z}}


%%%%%%Lie-Gruppen%%%%%%%%%
\newcommand\ad{\mathrm{ad}}
\newcommand\Ad{\mathrm{Ad}}
\newcommand{\kak}{K\exp\overline{\lier{a}^+}K}              %%%%Cartan-Zerlegung
\newcommand*\Rang{\mathrm{Rang}}
\newcommand*\glnr{\mathrm{\it GL}(n,\R)}
\newcommand*\glnc{\mathrm{\it GL}(n,\C)}
\newcommand*\slnr{\mathrm{\it SL}(n,\R)}
\newcommand*\on{\mathrm{\it O}(n)}
\newcommand*\son{\mathrm{\it SO}(n)}
\newcommand*\SLzr{\mathrm{\it SL}(2,\R)}
\newcommand*\SOzr{\mathrm{\it SO}(2,\R)}

%%%%%%%%%%%%%Algebraische Gruppen
\newcommand\bG{{\bf G}}
\newcommand\bT{{\bf T}}
\newcommand\bP{{\bf P}}
\newcommand\bN{{\bf N}}
\newcommand\bL{{\bf L}}
\newcommand\bS{{\bf S}}
\newcommand\bM{{\bf M}}

\newcommand\Mor{\mathrm{Mor}}



%%%%%%Geometry%%%%%%%%%%%
\newcommand{\Si}{\mathcal{S}}


%%%%%%Ableitungsoperatoren%%%%%%%%%%%
\newcommand*\pddt{\frac{\partial}{\partial t}}
\newcommand*\pddx{\frac{\partial}{\partial x}}
\newcommand*\pddxio{\frac{\partial}{\partial x^i}}
\newcommand*\pddxjo{\frac{\partial}{\partial x^j}}
\newcommand*\pddxko{\frac{\partial}{\partial x^k}}
\newcommand*\pddxlo{\frac{\partial}{\partial x^l}}

\newcommand*\pddy{\frac{\partial}{\partial y}}
\newcommand*\pddyio{\frac{\partial}{\partial y^i}}
\newcommand*\pddyjo{\frac{\partial}{\partial y^j}}
\newcommand*\pddyko{\frac{\partial}{\partial y^k}}
\newcommand*\pddylo{\frac{\partial}{\partial y^l}}

\newcommand*\pddyq{\frac{\partial^2}{\partial y^2}}
\newcommand*\pddyj{\frac{\partial}{\partial y_j}}
\newcommand*\pddyjq{\frac{\partial^2}{\partial y_j^2}}
\newcommand*\pddxiq{\frac{\partial^2}{\partial x_i^2}}
\newcommand*\pddxi{\frac{\partial}{\partial x_i}}
\newcommand*\ddt{\frac{d}{dt}}

\newcommand*\dx{\,dvol(x)}
\newcommand*\dy{\,dvol(y)}
\newcommand*\dty{\,dvol(\ty)}

\newcommand*\DMp{\Delta_{M,p}}                 %%%%Laplace-Operatoren
\newcommand*\DMq{\Delta_{M,q}}
\newcommand*\DM{\Delta_M}
\newcommand*\DX{\Delta_X}
\newcommand*\DXp{\Delta_{X,p}}
\newcommand*\DXq{\Delta_{X,q}}
\newcommand*\DAx{\Delta_{Ax_0}}
\newcommand*\Rad{\mathrm{Rad}}
\newcommand*\DXps{\Delta^{\#}_{X,p}}

\newcommand*\eDXps{\e^{-t(\Delta^{\#}_{X,p}-c)}} %%%%%% Semigroups
\newcommand*\LpsX{L^p_{\#}(X)}


%%%%%%%%%%%%%%%Komplexe Analysis
\newcommand*\Res{\mathrm{Res}}

%%%%%%%%%%%%%%%Definitionsmenge
\newcommand*\D{{\cal D}}







\author{Dr. Moritz Gruber\\ DHBW Karlsruhe}
\title{L\"osungen \"Ubungsaufgaben 4\\ 
	Ringe
}
\date{}

%%%%%%%%%
\begin{document}
%%%%%%%%%
\maketitle

%%%
\section{Ringe}
%%%

Sei
$$
	R := \{ a + b\sqrt{2} ~|~a,b \in \Z \}. 
$$
Zeigen Sie, dass $(R,+,\cdot)$ mit der \"ublichen Addition $+$ und Multiplikation $\cdot$ ein kommutativer Ring mit Einselement ist.


%%%
\paragraph{L\"osung:}
%%%

Addition und Multiplikation definieren eine Verkn\"upfung auf $R$:\\ 
Seien $a_1,a_2,b_1,b_2 \in \Z$.
$$
	(a_1 + b_1\sqrt{2}) + (a_2 + b_2\sqrt{2}) = (a_1+ a_2) + (b_1+ b_2)\sqrt{2}	\in R,
$$
$$
	(a_1 + b_1\sqrt{2}) \cdot (a_2 + b_2\sqrt{2}) = (a_1a_2 + 2b_1b_2) + (a_1b_2 + a_2b_1)\sqrt{2}	\in R.
$$

\begin{itemize}
	\item[(1)] $(R,+)$ ist eine abelsche Gruppe:
			\begin{itemize}
				\item[(a)] $+$ ist auf $R$ assoziativ \& kommutativ, da $+$ schon auf $\R$ assoziativ \& kommutativ ist.
						($R$ ist eine Teilmenge von $\R$!)
				\item[(b)] Das neutrale Element (Nullelement) ist $0 = 0 + 0\cdot\sqrt{2} \in R$.
				\item[(c)] Das inverse Element von $a + b\sqrt{2} \in R$  ist $(-a) + (-b)\sqrt{2} \in R$.
			\end{itemize}
	\item[(2)] Die Verkn\"upfung $\cdot$ ist assoziativ \& kommutativ, da $\cdot$ schon auf $\R$ assoziativ und kommutativ ist.
	\item[(3)] Die Distributivgesetze gelten in $(\R,+,\cdot)$ somit auch in $(R, +, \cdot)$.
	\item[(4)] Das Einselement (das neutrale Element bzgl. $\cdot$) ist $1 = 1 + 0\cdot\sqrt{2} \in R.$
\end{itemize}



%%%
\section{``Kleine'' Ringe *}
Es sei $n \in \N$. Wir notieren f\"ur $a \in \Z$ mit
$$
[a]_n:=\{m \in \Z \mid \exists k\in \Z: m=a+kn\}
$$
die Menge aller Zahlen, die beim Teilen durch $n$ den gleichen Rest wie $a$ haben.

\begin{itemize}
\item[(a)] Zeigen Sie, dass aus $b \in [a]_n$ folgt, dass $[a]_n=[b]_n$.
\end{itemize}
Wir definieren die Menge
$
\Z_n := \{[a]_n \mid a \in \Z\}
$
\begin{itemize}
\item[(b)] Zeigen Sie, dass $\Z_n:=\{[0]_n,[1]_n,[2]_n,...,[n-1]_n\}$ gilt.
\end{itemize}
Wir definieren weiter die Verkn\"upfungen $[a]_n+_n[b]_n=[a+b]_n$ und $[a]_n\cdot_n [b]_n:=[a\cdot b]_n$.
\begin{itemize}
\item[(c)] Zeigen Sie, dass $(\Z_n,+_n,\cdot_n)$ ein Ring ist.
\end{itemize}

%%%
\paragraph{L\"osung:}
%%%
\begin{itemize}
\item[(a)]
Da $b \in [a]_n$ gilt, gibt es ein $k_b\in \Z$ sodass $b=a+k_b n$. Sei nun $m \in [a]_n$ beliebig. Dann gibt es ein $k_m \in \Z$ sodass $m=a+k_m n$. Damit folgt:
$$
m=a+k_m n = a+k_m n + (b- (a+k_b n))=b + (a+k_m n -a- k_b n) =b + (k_m-k_b)n
$$
Mit $k:=k_m -k_b$ folgt also $m=b+kn$ und damit $m \in [b]_n$. Also gilt $[a]_n \subseteq [b]_n$.\\
Da insbesondere $a \in [a]_n$ und somit $a\in [b]_n$ gilt, folgert man mit vertauschten Rollen von $a$ und $b$ nun $[b]_n \subseteq [a]_n$ und damit $[a]_n=[b]_n$.

\item[(b)]
Es sei $m \in \Z$ beliebig. Dann gibt es ein $k \in \Z$ sodass $0\le m-kn \le n-1$. Wir definieren $a:=m-kn$. Dann gilt $m=a+kn$ und damit $m \in [a]_n$. Mit Aufgabenteil (b) folgt $[m]_n=[a]_n$. Und da $a \in \{0,1,2,...,n-1\}$ folgt somit die Behauptung.

\item[(c)]
Wir zeigen zuerst (i) $(\Z_n,+_n)$ ist eine abelsche Gruppe und dann (ii) die Verkn\"upfungen $+_n$ und $\cdot_n$ erf\"ullen die Distributivit\"at.
\begin{itemize}
\item[(i)]
Das neutrale Element ist $[0]_n$, da 
$$[0]_n+_n[a]_n=[0+a]_n=[a]_n=[a+0]_n=[a]_n+_n[0]_n$$
f\"ur alle $[a]_n \in \Z_n$.\\
F\"ur jedes Element $[a]_n \in \Z_n$ ist $[n-a]_n$ das Inverse, denn 
$$[a]_n+_n[n-a]_n=[a+n-a]_n=[n]_n=[0]_n=[n]_n=[n-a+a]_n=[n-a]_n+_n[a]_n$$
Die Assoziativ\"at und die Kommutativit\"at folgt direkt aus den entsprechenden Eigenschaften von $+$ in $\Z$.
\item[(ii)]
Es seien $[a]_n,[b]_n,[c]_n \in \Z_n$, dann gilt mit der Distributivit\"at von $(\Z,+,\cdot)$:
\begin{align*}
[a]_n \cdot_n ([b]_n +_n [c]_n)&=[a]_n\cdot_n [b+c]_n=[a\cdot (b+c)]_n=[ab+ac]_n=[ab]_n+_n[ac]_n\\
&=([a]_n\cdot_n[b]_n) +_n ([a]_n\cdot_n[c]_n)
\end{align*}
\end{itemize}
Somit ist $(\Z_n,+_n,\cdot_n)$ ein Ring.

\end{itemize}

\newpage
%%%
\section{Polynome}
%%%

Geben Sie alle Polynome in $\Z_3[X]$ mit Grad $1$ an. 

%%%
\paragraph{L\"osung:}
%%%

Als Koeffizienten kommen nur die Elemente aus $\Z_3$ in Frage:
$$
	[0], [1], [2].
$$
(F\"ur bessere \"Ubersicht verzichten wir auf den Index an den eckigen Klammern und den Verkn\"upfungen, wenn dieser aus dem Kontext klar ist.)\\

Ein Polynom mit Grad $1$ ist von der Form
$$
	a_1\cdot X + a_0	\qquad\text{mit}\qquad a_1\neq [0].
$$
Somit sind die folgenden Polynome alle Polynome mit Grad $1$ in $\Z_3[X]$:
\begin{eqnarray*}
	f_1	&=	& [1]\cdot X + [0]	\\
	f_2	&=	& [1]\cdot X + [1]	\\
	f_3	&=	& [1]\cdot X + [2]	\\
	f_4	&=	& [2]\cdot X + [0]	\\
	f_5	&=	& [2]\cdot X + [1]	\\
	f_6	&=	& [2]\cdot X + [2]	\\
\end{eqnarray*}

%%%
%%%
\section{Nullteiler und Einheiten}
%%%
Bestimmen Sie alle Nullteiler und alle Einheiten der folgenden Ringe:
\begin{itemize}
\item[i)] $\Z_6$
\item[ii)] $\Q[X]$
\item[iii)] $\R$
\end{itemize}
%%%
\paragraph{L\"osung:}
%%%
\begin{itemize}
\item[i)] $\Z_6:$\\
Wir erstellen eine Verkn\"upfungstafel von $(\Z_6, \cdot_6)$, lassen aber die Zeile und Spalte der Null weg, da dort immer nur die Null stehen wird.
$$
\begin{array}{c|c c c c c}
(\Z_6, \cdot_6)& [1]_6 & [2]_6 & [3]_6 & [4]_6 &[5]_6\\
\hline
\ [1]_6 & [1]_6 & [2]_6 & [3]_6 & [4]_6 & [5]_6\\
\ [2]_6 & [2]_6 & [4]_6 & [0]_6 & [2]_6 & [4]_6\\
\ [3]_6 & [3]_6 & [0]_6 & [3]_6 & [0]_6 & [3]_6\\
\ [4]_6 & [4]_6 & [2]_6 & [0]_6 & [4]_6 & [2]_6\\
\ [5]_6 & [5]_6 & [4]_6 & [3]_6 & [2]_6 & [1]_6
\end{array}
$$
Jedes Element in dessen Zeile das neutrale Element $[1]_6$ auftaucht ist eine Einheit, alle andern Elemente sind keine Einheiten, da keine Inverse für diese in $\Z_6$ existieren. Damit: $\Z_6^{\times}=\{[1]_6,[5]_6\}$.\\
\quad\\
Jedes Element in dessen Zeile das Nullelement $[0]_6$ auftaucht ist ein Nullteiler, alle andern Elemente sind keine Nullteiler. Damit sind die Nullteiler von $\Z_6$ gerade $[2]_6,[3]_6$ und $[4]_6$.
\item[ii)] $\Q[X]:$\\
Die rationalen Zahlen $\Q$ sind nullteilerfrei, damit ist auch der Polynomring $\Q[X]$ nullteilerfrei. Damit gilt auch in der Grad-Formel Gleichheit: $$deg(f\cdot g)=deg(f)+deg(g)$$
Au\ss erdem gilt $deg(q)=0 \Leftrightarrow q \in \Q\setminus\{0\}$ und damit auch insbesondere $deg(1)=0$.\\
Zusammen mit der Grad-Formel folgt nun f\"ur alle $f,g \in \Q[X]$:
$$
f\cdot g=1 \Rightarrow deg(f\cdot g)=\deg(1)=0 \Rightarrow deg(f)=0=deg(g) \Rightarrow f,g \in \Q
$$
Damit gilt $\Q[X]^\times \subseteq \Q$. Da jedes Element $q \in \Q, q\ne 0$ ein Inverses $\frac{1}{q}\in\Q \subseteq \Q[X]$ besitzt, folgt $\Q[X]^\times=\Q^\times = \Q \setminus \{0\}$.
\item[iii)] $\R:$\\
In $\R$ besitzt jedes Element $x \in \R, x \ne 0$ ein Inverses: $\frac{1}{x} \in \R$. Somit sind die Einheiten von $\R$ gerade $\R^\times = \R \setminus \{0\}$\\
\quad\\
Da eine Einheit niemals eine Nullteiler ist, und auch die $0$ per Definition kein Nullteiler ist, ist $\R$ nullteilerfrei.
\end{itemize}

\end{document}