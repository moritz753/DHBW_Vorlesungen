\documentclass[a4paper,11pt]{scrartcl}

\usepackage{graphicx}
\usepackage[T1]{fontenc}
\usepackage[utf8]{inputenc}
\usepackage{floatflt}
\usepackage{amsfonts}
\usepackage{amsmath}
\usepackage{amssymb}
\usepackage{amsthm}
\usepackage[ngerman]{babel}
%\usepackage[T1]{fontenc}


\usepackage{paralist}
\usepackage[usenames,dvipsnames]{color} 



\newcounter{auf}
\newcommand{\Aufgabe}%
        {\addtocounter{auf}{1} \subsubsection*{\rmfamily  Aufgabe \theauf \hspace{1em}} }









\pagestyle{empty}

\oddsidemargin0mm
\parindent0mm
\parskip2mm
\textheight24cm
\textwidth15.8cm
\unitlength1mm

\newcommand{\E}{\mathbb{E}}
\newcommand{\Hy}{\mathbb{H}}
\newcommand{\N}{\mathbb{N}}
\newcommand{\RR}{\mathbb{R}}
\newcommand{\Z}{\mathbb{Z}}
\newcommand{\Q}{\mathbb{Q}}
\newcommand{\C}{\mathbb{C}}
\newcommand{\K}{\mathbb{K}}
\newcommand*\e{\mathrm{e}}
\newcommand*\ii{\mathrm{i}}
\newcommand*\re{\mathrm{Re}}
\newcommand*\im{\mathrm{Im}}
\newcommand*\id{\mathrm{id}}
\newcommand*\glnr{\mathrm{\it GL}(n,\R)}
\newcommand*\slnr{\mathrm{\it SL}(n,\R)}
\newcommand*\on{\mathrm{\it O}(n)}
\newcommand*\son{\mathrm{\it SO}(n)}
\newcommand*\rang{\mathrm{Rang}}
\newcommand*\grad{\mathrm{grad~}}
\newcommand*\dive{\mathrm{div~}}
\newcommand*\sym{\mathrm{Sym}}
\newcommand*\spur{\mathrm{Spur}}
\newcommand*\isom{\mathrm{Isom}}
                  





\begin{document}





\Aufgabe

\begin{enumerate}[a)]

\item Seien $a$ und $b$  Aussagen. 
Bestimmen Sie die Wahrheitstafel der Aussage
$$
	(\neg a) \lor (b \land a).
$$
\item Negieren Sie die Aussage \ $\forall x \in \N\ \exists y \in \Z\ \forall z \in \N: x+y \le z$.\vspace{2mm}\\
(Hinweis: $\neg(\forall x \in \N\ \exists y \in \Z\ \forall z \in \N: x+y \le z)$ ist keine ausreichende Lösung!)\\
\item Geben Sie die Definition der Potenzmenge ${\cal P}(M)$ einer Menge $M$ an und bestimmen Sie die Potenzmenge für $M = \{1,2,3,4\}.$
\end{enumerate}



%%%%%
\newpage
\Aufgabe
Es seien die Abbildungen $$f:\RR \to \RR^3, x \mapsto (x, 0, -3x)$$ und $$g: \RR^3 \to \RR, (x,y,z) \mapsto 2x+y+z$$ gegeben. 
\begin{enumerate}[a)]
\item Sind $f$ und $g$ injektiv? Begründen Sie Ihre Antwort.
\item Sind $f$ und $g$ surjektiv? Begründen Sie Ihre Antwort.
\item Bestimmen Sie $g \circ f$ und $f\circ g$.
\end{enumerate}



%%%%%
\newpage
\Aufgabe
Es sei 
$$K:=\{a+b\sqrt{3} \mid a,b \in  \Q \} \subset \RR$$
zusammen mit den Verknüpfungen $+$ und $\cdot$ wie in $\RR$ gegeben.
\begin{enumerate}[a)]
\item Geben Sie die Definition eines Körpers an.

\item Zeigen Sie, dass $(K,+,\cdot)$ ein Körper ist.

\item Bestimmen Sie das multiplikative Inverse zu $z=5+2\sqrt{3} \in K$. 

\end{enumerate}



%%%%%
\newpage
\Aufgabe

Es seien $K$ ein Körper, $V$ ein $K$-Vektorraum und $U_1$ und $U_2$
Untervektorräume von $V.$

\begin{enumerate}[a)]
\item Geben Sie die Definition eines $K$-Vektorraums an.
\item Geben Sie die Definition eines Untervektorraums an.
\item Zeigen Sie: $V=U_1\cup U_2  \ \Longleftrightarrow \ V=U_1\ \text{ oder }\ V=U_2.$
\end{enumerate}




%%%%%
\newpage
\Aufgabe

Es seien die Vektoren $b_1=\begin{pmatrix} 2\\1\\0\end{pmatrix}$, $b_2=\begin{pmatrix} 0\\1\\2\end{pmatrix}$ und $b_3=\begin{pmatrix} 1\\0\\1\end{pmatrix} \in \RR^3$ sowie die lineare Abbildung 
$
\Phi: \RR^3 \to \RR^2, v \mapsto \Phi(v)
$
mit 
$$
\Phi(b_1)=\begin{pmatrix} 4\\0\end{pmatrix}, \ \Phi(b_2)=\begin{pmatrix} 0\\4\end{pmatrix} \text{ und } \Phi(b_3)=\begin{pmatrix} 4\\4\end{pmatrix}
$$
gegeben.

\begin{enumerate}[a)]

\item Zeigen Sie, dass die Vektoren $b_1,b_2$ und $b_3$ linear unabhängig sind.
\item Bestimmen Sie die Abbildungsmatrix $A$ von $\Phi$ bezüglich der Standardbasis.
\item Berechnen Sie $\Phi(\begin{pmatrix} 3\\2\\1 \end{pmatrix})$.

\end{enumerate}


%%%%%
\newpage
\Aufgabe
\begin{enumerate}[a)]
\item Bestimmen Sie alle $t\in\RR$, für die das folgende lineare Gleichungssystem lösbar ist:
\[\begin{array}{rrrrrrrrr}
2x_1&+&4x_2&+&2x_3&=&12t\\
2x_1&+&12x_2&+&7x_3&=&12t+7\\
x_1&+&10x_2&+&6x_3&=&7t+8
\end{array}\]
\item Geben Sie die Lösungsmenge des obigen linearen Gleichungssystems für $t=-1$ an.
\end{enumerate}

%%%%%
\newpage
\Aufgabe
Berechnen Sie die Determinanten der folgenden Matrizen und geben Sie jeweils an, ob die Matrix invertierbar ist.\\
%
\hspace*{10mm} a) \ $A=\begin{pmatrix} 7 & 5 \\ 1 & 7 \end{pmatrix}$ \qquad
b) \ $B=\begin{pmatrix} 1 & \frac{4}{7} & 93 \\ 0 & 2 & \frac{6}{47} \\ 0 & 0& 3 \end{pmatrix}$ \qquad
c) \ $C=\begin{pmatrix} 1 & 1 &  2 & 1 \\ 0 & 2 & 0 & 1 \\ 0 & 0& 3 & 4 \\ 1 & -1 & 5 & 4 \end{pmatrix}$



%%%%%
\newpage
\Aufgabe
Sei $A=\begin{pmatrix} -\frac{1}{3} & 0 & -\frac{2}{3} \\ 0 & 3 & 0 \\ \frac{2}{3} & 0 &\frac{4}{3} \end{pmatrix}$. 
\begin{enumerate}[a)]
\item Geben Sie die Definition eines Eigenvektors einer Matrix $B \in \RR^{n\times n}$ an.
\item Bestimmen Sie das charakteristische Polynom $CP_A(X)$ und alle Eigenwerte von $A$.
\item Bestimmen Sie eine invertierbare Matrix $D \in \RR^{3 \times 3}$, sodass $D^{-1}AD$ eine Diagonalmatrix ist.
\end{enumerate}


\quad\\

\vfill \hfill \textbf{Viel Erfolg!}
%%%
%Lösungsanfang...\\
%\\Wir bestimmen das charakteristische Polynom $CP_A(X)=\det(A-X\cdot I_3)$:\\
%$
%det(A-X\cdot I_3)=\det(\begin{pmatrix} -\frac{1}{3}-X & 0 & -\frac{2}{3} \\ 0 & 3-X & 0 \\ \frac{2}{3} & 0 &\frac{4}{3}-X \end{pmatrix})$\\
%\hspace{24.5mm}$=(3-X)\cdot \det(\begin{pmatrix} -\frac{1}{3}-X &  -\frac{2}{3} \\ \frac{2}{3} &\frac{4}{3}-X \end{pmatrix})$\\
%\hspace{24.5mm}$=(3-X)\cdot ((-\frac{1}{3}-X)(\frac{4}{3}-X)-(-\frac{2}{3})(\frac{2}{3}))$\\
%\hspace{24.5mm}$=(3-X)\cdot(-\frac{4}{9}-X+X^2+\frac{4}{9})$\\
%\hspace{24.5mm}$=(3-X)\cdot(X^2-X)$\\
%\hspace{24.5mm}$=-(3-X)(1-X)X
%$




\end{document}