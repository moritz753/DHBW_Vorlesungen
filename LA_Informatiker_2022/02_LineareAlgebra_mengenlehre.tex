\documentclass{beamer}

\usepackage{beamerthemesplit}

\usepackage{amsfonts}
\usepackage{amsmath}
\usepackage{amssymb}
\usepackage{amsthm}
\usepackage{amscd}

\usepackage{stmaryrd} 					%\lightning
\usepackage{algorithm2e}


\usepackage[ngerman]{babel}

\usepackage[utf8]{inputenc}
\usepackage[T1]{fontenc}
\usepackage{textcomp}


% Color Definitions
\definecolor{dhbwRed}{RGB}{226,0,26} 
\definecolor{dhbwGray}{RGB}{61,77,77}
\definecolor{lightBlue}{RGB}{28,134,230}

% Basic Theme
\usetheme{Malmoe}

% Color Re-Definitions
\usecolortheme[named=lightBlue]{structure}
\setbeamercolor*{alerted  text}{fg=dhbwRed, bg=white}
\setbeamercolor*{subsection in toc}{fg=dhbwGray, bg=white}

%\setbeamercolor*{palette primary}{fg=white,bg=lightBlue}
%\setbeamercolor*{palette secondary}{fg=white,bg=gray}
%\setbeamercolor*{palette tertiary}{fg=white,bg=gray}
%\setbeamercolor*{palette quaternary}{fg=white,bg=dhbwRed}

% no navigation symbols
\setbeamertemplate{navigation symbols}{}

% headline, footline
\setbeamertemplate{footline}{\color{dhbwGray} \hfill\insertframenumber\hspace{5mm}\vspace{2mm}}
\setbeamertemplate{headline}{}

% Title Page
\newcommand*{\makeTitlePage}{
	
	\begin{frame}[plain]
		
		\vfill
		\vfill
		\begin{center}
			{
				\usebeamerfont{title}
				\usebeamercolor[fg]{title}
				\Large
				\inserttitle
			}\\[3mm]
			{	
				\usebeamerfont{subtitle}
				\usebeamercolor[fg]{subtitle}
				\large
				\insertsubtitle
			}
		\end{center}
		%
		\vfill
		\vfill
		\vfill
		\vfill
		%
		\begin{columns}
			\begin{column}{0.5\textwidth}
				\begin{flushleft}
					{
						\usebeamerfont{normal text}
						\color{dhbwGray!80}
						\scriptsize
						Dr. Moritz Gruber\\
						DHBW Karlsruhe\\
						
					}
				\end{flushleft}
			\end{column}
			%
			\begin{column}{0.5\textwidth}
				\begin{flushright}
					\includegraphics[scale=0.06]{../DHBW.png}
				\end{flushright}
			\end{column}
		\end{columns}
		%
		\vspace{1mm}
		\begin{columns}
			\begin{column}{0.5\textwidth}
				\begin{flushleft}
					{
						\usebeamerfont{normal text}
						\color{dhbwGray!80}
						\scriptsize
						Version \today
					}
				\end{flushleft}
			\end{column}
			%
			\begin{column}{0.5\textwidth}
				% nothing (just a placeholder to be in line with the columns above
			\end{column}
		\end{columns}
	\end{frame}

}

% Section Divider Page
\newcommand*{\makeSectionDividerPage}{

	\begin{frame}[plain]
		\begin{center}
			\begin{flushleft}
				{				
					\usebeamercolor[fg]{frametitle}
					{\Large \insertsection} \\[3mm]
					{\large \insertsubsection}
				}
			\end{flushleft}
		\end{center}
        \end{frame}
	
}

% itemize
\setbeamertemplate{itemize items}[circle]
\setbeamertemplate{enumerate item}{(\theenumi)}




%--------------------------------------%
% Math ------------------------------%
%--------------------------------------%

% Mengen (Zahlen)
\newcommand{\N}{\mathbb{N}}
\newcommand{\Q}{\mathbb{Q}}
\newcommand{\R}{\mathbb{R}}
\newcommand{\Z}{\mathbb{Z}}
\newcommand{\C}{\mathbb{C}}

% Mengen (allgemein)
\newcommand{\K}{\mathbb{K}}
\newcommand\PX{{\cal P}(X)}

% Zahlentheorie
\newcommand{\ggT}{\mathrm{ggT}}


% Ableitungen
\newcommand{\ddx}{\frac{d}{dx}}
\newcommand{\pddx}{\frac{\partial}{\partial x}}
\newcommand{\pddy}{\frac{\partial}{\partial y}}
\newcommand{\grad}{\text{grad}}

%--------------------------------------%
% Layout Colors ------------------%
%--------------------------------------%
\newcommand*{\highlightDef}[1]{{\color{lightBlue}#1}}
\newcommand*{\highlight}[1]{{\color{lightBlue}#1}} % after theme for colours

%----------------------------------------------------------------------------------------------------
%--------- Document Title ---------------------------------------------------------------------
\title{Lineare Algebra\\[3mm] 
	\large Mengenlehre und Abbildungen
}
\author{Dr. Moritz Gruber} 
\institute{DHBW Karlsruhe}
\date{2022}

%%%%%%%%%%%%%%
\begin{document}

\AtBeginSection[]{
	\begin{frame}				
		\usebeamercolor[fg]{frametitle}
		{\Large \insertsection} 
        \end{frame}
}

%
\begin{frame}[plain] 
 \titlepage
\end{frame}
%
%
\begin{frame}\frametitle{Inhalt}
   \tableofcontents
\end{frame}
%%%
\section{Elementare Mengentheorie}
%%%
\subsection{Definition \& Beispiele}
%%%
%
\begin{frame}\frametitle{Menge}

	\begin{itemize}
		\item<1-> Eine \highlightDef{ Menge} $M$ ist eine Ansammlung von Objekten.
		\item<2-> Objekte der Menge werden \highlightDef{ Elemente} von $M$ genannt.
		\item<3-> Ist $x$ Element der Menge $M$, so schreiben wir 
			$$
				x\in M.
			$$
		\item<3-> Ist $x$ nicht in $M$ enthalten, so schreibt man entweder $x \notin M$ oder (selten) $\neg(x\in M)$. 
		\item<4-> Ein wichtiges Beispiel ist die \highlightDef{ leere Menge} $\emptyset$. Das ist die Menge, die kein Element enthält, 
			d.h.
			$$
				\forall x: x\notin \emptyset.
			$$
		\item<5-> Die Anzahl der Elemente einer Menge $M$ hei{\ss}t \highlightDef{ Mächtigkeit} oder \highlightDef{ Kardinalität} von $M$.
			Notation: $|M|$ oder auch $\#M$.
	\end{itemize}

\end{frame}
%
%
\begin{frame}\frametitle{Beispiele}
	
	``Kleine'' Mengen gibt man durch Aufzählen aller Elemente an: 
	$$
		M_1 = \{1,2,9,11\}
	$$
	\pause
	Es gilt:
	\begin{itemize}
		\item $1 \in M_1$.
		\item $7 \notin M_1$.
		\item $|M_1| = 4$.
	\end{itemize}
	
	\pause
	\vfill
	Elemente einer Menge können selbst auch Mengen sein:
	$$
		M_2 = \big\{ \{1,2\}, \{3,4\}, \{7\}, \{1,2,3,4\} \big\}.
	$$
	Ist $7\in M_2$?
	\pause
	(nein)
\end{frame}
%
%
\begin{frame}\frametitle{Beispiele}
	
	Ist das Aufzählen der Elemente einer Menge umständlich oder unmöglich, 
	definiert man die Menge durch Angabe einer gemeinsamen Eigenschaft der Elemente:
	$$
		M := \big\{p ~|~ p \text{~ist eine Primzahl}\big\}.
	$$

	
\end{frame}
%
%
\begin{frame}\frametitle{Wichtige Mengen}

	Wichtige Mengen:
	\begin{itemize}
		\item $\N := \{1,2,3,\ldots \}$, die \highlightDef{ natürlichen Zahlen}.
		\item $\Z := \{\ldots, -3,-2,-1,0,1,2,3,\ldots\}$, die \highlightDef{ ganzen Zahlen}.
		\item $\Q := \{ \frac{p}{q}~ |~ p,q\in \Z, q\neq 0 \}$, die \highlightDef{ rationalen Zahlen}.
		\item $\R$, die \highlightDef{ reellen Zahlen}.
	\end{itemize}
	\vfill
	Je nach Lehrbuch/Mathematiker wird auch $0$ zu den natürlichen Zahlen gezählt. Wir machen das mit $\N_0$ kenntlich, wenn wir die natürlichen Zahlen mit der $0$ meinen.
\end{frame}
%
%
\begin{frame}\frametitle{Russell’sche\footnote{Bertrand Russell, 1872 - 1970} Antinomie}
	
	Sei $M$ die Menge, die alle Mengen enthält, die sich nicht selbst enthalten:
	$$
		M := \{ A ~|~ A\text{~ist eine Menge und~}A\notin A \}.
	$$
	Ist $M$ in $M$ enthalten?
	\pause
	\begin{itemize}
		\item Annahme: $M$ ist in $M$ enthalten. Dann folgt aus der Definition von $M$: $M\notin M$. $\lightning$ \pause
		\item Annahme: $M$ ist nicht in $M$ enthalten. 
		Dann folgt aus der Definition von $M$, dass $M$ in $M$ enthalten ist. $\lightning$
	\end{itemize}
	\pause
	\vspace{1mm}
	Diese Mengenkonstruktion führt zu einem logischen Widerspruch.\\[1mm]
	Deshalb werden wir solche Mengenkonstruktionen (Mengen {\em aller} Mengen; Mengen, die sich selbst enthalten)  
	ausschlie{\ss}en.\\ 
	\vspace{2mm}
	 
\end{frame}
%
%
\begin{frame}\frametitle{Russell’sche Antinomie}
	\begin{itemize}
		\item Ein Barbier einer Stadt rasiert {\em genau} die Männer der Stadt, die sich nicht selbst rasiere\\[2mm]
				Rasiert sich der Barbier selbst?\\\vfill
	
		\item Um solche Probleme formal zu vermeiden wird in der modernen Mathematik mit dem Axiomensystem ZFC gearbeitet.
	\end{itemize} 
\end{frame}
%
%%%
\subsection{Teilmengen}
%%%
%
\begin{frame}\frametitle{Teilmenge}
	
	Eine Menge $A$ hei{\ss}t \highlightDef{ Teilmenge} der Menge $B$, wenn gilt\footnote{
		Damit ist gemeint, dass die Aussage ``$\forall x\in A: x\in B$'' wahr ist.}:
	$$
		\forall x\in A: x\in B.
	$$
	Schreibweise:
	$$
		A \subseteq B\qquad \text{oder} \qquad B \supseteq A.
	$$
	\vfill
	Ist $A$ eine Teilmenge von $B$, so nennt man $B$ auch \highlightDef{ Obermenge} der Menge $A$.
\end{frame}
%
%
\begin{frame}\frametitle{Beispiel}
	
	\begin{itemize}
		\item $\N \subseteq \Z \subseteq \Q \subseteq \R$.
		\pause
		\item Die leere Menge $\emptyset$ ist Teilmenge jeder Menge:\\
			Ist $M$ eine Menge, so gilt
			$$
				\emptyset \subseteq M.
			$$
			Dies gilt insbesondere für $M=\emptyset$, d.h. $\emptyset \subseteq \emptyset$.\\
			(Aber: $\emptyset \notin \emptyset$.)
	\end{itemize}
	
\end{frame}
%
%
\begin{frame}\frametitle{$A=B$}
	
	Seien $A$, $B$ Mengen.\\
	Wir definieren \highlightDef{ $A=B$}, wenn
	$$
		A \subseteq B \qquad\text{und}\qquad B\subseteq A.
	$$
	
\end{frame}
%

%
\begin{frame}\frametitle{Die Potenzmenge}
	
	Sei $M$ eine Menge. Die \highlightDef{ Potenzmenge} ${\cal P}(M)$ von $M$ ist die Menge, 
	die aus allen Teilmengen von $M$ besteht:
	$$
		{\cal P}(M) := \{ A ~|~ A \subseteq M \}.
	$$
\end{frame}
%
%
\begin{frame}\frametitle{Beispiel}

	Sei $M = \{ 1,2,3 \}$.  ${\cal P}(M) =$?\\[2mm]
	\pause
	Es gilt: \\\quad\\
	$
		{\cal P}(M) = \big\{$\pause$ \emptyset, $\pause$ \{1\}, \{2\}, \{3\},$\pause$ \{1,2\}, \{1,3\}, \{2,3\}, $\pause$ \{1,2,3\} \big\}.
	$

\end{frame}
%
%%%
\subsection{Mengendiagramme}
%%%
%
\begin{frame}\frametitle{Venn\footnote{John Venn Junior, 1834 - 1923}-Diagramme}
	
	Ein \highlightDef{ Venn-Diagramm} ist eine graphische Veranschaulichung, 
	die \textit{alle} möglichen Beziehungen zwischen einer gegebenen Anzahl von Mengen darstellt:
	
	\begin{columns}
		\begin{column}{0.5\textwidth}
			\begin{figure}
				\includegraphics[scale=0.6]{Grafiken/Venn2/venn2.pdf}
			\end{figure}
		\end{column}
		%
		\begin{column}{0.5\textwidth}
			\begin{figure}
				\includegraphics[scale=0.6]{Grafiken/Venn3/venn3.pdf}
			\end{figure}
		\end{column}
	\end{columns}
	
\end{frame}
%
%
\begin{frame}\frametitle{Euler\footnote{Leonhard Euler, 1707-1783}-Diagramme}
	
	\highlightDef{ Euler-Diagramme} veranschaulichen Teilmengenbeziehungen zwischen gegebenen Mengen:\\[3mm]
	
	\begin{figure}
		\includegraphics[scale=0.5]{Grafiken/Euler/euler.pdf}
	\end{figure}
	
\end{frame}
%
%
%%%
\subsection{Mengenoperationen}
%%%
%
\begin{frame}\frametitle{Schnittmenge}
	
	Seien $A,B$ Mengen. Dann ist die \highlightDef{ Schnittmenge} $A\cap B$ definiert durch
	$$
		A\cap B := \big\{ x ~|~ (x\in A) \land (x\in B) \big\}.
	$$
	
	\pause
	\vspace{5mm}
	$A$ und $B$ hei{\ss}en \highlightDef{ disjunkt}, wenn $A\cap B = \emptyset$.
	
\end{frame}
%
%
\begin{frame}\frametitle{Schnittmenge}
	
	Sei $I$ eine nichtleere (Index-)Menge und für jedes $i\in I$ sei eine Menge $A_i$ gegeben. \\
	Dann ist die Schnittmenge dieser Mengen definiert durch	
	$$
		\bigcap_{i\in I} A_i := \big\{ x ~|~  \forall i \in I: x \in A_i \big\}.
	$$
	
\end{frame}
%
%
\begin{frame}\frametitle{Vereinigung}

	Seien $A,B$ Mengen. Dann ist die \highlightDef{ Vereinigung} $A\cup B$ definiert durch
	$$
		A\cup B := \big\{ x ~|~ (x\in A) \lor (x\in B) \big\}.
	$$
	
\end{frame}
%
%
\begin{frame}\frametitle{Vereinigung}
	
	Sei $I$ eine nichtleere (Index-)Menge und für jedes $i\in I$ sei eine Menge $A_i$ gegeben. \\
	Dann ist die Vereinigung dieser Mengen definiert durch	
	$$
		\bigcup_{i\in I} A_i := \big\{ x ~|~  \exists i \in I: x \in A_i \big\}.
	$$
	
\end{frame}
%
%
\begin{frame}\frametitle{Differenzmenge}

	Seien $A,B$ Mengen. Dann ist die \highlightDef{ Differenzmenge} $A\backslash B$ definiert durch
	$$
		A\backslash B := \big\{ x ~|~ (x\in A) \land (x\notin B) \big\}.
	$$
	\vfill
	Man spricht oft auch von ``$A$ ohne $B$''.
\end{frame}
%
%
\begin{frame}\frametitle{Komplement}
	
	Seien $A,B$ Mengen mit $A\subseteq B$. Dann ist das \highlightDef{ Komplement} $A^{\complement}$ von $A$ in $B$ definiert durch
	$$
		A^\complement := B \backslash A.
	$$
	
\end{frame}
%


%%%
\subsection{Rechenregeln}
%%%
%
\begin{frame}\frametitle{De Morgan'sche\footnote{Augustus De Morgan, 1806 - 1871} Regeln}
	
	Seien $A,B$ Teilmengen einer Menge $M$. Dann gilt:
	\begin{itemize}
		\item[(1)] $(A \cup B)^\complement = A^\complement \cap B^\complement.$
		\item[(2)] $(A \cap B)^\complement = A^\complement \cup B^\complement.$
	\end{itemize}
	\vfill
	 Man vergleiche mit den De Morgan'schen Regeln der Aussagenlogik und beachte $A^\complement = M\backslash A = \{x\mid (x\in M) \land \neg (x \in A)\}$.
\end{frame}
%
%
\begin{frame}\frametitle{De Morgan'sche Regeln: Beweis}
	
	\vspace{-3mm}
	\begin{eqnarray*}
		(A \cup B)^\complement  	&=& M \backslash (A\cup B) \\
						&=& \{x ~|~ (x\in M) \land \neg(x \in A\cup B) \}\\
						&=& \{x ~|~ (x\in M) \land \neg( (x \in A) \lor (x\in B) ) \}\\
						&=& \{x ~|~ (x\in M) \land ( (x \notin A) \land (x \notin B) ) \}	\qquad (*)\\ 		
						&=& \{x ~|~ \big((x\in M) \land (x \notin A)\big) \land \big( (x \in M) \land (x \notin B) \big) \}	\\
						&=& \{x ~|~ \big( x\in A^\complement \big) \land \big( x\in B^\complement \big) \}\\
						&=& A^\complement \cap B^\complement
	\end{eqnarray*}
	$(*)$ In der vierten Zeile haben wir die de Morgan'sche Regel aus der Aussagenlogik verwendet.\\
	\vfill \pause
	Der Nachweis der zweiten Regel ist eine \"Ubungsaufgabe.
	
\end{frame}
%
%
\begin{frame}\frametitle{Assoziativgesetze und Distributivgesetze}
	
	\highlightDef{Assoziativgesetze}\\
	Seien $A,B,C$ Mengen. Dann gilt:
	\begin{itemize}
		\item[(1)] $A\cap(B \cap C) = (A \cap B) \cap C.$
		\item[(2)] $A\cup(B \cup C) = (A \cup B) \cup C.$
	\end{itemize}\pause
	\vfill

	\highlightDef{Distributivgesetze}\\
	Seien $A,B,C$ Mengen. Dann gilt:
	\begin{itemize}
		\item[(1)] $A\cup(B \cap C) = (A \cup B) \cap (A\cup C).$
		\item[(2)] $A\cap(B \cup C) = (A \cap B) \cup (A\cap C).$
	\end{itemize}
	
\end{frame}
%
%
\begin{frame}\frametitle{Assoziativ- und Distributivgesetze}

	Die Beweise der Assoziativ- und Distributivgesetze werden analog zum Beweis der de Morgan'schen Regeln geführt: \\
	Man führt diese Gesetze für Mengen auf die jeweiligen Gesetze der Aussagenlogik zurück.
	
\end{frame}
%
%
\begin{frame}\frametitle{Kartesisches Produkt}

	Seien $A,B$ Mengen. Dann ist das \highlightDef{ kartesische Produkt} $A\times B$ definiert durch
	$$
		A\times B := \big\{ (a,b) ~|~ (a\in A) \land (b\in B) \big\}.
	$$
	
	\vfill
	\pause
	Beispiel:\\
	$ \{a,b,c,d,e,f,g,h \} \times \{1,2,3,4,5,6,7,8\}$ wird beim Schach verwendet.
\end{frame}
%
%
\begin{frame}\frametitle{Kartesisches Produkt}

	Seien $A$ eine Menge und $k\in \N$ mit $k\geq 1$. Dann ist das kartesische Produkt $A^k$ definiert durch
	$$
		A^k := \left\{ 
				\begin{pmatrix} a_1\\a_2\\\vdots \\ a_k \end{pmatrix}
				 ~\Big|~ \forall i=1,\ldots,k : a_i \in A \right\}.
	$$
	\vfill
	Bekanntes Beispiel: $\R^n$.
\end{frame}
%
\section{Abbildungen}
%%%
%
%%%
\subsection{Definition}
%%%
%
%
\begin{frame}
	
	\begin{block}{Definition}
		Eine \highlightDef{ Abbildung} 
		$$
			f:M\to N
		$$ 
		zwischen zwei Mengen $M$ und $N$ ist eine Teilmenge 
		$$
			f \subseteq M\times N
		$$ 
		mit der Eigenschaft dass für alle $m\in M$ genau ein $n\in N$ existiert, so dass $(m,n)\in f$.\\
		Für dieses $n$ schreibt man $n = f(m)$.\\[2mm]
		
		\pause
		$M$ hei{\ss}t \highlightDef{ Definitionsmenge}, $N$ \highlightDef{ Wertebereich} von $f$ und 
		$f(M) := \{ f(m)~|~  m\in M \}$ \highlightDef{ Bildmenge} (auch mit \highlightDef{$Bild(f)$} notiert).\\[2mm]
		
		\pause
		Schreibweise für Abbildungen:
		\vspace{-2mm}
		$$ 
			f: M\to N, \quad m\mapsto f(m).
		$$
	\end{block}
	
\end{frame}
%
%
\begin{frame}\frametitle{Beispiele}

	\begin{itemize}
		\item $M=N=\{0,1\}, f = \{(0,1),(1,1)\}$:
			$$
				f(0) = 1,\quad f(1)=1.
			$$ \pause
		\item $f: \R\to \R, x\mapsto x^2$: \pause
			$$
				f = \{(x,x^2)~|~ x\in \R\}.
			$$\pause
		\item $f: \N\to \R, x\mapsto x^2$: \pause
			$$
				f = \{(x,x^2)~|~ x\in \N\}.
			$$\pause
		\item Die Abbildung 
			$$
				\text{Id}_M: M\to M, \quad m\mapsto m
			$$
			hei{\ss}t \highlightDef{ Identität} auf $M$.
	\end{itemize}

\end{frame}
%
%
%%%
\subsection{Eigenschaften}
%%%
%
\begin{frame}
	
	\begin{block}{Definition}
		Eine Abbildung $f: M\to N$ hei{\ss}t 
		\begin{itemize}
			\item \highlightDef{ injektiv}, wenn für alle $m_1, m_2 \in M$ gilt:
				$$	f(m_1) = f(m_2)	
					\quad
					\Rightarrow
					\quad
					m_1 = m_2.
				$$ \pause
			\item \highlightDef{ surjektiv}, wenn $f(M)=N$, d.h.:
				$$
					\forall n\in N\; \exists m\in M: f(m)=n.
				$$\pause
			\item \highlightDef{ bijektiv}, wenn $f$ injektiv und surjektiv ist.
		\end{itemize}
	\end{block}
	
\end{frame}
%
%
\begin{frame}
	
	\highlightDef{ Beispiele:}
	\begin{itemize}
		\item $f: \Z \to \Z,\, x\mapsto x+1$ 
			\pause 
			ist bijektiv.
		\pause
		\item $f: \N \to \N,\, x\mapsto x^2$ 
			\pause
			ist injektiv aber nicht surjektiv. 
		\pause
		\item $f:\Z \to \N,\, x\mapsto x^2$ 
			\pause
			ist weder injektiv noch surjektiv.
	\end{itemize}
	
\end{frame}
%
%%%
\subsection{Verkettung}
%%%
%
\begin{frame}
	
	\begin{block}{Definition}
		Seien $f: A\to B$ und $g:B\to C$ Abbildungen. 
		Die \highlightDef{ Verkettung} oder \highlightDef{ Komposition} von $f$ und $g$ ist die Abbildung
		$$
			g\circ f: A\to C,\; x\mapsto g\big( f(x) \big).
		$$
	\end{block}
	
	\pause
	\vfill
	\highlightDef{ Beispiel:} $f:\R\to\R,\, x\mapsto x^2$ und $g:\R\to\R,\, x\mapsto e^x$: 
	\begin{align*}
		g\circ f(x) 	&= g\big( f(x) \big) = g\big( x^2 \big) = e^{x^2},\\
		f\circ g(x)	&= f\big( g(x) \big) = f\big( e^x\big) = (e^x)^2 = e^{2x}.
	\end{align*}
	
\end{frame}
%
% 
\begin{frame}
	
	\begin{block}{Umkehrabbildung}
		Eine Abbildung $f: M\to N$ ist \highlightDef{ genau dann} bijektiv, \highlightDef{ wenn} 
		eine Abbildung $g: N \to M$ existiert mit
		$$
			g\circ f = \text{id}_M
			\qquad
			\text{und}
			\qquad
			f\circ g = \text{id}_N.
		$$  
		
		\pause
		\vspace{2mm}
		Die Abbildung $g$ hei{\ss}t \highlightDef{ Umkehrabbildung} von $f$. Statt $g$ schreibt man meist $f^{-1}$.
	\end{block}
	
\end{frame}
%
% 
\begin{frame}
	
	Eine Abbildung $f: M\to N$ ist \highlightDef{ genau dann} bijektiv, \highlightDef{ wenn} 
	eine Abbildung $g: N \to M$ existiert mit
	$$
		g\circ f = \text{id}_M
		\qquad
		\text{und}
		\qquad
		f\circ g = \text{id}_N.
	$$  
	
	\highlightDef{ Beweis $\Rightarrow$:}\\ 
	Voraussetzung: $f$ ist bijektiv.\\[1mm]
	$f$ surjektiv: $\forall n\in N\;  \exists m\in M: f(m) = n$.\\
	\pause
	Da $f$ injektiv, ist dieses $m$ eindeutig bestimmt.\\
	\pause
	Definition: $g:N\to M, g(n) = m$ für das eindeutige $m\in M$ mit $f(m) = n$.\\
	\pause
	Es folgt:
	$$
		g\circ f(m) = g\big( f(m) \big) = m, \qquad \text{d.h. } g\circ f = \text{id}_M
	$$
	und
	$$
		f\circ g(n) = f\big( g(n) \big) = n, \qquad \text{d.h. } f\circ g = \text{id}_N.
	$$	
	
\end{frame}
%
% 
\begin{frame}
	
	Eine Abbildung $f: M\to N$ ist \highlightDef{ genau dann} bijektiv, \highlightDef{ wenn} 
	eine Abbildung $g: N \to M$ existiert mit
	$$
		g\circ f = \text{id}_M
		\qquad
		\text{und}
		\qquad
		f\circ g = \text{id}_N.
	$$  
	
	\highlightDef{ Beweis $\Leftarrow$:} \\
	Voraussetzung: Es gibt eine Abbildung $g: N \to M$ mit $g\circ f = \text{id}_M$ und $f\circ g = \text{id}_N$.\\[1mm]
	\pause
	$f$ ist surjektiv, denn: Sei $n\in N$. Für $m:=g(n) \in M$ folgt 
	$$
		f(m) = f\big(g(n)\big) = f\circ g(n) = n.
	$$
	\pause
	$f$ ist injektiv, denn: Seien $m_1, m_2\in M$ mit $f(m_1)=f(m_2) = n \in N$. \\
	\pause
	Dann folgt:
	$$
		m_1 = g\circ f(m_1) =  g\big( f(m_1)\big) =g(n)= g\big( f(m_2)\big) = g\circ f(m_2) = m_2.
	$$
	\qed
	
\end{frame}
%
%

\end{document}