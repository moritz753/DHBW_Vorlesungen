\documentclass{beamer}

\usepackage{beamerthemesplit}

\usepackage{amsfonts}
\usepackage{amsmath}
\usepackage{amssymb}
\usepackage{amsthm}
\usepackage{amscd}

\usepackage{stmaryrd} 					%\lightning
\usepackage{algorithm2e}


\usepackage[ngerman]{babel}

\usepackage[utf8]{inputenc}
\usepackage[T1]{fontenc}
\usepackage{textcomp}


% Color Definitions
\definecolor{dhbwRed}{RGB}{226,0,26} 
\definecolor{dhbwGray}{RGB}{61,77,77}
\definecolor{lightBlue}{RGB}{28,134,230}

% Basic Theme
\usetheme{Malmoe}

% Color Re-Definitions
\usecolortheme[named=lightBlue]{structure}
\setbeamercolor*{alerted  text}{fg=dhbwRed, bg=white}
\setbeamercolor*{subsection in toc}{fg=dhbwGray, bg=white}

%\setbeamercolor*{palette primary}{fg=white,bg=lightBlue}
%\setbeamercolor*{palette secondary}{fg=white,bg=gray}
%\setbeamercolor*{palette tertiary}{fg=white,bg=gray}
%\setbeamercolor*{palette quaternary}{fg=white,bg=dhbwRed}

% no navigation symbols
\setbeamertemplate{navigation symbols}{}

% headline, footline
\setbeamertemplate{footline}{\color{dhbwGray} \hfill\insertframenumber\hspace{5mm}\vspace{2mm}}
\setbeamertemplate{headline}{}

% Title Page
\newcommand*{\makeTitlePage}{
	
	\begin{frame}[plain]
		
		\vfill
		\vfill
		\begin{center}
			{
				\usebeamerfont{title}
				\usebeamercolor[fg]{title}
				\Large
				\inserttitle
			}\\[3mm]
			{	
				\usebeamerfont{subtitle}
				\usebeamercolor[fg]{subtitle}
				\large
				\insertsubtitle
			}
		\end{center}
		%
		\vfill
		\vfill
		\vfill
		\vfill
		%
		\begin{columns}
			\begin{column}{0.5\textwidth}
				\begin{flushleft}
					{
						\usebeamerfont{normal text}
						\color{dhbwGray!80}
						\scriptsize
						Dr. Moritz Gruber\\
						DHBW Karlsruhe\\
						
					}
				\end{flushleft}
			\end{column}
			%
			\begin{column}{0.5\textwidth}
				\begin{flushright}
					\includegraphics[scale=0.06]{../DHBW.png}
				\end{flushright}
			\end{column}
		\end{columns}
		%
		\vspace{1mm}
		\begin{columns}
			\begin{column}{0.5\textwidth}
				\begin{flushleft}
					{
						\usebeamerfont{normal text}
						\color{dhbwGray!80}
						\scriptsize
						Version \today
					}
				\end{flushleft}
			\end{column}
			%
			\begin{column}{0.5\textwidth}
				% nothing (just a placeholder to be in line with the columns above
			\end{column}
		\end{columns}
	\end{frame}

}

% Section Divider Page
\newcommand*{\makeSectionDividerPage}{

	\begin{frame}[plain]
		\begin{center}
			\begin{flushleft}
				{				
					\usebeamercolor[fg]{frametitle}
					{\Large \insertsection} \\[3mm]
					{\large \insertsubsection}
				}
			\end{flushleft}
		\end{center}
        \end{frame}
	
}

% itemize
\setbeamertemplate{itemize items}[circle]
\setbeamertemplate{enumerate item}{(\theenumi)}




%--------------------------------------%
% Math ------------------------------%
%--------------------------------------%

% Mengen (Zahlen)
\newcommand{\N}{\mathbb{N}}
\newcommand{\Q}{\mathbb{Q}}
\newcommand{\R}{\mathbb{R}}
\newcommand{\Z}{\mathbb{Z}}
\newcommand{\C}{\mathbb{C}}

% Mengen (allgemein)
\newcommand{\K}{\mathbb{K}}
\newcommand\PX{{\cal P}(X)}

% Zahlentheorie
\newcommand{\ggT}{\mathrm{ggT}}


% Ableitungen
\newcommand{\ddx}{\frac{d}{dx}}
\newcommand{\pddx}{\frac{\partial}{\partial x}}
\newcommand{\pddy}{\frac{\partial}{\partial y}}
\newcommand{\grad}{\text{grad}}

%--------------------------------------%
% Layout Colors ------------------%
%--------------------------------------%
\newcommand*{\highlightDef}[1]{{\color{lightBlue}#1}}
\newcommand*{\highlight}[1]{{\color{lightBlue}#1}} % after theme for colours

%---------------------------------------------%
\title{Lineare Algebra}
\subtitle{Lineare Gleichungssysteme}

%---------------------------------------------%
\begin{document}

%---------------------------------------------%
\makeTitlePage

%---------------------------------------------%
\begin{frame}\frametitle{Inhalt}
   \tableofcontents
\end{frame}
%

%---------------------------------------------%
% Folien -----------------------------------%
%---------------------------------------------%
%

%--------------------------------------------
\section{Definition}
\makeSectionDividerPage
%%%
%
%
\begin{frame}\frametitle{Lineare Gleichungssysteme}

	Ein \highlightDef{Lineares Gleichungssystem (LGS)}	 mit $n$ Gleichungen und $m$ Unbekannten ist ein System
	$$
		\begin{array}{ccccccccc}
			a_{11}x_1	&+	&a_{12}x_2	&+	&\cdots	&a_{1m}x_m	&=	& b_1 \\
			a_{21}x_1	&+	&a_{22}x_2 	&+	&\cdots	&a_{2m}x_m	&=	& b_2 \\
			\vdots	&	&\vdots		&	&\vdots	&\vdots		&	& \vdots\\
			a_{n1}x_1	&+ 	&a_{n2}x_2 	&+	&\cdots	&a_{nm}x_m	&=	& b_n 
		\end{array}
	$$
	mit Koeffizienten $a_{ij}\in\R$ und $b_j\in \R$, für das eine Lösung
	$$	
		x
		=
		\begin{pmatrix} 
			x_1\\
			x_2\\
			\vdots\\
			x_m
		\end{pmatrix} 
		\in \R^m
	$$
	gesucht wird.
	
\end{frame}
%
%
\begin{frame}\frametitle{Lineare Gleichungssysteme}

	Mit
	$$
		A 
		=
		\begin{pmatrix}
			a_{11}	& a_{12}	&\cdots 	&a_{1m}\\
			a_{21}	& a_{22}	&\cdots 	&a_{2m}\\
			\vdots	& \vdots	&            	&\vdots \\
			a_{n1}	& a_{n2}	&\cdots	&a_{nm}
		\end{pmatrix}
		\in \R^{n\times m}
		\quad
		\text{und}
		\quad
		b
		=
		\begin{pmatrix}
			b_1\\
			\vdots\\
			b_n
		\end{pmatrix}		
		\in \R^n
	$$
	
	\vspace{2mm}
	können wir das LGS kompakter schreiben:
	$$
		A\cdot x = b.
	$$
	
\end{frame}
%
%
\begin{frame}\frametitle{Homogene/Inhomogene LGS}

	Ein lineares Gleichungssystem
	$$
		A\cdot x = 0
	$$
	hei{\ss}t \highlightDef{homogen},\\[2mm]
	
	ein lineares Gleichungssystem
	$$
		A\cdot x = b
	$$
	für $b\neq 0$ hei{\ss}t \highlightDef{inhomogen}.
	
	\pause
	\vspace{5mm}
	Ist ein inhomogenes LGS 
	$A\cdot x = b$ 
	gegeben, so hei{\ss}t das LGS $A\cdot x = 0$ das zugehörige homogene LGS.
	
\end{frame}
%

%------------------------------------------------
\section{Struktur der Lösungsmenge}
%%%

%------------------------------------------------
\subsection{Die Lösungsmenge eines homogenen LGS}
\makeSectionDividerPage
%%%
%
\begin{frame}\frametitle{Die Lösungsmenge eines homogenen LGS}

	Seien $A\in\R^{n\times m}$, $a \in\R$ und $x^{(1)}, x^{(2)} \in \R^{m}$ Lösungen des \highlight{homogenen LGS}
	$$
		A\cdot x = 0.
	$$
	Dann sind auch
	\begin{itemize}
		\item $x^{(1)} + x^{(2)}$ und
		\item $a\cdot x^{(1)}$  
	\end{itemize}
	Lösungen dieses LGS:\pause
	
	$$
		A\cdot(x^{(1)} + x^{(2)}) \pause = A\cdot x^{(1)} + A\cdot x^{(2)}\pause =0+0= 0,
	$$ 
\pause
	$$
		A\cdot (a\cdot x^{(1)}) \pause= a\cdot (A\cdot x^{(1)})\pause =a\cdot 0= 0.
	$$  
	
\end{frame}
%
%
\begin{frame}\frametitle{Die Lösungsmenge eines homogenen LGS}

	Die \highlight{Lösungsmenge}
	$$
		\mathcal{L} = \{ x \in \R^m ~|~ A\cdot x = 0\} \subseteq \R^m
	$$ 
	des LGS $A\cdot x=0$ ist somit abgeschlossen bezüglich 
	\begin{itemize}
		\item der Vektoraddition:\\ 
			$x^{(1)}, x^{(2)}\in \mathcal{L} \Rightarrow x^{(1)} + x^{(2)}\in \mathcal{L}$,
		\item der Multiplikation mit reellen Zahlen:\\ 
			$a\in \R, x^{(1)}\in \mathcal{L} \Rightarrow a\cdot x^{(1)}\in \mathcal{L}$. 
	\end{itemize}
	
	\pause
	\vspace{6mm}
	$(\mathcal{L},+,\cdot)$ erfüllt alle definierenden Eigenschaften eines Vektorraums.\\[2mm]
	
	Deshalb bildet
	\highlightDef{$\mathcal{L} \subseteq \R^m$} einen \highlightDef{Untervektorraum} des Vektorraums $\R^m$. 
	
\end{frame}
%
%
%------------------------------------------------
\subsection{Die Lösungsmenge eines inhomogenen LGS}
\makeSectionDividerPage
%%%
%
%
\begin{frame}\frametitle{Die Lösungsmenge eines inhomogenen LGS}

	Seien $A\in\R^{n\times m}$, $b\in \R^n$ und $x^{(1)}, x^{(2)} \in \R^{m}$ Lösungen des \highlight{inhomogenen LGS}
	$$
		A\cdot x = b.
	$$
	
	\pause
	Dann gilt:
	$$
		A\cdot ( x^{(1)} - x^{(2)} ) 
		=  
		A\cdot x^{(1)} - A\cdot x^{(2)} 
		= 
		b - b 
		= 0.
	$$
	
	\pause
	\vspace{5mm}
	Der Vektor $x^{(1)} - x^{(2)}\in \R^{m}$  ist somit eine Lösung des zugehörigen homogenen LGS $A\cdot x=0$.
	
\end{frame}
%
%
\begin{frame}\frametitle{Die Lösungsmenge eines inhomogenen LGS}

	Ist andererseits \highlight{$x_{h} \in \R^m$} eine Lösung des zugehörigen \highlight{homogenen} LGS 
	$$
		A\cdot x=0,
	$$ 
	dann folgt:
	$$
		A\cdot ( x^{(1)} + x_{h}) =  A\cdot x^{(1)} + A\cdot x_{h} = b + 0 = b.
	$$
	
	\pause
	\vspace{5mm}
	Der Vektor \highlight{$x^{(1)} + x_{h} \in \R^m$} ist somit ebenfalls eine Lösung des \highlight{inhomogenen} LGS $A\cdot x=b$.
	
\end{frame}
%
%
\begin{frame}\frametitle{Die Struktur der Lösungsmenge}

	Seien \highlight{$\mathcal{L}_{inh}$} die Lösungsmenge des inhomogenen LGS 
	$$
		A\cdot x = b,
	$$
	und
	\highlight{$\mathcal{L}_{h}$} die Lösungsmenge des zugehörigen homogenen LGS 
	$$
		A\cdot x=0
	$$ 
	sowie \highlight{$x_{sp} \in \mathcal{L}_{inh}$} eine (spezielle) Lösung des inhomogenen LGS.	
	
	\pause
	\vspace{5mm}
	Dann gilt:
	$$
 		\mathcal{L}_{inh} = \{ x_{sp} + x_h ~|~ x_h \in \mathcal{L}_{h} \}.
	$$
	
\end{frame}
%
%
\begin{frame}\frametitle{Die Struktur der Lösungsmenge}

	$$
 		\mathcal{L}_{inh} = \{ x_{sp} + x_h ~|~ x_h \in \mathcal{L}_{h} \}.
	$$
	
	\vspace{5mm}
	\highlight{Beweis}\\[1mm]
	
	\highlight{$\{ x_{sp} + x_h ~|~ x_h \in \mathcal{L}_{h} \} \subseteq \mathcal{L}_{inh}$:}\\[1mm] 
	s.o.\\[2mm]
	
	\pause
	
	\highlight{$\mathcal{L}_{inh} \subseteq \{ x_{sp} + x_h ~|~ x_h \in \mathcal{L}_{h} \}$:}\\[2mm]
	
	Sei $x \in \mathcal{L}_{inh}$.\\
	$\Rightarrow x_h := x - x_{sp} \in \mathcal{L}_{h}$ \quad (s.o.).\\
	$\Rightarrow  x = x_{sp} + x_h.$ 
	\qed
	
	\vfill
	
\end{frame}
%
%---------------------------------------
\section{Gau{\ss}-Algorithmus}
\makeSectionDividerPage
%%%
%
\begin{frame}\frametitle{Gau{\ss}-Algorithmus} 

	Der \highlight{Gau{\ss}-Algorithmus}\footnote{Johann Carl Friedrich Gau{\ss} (1777-1855)}
	ist ein Algorithmus zur Bestimmung der Lösungsmenge eines LGS
	$$
		Ax = b.
	$$
	
\end{frame}
%
%
\begin{frame}\frametitle{Invertierbare Matrizen}

	Seien $A\in\R^{n\times m}$, $b\in \R^n$ und $C\in \R^{n\times n}$ invertierbar. 
	Dann gilt:
	$$
		A\cdot x = b  \quad\Longleftrightarrow\quad C\cdot (A\cdot x) = C\cdot b. \vspace{5mm}
	$$

	\pause
	\vspace{5mm}
	Die Multiplikation mit einer invertierbaren Matrix $C$ ändert somit nicht die Lösungsmenge des LGS $A\cdot x=b$:
	$$
		\{ x \in \R^m ~|~ A\cdot x = b\}
		=
		\{ x \in \R^m ~|~ C\cdot A\cdot x = C\cdot b\}
	$$
	
	\pause
	\vspace{4mm}
	\highlight{Iterationsschritte des Gau{\ss}-Algorithmus:} 
	Multiplikationen eines LGS $A\cdot x = b$  mit speziellen invertierbaren Matrizen.

\end{frame}
%
%
\begin{frame}\frametitle{Erlaubte Operationen}

	\begin{itemize}
		\item[(1)] \highlight{Vertauschen zweier Gleichungen:}\\
			Multiplikation von $A\cdot x=b$ mit einer Vertauschungsmatrix. \pause
		\item[(2)] \highlight{Ersetzten einer Gleichung durch ihr $\alpha$-faches ($\alpha \neq 0$):}\\
			Multiplikation von $A\cdot x=b$ mit einer Diagonalmatrix wie z.B. $$\begin{pmatrix}\alpha & 0  \\ 0 & 1 \end{pmatrix}.$$ \pause
		\item[(3)] \highlight{Ersetzen der $i$-ten Gleichung durch die Summe der $i$-ten Gleichung mit dem 
			$\alpha$-fachen der $j$-ten Gleichung:}\\
			Multipikation von $A\cdot x=b$ mit einer Additionsmatrix. \pause
	\end{itemize}
	
	\vspace{3mm}
	Da diese Matrizen invertierbar sind, ändern diese Umformungen nicht die Lösungsmenge des LGS. 
	
\end{frame}
%
%
\begin{frame}\frametitle{Notation}

	$A\cdot x = b$ notieren wir als\\[3mm]
	
	$$	
		\left(
		\begin{array}{cccc|c}
			a_{11}	&a_{12}	&\cdots	&a_{1m}	&b_1\\
			a_{21} 	&a_{22} 	&\cdots 	&a_{2m} 	&b_2\\
			\vdots 	&\vdots   	&            	&\vdots   	&\vdots\\
			a_{n1} 	&a_{n2} 	&\cdots 	&a_{nm} 	&b_n
		\end{array}
		\right).
	$$
	
\end{frame}
%
%
\begin{frame}\frametitle{Schritt 1: \"Uberführung in Zeilenstufenform}

	Ist in der ersten Spalte ein Element $\neq 0$, 
	so lässt es sich durch Vertauschen (1) und ggf. Multiplizieren (2) sowie Addieren von Zeilen (3) erreichen, 
	dass das umgeformte LGS die folgende Form hat:\\[5mm]
	
	$$
		\left(
		\begin{array}{cccc | c}
			1		&a'_{12}	&\cdots	&a'_{1m}	&b'_1\\
			0 		&a'_{22}	&\cdots	&a'_{2m}	&b'_2\\
			\vdots	&\vdots   	&            	&\vdots   	&\vdots\\
			0 		&a'_{n2} 	&\cdots 	&a'_{nm} 	&b'_n
		\end{array}
		\right)
	$$
	
\end{frame}
%
%
\begin{frame}\frametitle{Schritt 1: \"Uberführung in Zeilenstufenform}

	Bestehen die ersten Spalten aus Nullen, 
	so führen wir diesen Schritt für die erste Spalte, die nicht nur aus Nullen besteht, durch:\\[5mm]
	
	$$
		\left(
		\begin{array}{ccccccc | c}
			0		&\cdots 	& 0 		& 1 		& a'_{1k} 	& \cdots 	& a'_{1m} 	& b'_1\\
			\vdots 	& 		& \vdots 	& 0 		& a'_{2k} 	& \cdots 	& a'_{2m} & b'_2\\
			\vdots 	&  		&\vdots 	& \vdots  	& \vdots    & 	   	& \vdots	& \vdots\\
			0		&\cdots 	& 0 		& 0 		& a'_{nk} 	& \cdots 	& a'_{nm} 	& b'_n
		\end{array}
		\right)
	$$
	
\end{frame}
%
%
\begin{frame}\frametitle{Schritt 1: \"Uberführung in Zeilenstufenform}
	
	Diesen Schritt wiederholen wir für die nächste Spalte, die ab Zeile 2 nicht nur aus Nullen besteht, und erhalten: \\[5mm]
	
	$$
		\left(
		\begin{array}{ccccccccccc | c}
			0		&\cdots 	& 0 		& 1 		& *		& \cdots 	& *		& *		& *		&  \cdots	& * 		& *\\
			\vdots 	& 		& \vdots 	& 0 		& 0		& \cdots	& 0		& 1		& * 		& \cdots 	& * 		& *\\
			\vdots 	&  		&\vdots 	& \vdots  	& \vdots    & 	   	& \vdots	& 0		& *		& \cdots	& *		& *\\
			\vdots 	&  		&\vdots 	& \vdots  	& \vdots    & 	   	& \vdots	& \vdots	& \vdots	& 		& \vdots	& \vdots\\			
			0		&\cdots 	& 0 		& 0 		& 0 		& \cdots 	& 0 		& 0		& *		& \cdots	& *		& *
		\end{array}
		\right)
	$$
	
\end{frame}
%
%
\begin{frame}\frametitle{Schritt 1: \"Uberführung in Zeilenstufenform}

	Diesen Schritt  wiederholen wir mit den nächsten Spalten bis wir bei der letzten Spalte angelangt sind:	
	$$
		\left(
		\begin{array}{ccccccccccccccc|c}
			0		&\cdots 	& 0 		& 1 		& *		& \cdots 	& *		& *		& *		&  \cdots	& *		& *		& *		& \cdots	& * 		& c_1\\
			\vdots 	& 		& \vdots 	& 0 		& 0		& \cdots	& 0		& 1		& * 		& \cdots 	& * 		& *		& *		& \cdots	& *		& c_2\\
			\vdots 	&  		&\vdots 	& \vdots  	& \vdots    & 	   	& \vdots	& 0		& *		& \cdots	& *		& *		& *		& \cdots	& *		& c_3\\			
			\vdots 	&  		&\vdots 	& \vdots  	& \vdots    & 	   	& \vdots	& \vdots	& \vdots	& 		& \vdots	& \vdots	& \vdots	&		& \vdots	& \vdots\\		
			\vdots 	&  		&\vdots 	& \vdots  	& \vdots    & 	   	& \vdots	& 0		& 0		& \cdots	& 0		& 1 		& *		&		& *		& c_r\\	
			\vdots 	&  		&\vdots 	& \vdots  	& \vdots    & 	   	& \vdots	& 0		& 0		& 		& 0		& 0		& \cdots	& \cdots	& 0		& c_{r+1}\\	
			\vdots 	&  		&\vdots 	& \vdots  	& \vdots    & 	   	& \vdots	& \vdots	& \vdots	& 		&\vdots	& \vdots	& 		& 		& \vdots	& \vdots\\				
			0		&\cdots 	& 0 		& 0 		& 0 		& \cdots 	& 0 		& 0		& 0		& \cdots	& 0		& 0		& \cdots	& \cdots	& 0		& c_n
		\end{array}
		\right)
	$$
	
	\pause
	\vspace{2mm}
	Das LGS ist genau dann lösbar, wenn $c_{r+1} = \ldots c_n = 0$.
	
\end{frame}
%
%
\begin{frame}\frametitle{Schritt 2: Gau{\ss}-Normalform}

	Durch weitere erlaubte Umformungen (Multiplizieren und Addieren von Zeilen) erreichen wir, dass oberhalb der Einsen ausschlie{\ss}lich Nullen stehen:
	$$
		\left(
		\begin{array}{ccccccccccccccc|c}
			0		&\cdots 	& 0 		& 1 		& *		& \cdots 	& *		& 0		& *		&  \cdots	& *		& 0		& *		& \cdots	& * 		& d_1\\
			\vdots 	& 		& \vdots 	& 0 		& 0		& \cdots	& 0		& 1		& * 		& \cdots 	& * 		& 0		& *		& \cdots	& *		& d_2\\
			\vdots 	&  		&\vdots 	& \vdots  	& \vdots    & 	   	& \vdots	& 0		& 0		& \cdots	& *		& 0		& *		& \cdots	& *		& d_3\\			
			\vdots 	&  		&\vdots 	& \vdots  	& \vdots    & 	   	& \vdots	& \vdots	& \vdots	& 		& \vdots	& \vdots	& \vdots	&		& \vdots	& \vdots\\		
			\vdots 	&  		&\vdots 	& \vdots  	& \vdots    & 	   	& \vdots	& 0		& 0		& \cdots	& 0		& 1 		& *		&		& *		& d_r\\	
			\vdots 	&  		&\vdots 	& \vdots  	& \vdots    & 	   	& \vdots	& 0		& 0		& 		& 0		& 0		& \cdots	& \cdots	& 0		& d_{r+1}\\	
			\vdots 	&  		&\vdots 	& \vdots  	& \vdots    & 	   	& \vdots	& \vdots	& \vdots	& 		&\vdots	& \vdots	& 		& 		& \vdots	& \vdots\\				
			0		&\cdots 	& 0 		& 0 		& 0 		& \cdots 	& 0 		& 0		& 0		& \cdots	& 0		& 0		& \cdots	& \cdots	& 0		& d_n
		\end{array}
		\right)
	$$
	
	\pause
	\vspace{2mm}
	An dieser \highlightDef{Gau{\ss}-Normalform} lässt sich sehr einfach die Lösungsmenge des LGS ablesen.
	
\end{frame}
%
%
%---------------------------------------
\subsection{Beispiel}
\makeSectionDividerPage
%%%
%
\begin{frame}\frametitle{Beispiel}
	
	$$
		\begin{array}{rcrcrcrcl}
			2x_1	&+	&4x_2	&+	&4x_3	&+	&2x_4	&= 4\\
				&	&x_2		&+	&x_3		&+	&x_4       	&= 1\\
			2x_1	&+	&5x_2	&+	&5x_3	&+	&3x_4 	&= 5	
		\end{array}
	$$
	
	\pause
	\vspace{10mm}
	$$
		\begin{pmatrix}
			2 & 4 & 4 & 2\\
			0 & 1 & 1 & 1\\
			2 & 5 & 5 & 3
		\end{pmatrix}
		\cdot 
		\begin{pmatrix}
			x_1\\
			x_2\\
			x_3\\
			x_4
		\end{pmatrix}
		=
		\begin{pmatrix}
			4\\ 
			1\\
			5
		\end{pmatrix}
	$$
	\pause
	\vspace{10mm}	
	$$
	\left(
	\begin{array}{cccc|c}
		2 & 4 & 4 & 2 & 4\\
		0 & 1 & 1 & 1 & 1\\
		2 & 5 & 5 & 3 & 5
	\end{array}
	\right)
	$$
	
\end{frame}
%
%
\begin{frame}\frametitle{Beispiel}

	$$
		\left(
		\begin{array}{cccc|c}
			2 & 4 & 4 & 2 & 4\\
			0 & 1 & 1 & 1 & 1\\
			2 & 5 & 5 & 3 & 5
		\end{array}
		\right)
	$$
	
	\pause
	\vspace{5mm}
	Multiplikation der ersten Zeile mit $1/2$ ergibt
	$$
		\sim>
		\left(
		\begin{array}{cccc|c}
			1 & 2 & 2 & 1 & 2\\
			0 & 1 & 1 & 1 & 1\\
			2 & 5 & 5 & 3 & 5
		\end{array}
		\right)
	$$
	
	\pause
	\vspace{5mm}
	Addition des $(-2)$-fachen der ersten Zeile zur dritten Zeile:
	$$
		\sim>
		\left(
		\begin{array}{cccc|c}
			1 & 2 & 2 & 1 & 2\\
			0 & 1 & 1 & 1 & 1\\
			0 & 1 & 1 & 1 & 1
		\end{array}
		\right)
	$$
	
\end{frame}
%
%
\begin{frame}\frametitle{Beispiel}
	
	\vspace{-3mm}
	$$
		\left(
		\begin{array}{cccc|c}
			1 & 2 & 2 & 1 & 2\\
			0 & 1 & 1 & 1 & 1\\
			0 & 1 & 1 & 1 & 1
		\end{array}
		\right)
	$$
	
	\pause
	\vspace{5mm}
	Addition des $(-1)$-fachen der zweiten Zeile zur dritten Zeile:
	$$
		\sim>
		\left(
		\begin{array}{cccc|c}
			1 & 2 & 2 & 1 & 2\\
			0 & 1 & 1 & 1 & 1\\
			0 & 0 & 0 & 0 & 0
		\end{array}
		\right)
	$$

	\pause
	\vspace{5mm}
	Addition des $(-2)$-fachen der zweiten Zeile zur ersten Zeile:
	$$
		\sim>
		\left(
		\begin{array}{cccc|c}
			1 & 0 & 0 & -1 & 0\\
			0 & 1 & 1 & 1 & 1\\
			0 & 0 & 0 & 0 & 0
		\end{array}
		\right)
	$$
	
	\vspace{5mm}
	Gau{\ss}-Normalform mit Stufen in den Spalten 1 und 2.
	
\end{frame}
%
%
\begin{frame}\frametitle{Beispiel}
	
	$$
		\left(
		\begin{array}{cccc|c}
			1 & 0 & 0 & -1 & 0\\
			0 & 1 & 1 & 1 & 1\\
			0 & 0 & 0 & 0 & 0
		\end{array}
		\right)
	$$
	
	Es folgt: \pause
	\begin{itemize}
		\item $x_3, x_4$ können beliebig in $\R$ gewählt werden\\ (Spalten 3 \& 4: Keine Stufen): \\
			\highlight{Freie Variablen} $x_3 = t_1, x_4 = t_2 \in \R$, \pause
		\item $x_1=x_4 = t_2$,	\pause	
		\item $x_2 = 1 - x_3 - x_4 = 1- t_1 - t_2$. \pause
	\end{itemize}
	
	\vspace{3mm}
	Somit ist für alle $t_1, t_2 \in \R$
	$$
		\begin{pmatrix}
			x_1\\
			x_2\\
			x_3\\
			x_4
		\end{pmatrix}		
		=
		\begin{pmatrix}
			t_2\\
			1-t_1-t_2\\
			t_1\\
			t_2
		\end{pmatrix}
		=		
		\begin{pmatrix}
			0\\
			1\\
			0\\
			0
		\end{pmatrix}
		+
		t_1
		\cdot
		\begin{pmatrix}
			0\\
			-1\\
			1\\
			0
		\end{pmatrix}
		+
		t_2
		\cdot
		\begin{pmatrix}
			1\\
			-1\\
			0\\
			1
		\end{pmatrix}		
	$$
	eine Lösung des LGS.
	
\end{frame}
%
%
\begin{frame}\frametitle{Beispiel: Lösungsmenge}

	$$
		{\mathcal L} =
		\left\{
					\begin{pmatrix}
						0\\
						1\\
						0\\
						0
					\end{pmatrix}
					+
					t_1
					\cdot
					\begin{pmatrix}
						0\\
						-1\\
						1\\
						0
					\end{pmatrix}
					+
					t_2
					\cdot
					\begin{pmatrix}
						1\\
						-1\\
						0\\
						1
					\end{pmatrix}	
					~|~
					t_1, t_2 \in \R
		\right\}
	$$

\end{frame}
%
%
%---------------------------------------
\subsection{Der $(-1)$-Trick}
\makeSectionDividerPage
%%%
%
%
\begin{frame}\frametitle{Satz}
Es sei $T =(t_{ij}) \in \R^{m\times n}$ eine Matrix in Gauß-Normalform mit $r$ Stufen in den Spalten $s_1,...,s_r$. Dann ist für jedes $j \in J:=\{1,...,n\}\setminus\{s_1,...,s_r\}$ der Vektor
$$
F^{(j)}:=\left(\sum_{i=1}^r t_{ij} e_{s_i} \right) - e_j
$$
eine Lösung des homogenen LGS $T\cdot x = 0$.\\
Außerdem gilt für die Lösungsmenge $\mathcal L$ des homogenen LGS $T\cdot x=0$
$$
\mathcal{L}=\langle \{F^{(j)} \mid j \in J\} \rangle
$$
\pause
\vfill
Die Vektoren $F^{(j)}$ nennt man die \highlightDef{Fundamentallösungen} des homogenen LGS.
\end{frame}
%
%
\begin{frame}\frametitle{Beweis}
Zuerst zeigen wir, dass die Vektoren $F^{(j)}$ Lösungen des homogenen LGS sind. \\\vspace{2mm}\pause
$
T\cdot F^{(j)}=\left(\sum_{i=1}^r t_{ij}\cdot T \cdot e_{s_i} \right) - T\cdot e_j $\\
\hspace{13.25mm}$= \pause \left(\sum_{i=1}^r t_{ij}\cdot e_i \right) - T\cdot e_j$\\
\hspace{13.25mm}$= \pause \sum_{i=1}^r t_{ij} e_i  - \sum_{k=1}^r t_{kj} e_k$\\
\hspace{13.25mm}$= \pause 0
$\\ \vspace{2mm}
Dadurch ist auch direkt die Inklusion $\mathcal{L} \supseteq \langle \{F^{(j)} \mid j \in J\} \rangle$ klar. Somit bleibt noch $\subseteq$ zu zeigen. 
\end{frame}
%
%
\begin{frame}\frametitle{Beweis (Fortsetzung)}
Sei also $x \in \mathcal L$ beliebig. Da $\mathcal L$ ein Untervektorraum des $\R^n$ ist, gilt
$$
v=x + \sum_{j\in J} x_j F^{(j)} \in \mathcal L
$$
Dieser Vektor hat als Eintrag an der Stelle $j$ eine $0$. Durch Einsetzen in das LGS und Ausnutzen von $v \in \mathcal L$ kann man zeigen, dass auch alle anderen Einträge von $v$ null sind. Somit gilt 
$$
x= -\sum_{j\in J} x_j F^{(j)} 
$$
und damit auch $\subseteq$. \hfill $\square$
\end{frame}
%
%
\begin{frame}\frametitle{Beispiel: Der $(-1)$-Trick}

	Ergänze in 
	$$
		\left(
		\begin{array}{cccc | c}
			1 & 0 & 0 & -1 & 0\\
			0 & 1 & 1 & 1 & 1\\
			0 & 0 & 0 & 0 & 0
		\end{array}
		\right)
	$$
	
	\vspace{4mm}
	eine Nullzeile an den Stellen ohne Stufe:
	$$
	\left(
		\begin{array}{cccc | c}
			1 & 0 & 0 & -1 & 0\\
			0 & 1 & 1 & 1 & 1\\
			0 & 0 & 0 & 0 & 0\\
			0 & 0 & 0 & 0 & 0
		\end{array}
	\right)
	$$
	
\end{frame}
%
%
\begin{frame}\frametitle{Beispiel: Der $(-1)$-Trick}
	
	\vspace{2mm}
	Ersetzte Nullen auf der Diagonalen durch eine $(-1)$:
	$$
	\left(
	\begin{array}{cccc | c}
		1 & 0 & 0 & -1 & 0\\
		0 & 1 & 1 & 1 & 1\\
		0 & 0 & 0 & 0 & 0\\
		0 & 0 & 0 & 0 & 0
	\end{array}
	\right)
	\sim>
	\left(
	\begin{array}{cccc | c}
		1 & 0 & 0 		           	& -1 			& 0\\
		0 & 1 & 1 			   	& 1 			& 1\\
		0 & 0 & \highlight{-1} 	& 0 			& 0\\
		0 & 0 & 0 				& \highlight{-1} 	& 0
	\end{array}
	\right)
	$$
	
	\pause
	Die rechte Seite 
	$
		\begin{pmatrix}
			0\\ 1\\0\\0
		\end{pmatrix}
	$
	ist eine spezielle Lösung des inhomogenen LGS und die beiden Spalten mit den 
	$(-1)$en erzeugen die Lösungsmenge des zugehörigen homogenen LGS:
	
	$$
		\mathcal{L}_h = \left\{ 
				t_1
				\cdot
				\begin{pmatrix}
					0\\1\\-1\\0
				\end{pmatrix}
				+
				t_2
				\cdot
				\begin{pmatrix}
					-1\\1\\0\\-1
				\end{pmatrix}
				~|~
				t_1,t_2 \in \R
			 \right\}
	$$	
	
\end{frame}
%
%
\begin{frame}\frametitle{Beispiel: Der $(-1)$-Trick}

	$$
		\left(
		\begin{array}{cccc | c}
			1 & 0 & 0 		           	& -1 			& 0\\
			0 & 1 & 1 			   	& 1 			& 1\\
			0 & 0 & \highlight{-1} 	& 0 			& 0\\
			0 & 0 & 0 				& \highlight{-1} 	& 0
		\end{array}
		\right)
	$$
	\vfill
	$$
		\mathcal{L} =   \left\{ 
				\begin{pmatrix}
				0\\ 1\\0\\0
				\end{pmatrix}
				+
				t_1
				\cdot
				\begin{pmatrix}
					0\\1\\-1\\0
				\end{pmatrix}
				+
				t_2
				\cdot
				\begin{pmatrix}
					-1\\1\\0\\-1
				\end{pmatrix}
				~|~
				t_1,t_2 \in \R
			 \right\}
	$$
	
\end{frame}
%
%---------------------------------------
\subsection{Inverse Matrix}
\makeSectionDividerPage
%%%

%
\begin{frame}\frametitle{Inverse Matrix}
	
	Sei $A\in\R^{n\times n}$ invertierbar. Dann gilt:
	$$
		A\cdot A^{-1} = E.
	$$
	D.h., die erste Spalte von $A^{-1}$ löst das LGS
	$$
		A\cdot x 
		= 
		\begin{pmatrix}
			1 \\ 0 \\ \vdots \\ 0
		\end{pmatrix}.
	$$

\end{frame}
%
%
\begin{frame}\frametitle{Inverse Matrix}
	
	Die zweite Spalte von $A^{-1}$ löst das LGS
	$$
		A\cdot x 
		= 
		\begin{pmatrix}
			0 \\ 1 \\0\\ \vdots \\ 0
		\end{pmatrix},
	$$
	usw.

\end{frame}
%
%
\begin{frame}\frametitle{Inverse Matrix}

	Die Inverse Matrix $A^{-1}$ lä{\ss}t sich somit mit dem Gau{\ss}-Algorithmus durch (simultanes) Lösen der linearen Gleichungssysteme
	
	$$
		\left(
		\begin{array}{ccc|ccccc}
			a_{11}	&\cdots	&a_{1n}	&1		&0		&0		&\cdots	&0 \\
			a_{21} 	&\cdots	&a_{2n}	&0		&1		&0		&\cdots	&0\\
			\vdots 	&            	&\vdots	&\vdots	& 		&\ddots	&		&\vdots \\
			a_{n1} 	&\cdots 	&a_{nn} 	&0         	&\cdots 	&\cdots	& 		& 1   
		\end{array}
		\right)
	$$
	
	berechnen.
	
\end{frame}
%

\begin{frame}\frametitle{Beispiel}
Es sei $A=\begin{pmatrix}1 & 2 & 0 \\ 1 & 0&0 \\ 2 & 0 & 2  \end{pmatrix} \in \R^{3\times 3}$	\\
\vfill\pause
	$$
		\left(
		\begin{array}{ccc|ccc}
			1&2&0 	&1&0&0\\
			1&0&0	&0&1&0\\
			2&0&2	&0&0&1
		\end{array}
		\right) \sim> \pause
				\left(
		\begin{array}{ccc|ccc}
			1&2&0 	&1&0&0\\
			0&-2&0	&-1&1&0\\
			0&-4&2	&-2&0&1
		\end{array}
		\right)
	$$	
		$$
		\sim> \pause\left(
		\begin{array}{ccc|ccc}
			1&2&0 	&1&0&0\\
			0&1&0	&\frac{1}{2}&-\frac{1}{2}&0\\
			0&0&2	&0&-2&1
		\end{array}
		\right) 	\sim>\pause
		\left(
		\begin{array}{ccc|ccc}
			1&0&0 	&0&1&0\\
			0&1&0	&\frac{1}{2}&-\frac{1}{2}&0\\
			0&0&1	&0&-1&\frac{1}{2}
		\end{array}
		\right) 
	$$\pause \vfill
	Somit ist $A^{-1}=\begin{pmatrix}0&1&0 \\ \frac{1}{2}&-\frac{1}{2}&0 \\0&-1&\frac{1}{2}  \end{pmatrix}$
\end{frame}
%
\begin{frame}\frametitle{Weiterer Anwendungsfall: Lineare Unabhängigkeit prüfen}
Gegeben seien Vektoren $v_1,...,v_m \in \R^n$ (wobei $m \le n$). \\\vfill
Die Lösungsmenge $\cal L$ des homogenen LGS
$$
a_1\cdot v_1 + a_2\cdot v_2 + ... +a_m\cdot v_m=0
$$
gibt darüber Aufschluss ob die Vektoren linear unabhängig sind:\\
\begin{itemize}
\item Ist $\cal L$ $=\{0\}$, so besitzt das LGS nur die triviale Lösung und die Vektoren sind linear unabhängig.
\item Ist $\cal L$ $\ne\{0\}$, so besitzt das LGS eine nicht-triviale Lösung und die Vektoren sind linear abhängig.
\end{itemize}
\end{frame}
%
\begin{frame}\frametitle{Beispiel}
Seien $v_1=\begin{pmatrix} 1\\0\\1\\0 \end{pmatrix},\ v_2=\begin{pmatrix} 1\\1\\1\\0 \end{pmatrix}, \ v_3=\begin{pmatrix} 0\\1\\0\\1 \end{pmatrix} \in \R^4$.\\\vfill \pause
Dann betrachten wir das LGS\\\vfill
		$\left(
		\begin{array}{ccc|c}
			1&1&0 	&0\\
			0&1&1	&0\\
			1&1&0	&0\\
			0&0&1   &0
		\end{array}
		\right) \sim>\pause \left(
		\begin{array}{ccc|c}
			1&1&0 	&0\\
			0&1&0	&0\\
			0&0&0	&0\\
			0&0&1   &0
		\end{array}
		\right) \sim>\pause\left(
		\begin{array}{ccc|c}
			1&0&0 	&0\\
			0&1&0	&0\\
			0&0&1	&0\\
			0&0&0   &0
		\end{array}
		\right)$\\\vfill
Damit gilt $\cal L$ $=\{0\}$ und die Vektoren sind linear unabhängig.
\end{frame}
%
%%
\end{document}