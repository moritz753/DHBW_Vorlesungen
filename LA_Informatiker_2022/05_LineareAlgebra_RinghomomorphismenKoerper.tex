\documentclass{beamer}

\usepackage{beamerthemesplit}

\usepackage{amsfonts}
\usepackage{amsmath}
\usepackage{amssymb}
\usepackage{amsthm}
\usepackage{amscd}

\usepackage{stmaryrd} 					%\lightning
\usepackage{algorithm2e}


\usepackage[ngerman]{babel}

\usepackage[utf8]{inputenc}
\usepackage[T1]{fontenc}
\usepackage{textcomp}


% Color Definitions
\definecolor{dhbwRed}{RGB}{226,0,26} 
\definecolor{dhbwGray}{RGB}{61,77,77}
\definecolor{lightBlue}{RGB}{28,134,230}

% Basic Theme
\usetheme{Malmoe}

% Color Re-Definitions
\usecolortheme[named=lightBlue]{structure}
\setbeamercolor*{alerted  text}{fg=dhbwRed, bg=white}
\setbeamercolor*{subsection in toc}{fg=dhbwGray, bg=white}

%\setbeamercolor*{palette primary}{fg=white,bg=lightBlue}
%\setbeamercolor*{palette secondary}{fg=white,bg=gray}
%\setbeamercolor*{palette tertiary}{fg=white,bg=gray}
%\setbeamercolor*{palette quaternary}{fg=white,bg=dhbwRed}

% no navigation symbols
\setbeamertemplate{navigation symbols}{}

% headline, footline
\setbeamertemplate{footline}{\color{dhbwGray} \hfill\insertframenumber\hspace{5mm}\vspace{2mm}}
\setbeamertemplate{headline}{}

% Title Page
\newcommand*{\makeTitlePage}{
	
	\begin{frame}[plain]
		
		\vfill
		\vfill
		\begin{center}
			{
				\usebeamerfont{title}
				\usebeamercolor[fg]{title}
				\Large
				\inserttitle
			}\\[3mm]
			{	
				\usebeamerfont{subtitle}
				\usebeamercolor[fg]{subtitle}
				\large
				\insertsubtitle
			}
		\end{center}
		%
		\vfill
		\vfill
		\vfill
		\vfill
		%
		\begin{columns}
			\begin{column}{0.5\textwidth}
				\begin{flushleft}
					{
						\usebeamerfont{normal text}
						\color{dhbwGray!80}
						\scriptsize
						Dr. Moritz Gruber\\
						DHBW Karlsruhe\\
						
					}
				\end{flushleft}
			\end{column}
			%
			\begin{column}{0.5\textwidth}
				\begin{flushright}
					\includegraphics[scale=0.06]{../DHBW.png}
				\end{flushright}
			\end{column}
		\end{columns}
		%
		\vspace{1mm}
		\begin{columns}
			\begin{column}{0.5\textwidth}
				\begin{flushleft}
					{
						\usebeamerfont{normal text}
						\color{dhbwGray!80}
						\scriptsize
						Version \today
					}
				\end{flushleft}
			\end{column}
			%
			\begin{column}{0.5\textwidth}
				% nothing (just a placeholder to be in line with the columns above
			\end{column}
		\end{columns}
	\end{frame}

}

% Section Divider Page
\newcommand*{\makeSectionDividerPage}{

	\begin{frame}[plain]
		\begin{center}
			\begin{flushleft}
				{				
					\usebeamercolor[fg]{frametitle}
					{\Large \insertsection} \\[3mm]
					{\large \insertsubsection}
				}
			\end{flushleft}
		\end{center}
        \end{frame}
	
}

% itemize
\setbeamertemplate{itemize items}[circle]
\setbeamertemplate{enumerate item}{(\theenumi)}




%--------------------------------------%
% Math ------------------------------%
%--------------------------------------%

% Mengen (Zahlen)
\newcommand{\N}{\mathbb{N}}
\newcommand{\Q}{\mathbb{Q}}
\newcommand{\R}{\mathbb{R}}
\newcommand{\Z}{\mathbb{Z}}
\newcommand{\C}{\mathbb{C}}

% Mengen (allgemein)
\newcommand{\K}{\mathbb{K}}
\newcommand\PX{{\cal P}(X)}

% Zahlentheorie
\newcommand{\ggT}{\mathrm{ggT}}


% Ableitungen
\newcommand{\ddx}{\frac{d}{dx}}
\newcommand{\pddx}{\frac{\partial}{\partial x}}
\newcommand{\pddy}{\frac{\partial}{\partial y}}
\newcommand{\grad}{\text{grad}}

%--------------------------------------%
% Layout Colors ------------------%
%--------------------------------------%
\newcommand*{\highlightDef}[1]{{\color{lightBlue}#1}}
\newcommand*{\highlight}[1]{{\color{lightBlue}#1}} % after theme for colours


%----------------------------------------------------------------------------------------------------
%--------- Document Title ---------------------------------------------------------------------
\title{Lineare Algebra\\[3mm] 
	\large Ring-Homomorphismen \& K\"orper
}
\author{Dr. Moritz Gruber } 
\institute{DHBW Karlsruhe}
\date{2022}
%%%%%%%%%%%%%%
\begin{document}

\AtBeginSection[]{
	\begin{frame}				
		\usebeamercolor[fg]{frametitle}
		{\Large \insertsection} 
        \end{frame}
}

%
\begin{frame}[plain] 
 \titlepage
\end{frame}
%
%
\begin{frame}\frametitle{Inhalt}
   \tableofcontents
\end{frame}
%
%%%


\section{Ring-Homomorphismen}
\subsection{Definition}
%%%
%
\begin{frame}\frametitle{Definition}
	
	Seien $(R,+_R,\cdot_R)$ und $(S, +_S, \cdot_S)$ zwei Ringe mit Einselementen $1_R$ und $1_S$.
	Dann hei{\ss}t eine Abbildung
	$$
		\Phi: (R,+_R,\cdot_R) \to (S, +_S, \cdot_S)
	$$
	\highlightDef{ (Ring-)Homomorphismus}, wenn f\"ur alle $x,y \in R$ gilt:
	\begin{itemize}
		\item $\Phi(x +_R y) = \Phi(x) +_S \Phi(y)$,
		\item $\Phi(x \cdot_R y) = \Phi(x) \cdot_S \Phi(y)$, 
		\item $\Phi(1_R) = 1_S$.
	\end{itemize}
	
	\vfill
	Einen \highlightDef{ bijektiven Homomorphismus} nennt man \highlightDef{ Isomorphismus}.\\\pause
	\vfill
	Aus der Definition folgt direkt, dass $\Phi(0_R)=0_S$ gilt und $\Phi$ insbesondere ein Gruppenhomomorphismus der additiven Gruppen $(R,+_R)$ und $(S,+_S)$ ist.
	
\end{frame}
%
%
\begin{frame}\frametitle{Beispiel}
	\begin{itemize}
	\item Die Abbildung
	$$
		\Phi: (\Z,+,\cdot) \to (\R,+,\cdot), \, x \mapsto x
	$$
	ein Homomorphismus der Ringe $(\Z,+,\cdot)$ und $(\R,+,\cdot)$. \pause
	\item Die Abbildung
	$$
		\Phi: (\R,+,\cdot) \to (\Z,+,\cdot), \, x \mapsto 1
	$$
	\underline{kein} Homomorphismus der Ringe $(\R,+,\cdot)$ und $(\Z,+,\cdot)$. \pause
	\item Die Abbildung
	$$
		\Phi: R[X] \to R, \, a_nX^n + ... + a_1X+a_0 \mapsto a_0 
	$$
	ein Homomorphismus des Polynomrings $R[X]$ und seinem Koeffizientenring $R$.
	\end{itemize}
	
\end{frame}
%
\subsection{Homomorphismen $\Z \to R$}
%
\begin{frame}\frametitle{Homomorphismen $\Z \to R$}
Ein Ring-Homomorphismus $\Phi$ von $\Z$ in einen beliebigen Ring $R$ mit Eins muss die Zahl $1$ auf $1_R$ abbilden. Damit folgt:\\\vspace{2mm} \pause
$\Phi(2)=\Phi(1+1)=1_R+1_R=:2\cdot 1_S$\\ \pause
$\Phi(3)=\Phi(1+1+1)=1_R+1_R+1_R=:3\cdot 1_R$\\
...\\\vspace{2mm}
und \pause\\\vspace{2mm}
$\Phi(-1)=-1_R$\\\pause
$\Phi(0)=\Phi(1+(-1))=1_R+(-1_R)=0_R$\\
\vspace{2mm}\pause
Damit sind alle Bildwerte allein durch die Definition des Begriffs Homomorphismus bereits festgelegt!\\\pause
Daher gibt es für jeden Ring $R$ genau einen Homomorphismus von $\Z$ nach $R$.




\end{frame}
%
%%%
\subsection{Produkte von Ringen}
%%%

%
\begin{frame}\frametitle{Produkt von Ringen}
	
	Seien $(R,+_R,\cdot_R)$ und $(S,+_S,\cdot_S)$ Ringe. \\
	Wir definieren auf dem kartesischen Produkt 
	$$
		R\times S
	$$ 
	die Verkn\"upfungen $+_{R\times S}$ und $\cdot_{R\times S}$: \\[1mm]
	F\"ur $(x_1,y_1), (x_2,y_2) \in R\times S$ gelte
	$$
		(x_1,y_1) +_{R\times S} (x_2,y_2) := (x_1 +_R x_2, y_1 +_S y_2)
	$$ 
	und
	$$
		(x_1,y_1) \cdot_{R\times S} (x_2,y_2) := (x_1 \cdot_R x_2, y_1 \cdot_S y_2).
	$$ 
		
\end{frame}
%
%
\begin{frame}\frametitle{Produkt von Ringen}
	
	Es gilt (\"Ubungsaufgabe):\\
	\begin{itemize}
		\item $(R\times S, +_{R\times S}, \cdot_{R\times S})$ ist ein Ring. \pause
		\item Sind $R$ und $S$ kommutative Ringe, so ist auch $R\times S$ kommutativ.
		\item Sind $R$ und $S$ Ringe mit Einselement, so ist auch $R\times S$ ein Ring mit Einselement. \pause
		\item Die Abbildungen
		$$p_1 : R\times S \to R, (r,s) \mapsto r \ \text{ und } \ p_2 : R\times S \to S, (r,s) \mapsto s$$
		sind Ring-Homomorphismen.
	\end{itemize}\vfill
	Analog für mehr als zwei Faktoren.
		
\end{frame}
%
\section{Polynomiale Abbildungen}
%
\begin{frame}\frametitle{Der Ring der Abbildungen}
Es sei $(R,+_R,\cdot_R)$ ein Ring. Dann ist die Menge $Abb(R,R)$ aller Abbildungen $R \to R$ mit den folgenden Verknüpfungen ein Ring: Für alle $f,g \in Abb(R,R)$ und alle $a\in R$ sei
\begin{itemize}
\item $(f+g)(a):=f(a)+_Rg(a)$
\item $(f\cdot g)(a):=f(a)\cdot_Rg(a)$
\end{itemize} \pause\vfill
\highlightDef{Beispiel}\\
$R=\Z$ und $f:\Z \to \Z, x \mapsto 2x$ und $g:\Z \to \Z, x \mapsto x^2$. \pause
\begin{itemize}
\item $f+g : \Z \to \Z, x \mapsto 2x+x^2$ \pause
\item $f\cdot g: \Z \to \Z, x\mapsto 2x \cdot x^2=2x^3$
\end{itemize}
\end{frame}
%
%
\begin{frame}\frametitle{Polynomiale Abbildungen}
Es sei $R$ kommutativer Ring mit Eins und $f=\sum a_kX^k \in R[X]$. Wir definieren die Abbildung
$$
\tilde f: R \to R, x \mapsto \sum a_kx^k
$$\pause
Die Abbildung
$$
T: R[X] \to Abb(R,R), f \mapsto \tilde f
$$
ist ein Ringhomomorphismus. \pause
\vfill
\highlightDef{Bemerkung:}\\
Wenn $\#R< \infty$, dann ist auch der Ring $Abb(R,R)$ endlich. Aber der Polynomring $R[X]$ ist trotzdem unendlich (solange $R\ne\{0\}$). Somit gibt es in diesem Fall (viel) mehr Polynome als polynomiale Abbildungen!
\end{frame}
%
%
\begin{frame}\frametitle{Beispiele}
\begin{itemize}
\item $R=\Z_2=\{[0]_2,[1]_2\}$, dann ist $\#Abb(\Z_2,\Z_2)=4$, \pause aber für jedes $n \in \N_0$ ist $f_n:=X^n=[1]_2X^n \in \Z_2[X]$ und für $n\ne m\in \N_0$ gilt $f_n \ne f_m$. \pause Somit $\#\Z_2[X] = \infty$.\pause

\item Mit $R=\Z_2$ ist der Homomorphismus
$$
T: \Z_2[X] \to Abb(\Z_2,\Z_2), f \mapsto \tilde f
$$
surjektiv: \pause
\begin{itemize}
\item $T([0]_2)=\widetilde{[0]_2} =g_1:x \mapsto [0]_2,$\pause
\item $T([1]_2)=\widetilde{[1]_2} =g_2:x \mapsto [1]_2,$\pause
\item $T(X)=\widetilde{X} =g_3=Id_{\Z_2},$ \pause
\item $T(X+[1]_2)=\widetilde{X+[1]_2} =g_4$ ist die Abbildung mit $g_4([0]_2)=[1]_2$ und $g_4([1]_2))=[0]_2$.
\end{itemize}
\end{itemize}
\end{frame}
%
\begin{frame}\frametitle{Die Einsetzabbildung}
Es sei $R$ ein Ring und $r \in R$. Dann ist
$$
E_r:R[X] \to R, f \mapsto \tilde f(r):=f(r)
$$
ein Ring-Homomorphismus. \\Man nennt ihn die \highlightDef{Einsetzabbildung bei $r$}.\pause\\
\vfill
Es macht nun Sinn von \highlightDef{Nullstellen} eines Polynoms $f \in R[X]$ zu sprechen: das sind die Elemente $r \in R$, sodass $E_r(f)=f(r)=0_R$ gilt.
\end{frame}
%
%
\section{Körper}
\subsection{Definition}
%
\begin{frame}\frametitle{Definition K\"orper}
	
	Ein Ring $(K,+,\cdot)$ bei dem $(K\backslash\{0\},\cdot)$ sogar eine abelsche Gruppe ist, nennt man \highlightDef{ K\"orper}.
	\pause
	\vfill
	Das heißt ein Körper ist ein kommutativer Ring mit Eins, in dem jedes von Null verschiedene Element ein multiplikatives Inverses besitzt.\\
	\vfill
	\pause
	\highlightDef{ Beispiele:}
	\begin{itemize}
		\item $(\R,+,\cdot)$
		\item $(\Q,+,\cdot)$
	\end{itemize}
	\pause
	\vfill
Für jeden Körper $(K,+,\cdot)$ gilt für die Einheitengruppe $K^\times =K\setminus \{0_K\}$.
\end{frame}
%
\begin{frame}\frametitle{Ring-Homomorphismen von Körpern}
Es seien $K$ und $L$ zwei Körper. Dann ist jeder Homomorphismus $\Phi:K \to L$ injektiv. \pause
\vfill
\highlightDef{Beweis:} Für jedes Element $k \in K^\times=K\setminus\{0_K\}$ gilt \pause
$$
\Phi(k) \in L^\times=L\setminus\{0_L\}
$$ \pause
denn 
$$1_L=\pause \Phi(1_K) \pause =\Phi(k\cdot_K k^{-1}) \pause=\Phi(k)\cdot_L \Phi(k^{-1})$$\pause
Somit gilt $Kern(\Phi)=\Phi^{-1}(\{0_L\})=\{0_K\}$, \pause und da $\Phi$ insbesondere ein Gruppenhomomorphismus der additiven Gruppen $(K,+_K)$ und $(L,+_L)$ ist, ist $\Phi$ injektiv. \hfill $\square$
\end{frame}
%
\subsection{Der Körper der komplexen Zahlen}
%
\begin{frame}\frametitle{Lösen von Gleichungen}
In einem Körper kann man beide Verknüfungen als Äquivalenzumformungen zum Lösen von Gleichungen verwenden, da für alle Elemente (ausgenommen Multiplikation mit $0$) Inverse gibt.\\\pause
\vfill
So lassen sich im Körper $\Q$ (dem kleinsten Körper der $\Z$ enthält) alle linearen Gleichungen $a\cdot x + b =0$ mit Koeffizienten $a,b \in \Q$ und in einer Variablen lösen, z.B.:\pause
\begin{align*}
&&a\cdot x + b &=0 &&| +(-b)\\
\Leftrightarrow &&a\cdot x &= -b &&| \cdot a^{-1}\\
\Leftrightarrow &&x&= -b \cdot a^{-1}&&
\end{align*} 
\end{frame}
%
\begin{frame}\frametitle{Existenz von Lösungen für (quadratische) Gleichungen}
Im Fall von Gleichungen von höherem Grad sieht das schon anders aus. Zum Beispiel hat nicht jede quadratische Gleichung $a\cdot x^2 +b\cdot x +c =0$ in eine Lösung:
\vfill
In $\Q$ hat schon $x^2=2$ kein Lösung, da $\sqrt{2} \notin \Q$.
\vfill \pause
In $\R$ lässt sich mit der ``Mitternachtsformel'' $x_{1/2} = \frac{-b \pm \sqrt{b^2-4ac}}{2a}$ ableiten, dass es zwei Lösungen gibt, wenn die Diskriminante $D=b^2-4ac$ positiv ist, es genau eine Lösung gibt, wenn $D=0$ gilt und es keine Lösung gibt, wenn $D$ negativ ist.\\
Dies liegt daran, dass sich jede positive Zahl in $\R$ auf zwei Wege als Quadrat einer anderen reellen Zahl darstellen lässt, es keine Nullteiler gibt und keine negative reelle Zahl das Quadrat einer reellen Zahl ist.

\end{frame}
%
\begin{frame}\frametitle{Ein noch größerer Körper}
Einen Ausweg bildet das Erweitern eines Körper um weitere Elemente, die fehlenden Lösungen. So ist z.B. 
$$
\Q[\sqrt{2}]:=\{a+b\sqrt{2} \mid a,b \in \Q\}
$$
zusammen mit der Addition und Multiplikation von $\Q$ und der Regel $\sqrt{2}\cdot \sqrt{2}=2$ der kleinste Körper, der $\Q$ enthält und in dem die Gleichung $x^2=2$ eine Lösung hat. \\\pause
\vfill
(Die Gleichung $x^2=3$ hat in diesem Körper noch immer keine Lösung, aber in $(\Q[\sqrt{2}])[\sqrt{3}]$.)
\end{frame}
%
\begin{frame}\frametitle{Die komplexen Zahlen}
Im Körper $\R$ haben wir durch die ``Mitternachtsformel'' verstanden, dass für die allgemeine Existenz von Lösungen von quadratischen Gleichungen, die negativen Zahlen ein Problem sind, da sich diese nicht als Quadrate darstellen lassen. \\\pause
Es genügt also $\R$ um $i:=\sqrt{-1}$ zu erweitern (warum?). Der so entstehende Körper heißt \highlightDef{der Körper der komplexen Zahlen}
$$
\C:=\R[i]=\{a+b\cdot i \mid a,b \in \R\}
$$ \pause
Für eine \highlightDef{komplexe Zahl} $z=a+b \cdot i$ wird $a$ als der \highlightDef{Realteil} $\frak{Re}(z)$ und $b$ als der \highlightDef{Imaginärteil} $\frak{Im}(z)$ bezeichnet.
\end{frame}
%
\begin{frame}\frametitle{Rechnen in $\C$}
Es seien $z_1=a+b\cdot i, z_2=c+d\cdot i \in \C$.
\vfill
\begin{itemize}
\item Addition: $z_1 + z_2 = (a+c)+(b+d)\cdot i$ \pause
\item[] Beispiel: $(2+3i)+(4-5i)=(2+4)+(3-5)i=6-2i$\pause
\vfill
\item Multiplikation: $z_1\cdot z_2=(ac-bd)+(ad+bc)\cdot i$ \pause
\item[] Beispiel: $(2+3i)\cdot(4-5i)=(2\cdot 4 - 3\cdot (-5))+(2\cdot (-5)+ 3\cdot 4)\cdot i=23+2i$
\end{itemize}
\vfill
Bei der Multiplikation wird die Assoziativität und die Definition von $i$ angewendet.
\end{frame}
%
\begin{frame}\frametitle{Fundamentalsatz der Algebra}
Im Körper der komplexen Zahlen hat jede algebraische Gleichung 
$$
a_nx^n +a_{n-1}x^{n-1} + ... a_1x+a_0 =0 
$$
mit $a_n,...,a_0 \in \C, a_n \ne 0, n\ge 1$ (mindestens) eine Lösung.
\vfill
Der Beweis dieses Satzes überschreitet den Rahmen dieser Vorlesung.
\end{frame}
%
\begin{frame}\frametitle{Algebraische Gleichungen und Polynome}
Algebraische Gleichungen und Polynome hängen eng zusammen. Genau genommen gibt eine 1-zu-1-Beziehung zwischen Polynomen $p=p(X) \in K[X]$ und algebraischen Gleichungen $p(x)=0$ über dem Körper $K$.
\vfill
Der Fundamentalsatz der Algebra lässt sich so auch als Aussage über die Nullstellen eines Polynoms interpretieren.
\end{frame}
%
%%%
%

\end{document}