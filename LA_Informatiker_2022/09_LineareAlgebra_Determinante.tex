\documentclass{beamer}

\usepackage{beamerthemesplit}

\usepackage{amsfonts}
\usepackage{amsmath}
\usepackage{amssymb}
\usepackage{amsthm}
\usepackage{amscd}

\usepackage{stmaryrd} 					%\lightning
\usepackage{algorithm2e}


\usepackage[ngerman]{babel}

\usepackage[utf8]{inputenc}
\usepackage[T1]{fontenc}
\usepackage{textcomp}


% Color Definitions
\definecolor{dhbwRed}{RGB}{226,0,26} 
\definecolor{dhbwGray}{RGB}{61,77,77}
\definecolor{lightBlue}{RGB}{28,134,230}

% Basic Theme
\usetheme{Malmoe}

% Color Re-Definitions
\usecolortheme[named=lightBlue]{structure}
\setbeamercolor*{alerted  text}{fg=dhbwRed, bg=white}
\setbeamercolor*{subsection in toc}{fg=dhbwGray, bg=white}

%\setbeamercolor*{palette primary}{fg=white,bg=lightBlue}
%\setbeamercolor*{palette secondary}{fg=white,bg=gray}
%\setbeamercolor*{palette tertiary}{fg=white,bg=gray}
%\setbeamercolor*{palette quaternary}{fg=white,bg=dhbwRed}

% no navigation symbols
\setbeamertemplate{navigation symbols}{}

% headline, footline
\setbeamertemplate{footline}{\color{dhbwGray} \hfill\insertframenumber\hspace{5mm}\vspace{2mm}}
\setbeamertemplate{headline}{}

% Title Page
\newcommand*{\makeTitlePage}{
	
	\begin{frame}[plain]
		
		\vfill
		\vfill
		\begin{center}
			{
				\usebeamerfont{title}
				\usebeamercolor[fg]{title}
				\Large
				\inserttitle
			}\\[3mm]
			{	
				\usebeamerfont{subtitle}
				\usebeamercolor[fg]{subtitle}
				\large
				\insertsubtitle
			}
		\end{center}
		%
		\vfill
		\vfill
		\vfill
		\vfill
		%
		\begin{columns}
			\begin{column}{0.5\textwidth}
				\begin{flushleft}
					{
						\usebeamerfont{normal text}
						\color{dhbwGray!80}
						\scriptsize
						Dr. Moritz Gruber\\
						DHBW Karlsruhe\\
						
					}
				\end{flushleft}
			\end{column}
			%
			\begin{column}{0.5\textwidth}
				\begin{flushright}
					\includegraphics[scale=0.06]{../DHBW.png}
				\end{flushright}
			\end{column}
		\end{columns}
		%
		\vspace{1mm}
		\begin{columns}
			\begin{column}{0.5\textwidth}
				\begin{flushleft}
					{
						\usebeamerfont{normal text}
						\color{dhbwGray!80}
						\scriptsize
						Version \today
					}
				\end{flushleft}
			\end{column}
			%
			\begin{column}{0.5\textwidth}
				% nothing (just a placeholder to be in line with the columns above
			\end{column}
		\end{columns}
	\end{frame}

}

% Section Divider Page
\newcommand*{\makeSectionDividerPage}{

	\begin{frame}[plain]
		\begin{center}
			\begin{flushleft}
				{				
					\usebeamercolor[fg]{frametitle}
					{\Large \insertsection} \\[3mm]
					{\large \insertsubsection}
				}
			\end{flushleft}
		\end{center}
        \end{frame}
	
}

% itemize
\setbeamertemplate{itemize items}[circle]
\setbeamertemplate{enumerate item}{(\theenumi)}




%--------------------------------------%
% Math ------------------------------%
%--------------------------------------%

% Mengen (Zahlen)
\newcommand{\N}{\mathbb{N}}
\newcommand{\Q}{\mathbb{Q}}
\newcommand{\R}{\mathbb{R}}
\newcommand{\Z}{\mathbb{Z}}
\newcommand{\C}{\mathbb{C}}

% Mengen (allgemein)
\newcommand{\K}{\mathbb{K}}
\newcommand\PX{{\cal P}(X)}

% Zahlentheorie
\newcommand{\ggT}{\mathrm{ggT}}


% Ableitungen
\newcommand{\ddx}{\frac{d}{dx}}
\newcommand{\pddx}{\frac{\partial}{\partial x}}
\newcommand{\pddy}{\frac{\partial}{\partial y}}
\newcommand{\grad}{\text{grad}}

%--------------------------------------%
% Layout Colors ------------------%
%--------------------------------------%
\newcommand*{\highlightDef}[1]{{\color{lightBlue}#1}}
\newcommand*{\highlight}[1]{{\color{lightBlue}#1}} % after theme for colours

%----------------------------------------------------------------------------------------------------
%--------- Document Title ---------------------------------------------------------------------
\title{Lineare Algebra\\[3mm] 
	\large Die Determinante
}
\author{Dr. Moritz Gruber} 
\institute{DHBW Karlsruhe}
\date{2022}
%%%%%%%%%%%%%%
\begin{document}

\AtBeginSection[]{
	\begin{frame}				
		\usebeamercolor[fg]{frametitle}
		{\Large \insertsection} 
        \end{frame}
}

%
\begin{frame}[plain] 
 \titlepage
\end{frame}
%
%
\begin{frame}\frametitle{Inhalt}
   \tableofcontents
\end{frame}
%
%%%
\section{Motivation}
%%%
%
\begin{frame}\frametitle{Eindeutige Löungen eines LGS}
Oft steht man vor dem Fall eines linearen Gleichungssystems mit ebensovielen Gleichungen wie Unbekannten, d.h. man möchte 
$$ 
A\cdot x = b \quad \text{ mit } \ A \in \R^{n\times n} \text{ und } b\in \R^n
$$
lösen.\\\pause
Ein solches Gleichungssystem hat genau dann eine eindeutige Lösung, wenn $A$ invertierbar ist, und die Lösung ist dann
$$
x= A^{-1} \cdot b
$$
Es ist daher von großen Interesse, zu wissen wann eine Matrix $A \in \R^{n\times n}$ invertierbar ist. 
\end{frame}
%
%
\begin{frame}\frametitle{$2 \times 2$-Matrizen}

Im Fall von $n=2$  lässt sich leicht das zeigen, dass 
$$
A=\begin{pmatrix} a & b \\ c & d \end{pmatrix} \in \R^{2 \times 2} \text{genau dann invertierbar ist, wenn } \ ad-bc \ne 0.
$$
Die Zahl $ad-bc$ heißt die \highlightDef{Determinante} von $A$.	
\end{frame}
%
%
\begin{frame}\frametitle{}
Geometrisch lässt sie sich (bis auf Vorzeichen) als der Flächeninhalt des von den Spalten
$$u=\begin{pmatrix} a  \\ c \end{pmatrix} \text{ und } v=\begin{pmatrix} b  \\ d \end{pmatrix}$$
von $A$ aufgespannten Parallelogramms verstehen.
	\begin{center}
		\includegraphics[scale=0.6]{Grafiken/Parallelogramm.pdf}
	\end{center}
\end{frame}
%
%
\begin{frame}\frametitle{}
	\begin{center}
		\includegraphics[scale=0.6]{Grafiken/Parallelogramm_Flaeche.pdf}
	\end{center}
gesamt $= \pause (a+b)\cdot(d+c)=ad+ac+bd+bc$\\
gelb $= \pause 2 \cdot a\cdot d$\\
grün $= \pause 2\cdot(\frac{1}{2}\cdot a \cdot c)=a\cdot c$\\
pink $= \pause 2\cdot(\frac{1}{2}\cdot b \cdot d)=b\cdot d$\\\pause
\vfill
Damit erhalten wir:\\\pause\vfill
Flächeninhalt-Parallelogramm = gesamt $-$ gelb $-$ grün $-$ pink $= \pause(ad+ac+bd+bc)-2ad-ac-bd=-ad+bc=-(ad-bc)$.
\end{frame}
%
\begin{frame}
Der Flächeninhalt des Parallelogramms ist genau dann $0$, wenn die beiden Vektoren $u$ und $v$ linear abhängig sind.\\\pause\vfill
Dieser geometrische Ansatz lässt sich so auf $n >2$ übertragen: Eine Matrix $A \in \R^{n\times n}$ ist genau dann invertierbar, wenn ihre Spalten linear unabhängig sind.\pause \\
Dies widerrum ist genau dann der Fall, wenn das Volumen des von den Spalten aufgespannten Parallelopipeds ungleich $0$ ist.
\end{frame}
%
%%
\section{Determinante}
%
\subsection{Definition}
%
\begin{frame}\frametitle{Definition: Determinante}
Die \highlightDef{Determinante} ist die eindeutig definierte Abbildung
$$
\det: \R^{n\times n} \to \R, A \mapsto \det(A)
$$
für die gilt:\pause
\begin{itemize}
\item[i)] $\det(I_n)=1$ \pause
\item[ii)] Die Determinante ist linear in jeder Spalte, d.h. 
$$
\det(a_1,...,a_j+w_j,...,a_n)=\det(a_1,...,a_j,...,a_n)+\det(a_1,...,w_j,...,a_n)
$$  \pause
\item[iii)] Die Determinante ist alternierend, d.h. durch Vertauschen zweier Spalten verändert sich das Vorzeichen:
$$
\det(a_1,...,a_i,...,a_j,...,a_n)=- \det(a_1,...,a_j,...,a_i,...,a_n)
$$ \pause
\end{itemize}
Wir betrachten hier die Matrix $A$ als das Tupel $(a_1,...,a_n)$ ihrer Spalten. 
\end{frame}
%
\subsection{Entwicklungssatz von Laplace}
%
\begin{frame}\frametitle{Entwicklungssatz von Laplace}
Es sei $n \in \N$ und  $A \in \R^{n\times n}$ sowie $1\le i,j \le n$. Weiter bezeichne $A^{ij}$ die $(n-1)\times(n-1)$-Matrix, die aus $A$ entsteht, wenn man die $i$-te Zeile und die $j$-te Spalte entfernt. Dann gilt für die Determinante\pause
$$\det(A)=a_{11} \quad \text{ falls } n=1$$ \pause
und für $n>1$
\vfill
$\det(A)=a_{i1}(-1)^{i+1}\det(A^{i1})+a_{i2}(-1)^{i+2}\det(A^{i2})+...$\\ \vspace{1mm}
\hspace{21mm}$...+a_{in}(-1)^{i+n}\det(A^{in})$\\\vspace{3mm}\pause
\hspace{11.5mm}$=a_{1j}(-1)^{1+j}\det(A^{1j})+a_{2j}(-1)^{2+j}\det(A^{2j})+...$\\ \vspace{1mm}
\hspace{21mm}$...+a_{nj}(-1)^{n+j}\det(A^{nj})$\\\vspace{3mm}
\end{frame}
%
\subsection{Beispiele}
%
\begin{frame}\frametitle{Beispiel: $n=2$}
\begin{itemize}
\item $A=\begin{pmatrix} a & b \\ c & d \end{pmatrix}$ und $i=1$:\pause \\ \vspace{1mm}$\det(A)=a(-1)^{1+1}\det(d)+b(-1)^{1+2}\det(c)\pause=ad-bc$
\pause \vfill
\item $B=\begin{pmatrix} 1 & 2 \\ 3 & 4 \end{pmatrix}$ und $i=1$:\pause \\ \vspace{1mm}$\det(B)=1(-1)^{1+1}\det(4)+2(-1)^{1+2}\det(3)=4-6=-2$
\pause \vfill
\item $A=\begin{pmatrix} a & b \\ c & d \end{pmatrix}$ und $j=2$:\pause \\ \vspace{1mm}$\det(A)=b(-1)^{1+2}\det(c)+d(-1)^{2+2}\det(a)\pause=-bc+da$\\\hspace{11.75mm}$=ad-bc$
\pause \vfill
\item $C=\begin{pmatrix} 1 & 2 \\ 2 & 4 \end{pmatrix}$ und $j=2$:\pause \\ \vspace{1mm}$\det(C)=2(-1)^{1+2}\det(2)+4(-1)^{2+2}\det(1)\pause=-4+4=0$
\end{itemize}
\end{frame}
%
%
\begin{frame}\frametitle{Beispiel: $n=3$}
$A=\begin{pmatrix} 1 & 2 & 3 \\ 0 & 1 & 0 \\ 4 & 2 & 0 \end{pmatrix}$\\\pause\vfill
Zuerst \highlightDef{entwickeln} wir die Determinante nach der 2. Zeile ($i=2$): \\\pause\vfill
\small$\det(A)=0\cdot(-1)^{2+1}\det(A^{21})+1\cdot(-1)^{2+2}\det(A^{22})+0\cdot(-1)^{2+3}\det(A^{23})$\pause\\
\hspace{10.75mm}$=1\cdot\det(\begin{pmatrix} 1&3\\4&0 \end{pmatrix})$\\\pause \vfill
\normalsize Als nächstes entwickeln wir nach der 2. Spalte (j=2):\\\pause\vfill
\small $ \det(A)=1\cdot\det(\begin{pmatrix} 1&3\\4&0 \end{pmatrix})=1\cdot (3\cdot(-1)^{2+1}\det(4)+0\cdot(-1)^{2+2}\det(1))$\\\hspace{10.75mm}$=1\cdot(-3\cdot4 +0)=-12$
\end{frame}
%
\subsection{Determinanten von Dreiecksmatrizen}
%
\begin{frame}\frametitle{Determinante von Dreiecksmatrizen und Diagonalmatrizen}
Eine Matrix $D =(d_{ij})\in \R^{n\times n}$ heißt obere (untere) Dreiecksmatrix, wenn für alle $i>j$ ($i<j$) gilt: $d_{ij}=0$.\\\pause
Beispiel: $D=\begin{pmatrix} 1 & 2 & 3 \\ 0 & 4 & 5 \\ 0&0&6 \end{pmatrix}$ ist eine obere Dreiecksmatrix. \\\vfill
Insbesondere ist jede Diagonalmatrix eine Dreiecksmatrix.\\
\vfill \pause
\highlightDef{Satz}\\
Ist $D=(d_{ij}) \in \R^{n\times n}$ eine Dreiecksmatrix, dann gilt
$$
\det(D)=\prod_{i=1}^n d_{ii} = d_{11} \cdot d_{22} \cdot ... \cdot d_{nn}
$$
\end{frame}
%
%
\begin{frame}\frametitle{Beweis}
Wir betrachten den Fall für eine obere Dreiecksmatrix (untere Dreiecksmatrix als Übungsaufgabe).\\\pause
Wir verwenden für den Beweis das Prinzip der vollständigen Induktion.\\\pause
\highlightDef{Induktionsanfang:} $n=1$, dann gilt $\det(D)=d_{11}$.\\\pause
\highlightDef{Induktionsvoraussetzung:} Für jede obere $(n-1)\times(n-1)$-Dreiecksmatrix $D$ gelte $\det(D)=\prod_{i=1}^{n-1} d_{ii}$.\\\pause
\highlightDef{Induktionsschritt:} Es sei $D \in \R^{n\times n}$ eine obere Dreiecksmatrix. Wir entwickeln die Determinante von $D$ nach der ersten Spalte:\\\pause
$\det(D)=d_{11}\cdot(-1)^{1+1}\det(D^{11})+d_{21}\cdot(-1)^{2+1}\det(D^{21})+...$\\
\hspace{15mm}$...+d_{n1}\cdot(-1)^{n+1}\det(D^{n1})$\\\pause
Da $D$ eine obere Dreiecksmatrix ist, gilt $d_{i1}=0$ für $i\ne1$, und somit bleibt nur der erste Summand stehen. Da weiter $D^{11}\in\R^{(n-1)\times(n-1)}$ eine obere Dreiecksmatrix ist, folgt mit der Induktionsvoraussetzung:\\\pause
$\det(D)=d_{11}\cdot \det(D^{11})=\pause d_{11}\cdot \prod_{i=2}^n d_{ii}= \pause \prod_{i=1}^n d_{ii}$ \hfill $\square$.
 

\end{frame}
%
\subsection{Eigenschaften der Determinante}
%
\begin{frame}\frametitle{Weitere Eigenschaften der Determinante}
Es seien $A,B \in \R^{n\times n}$ und $k \in \R$. Dann gilt: \pause\vfill
\begin{itemize}
\item[1)] $\det(A\cdot B)=\det(A) \cdot \det (B)$\pause\vfill
\item[2)] $\det(k\cdot A)=k^n\cdot \det(A)$\pause\vfill
\item[3)] $\det(A^T)=\det(A)$\pause\vfill
\item[4)] Falls $\det(A)\ne0$, dann $\det(A^{-1})=\frac{1}{\det(A)}$.
\end{itemize}
\end{frame}
%
\subsection{Determinanten von speziellen Matrizen}
%
\begin{frame}\frametitle{Determinanten von Additionsmatrizen}
Additionsmatrizen $A_{i,j}(\alpha)=I_n+\alpha\cdot E_{i,j}$ haben Determinante $\det(A_{i,j})=1$, denn sie sind Dreiecksmatrizen mit Einsen auf der Diagonalen. \pause\\\vfill
Es gilt also für jede Matrix $M \in \R^{n\times n}$: 
$$
\det(A_{i,j}(\alpha)\cdot M)=\det(A_{i,j}(\alpha))\cdot \det( M)=\det(M)
$$\pause\vfill
Damit können wir Matrizen vor bzw. während dem Entwickeln der Determinante noch vereinfachen.

\end{frame}
%
\begin{frame}\frametitle{Determinanten von Vertauschungsmatrizen}
Da die Determinante per Definition alternierend ist, muss für Vertauschungsmatrizen gelten, dass sie Determinante $-1$ haben:\\\vfill\pause
Sei $A \in \R^{n\times n}$ beliebig und $A(i,j)$ die Matrix, die aus $A$ entsteht, wenn die $i$-te und die $j$-te Spalte vertauscht. Dann gilt:\pause
\begin{itemize}
\item[1)] $\det(A^T)=\det(A)=-\det(A(i,j))=-\det(A(i,j)^T)$ \pause
\item[2)] $A(i,j)^T = V_{i,j}\cdot A^T$ \pause
\item[3)] $\det(A(i,j)^T)=\det(V_{i,j}\cdot A^T)=\det(V_{i,j})\cdot\det( A^T)$
\end{itemize}
\vfill
Betrachtet man nun $A$ mit $\det(A)=\det(A^T)=1 \ne 0$, so folgt aus der Kombination von 1) mit 3)
$$
-1=\det(A(i,j)^T)=\det(V_i,j)\cdot \det(A^T)=\det(V_i,j).
$$
\end{frame}
%
\begin{frame}\frametitle{Beispiel}
$A=\begin{pmatrix} 1&1&1&1&2 \\ 1&1&1&1&1 \\ 2&3&3&3&5 \\ 2&3&1&1&2 \\ 2&3&4&5&5 \end{pmatrix}$\\\pause\vfill
Wir addieren die zweite Zeile $(-1)$-mal auf die erste Zeile\\\vfill
$A_{1,2}(-1)\cdot A=\begin{pmatrix} 0&0&0&0&1 \\ 1&1&1&1&1 \\ 2&3&3&3&5 \\ 2&3&1&1&2 \\ 2&3&4&5&5 \end{pmatrix}$\\\pause\vfill
Dann entwickeln wir die Determinante nach der ersten Zeile:
$\det(A)=\det(A_{1,2}(-1)\cdot A)=1\cdot(-1)^{1+5}\cdot \det(\begin{pmatrix}  1&1&1&1\\ 2&3&3&3 \\ 2&3&1&1 \\ 2&3&4&5 \end{pmatrix})$

\end{frame}
%
%
\begin{frame}\frametitle{Beispiel}
Dann addieren wir das $(-3)$-fache der ersten Zeile auf die zweite Zeile und erhalten\\
$\det(A)=\det(\begin{pmatrix}  1&1&1&1\\ 2&3&3&3 \\ 2&3&1&1 \\ 2&3&4&5 \end{pmatrix})=\det(\begin{pmatrix}  1&1&1&1\\ -1&0&0&0 \\ 2&3&1&1 \\ 2&3&4&5 \end{pmatrix})$\\\pause
\vfill Jetzt entwickeln wir nach der zweiten Zeile:\\\vfill
$\det(A)=-1\cdot(-1)^{2+1}\cdot\det(\begin{pmatrix} 1&1&1 \\ 3&1&1 \\ 3&4&5\end{pmatrix})=\det(\begin{pmatrix} 1&1&1 \\ 3&1&1 \\ 3&4&5\end{pmatrix})$
\end{frame}
%
%
\begin{frame}\frametitle{Beispiel}
Als nächstes addieren wir das $(-1)$-fache der ersten Zeile auf die zweise Zeile und entwickeln dann nach der zweiten Zeile\\\vfill
$\det(A)=\det(\begin{pmatrix} 1&1&1 \\ 3&1&1 \\ 3&4&5\end{pmatrix})=\det(\begin{pmatrix} 1&1&1 \\ 2&0&0 \\ 3&4&5\end{pmatrix})$\\\pause
\hspace{11.75mm}$=2\cdot(-1)^{2+1}\cdot\det(\begin{pmatrix} 1&1 \\ 4 & 5 \end{pmatrix})=-2\cdot\det(\begin{pmatrix} 1&1 \\ 4 & 5 \end{pmatrix})$ \pause \\\vfill
Mit der Determinanten-Formel für $2\times2$-Matrizen erhalten wir schließlich\\ \vfill
$\det(A)=-2\cdot(1\cdot5 - 1\cdot 4)=-2 \cdot 1 =-2$.

\end{frame}
%
\subsection{Formel für inverse Matrix}
%
\begin{frame}\frametitle{Formel für die inverse Matrix}
Es sei $A \in \R^{n\times n}$ eine invertierbare Matrix, d.h. $\det(A)\ne 0$.
Mit Hilfe der Determinante kann man eine Formel für die inverse Matrix $A^{-1}$ angeben.\pause Dazu sei die \highlightDef{komplementäre Matrix} $\bar A := (\bar a_{jk})$ gegeben durch
$$
\bar a_{jk}=(-1)^{j+k}\det(A^{kj})
$$\pause
Dann gilt:
$$
A^{-1}=\frac{1}{\det(A)}\cdot \bar A
$$\pause\vfill
In der Praxis ist diese Formel aber oft sehr aufwendig auszuwerten.

\end{frame}
%
\begin{frame}\frametitle{Beispiel}
Sei $A=\begin{pmatrix} 1 & 2 \\ 3 & 4\end{pmatrix}$.\pause Dann ist $\det(A)=1\cdot 4 - 2 \cdot 3 =-2 \ne 0$.\\
Außerdem:
$$
A^{11}=4, \   A^{12}=3, \ A^{21}=2 \text{ und }  A^{22}=1
$$\pause
Da die Determinante einer $1\times1$-Matrix gerade deren Eintrag ist, folgt
$$
\bar A=\begin{pmatrix} 4 & -2 \\ -3 & 1\end{pmatrix}
$$\pause
Somit erhalten wir 
$$
A^{-1}=\frac{1}{\det(A)}\cdot \bar A=\frac{1}{-2}\begin{pmatrix} 4 & -2 \\ -3 & 1\end{pmatrix}=\begin{pmatrix}-2 & 1 \\ \frac{3}{2} & -\frac{1}{2}\end{pmatrix}
$$\pause
Probe: $A\cdot A^{-1}=\begin{pmatrix} 1 & 2 \\ 3 & 4\end{pmatrix} \cdot \begin{pmatrix}-2 & 1 \\ \frac{3}{2} & -\frac{1}{2}\end{pmatrix}$\\\hspace{24.5mm}$ =\pause\begin{pmatrix}1\cdot (-2)+2\cdot \frac{3}{2} & 1\cdot1 + 2\cdot(-\frac{1}{2}) \\ 3\cdot (-2)+4\cdot \frac{3}{2} & 3\cdot1+4\cdot(-\frac{1}{2}\end{pmatrix}=\pause\begin{pmatrix} 1 & 0 \\ 0 & 1\end{pmatrix}$
\end{frame}
%%
%
\end{document}