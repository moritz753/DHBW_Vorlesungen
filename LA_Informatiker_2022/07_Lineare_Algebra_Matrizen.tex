\documentclass{beamer}

\usepackage{beamerthemesplit}

\usepackage{amsfonts}
\usepackage{amsmath}
\usepackage{amssymb}
\usepackage{amsthm}
\usepackage{amscd}

\usepackage{stmaryrd} 					%\lightning
\usepackage{algorithm2e}


\usepackage[ngerman]{babel}

\usepackage[utf8]{inputenc}
\usepackage[T1]{fontenc}
\usepackage{textcomp}


% Color Definitions
\definecolor{dhbwRed}{RGB}{226,0,26} 
\definecolor{dhbwGray}{RGB}{61,77,77}
\definecolor{lightBlue}{RGB}{28,134,230}

% Basic Theme
\usetheme{Malmoe}

% Color Re-Definitions
\usecolortheme[named=lightBlue]{structure}
\setbeamercolor*{alerted  text}{fg=dhbwRed, bg=white}
\setbeamercolor*{subsection in toc}{fg=dhbwGray, bg=white}

%\setbeamercolor*{palette primary}{fg=white,bg=lightBlue}
%\setbeamercolor*{palette secondary}{fg=white,bg=gray}
%\setbeamercolor*{palette tertiary}{fg=white,bg=gray}
%\setbeamercolor*{palette quaternary}{fg=white,bg=dhbwRed}

% no navigation symbols
\setbeamertemplate{navigation symbols}{}

% headline, footline
\setbeamertemplate{footline}{\color{dhbwGray} \hfill\insertframenumber\hspace{5mm}\vspace{2mm}}
\setbeamertemplate{headline}{}

% Title Page
\newcommand*{\makeTitlePage}{
	
	\begin{frame}[plain]
		
		\vfill
		\vfill
		\begin{center}
			{
				\usebeamerfont{title}
				\usebeamercolor[fg]{title}
				\Large
				\inserttitle
			}\\[3mm]
			{	
				\usebeamerfont{subtitle}
				\usebeamercolor[fg]{subtitle}
				\large
				\insertsubtitle
			}
		\end{center}
		%
		\vfill
		\vfill
		\vfill
		\vfill
		%
		\begin{columns}
			\begin{column}{0.5\textwidth}
				\begin{flushleft}
					{
						\usebeamerfont{normal text}
						\color{dhbwGray!80}
						\scriptsize
						Dr. Moritz Gruber\\
						DHBW Karlsruhe\\
						
					}
				\end{flushleft}
			\end{column}
			%
			\begin{column}{0.5\textwidth}
				\begin{flushright}
					\includegraphics[scale=0.06]{../DHBW.png}
				\end{flushright}
			\end{column}
		\end{columns}
		%
		\vspace{1mm}
		\begin{columns}
			\begin{column}{0.5\textwidth}
				\begin{flushleft}
					{
						\usebeamerfont{normal text}
						\color{dhbwGray!80}
						\scriptsize
						Version \today
					}
				\end{flushleft}
			\end{column}
			%
			\begin{column}{0.5\textwidth}
				% nothing (just a placeholder to be in line with the columns above
			\end{column}
		\end{columns}
	\end{frame}

}

% Section Divider Page
\newcommand*{\makeSectionDividerPage}{

	\begin{frame}[plain]
		\begin{center}
			\begin{flushleft}
				{				
					\usebeamercolor[fg]{frametitle}
					{\Large \insertsection} \\[3mm]
					{\large \insertsubsection}
				}
			\end{flushleft}
		\end{center}
        \end{frame}
	
}

% itemize
\setbeamertemplate{itemize items}[circle]
\setbeamertemplate{enumerate item}{(\theenumi)}




%--------------------------------------%
% Math ------------------------------%
%--------------------------------------%

% Mengen (Zahlen)
\newcommand{\N}{\mathbb{N}}
\newcommand{\Q}{\mathbb{Q}}
\newcommand{\R}{\mathbb{R}}
\newcommand{\Z}{\mathbb{Z}}
\newcommand{\C}{\mathbb{C}}

% Mengen (allgemein)
\newcommand{\K}{\mathbb{K}}
\newcommand\PX{{\cal P}(X)}

% Zahlentheorie
\newcommand{\ggT}{\mathrm{ggT}}


% Ableitungen
\newcommand{\ddx}{\frac{d}{dx}}
\newcommand{\pddx}{\frac{\partial}{\partial x}}
\newcommand{\pddy}{\frac{\partial}{\partial y}}
\newcommand{\grad}{\text{grad}}

%--------------------------------------%
% Layout Colors ------------------%
%--------------------------------------%
\newcommand*{\highlightDef}[1]{{\color{lightBlue}#1}}
\newcommand*{\highlight}[1]{{\color{lightBlue}#1}} % after theme for colours

%---------------------------------------------%
\title{Lineare Algebra}
\subtitle{Lineare Abbildungen \& Matrizen}

%---------------------------------------------%
\begin{document}

%---------------------------------------------%
\makeTitlePage

%---------------------------------------------%
\begin{frame}\frametitle{Inhalt}
   \tableofcontents
\end{frame}
%

%---------------------------------------------%
% Folien -----------------------------------%
%---------------------------------------------%
%
\section{Matrizen I}
\makeSectionDividerPage
%
\subsection{Definition}
%
\begin{frame}\frametitle{Definition: $m \times n$-Matrix}
Es sei $K$ ein Körper und $n,m \in \N$. Eine \highlightDef{$m\times n$-Matrix} $A$ über $K$ ist eine Anordnung von Elementen $a_{ij} \in K$ in $m$ Zeilen und $n$ Spalten:
$$
A=(a_{ij})_{i , j=1}^{m,n}=\begin{pmatrix}
					a_{11} & a_{12} & ... &a_{1n} \\ 
					a_{21} & a_{22} & ... &a_{2n} \\ 
					... & ...&...&... \\ 
					a_{m1} & a_{m2} & ... &a_{mn} \\ 
			  \end{pmatrix}
$$
Den ersten Index, $i$, nennt man den \highlightDef{Zeilenindex}, den zweiten Index, $j$, den \highlightDef{Spaltenindex}.\\\pause
Die $i$-te Zeile von $A$ wird mit $A_{i\cdot}$ und die $j$-te Spalte mit $A_{\cdot j}$ bezeichnet.\\
Die Menge aller $m\times n$-Matrizen über $K$ notiert man als \highlightDef{$K^{m\times n}$} (oder auch $Mat(K,m,n)$).
\end{frame}
%
\begin{frame}\frametitle{Beispiele}
\begin{itemize}
\item $A=\begin{pmatrix} 1 & 2 & 3 \\ 1 & \sqrt{2} & \sqrt{3} \end{pmatrix} \in \R^{2\times 3}$ ist eine reelle $2\times 3$-Matrix.\pause
\vfill
\item $B=\begin{pmatrix} i & 2  \\ 1 & \sqrt{2} \\ 1 & 2+i \end{pmatrix} \in \C^{3\times 2}$ ist eine komplexe $3\times 2$-Matrix.\pause
\vfill
\item Spezialfall: Spaltenanzahl gleich 1
$$
C=\begin{pmatrix}  2  \\  \sqrt{2} \\ 1 \end{pmatrix} \in \R^{3\times 1}=\R^3
$$\pause
$\Rightarrow$ $n\times 1$-Matrizen über $K$ können als Vektoren aus $K^n$ verstanden werden.

\end{itemize}
\end{frame}
%
\subsection{Matrix-Vektor-Multiplikation}
\begin{frame}\frametitle{Matrix-Vektor-Multiplikation}
Es sei $A=(a_{ij})_{i,j=1}^{m,n} \in K^{m\times n}$ eine $m\times n$-Matrix über $K$ und $x=(x_j)_{j=1}^n\in K^n$ ein Vektor. Das \highlightDef{Matrix-Vektor-Produkt} von $A$ mit $x$ ist definert als
		$$
		A\cdot x 
		:=
		\begin{pmatrix}
			a_{11}x_1 + a_{12}x_2 + \ldots + a_{1n}x_n \\
			a_{21}x_1 + a_{22}x_2 + \ldots + a_{2n}x_n \\
			\vdots				\\
			a_{m1}x_1 + a_{m2}x_2 + \ldots + a_{mn}x_n\\
		\end{pmatrix}
		=
		\left(
			\sum_{j=1}^na_{ij}x_j
		\right)_{i=1}^{m} \in K^m
		$$\pause
\vfill
Man beachte, dass $x \in K^n$ und $A\cdot x \in K^m$.
\end{frame}
%
\begin{frame}\frametitle{Beispiele}
\begin{itemize}
\item $A=\begin{pmatrix} 1 & 2 & 3 \\  \frac{1}{2} & 1 & \frac{1}{3} \end{pmatrix} \in \R^{2\times 3}$ und $x=\begin{pmatrix} 4  \\ 5 \\ 6 \end{pmatrix} \in \R^{3}$\\\pause
$
A\cdot x=\begin{pmatrix} 1 & 2 & 3 \\  \frac{1}{2} & 1 & \frac{1}{3} \end{pmatrix} \cdot \begin{pmatrix} 4  \\ 5 \\ 6 \end{pmatrix}$\pause$=\begin{pmatrix} 1\cdot4 + 2\cdot5 + 3\cdot6 \\  \frac{1}{2}\cdot4 + 1\cdot5 +\frac{1}{3}\cdot6 \end{pmatrix}$\pause$ = \begin{pmatrix} 32 \\ 9 \end{pmatrix}\pause
$
\item $A=\begin{pmatrix} 1 & 2 & 3 \end{pmatrix} \in \R^{1\times 3}$ und $x=\begin{pmatrix} 4  \\ 5 \\ 6 \end{pmatrix} \in \R^{3}$\\\pause
$
A\cdot x=\begin{pmatrix} 1 & 2 & 3  \end{pmatrix} \cdot \begin{pmatrix} 4  \\ 5 \\ 6 \end{pmatrix}$\pause$=\begin{pmatrix} 1\cdot4 + 2\cdot5 + 3\cdot6 \end{pmatrix}$\pause$ = \begin{pmatrix} 32 \end{pmatrix} = 32
$
\end{itemize}
\end{frame}
%--------------------------------------------
\section{Lineare Abbildungen}
%%%
%
%--------------------------------------------
\subsection{Definition}
\makeSectionDividerPage
%%%
%
\begin{frame}\frametitle{Definition}

	Seien $V$ und $W$ zwei $K$-Vektorräume.
	Eine Abbildung
	$$
		\Phi: V \to W
	$$ 
	nennt man \highlightDef{lineare Abbildung} oder \highlightDef{Homomorphismus}, 
	wenn für alle $u,v \in V$ und $a\in K$ gilt:
	\begin{itemize}
		\item[i)] $\Phi(u+v)		= \Phi(u) + \Phi(v)$
		\item[ii)] $\Phi(a\cdot v)	= a\cdot \Phi(v).$
	\end{itemize}
	
	\pause
	\vspace{10mm}
	Durch i) ist $\Phi$ insbesondere ein Gruppenhomomorphismus von $(V,+)$ nach $(W,+)$.
	
\end{frame}
%
\begin{frame}\frametitle{Eigenschaften lineare Abbildungen}
\begin{itemize}
\item Ist $B=\{b_1,...,b_n\}$ eine Basis von $V$, so ist jeder Homomorphismus $\Phi: V \to W$ durch die Bilder der Basisvektoren eindeutig bestimmt.\\\pause 
Denn: Ist $v\in V$, so gibt es eine eindeutige Darstellung $v=\sum_{j=1}^n a_j\cdot b_j$ mit Koeffizienten $a_j \in K$.\\\pause
Damit gilt:\\ $\Phi(v)$\pause$=\Phi(\sum_{j=1}^n a_j\cdot b_j)$ \pause$=\sum_{j=1}^n \Phi(a_j\cdot b_j)$\pause$=\sum_{j=1}^n a_j\cdot \Phi(b_j)$.\pause
\vfill
\item Ein Homomorphismus $\Phi: V \to W$ ist genau dann injektiv, wenn das Urbild des Nullvektors nur aus dem Nullvektor besteht: $\Phi^{-1}(\{0_W\})=\{0_V\}$. \qquad (Übungsaufgabe)\\\pause
$\Phi^{-1}(\{0_W\})$ wird als \highlightDef{Kern} von $\Phi$ bezeichnet und ist ein Untervektorraum von $V$.
\end{itemize}
\end{frame}
%
%--------------------------------------------
\subsection{Beispiele}
%%%
%
\begin{frame}\frametitle{Beispiel $\Phi: \R^n\to \R^n$}


	Seien $c \in \R$ und $V=W=\R^n$ (Vektorraum über $\R$).
	$$
		\Phi: \R^n \to \R^n, \, v \mapsto \Phi(v) := c\cdot v.
	$$\pause
	Es gilt:
$$
		\Phi(u+v) 		= c\cdot(u+v) = c\cdot u + c\cdot v = \Phi(u) + \Phi(v)
$$\pause
und 
$$
		\Phi(a\cdot v)	= c\cdot (a\cdot v) = (c\cdot a)\cdot v = a\cdot (c\cdot v) = a\cdot \Phi(v).
$$
	
\end{frame}
%
%
\begin{frame}\frametitle{Beispiel $\Phi: \R^2\to \R^2$}

	Seien $a_{11}, a_{12}, a_{21}, a_{22} \in \R$ und $V=W=\R^2$.
	$$
		\Phi\left( 	
				\begin{pmatrix}
					v_1	\\
					v_2	
				\end{pmatrix}
			\right) 
		:=
		\begin{pmatrix}
			a_{11}v_1 + a_{12}v_2	\\
			a_{21}v_1 + a_{22}v_2	
		\end{pmatrix}.
	$$
	\pause
	\vspace{4mm}
	
	Es gilt für $a \in \R$,
	$
		u = 
		\begin{pmatrix}
			u_1	\\
			u_2	
		\end{pmatrix}
	$
	und
	$
		v = 
		\begin{pmatrix}
			v_1	\\
			v_2	
		\end{pmatrix}:
	$			\pause	
	\begin{align*}
		\Phi(u+v) 		
			&= 
			\begin{pmatrix}
				a_{11}(u_1+v_1) + a_{12}(u_2+v_2)	\\
				a_{21}(u_1+v_1) + a_{22}(u_2+v_2)		
			\end{pmatrix}	\\
			&=
			\begin{pmatrix}
				a_{11}u_1 + a_{12}u_2	\\
				a_{21}u_1 + a_{22}u_2		
			\end{pmatrix}
			+
			\begin{pmatrix}
				a_{11}v_1 + a_{12}v_2	\\
				a_{21}v_1 + a_{22}v_2		
			\end{pmatrix}	\\
			&=
			\Phi(u) + \Phi(v).	
	\end{align*}	\pause
	Analog:
	$$
		\Phi(a\cdot v) = a\cdot \Phi(v).
	$$	
	
\end{frame}
%
%
\begin{frame}\frametitle{Beispiel $\Phi_A: \R^2\to \R^2$}

	Mit 
	$$
		A =
		\begin{pmatrix}
			a_{11} 	& a_{12}	\\
			a_{21}	& a_{22}	
		\end{pmatrix}
		\in \R^{2 \times 2}
		%
		\quad
		\text{und} 
		\quad
		%
		v = 
		\begin{pmatrix}
			v_1	\\
			v_2	
		\end{pmatrix}
		\in\R^2
	$$
	haben wir definiert
	$$
		A\cdot v
		:=
		\begin{pmatrix}
			a_{11}v_1 + a_{12}v_2	\\
			a_{21}v_1 + a_{22}v_2	
		\end{pmatrix}.
	$$ \pause
	Das hei{\ss}t
	$$
		\Phi(v) = \Phi_A(v) := A\cdot v.
	$$

\end{frame}
%
%
\begin{frame}\frametitle{Spiegelung am Ursprung}
	
	\vspace{2mm}
	Seien
	$$
		A=
		\begin{pmatrix}
			-1	&	0	\\
			0	&	-1
		\end{pmatrix},
		\qquad
		v =
		\begin{pmatrix}
			v_1\\
			v_2
		\end{pmatrix}.
	$$
	Dann folgt:
	$$
		\Phi_A(v) = A\cdot v =
		\begin{pmatrix}
			-v_1\\
			-v_2
		\end{pmatrix} 
		=
		-v.
	$$
	
	
	\vspace{5mm}
	\begin{center}
		\includegraphics[scale=0.7]{Grafiken/Spiegelung/spiegelung.pdf}
	\end{center}

\end{frame}
%
%
\begin{frame}\frametitle{Drehung um $\pi/2$}
	
	\vspace{2mm}
	Seien
	$$
		A=
		\begin{pmatrix}
			0	&	-1	\\
			1	&	0
		\end{pmatrix},
		\qquad
		v =
		\begin{pmatrix}
			v_1\\
			v_2
		\end{pmatrix}.
	$$
	Dann folgt:
	$$
		\Phi_A(v) = A\cdot v =
		\begin{pmatrix}
			-v_2\\
			v_1
		\end{pmatrix}.
	$$
	
	\vspace{5mm}
	\begin{center}
		\includegraphics[scale=0.7]{Grafiken/Drehung_90/drehung_90.pdf}
	\end{center}

\end{frame}
%
%
%\begin{frame}\frametitle{Drehung um den Winkel $\varphi$}
%
%	Die lineare Abbildung
%	$$
%		\Phi: \R^2 \to \R^2,\,
%		\Phi(v) := 
%		\begin{pmatrix}
%			\cos(\varphi)	&-\sin(\varphi)	\\
%			\sin(\varphi)	&\cos(\varphi)
%		\end{pmatrix}
%		\cdot
%		v
%	$$
%	dreht den Vektor $v\neq 0$ um den Winkel $\varphi$.\\
%	(\"Ubungsaufgabe)
%
%\end{frame}
%
%
\begin{frame}\frametitle{Beispiel $\Phi_A: \R^n \to \R^m$}

	Eine Matrix $A\in \R^{m\times n}$ 
	$$
		A = 
		\begin{pmatrix}
			a_{11}	&a_{12}	&\cdots	&a_{1n}\\
			a_{21} 	&a_{22} 	&\cdots 	&a_{2n}\\
			\vdots 	& \vdots   	&            	& \vdots \\
			a_{m1} 	&a_{m2}	&\cdots 	&a_{mn}
		\end{pmatrix}
	$$
	definiert eine \highlight{lineare Abbildung $\Phi_A: \R^n \to \R^m$} durch
	$$
		\Phi_A
		\left(
			\begin{pmatrix}
				x_1\\
				\vdots\\
				x_n
			\end{pmatrix}
	     	\right) 
	      :=
		\begin{pmatrix}
			a_{11}x_1 + a_{12}x_2 + \ldots + a_{1n}x_n \\
			a_{21}x_1 + a_{22}x_2 + \ldots + a_{2n}x_n \\
			\vdots				\\
			a_{m1}x_1 + a_{m2}x_2 + \ldots + a_{mn}x_n\\
		\end{pmatrix}.
	$$
	(Beweis: Analog zum Beispiel $\Phi_A: \R^2\to \R^2$.)
	
\end{frame}
%
%
\begin{frame}\frametitle{Beispiel $\Phi_A: \R^n \to \R^m$}
	
	Mit 
	$$
		x 
		=
		\begin{pmatrix}
			x_1\\
			\vdots\\
			x_n
		\end{pmatrix}
	$$
	haben wir definiert
	$$
		A\cdot x 
		:=
		\begin{pmatrix}
			a_{11}x_1 + a_{12}x_2 + \ldots + a_{1n}x_n \\
			a_{21}x_1 + a_{22}x_2 + \ldots + a_{2n}x_n \\
			\vdots				\\
			a_{m1}x_1 + a_{m2}x_2 + \ldots + a_{mn}x_n\\
		\end{pmatrix}
		=
		\left(
			\sum_{j=1}^na_{ij}x_j
		\right)_{i=1,\ldots, m}	
	$$
	Das hei{\ss}t
	$$
		\Phi_A(x) = A\cdot x.
	$$
	
\end{frame}
%
%
\subsection{Abbildungsmatrix}
%
\begin{frame}\frametitle{Abbildungsmatrix}
Die obigen Beispiele waren (fast) alle von der Form $\Phi_A : \R^n \to \R^m, \Phi_A(x)=A \cdot x$ mit einer Matrix $A \in \R^{m\times n}$. \pause\\ \vfill
In der Tat ist jeder Homomorphismus $\Phi: \R^n \to \R^m$ von dieser Form:\\\quad\\
\pause
\highlightDef{Satz}\\
Es sei $\Phi:\R^n \to \R^m$ ein Homomorphismus und $B=\{e_1,...,e_n\}$ die Standardbasis von $\R^n$. Dann gilt:
$$
\Phi=\Phi_A \ \text{ für die Matrix } A \text{ mit den Spalten } A_{\cdot j}=\Phi(e_j)
$$
\end{frame}
%
\begin{frame}\frametitle{Beispiel: Abbildungsmatrix}
	Seien $c \in \R$ und $V=W=\R^n$ (Vektorraum über $\R$).
	$$
		\Phi: \R^n \to \R^n, \, v \mapsto \Phi(v) := c\cdot v.
	$$\pause
	Seien nun $e_1,...,e_n$ die Standardbasisvektoren des $\R^n$. Dann\pause
	$$
	\Phi(e_1)=c\cdot e_1=\begin{pmatrix} c \\ 0 \\ 0\\... \\0 \end{pmatrix}, \Phi(e_2)=\begin{pmatrix} 0 \\ c \\ 0 \\ ... \\0 \end{pmatrix},...,\Phi(e_n)=\begin{pmatrix} 0 \\ 0 \\ ... \\ 0 \\c \end{pmatrix}
	$$\pause
	Das ergibt die Abbildungsmatrix
	$$
	A=\begin{pmatrix} c & 0 &0&... &0\\ 0&c&0&...&0 \\ 0&0 & ...&...&...\\... &...&...&c&0\\0&...&...&0&c \end{pmatrix}
	$$
\end{frame}
%
%--------------------------------------------
\section{Verkettung \& Matrixmultiplikation}
\makeSectionDividerPage
%%%
%
\begin{frame}\frametitle{Verkettung}
	
	\vspace{2mm}
	Seien $V_1, V_2, V_3$ Vektorräume über $K$ und 
	$\Phi_1: V_1 \to V_2$ sowie $\Phi_2:V_2\to V_3$ lineare Abbildungen.
	Dann ist auch
	$$
		\Phi_2\circ\Phi_1: V_1 \to V_3
	$$
	eine lineare Abbildung.\\[3mm]
	\pause
	\highlight{Beweis}
	\begin{align*}
		\Phi_2\circ\Phi_1(u+v) 
			&= \Phi_2\big( \Phi_1(u+v) \big) 			\\
			&= \Phi_2\big( \Phi_1(u) + \Phi_1(v) \big)	\\
			&= \Phi_2\big( \Phi_1(u) \big) + \Phi_2\big( \Phi_1(v) \big)	\\
			&= \Phi_2\circ\Phi_1(u) + \Phi_2\circ\Phi_1(v)	\\[1mm]
		\Phi_2\circ\Phi_1(a\cdot v) 
			&= \Phi_2\big( \Phi_1(a\cdot v) \big) 		\\
			&= \Phi_2\big( a\cdot \Phi_1(v) \big) 		\\
			&= a\cdot\Phi_2\big( \Phi_1(v) \big) 		\\
			&= a\cdot\Phi_2\circ\Phi_1(v)  		
	\end{align*}	
	
	\vspace{-3mm}\qed
	
\end{frame}
%
%
\begin{frame}\frametitle{Verkettung \& Matrixmultiplikation}
	
	Seien $A\in \R^{m\times n}$ und $B \in \R^{p\times m}$. 
	Dann sind
	$$
		\Phi_A: \R^n \to \R^m,\, x\mapsto A\cdot x
	$$
	und
	$$
		\Phi_B: \R^m\to \R^p, y\mapsto B\cdot y
	$$
	lineare Abbildungen.\\[3mm]
	\pause
	Dann ist $\Phi_B \circ \Phi_A: \R^n \to \R^p$ eine lineare Abbildung und besitzt daher eine Abbildungsmatrix $C \in \R^{p \times n}$.\pause\\
	Wie sieht die Matrix $C \in \R^{p\times n}$ mit
	$$
		\Phi_C = \Phi_B\circ \Phi_A\, 
	$$
	aus?
	


\end{frame}
%
%
\begin{frame}\frametitle{Verkettung \& Matrixmultiplikation}
	
$\Phi_B\circ \Phi_A(x) = \Phi_B\big(\Phi_A(x)\big)= \Phi_B(A\cdot x) $ \pause\\
\hspace{20mm}$=\Phi_B\left(\left( \sum_{j=1}^na_{ij}x_j \right)_{i=1}^{m}\right)	= B \cdot \left(\sum_{j=1}^na_{ij}x_j \right)_{i=1}^{m}$ \pause	\\
\hspace{20mm}$= \left( \sum_{i=1}^m b_{ki}\sum_{j=1}^na_{ij}x_j \right)_{k=1}^{p}$\\ \pause 
\hspace{20mm}$= \left( \sum_{j=1}^n (\sum_{i=1}^m b_{ki}a_{ij}) x_j \right)_{k=1}^{p}$ \pause	\\
\hspace{20mm}$= \Phi_C(x)$\\
	
	für die Matrix
	$$
		C:= \left(\sum_{i=1}^m b_{ki}a_{ij}\right)_{\substack{k=1\ldots p,\\ j=1\ldots n} } \in \R^{p \times n}.
	$$

\end{frame}
%
%
\begin{frame}\frametitle{Verkettung \& Matrixmultiplikation}
	
$\Phi_B\circ \Phi_A(x) = \Phi_B\big(\Phi_A(x)\big)= \Phi_B(A\cdot x) $ \\
\hspace{20mm}$=\Phi_B\left(\left( \sum_{j=1}^na_{ij}x_j \right)_{i=1}^{m}\right)	= B \cdot \left(\sum_{j=1}^na_{ij}x_j \right)_{i=1}^{m}$ 	\\
\hspace{20mm}$= \left( \sum_{i=1}^m b_{ki}\sum_{j=1}^na_{ij}x_j \right)_{k=1}^{p}$\\  
\hspace{20mm}$= \left( \sum_{j=1}^n (\textcolor{red}{\sum_{i=1}^m b_{ki}a_{ij}}) x_j \right)_{k=1}^{p}$ 	\\
\hspace{20mm}$= \Phi_C(x)$\\
	
	für die Matrix
	$$
		C:= \left(\textcolor{red}{\sum_{i=1}^m b_{ki}a_{ij}}\right)_{k=1, j=1}^{p, n}  \in \R^{p \times n}.
	$$

\end{frame}
%
%
\begin{frame}\frametitle{Definition: Matrixmultiplikation}
Seien $A\in \R^{m\times n}$ und $B \in \R^{p\times m}$. Das \highlightDef{Matrizenprodukt} von $B$ mit $A$ definiert man als

$B\cdot A = \begin{pmatrix}
			b_{11}	&b_{12}	&\ldots	&b_{1m}	\\
			\vdots	&		&		&\vdots	\\
			b_{p1}	&b_{p2}	&\ldots	&b_{pm}	\\
		\end{pmatrix}
		\cdot
		\begin{pmatrix}
			a_{11}	&a_{12}	&\ldots	&a_{1n}	\\
			a_{21}	&a_{22}	&\ldots	&a_{2n}	\\
			\vdots	&		&		&\vdots	\\
			a_{m1}	&a_{m2}	&\ldots	&a_{mn}	\\
		\end{pmatrix}	
		:=	
	$ \pause $	
		\begin{pmatrix}
			b_{11}a_{11} + b_{12}a_{21} + \ldots + b_{1m}a_{m1} & \cdots & b_{11}a_{1n}  + \ldots + b_{1m}a_{mn}\\
			\vdots &           & \vdots \\
			b_{p1}a_{11} + b_{p2}a_{21} + \ldots + b_{pm}a_{m1} & \cdots & b_{p1}a_{1n}  + \ldots + b_{pm}a_{mn}
		\end{pmatrix} 
	$
	$$
		= \left(\sum_{i=1}^m b_{ki}a_{ij}\right)_{k=1, j=1}^{p, n}
	$$

\end{frame}
%
%
\begin{frame}\frametitle{Verkettung \& Matrixmultiplikation}
	
	Es folgt:
	\begin{align*}
		\Phi_B\circ\Phi_A(x) 
			&= B\cdot (A\cdot x)	\\
			&= (B\cdot A)\cdot x	\\
			&= \Phi_{B\cdot A}(x).
	\end{align*}
	
\end{frame}
%
%
\begin{frame}\frametitle{Beispiele}

	\begin{itemize}
		\item[(1)]
			$$
				\begin{pmatrix}
					1			&2			&3	\\
					{4}	&{5}	&{6}	\\		
				\end{pmatrix}
				\cdot
				\begin{pmatrix}
					1	&{2}	\\
					3	&{4}	\\
					5	&{6}	
				\end{pmatrix}
				=
				\begin{pmatrix}
					22	&28	\\
					49	&{64}
				\end{pmatrix}
			$$


	
	\end{itemize}
\end{frame}
%
%
\begin{frame}\frametitle{Beispiele}

	\begin{itemize}
		\item[(1)]
			$$
				\begin{pmatrix}
					1			&2			&3	\\
					\highlight{4}	&\highlight{5}	&\highlight{6}	\\		
				\end{pmatrix}
				\cdot
				\begin{pmatrix}
					1	&\highlight{2}	\\
					3	&\highlight{4}	\\
					5	&\highlight{6}	
				\end{pmatrix}
				=
				\begin{pmatrix}
					22	&28	\\
					49	&\highlight{64}
				\end{pmatrix}
			$$
	
			Hierbei:
			$$
				\highlight{64 = 4\cdot 2 + 5\cdot 4 + 6\cdot 6.}
			$$

	
	\end{itemize}
\end{frame}
%
%
\begin{frame}\frametitle{Beispiele}

	\begin{itemize}
		\item[(1)]
			$$
				\begin{pmatrix}
					1			&2			&3	\\
					{4}	&{5}	&{6}	\\		
				\end{pmatrix}
				\cdot
				\begin{pmatrix}
					1	&{2}	\\
					3	&{4}	\\
					5	&{6}	
				\end{pmatrix}
				=
				\begin{pmatrix}
					22	&28	\\
					49	&{64}
				\end{pmatrix}
			$$
	
		\item[(2)]
			$$
				\begin{pmatrix}
					\highlight{1}			&\highlight{2}			&\highlight{3}	\\
					{4}	&{5}	&{6}	\\		
				\end{pmatrix}
				\cdot
				\begin{pmatrix}
					\highlight{1}	\\
					\highlight{3}	\\
					\highlight{5}		
				\end{pmatrix}
				=
				\begin{pmatrix}
					\highlight{22}	\\
					{49}
				\end{pmatrix}
			$$
	
	\end{itemize}
\end{frame}
%
%
\begin{frame}\frametitle{Beispiele}

	\begin{itemize}
		\item[(1)]
			$$
				\begin{pmatrix}
					1			&2			&3	\\
					{4}	&{5}	&{6}	\\		
				\end{pmatrix}
				\cdot
				\begin{pmatrix}
					1	&{2}	\\
					3	&{4}	\\
					5	&{6}	
				\end{pmatrix}
				=
				\begin{pmatrix}
					22	&28	\\
					49	&{64}
				\end{pmatrix}
			$$

		\item[(2)]
			$$
				\begin{pmatrix}
					1			&2			&3	\\
					\highlight{4}	&\highlight{5}	&\highlight{6}	\\		
				\end{pmatrix}
				\cdot
				\begin{pmatrix}
					\highlight{1}	\\
					\highlight{3}	\\
					\highlight{5}		
				\end{pmatrix}
				=
				\begin{pmatrix}
					22	\\
					\highlight{49}
				\end{pmatrix}
			$$
	
	\end{itemize}
\end{frame}
%
%--------------------------------------------
\section{Matrizen II}
%%%
%
%--------------------------------------------
\subsection{Eigenschaften der Matrixmultiplikation}
\makeSectionDividerPage
%%%
%
\begin{frame}\frametitle{Matrixmultiplikation ist assoziativ}

	Seien  $A\in \R^{m\times n}, B \in \R^{p\times m}, C \in \R^{q\times p}$.
	Da die Verkettung der Abbildungen $\Phi_A, \Phi_B, \Phi_C$ assoziativ ist, folgt für alle $x\in \R^n$: 
	\begin{align*}
		C\cdot\big((B\cdot A)\cdot x\big) 
			&= \Phi_C\big( \Phi_{B\cdot A}(x) \big)		\\
			&= \Phi_C\big( \Phi_B\circ\Phi_A(x)\big)	\\
			&= \Phi_C\circ\Phi_B\big(\Phi_A(x)\big)	\\
			&= \Phi_{C\cdot B}\big(\Phi_A(x)\big)		\\
			&= (C\cdot B)\cdot (A\cdot x) 
	\end{align*}
	und somit
	$$
		C\cdot(B\cdot A) = (C\cdot B)\cdot A,
	$$
	denn eine Matrix $A$ legt die zugehörige lineare Abbildung $\Phi_A$ eindeutig fest. 
	
\end{frame}
%
%--------------------------------------------
\subsection{Quadratische Matrizen}
\makeSectionDividerPage
%%%
%
\begin{frame}\frametitle{Einheitsmatrix}
	
	Die \highlightDef{Einheitsmatrix}
	$$
		I_n = 
		\begin{pmatrix}
			1 & 0 & 0 & \cdots & 0 \\
			0 & 1 & 0 & \cdots & 0\\
			0 & 0 & 1 & \cdots & 0\\
			\vdots & & & & \vdots\\
			0 & 0 & 0 & \cdots & 1
		\end{pmatrix} \in \R^{n\times n}
	$$
	ist das neutrale Element der Matrixmultiplikation in $\R^{n\times n}$ 	
	
\end{frame}
%
%
\begin{frame}\frametitle{Inverse Matrix}

	Sei $A\in \R^{n\times n}$.
	Die zu $A$ \highlightDef{inverse Matrix} $A^{-1}\in\R^{n\times n}$ ist diejenige Matrix, für die 
	$$
		A^{-1}\cdot A = A\cdot A^{-1} = I_n
	$$
	gilt.
	\pause
	\vfill
	\highlight{Bemerkungen}
	\begin{itemize}
		\item[(a)] 
			Die Abbildung $\Phi_{A^{-1}}$ ist die Umkehrabbildung von $\Phi_A$,
			denn für alle $x\in \R^n$ gilt:
			$$
				\Phi_{A^{-1}} \circ \Phi_A(x) 
				= A^{-1}\cdot(A\cdot x) 
				= (A^{-1}\cdot A)\cdot x
				= I_n\cdot x
				= x.
			$$\pause
	\item[(b)] Es gibt nicht zu jeder Matrix $A\in\R^{n\times n}$ eine inverse Matrix. 
	\end{itemize}
	
\end{frame}
%
%--------------------------------------------
\subsection{Matrixaddition}
\makeSectionDividerPage
%%%
%
\begin{frame}\frametitle{Matrixaddition}

	Für 
	$$
		A = 
			\begin{pmatrix}
				a_{11}	&a_{12}	&\cdots	&a_{1n}\\
				a_{21} 	&a_{22} 	&\cdots 	&a_{2n}\\
				\vdots 	&\vdots   	&            	&\vdots \\
				a_{m1} 	&a_{m2} 	&\cdots 	&a_{mn}
			\end{pmatrix},
		\;
		B = 
			\begin{pmatrix}
				b_{11}	&b_{12}	&\cdots	&b_{1n}\\
				b_{21} 	&b_{22} 	&\cdots 	&b_{2n}\\
				\vdots 	&\vdots   	&            	& \vdots \\
				b_{m1} 	&b_{m2} 	&\cdots 	&b_{mn}
			\end{pmatrix}
		\in \R^{m\times n}
	$$
	definieren wir
	$$
		A+B := 
			\begin{pmatrix}
				a_{11} + b_{11}		&a_{12} + b_{12}	&\cdots	&a_{1n} + b_{1n}\\
				a_{21} + b_{21}		&a_{22} + b_{22}	&\cdots	& a_{2n} + b_{2n}\\
				\vdots 			&\vdots   			&            	& \vdots \\
				a_{m1} + b_{m1}	&a_{m2} + b_{m2}	&\cdots	& a_{mn} + b_{mn}
			\end{pmatrix} \in \R^{m \times n}
	$$
	\pause
	Es gilt für alle $x\in \R^n$:
	$$
		\Phi_A(x) + \Phi_B(x) = \Phi_{A+B}(x). 
	$$
	
\end{frame}
%
%
\begin{frame}\frametitle{Nullmatrix}

	Die \highlightDef{Nullmatrix}
	$$
		\begin{pmatrix}
			0		&\cdots	&0		\\
			\vdots	&		&\vdots	\\
			0		&\cdots	&0
		\end{pmatrix}
		\in \R^{m\times n}
	$$
	ist das neutrale Element der Matrixaddition  in $\R^{m\times n}$ und	
	$(\R^{m\times n}, +)$ ist eine abelsche Gruppe.

\end{frame}
%
%--------------------------------------------
\subsection{Matrizen als Vektorraum}
\makeSectionDividerPage
%%%
%
\begin{frame}\frametitle{Multiplikation mit einem Körperelement}

	Für $c\in\R$ und 
	$$
		A = 
			\begin{pmatrix}
				a_{11}	&a_{12}	&\cdots	&a_{1n}\\
				a_{21} 	&a_{22} 	&\cdots 	&a_{2n}\\
				\vdots 	&\vdots   	&            	&\vdots \\
				a_{m1} 	&a_{m2} 	&\cdots 	&a_{mn}
			\end{pmatrix}
			=(a_{ij})_{i,j}\in \R^{m\times n}
	$$
	definieren wir
	$$
		c\cdot A := 
			\begin{pmatrix}
				ca_{11}	&ca_{12}	&\cdots	&ca_{1n}\\
				ca_{21} 	&ca_{22} 	&\cdots 	&ca_{2n}\\
				\vdots 	&\vdots   	&            	&\vdots \\
				ca_{m1} 	&ca_{m2} 	&\cdots 	&ca_{mn}
			\end{pmatrix}
			=(ca_{ij})_{i,j}.
	$$
\pause
	Es gilt für alle $x\in \R^n$:
	$$
		\Phi_{c\cdot A}(x) = (c\cdot A)\cdot x = c\cdot(A\cdot x) = c\cdot \Phi_A(x). 
	$$
	
\end{frame}
%
%
\begin{frame}\frametitle{$\R^{m\times n}$ als Vektorraum}

	$(\R^{m\times n},+,\cdot)$ ist ein Vektorraum über $\R$ mit $\dim(\R^{m\times n}) = m\cdot n$.\\[10mm]
	
	\pause
	\highlight{Beweisskizze} 
	\begin{itemize}
		\item[(1)] $(\R^{m\times n},+,\cdot)$ ist ein Vektorraum: 
			\"Uberprüfung der definierenden Eigenschaften eines Vektorraums.\pause
		\item[(2)] Die Matrizen $E_{ij}$, die in der Stelle $(i,j)$ (Zeile $i$, Spalte $j$) den Wert $1$ haben 
			und an den anderen Stellen eine $0$, bilden eine Basis.
	\end{itemize}

\end{frame}
%
%--------------------------------------------
\subsection{Spezielle Matrizen}
\makeSectionDividerPage
%%%
%
\begin{frame}\frametitle{Elementarmatrizen}
Die Matrizen $E_{ij}$, die in der Stelle $(i,j)$ (Zeile $i$, Spalte $j$) den Wert $1$ haben und an den anderen Stellen eine $0$, heißen \highlightDef{Elementarmatrizen}.
			
$$
E_{ij}:=\begin{pmatrix}
0 & 0& ... &0& 0\\
0 & 0& ... &0& 0\\
...&...&0&...&...\\
...& 0& 1 &0& ...\\
...&...&0&...&...\\
0& 0&...& 0
\end{pmatrix}
$$

\end{frame}
%
\begin{frame}\frametitle{Transponierte Matrix}
	
	Sei
	$$
		A = 
			\begin{pmatrix}
				a_{11}	&a_{12}	&\cdots	&a_{1n}\\
				a_{21} 	&a_{22} 	&\cdots 	&a_{2n}\\
				\vdots 	&\vdots   	&            	&\vdots \\
				a_{m1} 	&a_{m2} 	&\cdots 	&a_{mn}
			\end{pmatrix}
			\in \R^{m\times n}.
	$$
	
	\vspace{3mm}
	Die zu $A$  \highlightDef{transponierte} Matrix $A^{\top}$ ist definiert durch
	$$
	A^{\top} := 
			\begin{pmatrix}
				a_{11}	&a_{21}	&\cdots	&a_{m1}\\
				a_{12} 	&a_{22} 	&\cdots 	&a_{m2}\\
				\vdots 	&\vdots   	&            	&\vdots \\
				a_{1n} 	&a_{2n} 	&\cdots 	&a_{mn}
			\end{pmatrix}
			\in \R^{n\times m}.
	$$
	
\end{frame}
%
%
\begin{frame}\frametitle{Beispiele}

	$$
		A =
		\begin{pmatrix}
			1	&2	&3	\\
			4	&5	&6	
		\end{pmatrix},
		\qquad
		A^{\top} =
		\begin{pmatrix}
			1	&4	\\
			2	&5	\\
			3	&6	
		\end{pmatrix}.
	$$
	
	$$
		v = 
		\begin{pmatrix}
			1	\\
			2	\\
			3
		\end{pmatrix},
		\qquad
		v^{\top} =
		\begin{pmatrix}
			1	&2	&3
		\end{pmatrix}
		.
	$$
\end{frame}
%
\begin{frame}\frametitle{Additionsmatrix}
Für $1\le i \ne j \le n$ und $\alpha \in \R$ definieren wir die Matrix $A_{i,j}(\alpha) \in \R^{n\times n}$ durch
$$
A_{i,j}(\alpha):=I_n + \alpha \cdot E_{i,j}
$$
Die Matrix $A_{i,j}(\alpha)$ heißt \highlightDef{Additionsmatrix} und für jede Matrix $M \in \R^{n\times m}$ ist $A_{i,j}(\alpha)\cdot M$ die Matrix, die aus $M$ entsteht, wenn man zur $i$-ten Zeile das $\alpha$-fache der $j$-ten Zeile addiert.
\vfill
Additionsmatrizen sind invertierbar mit inverser Matrix $A_{i,j}(\alpha)^{-1}=A_{i,j}(-\alpha)$.
\end{frame}
%
\begin{frame}\frametitle{Beispiel für Additionsmatrix}
Es sei $n=2$ und 
	$$
		A_{1,2}(\alpha)=\begin{pmatrix}
			1 & \alpha\\
			0 & 1
		\end{pmatrix}
	$$
	ist invertierbar mit inverser Matrix
	$$
		\begin{pmatrix}
			1 & -\alpha\\
			0 & 1
		\end{pmatrix}.
	$$ \pause
	Es gilt:
	$$
		\begin{pmatrix}
			1 & \alpha\\
			0 & 1
		\end{pmatrix} 
		\cdot
		\begin{pmatrix}
			a & b & \ldots & c\\
			d & e  & \ldots & f
		\end{pmatrix}
		=
		\begin{pmatrix}
			a +\alpha d& b + \alpha e & \ldots & c + \alpha f\\
			d & e  & \ldots & f
		\end{pmatrix}.
	$$
	
\end{frame}
%
%
\begin{frame}\frametitle{Vertauschungsmatrix}
Für $1\le i \ne j \le n$ definieren wir die Matrix $V_{i,j} \in \R^{n\times n}$ durch
$$
V_{i,j}:=I_n +  E_{i,j}+  E_{j,i}-  E_{i,i}-  E_{j,j}
$$
Die Matrix $V_{i,j}$ heißt \highlightDef{Vertauschungsmatrix} und für jede Matrix $M \in \R^{n\times m}$ ist $V_{i,j}\cdot M$ die Matrix, die aus $M$ entsteht, wenn man in $M$ die $i$-te und die $j$-te Zeile vertauscht.
\vfill
Vertauschungsmatrizen sind invertierbar mit Inversermatrix $V_{i,j}^{-1}=V_{i,j}$.
\end{frame}
%
%
\begin{frame}\frametitle{Beispiel für Vertauschungsmatrix}

	Es sei $n=2$ und 
	$$ 
		V_{1,2} =
		\begin{pmatrix}
			0 & 1\\
			1 & 0
		\end{pmatrix}
	$$
	ist invertierbar mit inverser Matrix $V_{1,2}$.\\
\pause \vfill
	Es gilt:
	$$
		\begin{pmatrix}
			0 & 1\\
			1 & 0
		\end{pmatrix} 
		\cdot
		\begin{pmatrix}
			a & b & \ldots & c\\
			d & e  & \ldots & f
		\end{pmatrix}
		=
		\begin{pmatrix}
			d & e  & \ldots & f\\
			a & b & \ldots & c
		\end{pmatrix}.
	$$
	
\end{frame}
%
\begin{frame}\frametitle{Diagonalmatrizen}
Für $\alpha_1,...,\alpha_n \in \R$ definiert man die \highlightDef{Diagonalmatrix} $diag(\alpha_1,...,\alpha_n)$ als
$$
diag(\alpha_1,...,\alpha_n):= \sum_{i=1}^n \alpha_i E_{i,i}
$$
für jede Matrix $M \in \R^{n\times m}$ ist $V_{i,j}\cdot M$ die Matrix, die aus $M$ entsteht, wenn man in $M$ für alle $1\le i \le n$ die $i$-te Zeile mit $\alpha_i$ multipliziert.
\vfill
Wenn alle $\alpha_i$ ungleich $0$ sind, dann ist $diag(\alpha_1,...,\alpha_n)$ invertierbar mit inverser Matrix $diag(\alpha_1,...,\alpha_n)^{-1}=diag(\alpha_1^{-1},...,\alpha_n^{-1})$.
\end{frame}
%
\begin{frame}\frametitle{Beispiel für Diagonalmatrix}

	Für $\alpha, \beta \in \R\backslash\{0\}$ ist die Matrix
	$$ 
		diag(\alpha,\beta)=\begin{pmatrix}
			\alpha & 0\\
			0         & \beta
		\end{pmatrix}
	$$
	invertierbar mit inverser Matrix 
	$$ 
		\begin{pmatrix}
			1/\alpha & 0\\
			0         & 1/\beta
		\end{pmatrix}.
	$$
	\pause
	Es gilt:
	$$
		\begin{pmatrix}
			\alpha & 0\\
			0         & \beta
		\end{pmatrix}
		\cdot
		\begin{pmatrix}
			a & b & \ldots & c\\
			d & e  & \ldots & f
		\end{pmatrix}
		=
		\begin{pmatrix}
			\alpha a & \alpha b  & \ldots & \alpha c\\
			\beta d & \beta e & \ldots & \beta f
		\end{pmatrix}.
	$$
	
\end{frame}
%



\end{document}
