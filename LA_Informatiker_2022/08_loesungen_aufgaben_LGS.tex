\documentclass[
				a4paper,
				10pt
			]
			{scrartcl}

\parindent0mm

\usepackage{amsfonts}
\usepackage{amsmath}
\usepackage{amssymb}
\usepackage{amsthm}
\usepackage[ngerman]{babel}

\usepackage[utf8]{inputenc}
\usepackage[T1]{fontenc}
\usepackage{textcomp}

\usepackage{graphicx}
\usepackage{xcolor}

\usepackage[
			pdftex,
			colorlinks,
			breaklinks,
			linkcolor=blue,
			citecolor=blue,
			filecolor=black,
			menucolor=black,
			urlcolor=black,
			pdfauthor={Andreas Weber},
			pdftitle={Aufgaben zu Analysis und Lineare Algebra},
			plainpages=false,
			pdfpagelabels,
			bookmarksnumbered=true
		   ]{hyperref}


%--------------------------------------%
% Math ------------------------------%
%--------------------------------------%

% Mengen (Zahlen)
\newcommand{\N}{\mathbb{N}}
\newcommand{\Q}{\mathbb{Q}}
\newcommand{\R}{\mathbb{R}}
\newcommand{\Z}{\mathbb{Z}}
\newcommand{\C}{\mathbb{C}}

% Mengen (allgemein)
\newcommand{\K}{\mathbb{K}}
\newcommand\PX{{\cal P}(X)}

% Zahlentheorie
\newcommand{\ggT}{\mathrm{ggT}}


% Ableitungen
\newcommand{\ddx}{\frac{d}{dx}}
\newcommand{\pddx}{\frac{\partial}{\partial x}}
\newcommand{\pddy}{\frac{\partial}{\partial y}}
\newcommand{\grad}{\text{grad}}

%--------------------------------------%
% Layout Colors ------------------%
%--------------------------------------%
\newcommand*{\highlightDef}[1]{{\color{lightBlue}#1}}
\newcommand*{\highlight}[1]{{\color{lightBlue}#1}}
% Color Definitions
\definecolor{dhbwRed}{RGB}{226,0,26} 
\definecolor{dhbwGray}{RGB}{61,77,77}
\definecolor{lightBlue}{RGB}{28,134,230}


%-------------------------------------------------------------------
\begin{document}

\vspace*{-20mm}
{
	%\usekomafont{title}
	\color{dhbwGray}
	Dr. Moritz Gruber	\hfill Version \today\\
	DHBW Karlsruhe\\
}

\vspace{10mm}
\begin{center}
	{
		\usekomafont{title}
		\color{lightBlue}
		{ \LARGE L\"osungen Übungsaufgaben 8}\\[3mm]
		{\Large Lineare Gleichungssysteme}
	}
\end{center}

\vspace{5mm}

%-------------------------------------------------------------------


%-------------------------
\section{LGS}
%%%

Berechnen Sie die L\"osungsmenge des LGS

\begin{align*}
	x_1 - x_2 + 6x_3 + 8x_4 &= 1,\\
	-x_1 +2x_2 - 7x_3-2x_4 &= 1.
\end{align*}

%-------------------------
\subsection*{L\"osung}
%%%

$$
	\left(
		\begin{array}{rrrr | r}
			1	&-1	&6	&8	&1	\\
			-1	&2	&-7	&-2	&1
		\end{array}
	\right)
$$

Addition der ersten Zeile zur zweiten Zeile:
$$
	\sim>
	\left(
		\begin{array}{rrrr | r}
			1	&-1	&6	&8	&1	\\
			0	&1	&-1	&6	&2
		\end{array}
	\right)
$$

Addition der zweiten Zeile zur ersten Zeile:
$$
	\sim>
	\left(
		\begin{array}{rrrr | r}
			1	&0	&5	&14	&3	\\
			0	&1	&-1	&6	&2
		\end{array}
	\right)
$$

Das ist die Gau{\ss}-Normalform mit Stufen in den Spalten 1 und 2.\\

Freie Variablen $x_3 = t_1 \in \R$ und $x_4 = t_2 \in \R$,

\begin{align*}
	x_1 &= 3 - 5x_3 - 14x_4 = 3 - 5t_1 - 14t_2,\\
	x_2 &= 2 +x_3 - 6x_4 = 2 + t_1 -6t_2.
\end{align*}

F\"ur alle $t_1, t_2 \in \R$ ist also

$$
	\begin{pmatrix}
		x_1\\
		x_2\\
		x_3\\
		x_4
	\end{pmatrix}
	=
	\begin{pmatrix}
		3\\
		2\\
		0\\
		0
	\end{pmatrix}
	+
	t_1 \cdot
	\begin{pmatrix}
		-5\\
		1\\
		1\\
		0
	\end{pmatrix}	
	+t_2\cdot
	\begin{pmatrix}
		-14\\
		-6\\
		0\\
		1
	\end{pmatrix}	
$$
eine L\"osung des LGS. \\

L\"osungsmenge:
$$
	\left\{
		\begin{pmatrix}
			3\\
			2\\
			0\\
			0
		\end{pmatrix}
		+
		t_1 \cdot
		\begin{pmatrix}
			-5\\
			1\\
			1\\
			0
		\end{pmatrix}	
		+t_2\cdot
		\begin{pmatrix}
			-14\\
			-6\\
			0\\
			1
		\end{pmatrix}		
		~|~
		t_1, t_2 \in \R	
	\right\}.
$$

%-------------------------------------
\subsection*{Der $(-1)$-Trick}
%%%

Erg\"anze in der Gau{\ss}-Normalform (!)
$$
	\left(
		\begin{array}{rrrr | r}
			1	&0	&5	&14	&3	\\
			0	&1	&-1	&6	&2
		\end{array}
	\right)
$$

Nullzeilen, so dass die Stufen auf der Diagonalen liegen:

$$
	\left(
		\begin{array}{rrrr | r}
			1	&0	&5	&14	&3	\\
			0	&1	&-1	&6	&2	\\
			0	&0	&0	&0	&0	\\
			0	&0	&0	&0	&0
		\end{array}
	\right)
$$

Die rechte Seite 
$$
	\begin{pmatrix}
		3\\
		2\\
		0\\
		0
	\end{pmatrix}
$$
ist eine (spezielle) L\"osung des LGS.\\[1mm]

Ersetze auf der Diagonalen Nullen durch $-1$:

$$
	\left(
		\begin{array}{rrrr | r}
			1	&0	&5			&14			&3	\\
			0	&1	&-1			&6			&2	\\
			0	&0	&\highlight{-1}	&0			&0	\\
			0	&0	&0			&\highlight{-1}	&0
		\end{array}
	\right)
$$

Die Spalten mit den $(-1)$en auf der Diagonale erzeugen die L\"osungsmenge des zugeh\"origen homogenen LGS.\\[2mm]

Wir erhalten als L\"osungsmenge des inhomogenen LGS:

$$
	\left\{
		\begin{pmatrix}
			3\\
			2\\
			0\\
			0
		\end{pmatrix}
		+
		t_1 \cdot
		\begin{pmatrix}
			5\\
			-1\\
			-1\\
			0
		\end{pmatrix}	
		+t_2\cdot
		\begin{pmatrix}
			14\\
			6\\
			0\\
			-1
		\end{pmatrix}		
		~|~
		t_1, t_2 \in \R	
	\right\}.
$$

%-------------------------
\newpage
\section{Noch ein LGS}
%%%

Berechnen Sie die L\"osungsmenge des LGS

$$
	\left(
	\begin{array}{rrrrrr | r}
		0	& 2	& 4	& -2 	& 1	& 7	& -1\\
		1	& 0	& 1	& 3	& 0	& -1	& 1\\
		1	& 1	& 3	& 2	& 0	& 1	& 1\\
		0	& 1	& 2	& -1	& -1	& -1	& 1\\
		3	& 2	& 7	& 7	& -1	& -2	& 4
	\end{array}
	\right).
$$

%-------------------------
\subsection*{L\"osung}
%%%

$$
	\left(
	\begin{array}{rrrrrr | r}
		0	& 2	& 4	& -2 	& 1	& 7	& -1\\
		1	& 0	& 1	& 3	& 0	& -1	& 1\\
		1	& 1	& 3	& 2	& 0	& 1	& 1\\
		0	& 1	& 2	& -1	& -1	& -1	& 1\\
		3	& 2	& 7	& 7	& -1	& -2	& 4
	\end{array}
	\right).
$$
Vertauschen von Zeile 1 und 2:
$$
	\sim>
	\left(
	\begin{array}{rrrrrr | r}
		1	& 0	& 1	& 3	& 0	& -1	& 1\\	
		0	& 2	& 4	& -2 	& 1	& 7	& -1\\
		1	& 1	& 3	& 2	& 0	& 1	& 1\\
		0	& 1	& 2	& -1	& -1	& -1	& 1\\
		3	& 2	& 7	& 7	& -1	& -2	& 4
	\end{array}
	\right).
$$

Addition des $(-1)$-fachen von Zeile 1 zu Zeile 3 und des $(-3)$-fachen zu Zeile 5:
$$
	\sim>
	\left(
	\begin{array}{rrrrrr | r}
		1	& 0	& 1	& 3	& 0	& -1	& 1\\	
		0	& 2	& 4	& -2 	& 1	& 7	& -1\\
		0	& 1	& 2	& -1	& 0	& 2	& 0\\
		0	& 1	& 2	& -1	& -1	& -1	& 1\\
		0	& 2	& 4	& -2	& -1	& 1	& 1
	\end{array}
	\right).
$$

Vertauschen von  Zeile 2 und 3:
$$
	\sim>
	\left(
	\begin{array}{rrrrrr | r}
		1	& 0	& 1	& 3	& 0	& -1	& 1\\	
		0	& 1	& 2	& -1	& 0	& 2	& 0\\
		0	& 2	& 4	& -2 	& 1	& 7	& -1\\
		0	& 1	& 2	& -1	& -1	& -1	& 1\\
		0	& 2	& 4	& -2	& -1	& 1	& 1
	\end{array}
	\right).
$$

Addition des $(-2)$-fachen von Zeile 2 zu Zeile 3 und 5, des $(-1)$-fachen zu Zeile 4:
$$
	\sim>
	\left(
	\begin{array}{rrrrrr | r}
		1	& 0	& 1	& 3	& 0	& -1	& 1\\	
		0	& 1	& 2	& -1	& 0	& 2	& 0\\
		0	& 0	& 0	& 0 	& 1	& 3	& -1\\
		0	& 0	& 0	& 0	& -1	& -3	& 1\\
		0	& 0	& 0	& 0	& -1	& -3	& 1
	\end{array}
	\right).
$$

Addition von Zeile 3 zu Zeilen 4 und 5:
$$
	\sim>
	\left(
	\begin{array}{rrrrrr | r}
		1	& 0	& 1	& 3	& 0	& -1	& 1\\	
		0	& 1	& 2	& -1	& 0	& 2	& 0\\
		0	& 0	& 0	& 0 	& 1	& 3	& -1\\
		0	& 0	& 0	& 0	& 0	& 0	& 0\\
		0	& 0	& 0	& 0	& 0	& 0	& 0
	\end{array}
	\right).
$$

Gau{\ss}-Normalform mit Stufen in den Spalten 1, 2 und 5.\\

Freie Variablen
$$
	x_3 = t_1,\quad x_4 = t_2,\quad x_6 = t_3
$$
und
\begin{align*}
	x_1	&= 1 - x_3 - 3x_4 + x_6	= 1 - t_1 - 3t_2 + t_3,	\\
	x_2	&= -2x_3 + x_4-2x_6 	= -2t_1 + t_2 -2t_3,	\\
	x_5	&= -1 - 3x_6 			= -1 - 3t_3							
\end{align*}

F\"ur alle $t_1, t_2, t_3 \in \R$ ist also

$$
	\begin{pmatrix}
		x_1\\
		x_2\\
		x_3\\
		x_4\\
		x_5\\
		x_6
	\end{pmatrix}
	=
	\begin{pmatrix}
		1\\
		0\\
		0\\
		0\\
		-1\\
		0
	\end{pmatrix}
	+
	t_1 \cdot
	\begin{pmatrix}
		-1\\
		-2\\
		1\\
		0\\
		0\\
		0
	\end{pmatrix}	
	+t_2\cdot
	\begin{pmatrix}
		-3\\
		1\\
		0\\
		1\\
		0\\
		0
	\end{pmatrix}	
	+t_3\cdot
	\begin{pmatrix}
		1\\
		-2\\
		0\\
		0\\
		-3\\
		1
	\end{pmatrix}	
$$
eine L\"osung des LGS. \\

L\"osungsmenge:
$$
	\left\{
		\begin{pmatrix}
			1\\
			0\\
			0\\
			0\\
			-1\\
			0
		\end{pmatrix}
		+
		t_1 \cdot
		\begin{pmatrix}
			-1\\
			-2\\
			1\\
			0\\
			0\\
			0
		\end{pmatrix}	
		+t_2\cdot
		\begin{pmatrix}
			-3\\
			1\\
			0\\
			1\\
			0\\
			0
		\end{pmatrix}	
		+t_3\cdot
		\begin{pmatrix}
			1\\
			-2\\
			0\\
			0\\
			-3\\
			1
		\end{pmatrix}	
		~|~
		t_1, t_2, t_3 \in \R	
	\right\}.
$$

%-------------------------------------
\newpage
\subsection*{Der $(-1)$-Trick}
%%%

Erg\"anze in der Gau{\ss}-Normalform 
$$
	\left(
	\begin{array}{rrrrrr | r}
		1	& 0	& 1	& 3	& 0	& -1	& 1\\	
		0	& 1	& 2	& -1	& 0	& 2	& 0\\
		0	& 0	& 0	& 0 	& 1	& 3	& -1\\
		0	& 0	& 0	& 0	& 0	& 0	& 0\\
		0	& 0	& 0	& 0	& 0	& 0	& 0
	\end{array}
	\right).
$$

eine Nullzeile und vertausche Zeilen so, dass die $1$en der Stufen auf der Diagonalen stehen:

$$
	\left(
	\begin{array}{rrrrrr | r}
		1	& 0	& 1	& 3	& 0	& -1	& 1\\	
		0	& 1	& 2	& -1	& 0	& 2	& 0\\
		0	& 0	& 0	& 0	& 0	& 0	& 0\\
		0	& 0	& 0	& 0	& 0	& 0	& 0\\
		0	& 0	& 0	& 0 	& 1	& 3	& -1\\
		0	& 0	& 0	& 0	& 0	& 0	& 0
	\end{array}
	\right).
$$

Die rechte Seite
$$
	\begin{pmatrix}
		1\\
		0\\
		0\\
		0\\
		-1\\
		0
	\end{pmatrix}
$$
ist eine (spezielle) L\"osung des inhomogenen LGS.\\[2mm]

Erg\"anze $(-1)$en:
$$
	\left(
	\begin{array}{rrrrrr | r}
		1	& 0	& 1			& 3			& 0	& -1			& 1\\	
		0	& 1	& 2			& -1			& 0	& 2			& 0\\
		0	& 0	& \highlight{-1}	& 0			& 0	& 0			& 0\\
		0	& 0	& 0			&  \highlight{-1}	& 0	& 0			& 0\\
		0	& 0	& 0			& 0 			& 1	& 3			& -1\\
		0	& 0	& 0			& 0			& 0	&  \highlight{-1}	& 0
	\end{array}
	\right).
$$

L\"osungsmenge des inhomogenen LGS:
$$
	\left\{
		\begin{pmatrix}
			1\\
			0\\
			0\\
			0\\
			-1\\
			0
		\end{pmatrix}
		+
		t_1 \cdot
		\begin{pmatrix}
			1\\
			2\\
			-1\\
			0\\
			0\\
			0
		\end{pmatrix}	
		+t_2\cdot
		\begin{pmatrix}
			3\\
			-1\\
			0\\
			-1\\
			0\\
			0
		\end{pmatrix}	
		+t_3\cdot
		\begin{pmatrix}
			-1\\
			2\\
			0\\
			0\\
			3\\
			-1
		\end{pmatrix}	
		~|~
		t_1, t_2, t_3 \in \R	
	\right\}.
$$

%-------------------------
\newpage
\section{L\"osbarkeit}
%%%

Seien $\alpha,\beta \in \R$ und
$$
	A =
	\begin{pmatrix}
		1	&5	&\alpha	\\
		0	&-2	&1	\\
		-1	&1	&3
	\end{pmatrix},
	\qquad
	b =
	\begin{pmatrix}
		-1	\\
		\beta	\\
		1
	\end{pmatrix}.
$$

F\"ur welche Werte von $\alpha$ und $\beta$ ist das LGS $A\cdot x = b$
\begin{itemize}
	\item eindeutig l\"osbar,
	\item nicht l\"osbar,
	\item l\"osbar, aber nicht eindeutig l\"osbar?
\end{itemize}

%-------------------------
\subsection*{L\"osung}
%%%

$$
	\left(
		\begin{array}{rrr | r}
			1	&5	&\alpha	&-1\\
			0	&-2	&1	&\beta\\
			-1	&1	&3	&1
		\end{array}
	\right)
$$

Addition von Zeile 1 zu Zeile 3, Multiplikation von Zeile 2 mit $-\frac{1}{2}$:
$$
	\sim>
	\left(
		\begin{array}{rrr | r}
			1	&5	&\alpha		&-1\\
			0	&1	&-1/2	&-\beta/2\\
			0	&6	&\alpha+3	&0
		\end{array}
	\right)
$$
Addition des $(-6)$-fachen von Z2 zu Z3:
$$
	\sim>
	\left(
		\begin{array}{rrr | r}
			1	&5	&\alpha		&-1\\
			0	&1	&-1/2	&-\beta/2\\
			0	&0	&\alpha+6	&3\beta
		\end{array}
	\right)
$$

Das LGS ist somit
\begin{itemize}
	\item eindeutig l\"osbar, wenn $\alpha\neq -6$,
	\item nicht l\"osbar, wenn $\alpha=-6$ und $\beta\neq 0$
	\item l\"osbar, aber nicht eindeutig l\"osbar, wenn $\alpha=-6$ und $\beta=0$.
\end{itemize}



%-------------------------
\newpage
\section{Spezielle Matrizen II}
%%%

Sei 
$$
	D = 
	\begin{pmatrix}
		d_{11}	&\cdots	&d_{15}\\
		d_{21}	&\cdots	&d_{25}\\		
		d_{31}	&\cdots	&d_{35}\\
		d_{41}	&\cdots	&d_{45}\\
	\end{pmatrix}
	\in \R^{4\times 5}.
$$

\begin{itemize}
	\item[(a)] Geben Sie eine invertierbare Matrix $A\in\R^{4\times 4}$ an, so dass
			$$
				A\cdot D = 
				\begin{pmatrix}
					d_{31}	&\cdots	&d_{35}\\
					d_{21}	&\cdots	&d_{25}\\		
					d_{11}	&\cdots	&d_{15}\\
					d_{41}	&\cdots	&d_{45}\\
				\end{pmatrix}.
			$$
			(Vertauschung von Zeilen 1 und 3.)
		%-------------------------
		\subsection*{L\"osung}
		%%%

		$$	
			A =
			\begin{pmatrix}
				0	&0	&1	&0	\\
				0	&1	&0	&0	\\
				1	&0	&0	&0	\\
				0	&0	&0	&1	
			\end{pmatrix}.
		$$
		$$
			A^{-1} = A.
		$$
		(Zweimal vertauschen von Zeilen 1 und 3 ergibt wieder $A$.)			
	\item[(b)] Geben Sie eine invertierbare Matrix $B\in\R^{4\times 4}$ an, so dass
			$$
				B\cdot D = 
				\begin{pmatrix}
					d_{11} 			&\cdots	&d_{15} \\
					d_{21}			&\cdots	&d_{25}\\		
					2d_{31}			&\cdots	&2d_{35}\\
					d_{41}			&\cdots	&d_{45}\\
				\end{pmatrix}.
			$$
			(Multiplikation von Zeile 3 mit 2.)
		%-------------------------
		\subsection*{L\"osung}
		%%%			
		$$	
			B =
			\begin{pmatrix}
				1	&0	&0	&0	\\
				0	&1	&0	&0	\\
				0	&0	&2	&0	\\
				0	&0	&0	&1	
			\end{pmatrix}.
		$$	
		
		$$	
			B^{-1} =
			\begin{pmatrix}
				1	&0	&0	&0	\\
				0	&1	&0	&0	\\
				0	&0	&1/2	&0	\\
				0	&0	&0	&1	
			\end{pmatrix}.
		$$			
		(Multiplikation von Zeile 3 mit $1/2$ invertiert die Operation.)
	\item[(c)] Geben Sie eine invertierbare Matrix $C\in\R^{4\times 4}$ an, so dass
			$$
				C\cdot D = 
				\begin{pmatrix}
					d_{11} 			&\cdots	&d_{15} \\
					d_{21}			&\cdots	&d_{25}\\		
					2d_{11} + d_{31}	&\cdots	&2d_{15} + d_{35}\\
					d_{41}			&\cdots	&d_{45}\\
				\end{pmatrix}.
			$$
			(Addition des 2-fachen von Zeile 1 zu Zeile 3.)

		%-------------------------
		\subsection*{L\"osung}
		%%%			
		$$	
			C =
			\begin{pmatrix}
				1	&0	&0	&0	\\
				0	&1	&0	&0	\\
				2	&0	&1	&0	\\
				0	&0	&0	&1	
			\end{pmatrix}.
		$$	
		
		$$	
			C^{-1} =
			\begin{pmatrix}
				1	&0	&0	&0	\\
				0	&1	&0	&0	\\
				-2	&0	&1	&0	\\
				0	&0	&0	&1	
			\end{pmatrix}.
		$$	
		(Addition des $(-2)$-fachen von Zeile 1 zu Zeile 3 invertiert die Operation.)
\end{itemize}
\section{Nochmal die Abbildungsmatrix}
%%%
Es sei $\Psi: \R^3 \to \R^3$ eine lineare Abbildung mit
$$
\Psi(\begin{pmatrix}1\\1\\1 \end{pmatrix}) = \begin{pmatrix}2 \\ 1 \\ 2 \end{pmatrix}, \quad \Psi(\begin{pmatrix} 2 \\ 0\\ 2 \end{pmatrix}) = \begin{pmatrix}0\\2\\4 \end{pmatrix} \quad \text{ und } \quad \Psi(\begin{pmatrix} 0\\1\\1\end{pmatrix} )= \begin{pmatrix} 2 \\ 0 \\ 2\end{pmatrix}.
$$
Bestimmen Sie eine Matrix $A \in \R^{3\times 3}$, sodass $\Psi=\Phi_A$ gilt, indem Sie das LGS
$$
A\cdot \begin{pmatrix}1\\1\\1 \end{pmatrix}= \begin{pmatrix}2 \\ 1 \\ 2 \end{pmatrix},\ 
A\cdot \begin{pmatrix}2\\0\\2 \end{pmatrix}= \begin{pmatrix}0 \\ 2 \\ 4 \end{pmatrix} \text{ und }
A\cdot \begin{pmatrix}0\\1\\1 \end{pmatrix}= \begin{pmatrix}2 \\ 0 \\ 2 \end{pmatrix} 
$$
lösen. 

%-------------------------
\subsection*{L\"osung}
%%%	
Aus $A\cdot \begin{pmatrix}1\\1\\1 \end{pmatrix}= \begin{pmatrix}2 \\ 1 \\ 2 \end{pmatrix}$ erhalten wir die Gleichungen
\begin{align*}
a_{11} + a_{12} + a_{13}&=2 \\
a_{21}+a_{22}+a_{23}&=1\\
a_{31}+a_{32}+a_{33}&=2
\end{align*}

Aus $A\cdot \begin{pmatrix}2\\0\\2 \end{pmatrix}= \begin{pmatrix}0 \\ 2 \\ 4 \end{pmatrix}$ erhalten wir die Gleichungen
\begin{align*}
2a_{11} +2 a_{13}&=0 \\
2a_{21}+2a_{23}&=2\\
2a_{31}+2a_{33}&=4
\end{align*}

Aus $A\cdot \begin{pmatrix}0\\1\\1 \end{pmatrix}= \begin{pmatrix}2 \\ 0 \\ 2 \end{pmatrix}$ erhalten wir die Gleichungen
\begin{align*}
a_{12} + a_{13}&=2 \\
a_{22}+a_{23}&=0\\
a_{32}+a_{33}&=2
\end{align*}

Insgesamt ergibt das somit das LGS
$$
	\left(
		\begin{array}{rrrrrrrrr | r}
			a_{11} & a_{12} & a_{13}&a_{21}&a_{22}&a_{23}&a_{31}&a_{32}&a_{33}&\\
			\hline
			1&1&1&0&0&0&0&0&0&2\\
			0&0&0&1&1&1&0&0&0&1\\
			0&0&0&0&0&0&1&1&1&2\\
			2&0&2&0&0&0&0&0&0&0\\
			0&0&0&2&0&2&0&0&0&2\\
			0&0&0&0&0&0&2&0&2&4\\
			0&1&1&0&0&0&0&0&0&2\\
			0&0&0&0&1&1&0&0&0&0\\
			0&0&0&0&0&0&0&1&1&2\\
		\end{array}
	\right)
$$
Wir multiplizieren die 4. Zeile mit $\frac{1}{2}$ und ziehen sie dann von der 1.Zeile ab. Außerdem multiplizieren die 5. Zeile mit $\frac{1}{2}$ und ziehen sie dann von der 2.Zeile ab und multiplizieren die 6. Zeile mit $\frac{1}{2}$ und ziehen sie dann von der 3.Zeile ab. Das ergibt
$$
	\left(
		\begin{array}{rrrrrrrrr | r}
			a_{11} & a_{12} & a_{13}&a_{21}&a_{22}&a_{23}&a_{31}&a_{32}&a_{33}&\\
			\hline
			0&1&0&0&0&0&0&0&0&2\\
			0&0&0&0&1&0&0&0&0&0\\
			0&0&0&0&0&0&0&1&0&0\\
			1&0&1&0&0&0&0&0&0&0\\
			0&0&0&1&0&1&0&0&0&1\\
			0&0&0&0&0&0&1&0&1&2\\
			0&1&1&0&0&0&0&0&0&2\\
			0&0&0&0&1&1&0&0&0&0\\
			0&0&0&0&0&0&0&1&1&2\\
		\end{array}
	\right)
$$
Dann ziehen wir die 1.Zeile von der 7.Zeile, die 2.Zeile von der 8.Zeile und die 3.Zeile von der 9.Zeile ab. Das ergibt
$$
	\left(
		\begin{array}{rrrrrrrrr | r}
			a_{11} & a_{12} & a_{13}&a_{21}&a_{22}&a_{23}&a_{31}&a_{32}&a_{33}&\\
			\hline
			0&1&0&0&0&0&0&0&0&2\\
			0&0&0&0&1&0&0&0&0&0\\
			0&0&0&0&0&0&0&1&0&0\\
			1&0&1&0&0&0&0&0&0&0\\
			0&0&0&1&0&1&0&0&0&1\\
			0&0&0&0&0&0&1&0&1&2\\
			0&0&1&0&0&0&0&0&0&0\\
			0&0&0&0&0&1&0&0&0&0\\
			0&0&0&0&0&0&0&0&1&2\\
		\end{array}
	\right)
$$
Dann ziehen wir die 7.Zeile von der 4.Zeile, die 8.Zeile von der 5.Zeile und die 9.Zeile von der 6.Zeile ab. Das ergibt
$$
	\left(
		\begin{array}{rrrrrrrrr | r}
			a_{11} & a_{12} & a_{13}&a_{21}&a_{22}&a_{23}&a_{31}&a_{32}&a_{33}&\\
			\hline
			0&1&0&0&0&0&0&0&0&2\\
			0&0&0&0&1&0&0&0&0&0\\
			0&0&0&0&0&0&0&1&0&0\\
			1&0&0&0&0&0&0&0&0&0\\
			0&0&0&1&0&0&0&0&0&1\\
			0&0&0&0&0&0&1&0&0&0\\
			0&0&1&0&0&0&0&0&0&0\\
			0&0&0&0&0&1&0&0&0&0\\
			0&0&0&0&0&0&0&0&1&2\\
		\end{array}
	\right)
$$
Als letztes sortieren wir die Zeilen zur Stufenform und lesen dann die Lösung ab.
$$
	\left(
		\begin{array}{rrrrrrrrr | r}
			a_{11} & a_{12} & a_{13}&a_{21}&a_{22}&a_{23}&a_{31}&a_{32}&a_{33}&\\
			\hline
			1&0&0&0&0&0&0&0&0&0\\
			0&1&0&0&0&0&0&0&0&2\\
			0&0&1&0&0&0&0&0&0&0\\
			0&0&0&1&0&0&0&0&0&1\\
			0&0&0&0&1&0&0&0&0&0\\
			0&0&0&0&0&1&0&0&0&0\\
			0&0&0&0&0&0&1&0&0&0\\
			0&0&0&0&0&0&0&1&0&0\\
			0&0&0&0&0&0&0&0&1&2\\
		\end{array}
	\right)
$$
Damit ist die gesuchte Matrix
$$
A= \begin{pmatrix}0&2&0\\1&0&0\\0&0&2 \end{pmatrix}
$$

\section{Inverse Matrix}
%%%
Bestimmen Sie mit Hilfe des Gauß-Verfahrens die inverse Matrix zu
$$
A= \begin{pmatrix}1&2 & 0&0 \\ 0 & 2 & 4 & 0 \\ 0& 2 & 0 & 1 \\ 5 & 0& 5 & 0  \end{pmatrix}
$$
\subsection{Lösung}
$
\left(
		\begin{array}{rrrr|rrrr}
1&2 & 0&0 & 1&0&0&0 \\ 0 & 2 & 4 & 0 &0&1&0&0 \\ 0& 2 & 0 & 1 & 0&0&1&0 \\ 5 & 0& 5 & 0 &0&0&0&1
		\end{array}
	\right)
	\stackrel{1)}{\sim>}
	\left(
		\begin{array}{rrrr|rrrr}
1&2 & 0&0 & 1&0&0&0 \\ 0 & 2 & 4 & 0 &0&1&0&0 \\ 0& 2 & 0 & 1 & 0&0&1&0 \\ 0 & -10& 5 & 0 &-5&0&0&1
		\end{array}
	\right)\\\vspace{5mm}\\
	\stackrel{2)}{\sim>}
		\left(
		\begin{array}{rrrr|rrrr}
1&2 & 0&0 & 1&0&0&0 \\ 0 & 2 & 4 & 0 &0&1&0&0 \\ 0& 0 & -4 & 1 & 0&-1&1&0 \\ 0 & 0& 25 & 0 &-5&5&0&1
		\end{array}
	\right)
		\stackrel{3)}{\sim>}
		\left(
		\begin{array}{rrrr|rrrr}
1&2 & 0&0 & 1&0&0&0 \\ 0 & 2 & 4 & 0 &0&1&0&0 \\ 0& 0 & -4 & 1 & 0&-1&1&0 \\ 0 & 0& 0 & \frac{25}{4} &-5&\frac{-5}{4}&\frac{25}{4}&1
		\end{array}
	\right)\\\vspace{5mm}\\
	\stackrel{4)}{\sim>}
		\left(
		\begin{array}{rrrr|rrrr}
1&2 & 0&0 & 1&0&0&0 \\ 0 & 1 & 2 & 0 &0&\frac{1}{2}&0&0 \\ 0& 0 & 1 & -\frac{1}{4} & 0&\frac{1}{4}&-\frac{1}{4}&0 \\ 0 & 0& 0 & 1 &-\frac{4}{5}&\frac{-1}{5}&1&\frac{4}{25}
		\end{array}
	\right)
	\stackrel{5)}{\sim>}
		\left(
		\begin{array}{rrrr|rrrr}
1&0 & 0&0 &\frac{1}{5}&-\frac{1}{5}&0&\frac{4}{25}  \\ 0 & 1 & 0 & 0 &\frac{2}{5}&\frac{1}{10}&0&-\frac{2}{25} \\ 0& 0 & 1 & 0 & -\frac{1}{5}&\frac{1}{5}&0&\frac{1}{25} \\ 0 & 0& 0 & 1 &-\frac{4}{5}&\frac{-1}{5}&1&\frac{4}{25}
		\end{array}
	\right)
$


\begin{itemize}
\item[1)] 4. Zeile minus 5mal 1.Zeile 
\item[2)] 3.Zeile minus 2.Zeile und 4.Zeile plus 5mal 2. Zeile
\item[3)] 4.Zeile plus 25/4 mal 3.Zeile
\item[4)] 2.Zeile mal 1/2, 3.Zeile mal -1/4 und 4.Zeile mal 4/25
\item[5)] 3.Zeile plus 1/4 mal 4.Zeile, danach 2.Zeile minus 2mal 3.Zeile und danach 1.Zeile minus 2mal 2.Zeile
\end{itemize}

Somit erhalten wir
$$
A^{-1}=\begin{pmatrix}\frac{1}{5}&-\frac{1}{5}&0&\frac{4}{25}  \\ \frac{2}{5}&\frac{1}{10}&0&-\frac{2}{25} \\  -\frac{1}{5}&\frac{1}{5}&0&\frac{1}{25} \\ -\frac{4}{5}&\frac{-1}{5}&1&\frac{4}{25}\end{pmatrix}
$$
%
%
%
%
\end{document}
