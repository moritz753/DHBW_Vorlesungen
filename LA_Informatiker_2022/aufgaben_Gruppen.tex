\documentclass[
				a4paper,
				10pt
			]
			{scrartcl}

\parindent0mm

\usepackage{amsfonts}
\usepackage{amsmath}
\usepackage{amssymb}
\usepackage{amsthm}
\usepackage[ngerman]{babel}

\usepackage[
			pdftex,
			colorlinks,
			breaklinks,
			linkcolor=blue,
			citecolor=blue,
			filecolor=black,
			menucolor=black,
			urlcolor=black,
			pdfauthor={Andreas Weber},
			pdftitle={Aufgaben zur Logik und Algebra},
			plainpages=false,
			pdfpagelabels,
			bookmarksnumbered=true
		   ]{hyperref}


 %%%%%%%%%%%%%%%Schriften%%%%%%%%%%%%%
\DeclareMathAlphabet{\lier}{U}{eur}{m}{n}  %% Gothisch/Fraktur - Roman



\newcommand{\M}{\mathbb{M}}
\newcommand{\E}{\mathbb{E}}
\newcommand{\Hy}{\mathbb{H}}
\newcommand{\N}{\mathbb{N}}
\newcommand{\Q}{\mathbb{Q}}
\newcommand{\R}{\mathbb{R}}
\newcommand{\Z}{\mathbb{Z}}
\newcommand{\C}{\mathbb{C}}
\newcommand{\K}{\mathbb{K}}
\newcommand*\ee{\mathrm{e}}
\newcommand*\e{\mathrm{e}}

\newcommand*\ii{\mathrm{i}}
\newcommand*\re{\mathrm{Re}}
\newcommand*\im{\mathrm{Im}}
\newcommand*\id{\mathrm{id}}
\newcommand*\rang{\mathrm{rang}}
\newcommand*\grad{\mathrm{grad}}
\newcommand*\dive{\mathrm{div}}
\newcommand*\sym{\mathrm{Sym}}
\newcommand*\spur{\mathrm{Spur}}
\newcommand*\isom{\mathrm{Isom}}
\newcommand*\vol{\mathrm{vol\,}}
\newcommand*\supp{\mathrm{supp}}
\newcommand*\inj{\mathrm{inj}}
\newcommand*\rank{\mathrm{rank}}
\newcommand*\qrank{\Q\mbox{-}\mathrm{rank}}
\newcommand*\rrank{\R\mbox{-}\mathrm{rank}}
\newcommand*\dom{\mathrm{dom}}
\newcommand*\tr{\mathrm{tr}}
\newcommand*\spa{\mathrm{span}}
\newcommand*\diam{\mathrm{diam}}

\newcommand*\ric{\mathrm{Ric}}

\newcommand*\con{\mathrm{con}}
\newcommand*\dis{\mathrm{dis}}

\newcommand\PX{{\cal P}(X)}
\newcommand\T{{\cal T}}
\newcommand\B{{\cal B}}



\newcommand\scp{\langle\cdot,\cdot\rangle}     %% Metric

%%%%%%%%%%%%Tilde%%%%%%%
\newcommand\tx{\tilde{x}}
\newcommand\ty{\tilde{y}}
\newcommand\tu{\tilde{u}}
\newcommand\tk{\tilde{k}}
\newcommand\td{\tilde{d}}
\newcommand\tD{\tilde{D}}
\newcommand\tX{\tilde{X}}
\newcommand\tY{\tilde{Y}}
\newcommand\tZ{\tilde{Z}}


%%%%%%Lie-Gruppen%%%%%%%%%
\newcommand\ad{\mathrm{ad}}
\newcommand\Ad{\mathrm{Ad}}
\newcommand{\kak}{K\exp\overline{\lier{a}^+}K}              %%%%Cartan-Zerlegung
\newcommand*\Rang{\mathrm{Rang}}
\newcommand*\glnr{\mathrm{\it GL}(n,\R)}
\newcommand*\glnc{\mathrm{\it GL}(n,\C)}
\newcommand*\slnr{\mathrm{\it SL}(n,\R)}
\newcommand*\on{\mathrm{\it O}(n)}
\newcommand*\son{\mathrm{\it SO}(n)}
\newcommand*\SLzr{\mathrm{\it SL}(2,\R)}
\newcommand*\SOzr{\mathrm{\it SO}(2,\R)}

%%%%%%%%%%%%%Algebraische Gruppen
\newcommand\bG{{\bf G}}
\newcommand\bT{{\bf T}}
\newcommand\bP{{\bf P}}
\newcommand\bN{{\bf N}}
\newcommand\bL{{\bf L}}
\newcommand\bS{{\bf S}}
\newcommand\bM{{\bf M}}

\newcommand\Mor{\mathrm{Mor}}



%%%%%%Geometry%%%%%%%%%%%
\newcommand{\Si}{\mathcal{S}}


%%%%%%Ableitungsoperatoren%%%%%%%%%%%
\newcommand*\pddt{\frac{\partial}{\partial t}}
\newcommand*\pddx{\frac{\partial}{\partial x}}
\newcommand*\pddxio{\frac{\partial}{\partial x^i}}
\newcommand*\pddxjo{\frac{\partial}{\partial x^j}}
\newcommand*\pddxko{\frac{\partial}{\partial x^k}}
\newcommand*\pddxlo{\frac{\partial}{\partial x^l}}

\newcommand*\pddy{\frac{\partial}{\partial y}}
\newcommand*\pddyio{\frac{\partial}{\partial y^i}}
\newcommand*\pddyjo{\frac{\partial}{\partial y^j}}
\newcommand*\pddyko{\frac{\partial}{\partial y^k}}
\newcommand*\pddylo{\frac{\partial}{\partial y^l}}

\newcommand*\pddyq{\frac{\partial^2}{\partial y^2}}
\newcommand*\pddyj{\frac{\partial}{\partial y_j}}
\newcommand*\pddyjq{\frac{\partial^2}{\partial y_j^2}}
\newcommand*\pddxiq{\frac{\partial^2}{\partial x_i^2}}
\newcommand*\pddxi{\frac{\partial}{\partial x_i}}
\newcommand*\ddt{\frac{d}{dt}}

\newcommand*\dx{\,dvol(x)}
\newcommand*\dy{\,dvol(y)}
\newcommand*\dty{\,dvol(\ty)}

\newcommand*\DMp{\Delta_{M,p}}                 %%%%Laplace-Operatoren
\newcommand*\DMq{\Delta_{M,q}}
\newcommand*\DM{\Delta_M}
\newcommand*\DX{\Delta_X}
\newcommand*\DXp{\Delta_{X,p}}
\newcommand*\DXq{\Delta_{X,q}}
\newcommand*\DAx{\Delta_{Ax_0}}
\newcommand*\Rad{\mathrm{Rad}}
\newcommand*\DXps{\Delta^{\#}_{X,p}}

\newcommand*\eDXps{\e^{-t(\Delta^{\#}_{X,p}-c)}} %%%%%% Semigroups
\newcommand*\LpsX{L^p_{\#}(X)}


%%%%%%%%%%%%%%%Komplexe Analysis
\newcommand*\Res{\mathrm{Res}}

%%%%%%%%%%%%%%%Definitionsmenge
\newcommand*\D{{\cal D}}







\author{Dr. Moritz Gruber\\ DHBW Karlsruhe}
\title{\"Ubungsaufgaben 3\\ 
	Gruppen
}
\date{}

%%%%%%%%%
\begin{document}
%%%%%%%%%
\maketitle

%%%
\section{Die symmetrische Gruppe}
%%%
Es sei $D$ eine endliche Menge und 
$$
Sym(D):=\{f:D \to D \mid f \text{ bijektiv}\}.
$$
Zeigen Sie, dass $(Sym(D),\circ)$ eine Gruppe ist.

%%%
\section{Assoziativ?}
%%%

Sei $M = \{a,b\}$ mit $a\neq b$. Wir definieren eine Verkn\"upfung $\ast$ auf $M$ durch
$$
	a\ast a = b, \quad b\ast b = a, \quad a \ast b = a \quad \text{und} \quad b\ast a = a.
$$ 
Zeigen Sie, dass die Verkn\"upfung kommutativ aber nicht assoziativ ist.

%%%
\section{Abel'sche Gruppe}
%%%
Wir definieren auf der Menge $G= \R\backslash\{-\frac{1}{2}\}$ die Verkn\"upfung $\ast$ durch
$$
	a\ast b := 2ab +a +b.
$$
\begin{itemize}
	\item[(a)] Zeigen Sie, dass $(G,\ast)$ eine abelsche Gruppe ist.
	\item[(b)] Zeigen Sie, dass die Abbildung
			$$
				f: G \to \R\backslash\{0\},\, x\mapsto 2x + 1
			$$
			ein Isomorphismus (d.h. ein bijektiver Homomorphismus) der Gruppen $(G,\ast)$ und $(\R\backslash\{0\}, \cdot)$ ist.\\
			(Die Gruppe $(G,\ast)$ ist also die Gruppe $(\R\backslash\{0\}, \cdot)$ in einem neuen Gewand.)
	\item[(c)] L\"osen Sie die Gleichung
			$$
				2\ast x\ast x = 1.
			$$
			D.h., bestimmen Sie alle $x\in G$, f\"ur die diese Gleichung erf\"ullt ist.
\end{itemize}

%%%
\section{Matrixmultiplikation von reellen $(2\times 2)$-Matrizen *}
%%%

\begin{itemize}
	\item[(a)] Zeigen Sie, dass die Matrixmultiplikation assoziativ ist.
	\item[(b)] Zeigen Sie, dass die Matrixmultiplikation nicht kommutativ ist: 
			Finden Sie hierzu $A,B \in \R^{2\times 2}$ mit
			$$
				AB \neq BA.
			$$
	\item[(c)] Zeigen Sie, dass  
			$$
				E
				=
				\begin{pmatrix}
					1	&	0\\
					0	&	1
				\end{pmatrix}
			$$ 
			das neutrale Element der Matrixmultiplikation ist. 
	
\end{itemize}
%%%
\section{Matrixgruppe}
%%%

Sei 
$$
	G := \{ A \in \R^{2\times 2}~|~ \exists B \in \R^{2\times 2} \text{ mit } AB=BA=E \}.
$$

\begin{itemize}
	\item[(a)] Zeigen Sie, dass $(G,\cdot)$ eine Gruppe ist (``$\cdot$'' bezeichnet die Matrixmultiplikation).
	\item[(b)] Wir bezeichnen mit $A^{-1}$ das inverse Element zu der Matrix $A \in G$.\\[1mm] 
			Zeigen Sie, dass f\"ur alle $A,B \in G$ gilt:
			$$
				(A\cdot B)^{-1} = B^{-1}\cdot A^{-1}.
			$$
\end{itemize}

%%%
\section{Rekursive Folge (optional)}
%%%
Sei die Folge $(a_n)_{n\in \N}$ rekursiv definiert:
\begin{eqnarray*}
	a_0 		&=& 1\\
	a_{n+1}	&=& a_n + 2^{n+1}, \qquad n\in \N.
\end{eqnarray*}
Die ersten Folgeglieder berechnen sich als
$$
	a_1 = 1+2= 3, \qquad a_2 = 3+4= 7, \qquad a_3 = 7 + 8 =15,\qquad a_4 = 15 + 16 = 31, \ldots
$$
Hieraus k\"onnten wir die Vermutung ableiten, dass im Allgemeinen gilt:
$$
	a_n = 2^{n+1} - 1.
$$
Beweisen Sie diese Vermutung durch vollst\"andige Induktion.




\end{document}