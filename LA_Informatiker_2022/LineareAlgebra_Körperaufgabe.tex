\documentclass{beamer}

\usepackage{beamerthemesplit}

\usepackage{amsfonts}
\usepackage{amsmath}
\usepackage{amssymb}
\usepackage{amsthm}
\usepackage{amscd}

\usepackage{stmaryrd} 					%\lightning
\usepackage{algorithm2e}


\usepackage[ngerman]{babel}

\usepackage[utf8]{inputenc}
\usepackage[T1]{fontenc}
\usepackage{textcomp}


% Color Definitions
\definecolor{dhbwRed}{RGB}{226,0,26} 
\definecolor{dhbwGray}{RGB}{61,77,77}
\definecolor{lightBlue}{RGB}{28,134,230}

% Basic Theme
\usetheme{Malmoe}

% Color Re-Definitions
\usecolortheme[named=lightBlue]{structure}
\setbeamercolor*{alerted  text}{fg=dhbwRed, bg=white}
\setbeamercolor*{subsection in toc}{fg=dhbwGray, bg=white}

%\setbeamercolor*{palette primary}{fg=white,bg=lightBlue}
%\setbeamercolor*{palette secondary}{fg=white,bg=gray}
%\setbeamercolor*{palette tertiary}{fg=white,bg=gray}
%\setbeamercolor*{palette quaternary}{fg=white,bg=dhbwRed}

% no navigation symbols
\setbeamertemplate{navigation symbols}{}

% headline, footline
\setbeamertemplate{footline}{\color{dhbwGray} \hfill\insertframenumber\hspace{5mm}\vspace{2mm}}
\setbeamertemplate{headline}{}

% Title Page
\newcommand*{\makeTitlePage}{
	
	\begin{frame}[plain]
		
		\vfill
		\vfill
		\begin{center}
			{
				\usebeamerfont{title}
				\usebeamercolor[fg]{title}
				\Large
				\inserttitle
			}\\[3mm]
			{	
				\usebeamerfont{subtitle}
				\usebeamercolor[fg]{subtitle}
				\large
				\insertsubtitle
			}
		\end{center}
		%
		\vfill
		\vfill
		\vfill
		\vfill
		%
		\begin{columns}
			\begin{column}{0.5\textwidth}
				\begin{flushleft}
					{
						\usebeamerfont{normal text}
						\color{dhbwGray!80}
						\scriptsize
						Dr. Moritz Gruber\\
						DHBW Karlsruhe\\
						
					}
				\end{flushleft}
			\end{column}
			%
			\begin{column}{0.5\textwidth}
				\begin{flushright}
					\includegraphics[scale=0.06]{../DHBW.png}
				\end{flushright}
			\end{column}
		\end{columns}
		%
		\vspace{1mm}
		\begin{columns}
			\begin{column}{0.5\textwidth}
				\begin{flushleft}
					{
						\usebeamerfont{normal text}
						\color{dhbwGray!80}
						\scriptsize
						Version \today
					}
				\end{flushleft}
			\end{column}
			%
			\begin{column}{0.5\textwidth}
				% nothing (just a placeholder to be in line with the columns above
			\end{column}
		\end{columns}
	\end{frame}

}

% Section Divider Page
\newcommand*{\makeSectionDividerPage}{

	\begin{frame}[plain]
		\begin{center}
			\begin{flushleft}
				{				
					\usebeamercolor[fg]{frametitle}
					{\Large \insertsection} \\[3mm]
					{\large \insertsubsection}
				}
			\end{flushleft}
		\end{center}
        \end{frame}
	
}

% itemize
\setbeamertemplate{itemize items}[circle]
\setbeamertemplate{enumerate item}{(\theenumi)}




%--------------------------------------%
% Math ------------------------------%
%--------------------------------------%

% Mengen (Zahlen)
\newcommand{\N}{\mathbb{N}}
\newcommand{\Q}{\mathbb{Q}}
\newcommand{\R}{\mathbb{R}}
\newcommand{\Z}{\mathbb{Z}}
\newcommand{\C}{\mathbb{C}}

% Mengen (allgemein)
\newcommand{\K}{\mathbb{K}}
\newcommand\PX{{\cal P}(X)}

% Zahlentheorie
\newcommand{\ggT}{\mathrm{ggT}}


% Ableitungen
\newcommand{\ddx}{\frac{d}{dx}}
\newcommand{\pddx}{\frac{\partial}{\partial x}}
\newcommand{\pddy}{\frac{\partial}{\partial y}}
\newcommand{\grad}{\text{grad}}

%--------------------------------------%
% Layout Colors ------------------%
%--------------------------------------%
\newcommand*{\highlightDef}[1]{{\color{lightBlue}#1}}
\newcommand*{\highlight}[1]{{\color{lightBlue}#1}} % after theme for colours

%----------------------------------------------------------------------------------------------------
%--------- Document Title ---------------------------------------------------------------------
\title{Lineare Algebra\\[3mm] 
	\large Körper-Aufgabe: Körpererweiterung von $\Q$
}
\author{Dr. Moritz Gruber} 
\institute{DHBW Karlsruhe}
\date{2022}
%%%%%%%%%%%%%%
\begin{document}

\AtBeginSection[]{
	\begin{frame}				
		\usebeamercolor[fg]{frametitle}
		{\Large \insertsection} 
        \end{frame}
}

%
\begin{frame}[plain] 
 \titlepage
\end{frame}
%
\begin{frame}\frametitle{Aufgabe: Körpererweiterung von $\Q$}
Sei
$$
	K := \{ a + b\sqrt{2} ~|~a,b \in \Q \}. 
$$
Zeigen Sie, dass $(K,+,\cdot)$ mit der \"ublichen Addition $+$ und Multiplikation $\cdot$ ein  K\"orper ist.
%%
\end{frame}
%
%
\begin{frame}\frametitle{Lösung: }
%
Alle Eigenschaften eines kommutativen Ringes zeigt man wie in Aufgabe 1 auf  \"Ubungsblatt 4. Es bleibt also nur zu zeigen, dass jedes Element in $K \setminus \{0\}$ ein multiplikatives Inverses besitzt. \\ \pause
\quad\\
Seien dazu $z=a_1 + b_1 \sqrt{2}, \tilde z=a_2+b_2\sqrt{2} \in K$. Dann gilt: \pause
$$
z \cdot \tilde z =(a_1 + b_1\sqrt{2}) \cdot (a_2 + b_2\sqrt{2}) = (a_1a_2 + 2b_1b_2) + (a_1b_2 + a_2b_1)\sqrt{2}
$$ \pause
Setzt man nun $z \cdot \tilde z=1$, so erh\"alt man zwei Gleichungen: \pause
\begin{align*}
a_1a_2 + 2b_1b_2&=1\\ 
a_1b_2 + a_2b_1&=0
\end{align*}
\end{frame}
%
\begin{frame}
\begin{align*}
a_1a_2 + 2b_1b_2&=1\\
a_1b_2 + a_2b_1&=0
\end{align*}
Fall $b_1=0$: \\Dann ist $z=a_1+0\cdot \sqrt{2}=a_1 \in \Q$ und mit $\tilde z=\frac{1}{a_1}$ gilt $z\cdot \tilde z=1$.\\\quad\\
Fall $b_1 \ne 0$: \\
L\"ost man die zweite Gleichung nach $a_2$ auf, so erh\"alt man $a_2=\frac{-a_1b_2}{b_1}$.  \pause Dies kann man wiederum in die erste Gleichung einsetzen und erh\"alt
$$
1=a_1\cdot \frac{-a_1b_2}{b_1} + 2b_1b_2=b_2 \left( \frac{-a_1^2+2b_1^2}{b_1}\right)
$$ \pause
L\"ost man diese Gleichung nun nach $b_2$, so ergibt das $b_2=\frac{b_1}{2b_1^2-a_1^2}$. \\
(Aus $2b_1^2-a_1^2=0$ würde $a_1=\sqrt{2}b_1 \notin \Q$ folgen.)
\\ \pause Setzt man $b_2$ nun zur\"uck in die Formel von $a_2$ ein, erh\"alt man $a_2=\frac{-a_1}{2b_1^2-a_1^2}$.\\ 
\end{frame}
%
\begin{frame}
Somit ist $\tilde z=\frac{-a_1}{2b_1^2-a_1^2} + \frac{b_1}{2b_1^2-a_1^2} \cdot \sqrt{2}$ das zu $z$ Inverse Element und es gilt auch offensichtlich $\tilde z \in K$. \\
\vfill
Somit haben wir beide Fälle, $b_1=0$ und $b_1\ne 0$ ein multiplikatives Inverses zu $z$ gefunden.\\
\vfill
Damit ist $K$ ein Körper.
\end{frame}
%
%


\end{document}