\documentclass[
				a4paper,
				10pt
			]
			{scrartcl}

\parindent0mm

\usepackage{amsfonts}
\usepackage{amsmath}
\usepackage{amssymb}
\usepackage{amsthm}
\usepackage[ngerman]{babel}

\usepackage[
			pdftex,
			colorlinks,
			breaklinks,
			linkcolor=blue,
			citecolor=blue,
			filecolor=black,
			menucolor=black,
			urlcolor=black,
			pdfauthor={Andreas Weber},
			pdftitle={Aufgaben zur Logik und Algebra},
			plainpages=false,
			pdfpagelabels,
			bookmarksnumbered=true
		   ]{hyperref}


 %%%%%%%%%%%%%%%Schriften%%%%%%%%%%%%%
\DeclareMathAlphabet{\lier}{U}{eur}{m}{n}  %% Gothisch/Fraktur - Roman



\newcommand{\M}{\mathbb{M}}
\newcommand{\E}{\mathbb{E}}
\newcommand{\Hy}{\mathbb{H}}
\newcommand{\N}{\mathbb{N}}
\newcommand{\Q}{\mathbb{Q}}
\newcommand{\R}{\mathbb{R}}
\newcommand{\Z}{\mathbb{Z}}
\newcommand{\C}{\mathbb{C}}
\newcommand{\K}{\mathbb{K}}
\newcommand*\ee{\mathrm{e}}
\newcommand*\e{\mathrm{e}}

\newcommand*\ii{\mathrm{i}}
\newcommand*\re{\mathrm{Re}}
\newcommand*\im{\mathrm{Im}}
\newcommand*\id{\mathrm{id}}
\newcommand*\rang{\mathrm{rang}}
\newcommand*\grad{\mathrm{grad}}
\newcommand*\dive{\mathrm{div}}
\newcommand*\sym{\mathrm{Sym}}
\newcommand*\spur{\mathrm{Spur}}
\newcommand*\isom{\mathrm{Isom}}
\newcommand*\vol{\mathrm{vol\,}}
\newcommand*\supp{\mathrm{supp}}
\newcommand*\inj{\mathrm{inj}}
\newcommand*\rank{\mathrm{rank}}
\newcommand*\qrank{\Q\mbox{-}\mathrm{rank}}
\newcommand*\rrank{\R\mbox{-}\mathrm{rank}}
\newcommand*\dom{\mathrm{dom}}
\newcommand*\tr{\mathrm{tr}}
\newcommand*\spa{\mathrm{span}}
\newcommand*\diam{\mathrm{diam}}

\newcommand*\ric{\mathrm{Ric}}

\newcommand*\con{\mathrm{con}}
\newcommand*\dis{\mathrm{dis}}

\newcommand\PX{{\cal P}(X)}
\newcommand\T{{\cal T}}
\newcommand\B{{\cal B}}



\newcommand\scp{\langle\cdot,\cdot\rangle}     %% Metric

%%%%%%%%%%%%Tilde%%%%%%%
\newcommand\tx{\tilde{x}}
\newcommand\ty{\tilde{y}}
\newcommand\tu{\tilde{u}}
\newcommand\tk{\tilde{k}}
\newcommand\td{\tilde{d}}
\newcommand\tD{\tilde{D}}
\newcommand\tX{\tilde{X}}
\newcommand\tY{\tilde{Y}}
\newcommand\tZ{\tilde{Z}}


%%%%%%Lie-Gruppen%%%%%%%%%
\newcommand\ad{\mathrm{ad}}
\newcommand\Ad{\mathrm{Ad}}
\newcommand{\kak}{K\exp\overline{\lier{a}^+}K}              %%%%Cartan-Zerlegung
\newcommand*\Rang{\mathrm{Rang}}
\newcommand*\glnr{\mathrm{\it GL}(n,\R)}
\newcommand*\glnc{\mathrm{\it GL}(n,\C)}
\newcommand*\slnr{\mathrm{\it SL}(n,\R)}
\newcommand*\on{\mathrm{\it O}(n)}
\newcommand*\son{\mathrm{\it SO}(n)}
\newcommand*\SLzr{\mathrm{\it SL}(2,\R)}
\newcommand*\SOzr{\mathrm{\it SO}(2,\R)}

%%%%%%%%%%%%%Algebraische Gruppen
\newcommand\bG{{\bf G}}
\newcommand\bT{{\bf T}}
\newcommand\bP{{\bf P}}
\newcommand\bN{{\bf N}}
\newcommand\bL{{\bf L}}
\newcommand\bS{{\bf S}}
\newcommand\bM{{\bf M}}

\newcommand\Mor{\mathrm{Mor}}



%%%%%%Geometry%%%%%%%%%%%
\newcommand{\Si}{\mathcal{S}}


%%%%%%Ableitungsoperatoren%%%%%%%%%%%
\newcommand*\pddt{\frac{\partial}{\partial t}}
\newcommand*\pddx{\frac{\partial}{\partial x}}
\newcommand*\pddxio{\frac{\partial}{\partial x^i}}
\newcommand*\pddxjo{\frac{\partial}{\partial x^j}}
\newcommand*\pddxko{\frac{\partial}{\partial x^k}}
\newcommand*\pddxlo{\frac{\partial}{\partial x^l}}

\newcommand*\pddy{\frac{\partial}{\partial y}}
\newcommand*\pddyio{\frac{\partial}{\partial y^i}}
\newcommand*\pddyjo{\frac{\partial}{\partial y^j}}
\newcommand*\pddyko{\frac{\partial}{\partial y^k}}
\newcommand*\pddylo{\frac{\partial}{\partial y^l}}

\newcommand*\pddyq{\frac{\partial^2}{\partial y^2}}
\newcommand*\pddyj{\frac{\partial}{\partial y_j}}
\newcommand*\pddyjq{\frac{\partial^2}{\partial y_j^2}}
\newcommand*\pddxiq{\frac{\partial^2}{\partial x_i^2}}
\newcommand*\pddxi{\frac{\partial}{\partial x_i}}
\newcommand*\ddt{\frac{d}{dt}}

\newcommand*\dx{\,dvol(x)}
\newcommand*\dy{\,dvol(y)}
\newcommand*\dty{\,dvol(\ty)}

\newcommand*\DMp{\Delta_{M,p}}                 %%%%Laplace-Operatoren
\newcommand*\DMq{\Delta_{M,q}}
\newcommand*\DM{\Delta_M}
\newcommand*\DX{\Delta_X}
\newcommand*\DXp{\Delta_{X,p}}
\newcommand*\DXq{\Delta_{X,q}}
\newcommand*\DAx{\Delta_{Ax_0}}
\newcommand*\Rad{\mathrm{Rad}}
\newcommand*\DXps{\Delta^{\#}_{X,p}}

\newcommand*\eDXps{\e^{-t(\Delta^{\#}_{X,p}-c)}} %%%%%% Semigroups
\newcommand*\LpsX{L^p_{\#}(X)}


%%%%%%%%%%%%%%%Komplexe Analysis
\newcommand*\Res{\mathrm{Res}}

%%%%%%%%%%%%%%%Definitionsmenge
\newcommand*\D{{\cal D}}







\author{Dr. Moritz Gruber\\ DHBW Karlsruhe}
\title{L\"osungen \"Ubungsaufgaben 3\\ 
	Gruppen
}
\date{}

%%%%%%%%%
\begin{document}
%%%%%%%%%
\maketitle
%%%
\section{Die symmetrische Gruppe}
%%%
Es sei $D$ eine endliche Menge und 
$$
Sym(D):=\{f:D \to D \mid f \text{ bijektiv}\}.
$$
Zeigen Sie, dass $(Sym(D),\circ)$ eine Gruppe ist.
%%%
\paragraph{L\"osung:}
%%%
\begin{itemize}
\item[1)] \underline{Die Verkettung $\circ$ ist eine assoziative Verkn\"upfung auf $Sym(D)$:}\\
Die Verkettung $\circ$ ist eine assoziative Verkn\"upfung auf der Menge aller Abbildungen $D \to D$ und somit auch assoziativ f\"ur bijektive Abbildungen $D \to D$.\\
Es bleibt zu zeigen, dass die Verkettung zweier bijektiver Abbildungen wieder bijektiv ist.
Es seien $f,g \in Sym(D)$. Da $f$ und $g$ bijektiv sind, gibt es Umkehrabbildungen $f^{-1}$ und $g^{-1}$. Damit folgt
$$
(f \circ g) \circ (g^{-1} \circ f^{-1})=f \circ (g \circ g^{-1}) \circ f^{-1}=f\circ Id_D \circ f^{-1}=Id_D
$$
Somit ist $g^{-1} \circ f^{-1}$ die Umkehrabbildung zu  $f \circ g$ und damit ist $f \circ g$ bijektiv. Da auch $f\circ g: D \to D$ ist $f\circ g \in Sym(D)$.

\item[2)] \underline{Es gibt ein neutrales Element:}\\
$Id_D$ ist bijektiv und f\"ur alle $f \in Sym(D)$ gilt
$$
f \circ Id_D = Id_D \circ f = f
$$

\item[3)] \underline{F\"ur jedes Element gibt es ein Inverses:}\\
Da jedes Element $f\in Sym(D)$ bijektiv ist, existiert f\"ur jedes $f \in Sym(D)$ die Umkehrabbildung $f^{-1}:D \to D$. Da andersherum $f$ die Umkehrabbildung zu $f^{-1}$ ist, ist auch $f^{-1}$ bijektiv und daher $f^{-1}\in Sym(D)$. Au\ss erdem gilt
$$
f \circ f^{-1} = f^{-1} \circ f = Id_D
$$
Damit ist $f^{-1}$ das Inverse zu $f$ in $Sym(D)$.
\end{itemize}

%%%
\section{Assoziativ?}
%%%

Sei $M = \{a,b\}$ mit $a\neq b$. Wir definieren eine Verkn\"upfung $\ast$ auf $M$ durch
$$
	a\ast a = b, \quad b\ast b = a, \quad a \ast b =a \quad \text{und} \quad b\ast a = a.
$$ 
Zeigen Sie, dass die Verkn\"upfung kommutativ aber nicht assoziativ ist.

%%%
\paragraph{L\"osung:}
%%%

Verkn\"upfungstabelle:
$$
	\begin{array}{c || c | c}
		\ast	&	a	&	b\\
		\hline
		\hline
		a	&	b	& 	a\\
		b	&	a	&	a
	\end{array}
$$

\begin{itemize}
	\item[(a)] Die Verkn\"upfung ist kommutativ:
			$$
				a \ast b = b \ast a.
			$$
	\item[(b)] Die Verkn\"upfung ist nicht assoziativ:\\
			$$
				(a\ast a)\ast b = b\ast b = a.
			$$
			$$
				a\ast(a \ast b) = a\ast a = b.
			$$
\end{itemize}

%%%
\newpage
\section{Abel'sche Gruppe}
%%%
Wir definieren auf der Menge $G= \R\backslash\{-\frac{1}{2}\}$ die Verkn\"upfung $\ast$ durch
$$
	a\ast b := 2ab +a +b.
$$
\begin{itemize}
	\item[(a)] Zeigen Sie, dass $(G,\ast)$ eine abelsche Gruppe ist.
	\item[(b)] Zeigen Sie, dass die Abbildung
			$$
				f: G \to \R\backslash\{0\},\, x\mapsto 2x + 1
			$$
			ein Isomorphismus der Gruppen $(G,\ast)$ und $(\R\backslash\{0\}, \cdot)$ ist.\\
			(D.h., die Gruppe $(G,\ast)$ ist die Gruppe $(\R\backslash\{0\}, \cdot)$ in einem neuen Gewand.)
	\item[(c)] L\"osen Sie die Gleichung
			$$
				2\ast x\ast x = 1.
			$$
			D.h., bestimmen Sie alle $x\in G$, f\"ur die diese Gleichung erf\"ullt ist.
\end{itemize}

%%%
\paragraph{L\"osung:}
%%%

Vorbemerkung:
Durch $\ast$ wird tats\"achlich eine Verkn\"upfung definiert, d.h. f\"ur $a,b \in G$ gilt auch $a\ast b \in G$. Denn f\"ur $a,b\in G$ folgt:
$$
	a\ast b = -\frac{1}{2}
$$ 
gilt genau dann, wenn
$$
	-\frac{1}{2} = 2ab + a +b = (2b+1)a + b.
$$
Dies ist \"aquivalent zu
$$
	-\frac{1}{2}( 1 + 2b ) = (2b+1)a.
$$
Da $b\in G$ gilt $(2b+1)\neq 0$ und wir k\"onnen die Gleichung durch $(2b+1)$ dividieren. Wir erhalten den Widerspruch
$$
	-\frac{1}{2} = a.
$$

\begin{itemize}
	\item[(a)]
		\begin{itemize}
			\item[(a1)] Die Verkn\"upfung ist assoziativ:\\
					Seien $a,b,c \in G$. Dann gilt:
					\begin{align*}
						(a\ast b)\ast c	&=	 (2ab + a + b)\ast c\\
									&=	 2(2ab + a + b)c + (2ab + a + b) + c\\
									&=	 4abc + 2ac + 2bc + 2ab + a+ b + c
					\end{align*}
					%
					\begin{align*}
						a\ast(b\ast c)	&=	 a\ast(2bc + b+ c)	\\
									&=	 2a(2bc + b + c) + a + (2bc + b + c)\\
									&=	 4abc + 2ab + 2ac + a + 2bc + b +c
					\end{align*}
					Somit:
					$$
						(a\ast b)\ast c = a\ast(b\ast c).
					$$
					
			\item[(a2)] Das neutrale Element ist $e = 0 \in G$, da f\"ur alle $x \in G$ gilt
					$$
						x\ast 0 = x	\qquad\text{und}\qquad	0\ast x = x.
					$$		
			\item[(a3)] Sei $a\in G$. Wir berechnen das zu $a$ inverse Element. Gesucht ist $x\in G$ mit
					$$
						a\ast x = x\ast a = e.
					$$
					Das hei{\ss}t
					$$
						0 = a\ast x = 2ax + a + x. 
					$$
					Dies ist \"aquivalent zu
					$$
						(2a+1)x = -a.
					$$
					Somit
					$$
						x = - \frac{a}{2a+1}.
					$$
					Da $a\neq -\frac{1}{2}$, existiert f\"ur alle $a\in G$ dieser Bruch.
					Wegen
					$$
						a\ast x = 0
					$$
					und 
					\begin{align*}
						x \ast a	&=	- \frac{a}{2a+1} \ast a	\\
								&=	 -\frac{2a}{2a+1}\cdot a - \frac{a}{2a+1}  + a \\
								&=	\frac{-2a^2 - a + a\cdot(2a+1)}{2a+1}		\\
								&=	0
					\end{align*}
					ist $x$ das inverse Element zu $a$.\\
					
					Aus (a1),(a2), (a3) folgt, dass $(G,\ast)$ eine Gruppe ist.
			\item[(a4)] Es gilt
					$$
						b\ast a = 2ba + b + a = a \ast b.
					$$
					Die Verkn\"upfung ist somit kommutativ und die Gruppe hiermit abelsch.
		\end{itemize}
		
		\item[(b)]
			F\"ur $a,b \in G$ gilt:
			\begin{align*}
				f(a\ast b)	&=	f(2ab+a+b)	\\
						&=	4ab + 2a + 2b + 1. 
			\end{align*}
			Andererseits
			\begin{align*}
				f(a)\cdot f(b)	&= (2a+1)\cdot(2b+1)	\\
							&= 4ab + 2a + 2b + 1. 
			\end{align*}
			Es gilt also
			$$
				f(a \ast b) = f(a)\cdot f(b).
			$$
			Die Abbildung $f$ ist somit ein Homomorphismus. \\
			Da die Abbildung offensichtlich bijektiv ist, handelt es sich um einen Isomorphismus.
		\item[(c)]
			Es gilt
			$$
				f(2\ast x \ast x) = f(2)\cdot f(x) \cdot f(x) = 5f(x)^2
			$$ 
			und 
			$$
				f(1) = 3.
			$$
			Die Gleichung 
			$$
				2\ast x\ast x = 1.
			$$
			in $(G,\ast)$ wird durch $f$ \"ubersetzt in die Gleichung
			$$
				5f(x)^2 = 3
			$$			
			in der Gruppe $(\R\backslash\{0\},\cdot)$.
			Es folgt also
			$$
				f(x) = \pm \sqrt{ \frac{3}{5}}.
			$$
			F\"ur $x$ folgt dann
			$$
				2x+1 = \pm \sqrt{ \frac{3}{5}}
			$$
			bzw.
			$$
				x = -\frac{1}{2} \pm \frac{1}{2}\cdot \sqrt{ \frac{3}{5}}.
			$$
\end{itemize}

%%%
\newpage
\section{Matrixmultiplikation von reellen $(2\times 2)$-Matrizen}
%%%

\begin{itemize}
	\item[(a)] Zeigen Sie, dass die Matrixmultiplikation assoziativ ist.
	\item[(b)] Zeigen Sie, dass die Matrixmultiplikation nicht kommutativ ist: 
			Finden Sie hierzu $A,B \in \R^{2\times 2}$ mit
			$$
				AB \neq BA.
			$$
	\item[(c)] Zeigen Sie, dass  
			$$
				E
				=
				\begin{pmatrix}
					1	&	0\\
					0	&	1
				\end{pmatrix}
			$$ 
			das neutrale Element der Matrixmultiplikation ist. 
%	\item[(d)] Sei
%			$$
%				A 
%				=
%				\begin{pmatrix}
%					a_{11}	&a_{12}	\\
%					a_{21}	&a_{22}
%				\end{pmatrix} 			
%			$$
%			mit $a_{11}a_{22} - a_{12}a_{21} \neq 0$.
%			Zeigen Sie, dass die Matrix
%			$$	
%				B:= 
%				\frac{1}{a_{11}a_{22} - a_{12}a_{21}}\cdot
%				\begin{pmatrix}
%					a_{22}	&-a_{12}	\\
%					-a_{21}	&a_{11}
%				\end{pmatrix} 	
%				:= 
%				\begin{pmatrix}
%					\frac{a_{22}}{a_{11}a_{22} - a_{12}a_{21}}	& \frac{-a_{12}}{a_{11}a_{22} - a_{12}a_{21}}	\\
%					\frac{-a_{21}}{a_{11}a_{22} - a_{12}a_{21}}	&\frac{a_{11}}{a_{11}a_{22} - a_{12}a_{21}}
%				\end{pmatrix} 								
%			$$
%			das inverse Element zu $A$ bzgl. der Matrixmultiplikation ist, d.h. dass
%			$$
%				BA = AB = E
%			$$
%			gilt. (Man schreibt \"ublicherweise $A^{-1}$ f\"ur das inverse Element zu $A$.)\\[1mm]
%			
%			\textbf{Bemerkung:}\\ 
%			Eine reelle $(2\times 2)$-Matrix 
%			$$
%				\begin{pmatrix}
%					a_{11}	&a_{12}	\\
%					a_{21}	&a_{22}
%				\end{pmatrix} 	
%			$$
%			ist \textit{genau dann} invertierbar, \textit{wenn} 
%			$$
%				a_{11}a_{22} - a_{12}a_{21} \neq 0.
%			$$
\end{itemize}


%%%
\paragraph{L\"osung:}
%%%

\begin{itemize}
	\item[(a)]	Seien
		$$
			A 
			=
			\begin{pmatrix}
				a_{11}	&a_{12}	\\
				a_{21}	&a_{22}
			\end{pmatrix}, \quad
			B 
			=
			\begin{pmatrix}
				b_{11}	&b_{12}	\\
				b_{21}	&b_{22}
			\end{pmatrix}, \quad		
			C 
			=
			\begin{pmatrix}
				c_{11}	&c_{12}	\\
				c_{21}	&c_{22}
			\end{pmatrix}
			\in \R^{2\times 2}.		
		$$
		Es folgt:
		$$
			(A\cdot B) \cdot C	
			=
			\begin{pmatrix}
				a_{11}b_{11} + a_{12}b_{21}	&	a_{11}b_{12} + a_{12}b_{22}	\\
				a_{21}b_{11} + a_{22}b_{21}	&	a_{21}b_{12} + a_{22}b_{22}	\\
			\end{pmatrix}
			\cdot C
			=
		$$
		$$													
			\begin{pmatrix}
				a_{11}b_{11}c_{11} + a_{12}b_{21}c_{11} +  a_{11}b_{12}c_{21} + a_{12}b_{22}c_{21}	
					&	a_{11}b_{11}c_{12} + a_{12}b_{21}c_{12} +  a_{11}b_{12}c_{22} + a_{12}b_{22}c_{22}	\\
				a_{21}b_{11}c_{11} + a_{22}b_{21}c_{11} +  a_{21}b_{12}c_{21} + a_{22}b_{22}c_{21}	
					&	a_{21}b_{11}c_{12} + a_{22}b_{21}c_{12} +  a_{21}b_{12}c_{22} + a_{22}b_{22}c_{22}
			\end{pmatrix}			
		$$
		$$
			= \ldots
			= A\cdot(B\cdot C).
		$$
	\item[(b)]
		F\"ur 
		$$
			A
			=
			\begin{pmatrix}
				1	&	0	\\
				0	&	0
			\end{pmatrix}, \quad			
			B
			=
			\begin{pmatrix}
				0	&	1	\\
				0	&	0
			\end{pmatrix}			
		$$
		folgt:
		$$
			A \cdot B = 
			\begin{pmatrix}
				0	&	1	\\
				0	&	0
			\end{pmatrix}
		$$
		und
		$$
			B \cdot A = 
			\begin{pmatrix}
				0	&	0	\\
				0	&	0
			\end{pmatrix}.
		$$		
	\item[(c)]
		F\"ur alle
		$$
			A 
			=
			\begin{pmatrix}
				a_{11}	&a_{12}	\\
				a_{21}	&a_{22}
			\end{pmatrix} 
			\in \R^{2\times 2}
		$$
		folgt:
		$$
			A\cdot E 
			=
			\begin{pmatrix}
				a_{11}	&a_{12}	\\
				a_{21}	&a_{22}
			\end{pmatrix} 
			\cdot
			\begin{pmatrix}
				1	&	0\\
				0	&	1
			\end{pmatrix}			
			=
			A
		$$
		und
		$$
			E\cdot A = A.
		$$
%	\item[(d)]
%		F\"ur
%			$$
%				A 
%				=
%				\begin{pmatrix}
%					a_{11}	&a_{12}	\\
%					a_{21}	&a_{22}
%				\end{pmatrix} 			
%			$$
%			mit $a_{11}a_{22} - a_{12}a_{21} \neq 0$ 
%			und
%			$$	
%				B:= 
%				\frac{1}{a_{11}a_{22} - a_{12}a_{21}}\cdot
%				\begin{pmatrix}
%					a_{22}	&-a_{12}	\\
%					-a_{21}	&a_{11}
%				\end{pmatrix} 	
%				:= 
%				\begin{pmatrix}
%					\frac{a_{22}}{a_{11}a_{22} - a_{12}a_{21}}	& \frac{-a_{12}}{a_{11}a_{22} - a_{12}a_{21}}	\\
%					\frac{-a_{21}}{a_{11}a_{22} - a_{12}a_{21}}	&\frac{a_{11}}{a_{11}a_{22} - a_{12}a_{21}}
%				\end{pmatrix} 								
%			$$
%			folgt:
%			$$
%				A \cdot B 
%				=
%				\frac{1}{a_{11}a_{22} - a_{12}a_{21}}
%				\cdot
%				\begin{pmatrix}
%					a_{11}	&a_{12}	\\
%					a_{21}	&a_{22}
%				\end{pmatrix} 	
%				\cdot 
%				\begin{pmatrix}
%					a_{22}	&-a_{12}	\\
%					-a_{21}	&a_{11}
%				\end{pmatrix} 	
%				=
%			$$
%			$$
%				=
%				\frac{1}{a_{11}a_{22} - a_{12}a_{21}}
%				\cdot
%				\begin{pmatrix}
%					a_{11}a_{22} - a_{12}a_{21}	& -a_{11}a_{12} + a_{12}a_{11}	\\
%					a_{21}a_{22} - a_{22}a_{21}	& -a_{21}a_{12} + a_{22}a_{11}
%				\end{pmatrix} 
%				=
%				E.				
%			$$
%			Eine analoge Rechnung ergibt auch $B\cdot A = E$.
\end{itemize}


%%%
\newpage
\section{Matrixgruppe}
%%%

Sei 
$$
	G := \{ A \in \R^{2\times 2}~|~ \exists B \in \R^{2\times 2} \text{ mit } AB=BA=E \}.
$$

\begin{itemize}
	\item[(a)] Zeigen Sie, dass $(G,\cdot)$ eine Gruppe ist (``$\cdot$'' bezeichnet die Matrixmultiplikation).
	\item[(b)] Wir bezeichnen mit $A^{-1}$ das inverse Element zu der Matrix $A \in G$.\\[1mm] 
			Zeigen Sie, dass f\"ur alle $A,B \in G$ gilt:
			$$
				(A\cdot B)^{-1} = B^{-1}\cdot A^{-1}.
			$$
\end{itemize}


%%%
\paragraph{L\"osung:}
%%%

\begin{itemize}
	\item[(a)] 
		\begin{itemize}
			\item[(a1)] Die Verkn\"upfung ``$\cdot$'' ist assoziativ (vgl. \"Ubungsaufgaben 5). 
			\item[(a2)] Das neutrale Element bzgl. ``$\cdot$'' ist die Matrix
					$$
						E =
							\begin{pmatrix}
								1	& 0	\\
								0	& 1	
							\end{pmatrix},
					$$
					vgl.  \"Ubungsaufgaben 5. \\
					Zu \"uberpr\"ufen ist allerdings noch, ob $E\in G$.\\
					F\"ur $B=E$ folgt jedoch:
					$$
						E\cdot B = B\cdot E = E
					$$
					und somit $E\in G$.
			\item[(a3)] Jede Matrix $A\in G$ besitzt ein inverses Element: Die Matrix $B$. 
		\end{itemize}
		
		Damit ist  $(G,\cdot)$  eine Gruppe.\\
		
		\textbf{Bemerkung:} Diese Gruppe ist nicht abelsch: 
		$$
			A =
				\begin{pmatrix}
					1	&	1\\
					0	&	1
				\end{pmatrix},
			B =
				\begin{pmatrix}
					1	&	0\\
					1	&	1
				\end{pmatrix}
			\in G
		$$
		(vgl. \"Ubungsaufgaben 5).\\
		Es gilt
		$$
			AB =
				\begin{pmatrix}
					2	&	1\\
					1	&	1
				\end{pmatrix},				
		$$
		aber
		$$
			BA =
				\begin{pmatrix}
					1	&	1\\
					1	&	2
				\end{pmatrix}.			
		$$	
	\item[(b)]  Es gilt:
			$$
				(A\cdot B)\cdot(B^{-1}\cdot A^{-1} ) 
				=
				A\cdot (B\cdot B^{-1})\cdot A^{-1} 
				=
				A \cdot E \cdot A^{-1} 
				= 
				E
			$$
			und
			$$
				(B^{-1}\cdot A^{-1} ) \cdot (A\cdot B)
				=
				B^{-1}\cdot (A^{-1}  \cdot A) \cdot B
				=
				B^{-1}\cdot E \cdot B
				=
				E.
			$$
			Somit folgt
			$$
				(A\cdot B)^{-1} = B^{-1}\cdot A^{-1}.
			$$
			
			\textbf{Bemerkung:}\\ 
			Da die Gruppe nicht abelsch ist, gilt im Allgemeinen \textbf{nicht} $(A\cdot B)^{-1} = A^{-1}\cdot B^{-1}$.
			
\end{itemize}

%%%
\newpage
\section{Rekursive Folge}
%%%
Sei die Folge $(a_n)_{n\in \N}$ rekursiv definiert:
\begin{eqnarray*}
	a_0 		&=& 1\\
	a_{n+1}	&=& a_n + 2^{n+1}, \qquad n\in \N.
\end{eqnarray*}
Die ersten Folgeglieder berechnen sich als
$$
	a_1 = 1+2= 3, \qquad a_2 = 3+4= 7, \qquad a_3 = 7 + 8 =15,\qquad a_4 = 15 + 16 = 31, \ldots
$$
Hieraus k\"onnten wir die Vermutung ableiten, dass im Allgemeinen gilt:
$$
	a_n = 2^{n+1} - 1.
$$
Beweisen Sie diese Vermutung durch vollst\"andige Induktion.

%%%
\paragraph{L\"osung:}
%%%

Induktionsanfang ($n=0$):
\begin{eqnarray*}
	a_0 &=& 1.\\
	2^{0+1}-1 &=& 1. \quad\checkmark
\end{eqnarray*}

Induktionsschritt ($n \curvearrowright n+1$): \\
Wir nehmen an, dass 
$$
	a_n = 2^{n+1} - 1
$$
und m\"ussen folgern, dass dann auch
$$
	a_{n+1} = 2^{(n+1)+1} - 1.
$$

Es gilt:
\begin{eqnarray*}
	a_{n+1} &=& a_n + 2^{n+1} \qquad \text{(laut Definition von $a_n$)}\\
		     &=& 2^{n+1} - 1 + 2^{n+1} \qquad \text{(Verwendung der Induktionsannahme)}\\
		     &=& 2^{(n+1) + 1} - 1.	
\end{eqnarray*}
\hfill$\qed$


\end{document}