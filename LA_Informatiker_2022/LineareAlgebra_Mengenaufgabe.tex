\documentclass{beamer}

\usepackage{beamerthemesplit}

\usepackage{amsfonts}
\usepackage{amsmath}
\usepackage{amssymb}
\usepackage{amsthm}
\usepackage{amscd}

\usepackage{stmaryrd} 					%\lightning
\usepackage{algorithm2e}


\usepackage[ngerman]{babel}

\usepackage[utf8]{inputenc}
\usepackage[T1]{fontenc}
\usepackage{textcomp}


% Color Definitions
\definecolor{dhbwRed}{RGB}{226,0,26} 
\definecolor{dhbwGray}{RGB}{61,77,77}
\definecolor{lightBlue}{RGB}{28,134,230}

% Basic Theme
\usetheme{Malmoe}

% Color Re-Definitions
\usecolortheme[named=lightBlue]{structure}
\setbeamercolor*{alerted  text}{fg=dhbwRed, bg=white}
\setbeamercolor*{subsection in toc}{fg=dhbwGray, bg=white}

%\setbeamercolor*{palette primary}{fg=white,bg=lightBlue}
%\setbeamercolor*{palette secondary}{fg=white,bg=gray}
%\setbeamercolor*{palette tertiary}{fg=white,bg=gray}
%\setbeamercolor*{palette quaternary}{fg=white,bg=dhbwRed}

% no navigation symbols
\setbeamertemplate{navigation symbols}{}

% headline, footline
\setbeamertemplate{footline}{\color{dhbwGray} \hfill\insertframenumber\hspace{5mm}\vspace{2mm}}
\setbeamertemplate{headline}{}

% Title Page
\newcommand*{\makeTitlePage}{
	
	\begin{frame}[plain]
		
		\vfill
		\vfill
		\begin{center}
			{
				\usebeamerfont{title}
				\usebeamercolor[fg]{title}
				\Large
				\inserttitle
			}\\[3mm]
			{	
				\usebeamerfont{subtitle}
				\usebeamercolor[fg]{subtitle}
				\large
				\insertsubtitle
			}
		\end{center}
		%
		\vfill
		\vfill
		\vfill
		\vfill
		%
		\begin{columns}
			\begin{column}{0.5\textwidth}
				\begin{flushleft}
					{
						\usebeamerfont{normal text}
						\color{dhbwGray!80}
						\scriptsize
						Dr. Moritz Gruber\\
						DHBW Karlsruhe\\
						
					}
				\end{flushleft}
			\end{column}
			%
			\begin{column}{0.5\textwidth}
				\begin{flushright}
					\includegraphics[scale=0.06]{../DHBW.png}
				\end{flushright}
			\end{column}
		\end{columns}
		%
		\vspace{1mm}
		\begin{columns}
			\begin{column}{0.5\textwidth}
				\begin{flushleft}
					{
						\usebeamerfont{normal text}
						\color{dhbwGray!80}
						\scriptsize
						Version \today
					}
				\end{flushleft}
			\end{column}
			%
			\begin{column}{0.5\textwidth}
				% nothing (just a placeholder to be in line with the columns above
			\end{column}
		\end{columns}
	\end{frame}

}

% Section Divider Page
\newcommand*{\makeSectionDividerPage}{

	\begin{frame}[plain]
		\begin{center}
			\begin{flushleft}
				{				
					\usebeamercolor[fg]{frametitle}
					{\Large \insertsection} \\[3mm]
					{\large \insertsubsection}
				}
			\end{flushleft}
		\end{center}
        \end{frame}
	
}

% itemize
\setbeamertemplate{itemize items}[circle]
\setbeamertemplate{enumerate item}{(\theenumi)}




%--------------------------------------%
% Math ------------------------------%
%--------------------------------------%

% Mengen (Zahlen)
\newcommand{\N}{\mathbb{N}}
\newcommand{\Q}{\mathbb{Q}}
\newcommand{\R}{\mathbb{R}}
\newcommand{\Z}{\mathbb{Z}}
\newcommand{\C}{\mathbb{C}}

% Mengen (allgemein)
\newcommand{\K}{\mathbb{K}}
\newcommand\PX{{\cal P}(X)}

% Zahlentheorie
\newcommand{\ggT}{\mathrm{ggT}}


% Ableitungen
\newcommand{\ddx}{\frac{d}{dx}}
\newcommand{\pddx}{\frac{\partial}{\partial x}}
\newcommand{\pddy}{\frac{\partial}{\partial y}}
\newcommand{\grad}{\text{grad}}

%--------------------------------------%
% Layout Colors ------------------%
%--------------------------------------%
\newcommand*{\highlightDef}[1]{{\color{lightBlue}#1}}
\newcommand*{\highlight}[1]{{\color{lightBlue}#1}} % after theme for colours

%----------------------------------------------------------------------------------------------------
%--------- Document Title ---------------------------------------------------------------------
\title{Lineare Algebra\\[3mm] 
	\large Mengenaufgabe: Potenzmenge
}
\author{Dr. Moritz Gruber} 
\institute{DHBW Karlsruhe}
\date{2022}
%%%%%%%%%%%%%%
\begin{document}

\AtBeginSection[]{
	\begin{frame}				
		\usebeamercolor[fg]{frametitle}
		{\Large \insertsection} 
        \end{frame}
}

%
\begin{frame}[plain] 
 \titlepage
\end{frame}
%
%
%%%
\begin{frame} \frametitle{Aufgabe}


\begin{itemize}
	\item[(a)]Sei $M = \{0,1,2,3\}.$ Bestimmen Sie ${\cal P}(M)$.
	\item[(b)] Was ist ${\cal P}(\emptyset)$?
	\item[(c)] Seien $A, B$ zwei Mengen. Zeigen Sie:
			$$
				{\cal P}(A) \cap {\cal P}(B) = {\cal P}(A\cap B).
			$$
	\item[(d)] Finden Sie zwei Mengen $A,B$, so dass
			$$
				{\cal P}(A) \cup {\cal P}(B) \neq {\cal P}(A\cup B).
			$$
\end{itemize}
%%%
\end{frame}

%
%
\begin{frame}\frametitle{Lösung: Teil a)}
%%
Sei $M = \{0,1,2,3\}.$ Bestimmen Sie ${\cal P}(M)$. \pause
\vfill
${\cal P}(M) = \big\{$ \pause				$ \emptyset,$\\  \pause
				\hspace{17mm}$\{0\}, \{1\}, \{2\}, \{3\},$\\ \pause
				\hspace{17mm}$\{0,1\}, \{0,2\}, \{0,3\}, \{1,2\}, \{1,3\}, \{2,3\},$\\  \pause
				\hspace{17mm}$\{0,1,2\}, \{0,1,3\}, \{0,2,3\}, \{1,2,3\},$\\  \pause
				\hspace{17mm}$\{0,1,2,3\}$ \pause$\ \big\}$
%
\end{frame}
%
%
\begin{frame}\frametitle{Lösung: Teil b)}
%
Was ist ${\cal P}(\emptyset)$? \pause
\vfill
%%
${\cal P}(\emptyset) = \{ \emptyset \}$
\vfill
Achtung: Dies ist \textbf{nicht} die Leere Menge!
\end{frame}
%
%
%
\begin{frame}\frametitle{Lösung: Teil c)}
%
Seien $A, B$ zwei Mengen. Zeigen Sie:
			$$
				{\cal P}(A) \cap {\cal P}(B) = {\cal P}(A\cap B).
			$$\pause
%%
1. Schritt:\\ 
			Wir zeigen ${\cal P}(A) \cap {\cal P}(B) \subseteq  {\cal P}(A\cap B)$:\\ \pause
			Sei $X \in {\cal P}(A) \cap {\cal P}(B).$ Dann folgt: \pause
			$$
				X \subseteq A\quad \land \quad X \subseteq B.
			$$\pause
			Das hei{\ss}t
			$$
				X \subseteq A \cap B
			$$\pause
			und somit
			$$
				X \in  {\cal P}(A\cap B).
			$$
			

			

\end{frame}
%
%%
\begin{frame}\frametitle{Lösung: Teil c)}
%%
			2. Schritt:\\
			Wir zeigen $ {\cal P}(A\cap B) \subseteq  {\cal P}(A) \cap {\cal P}(B) $:\\ \pause
			Sei $X \in {\cal P}(A\cap B)$. Dann folgt: \pause
			$$
				X \subseteq A\cap B.
			$$ \pause
			Das hei{\ss}t
			$$
				X \subseteq A \quad \land \quad X \subseteq B
			$$ \pause
			und somit
			$$
				X \in {\cal P}(A) \cap {\cal P}(B).
			$$\pause
			
			Und da nun $ {\cal P}(A\cap B) \subseteq  {\cal P}(A) \cap {\cal P}(B) $ und $ {\cal P}(A\cap B) \supseteq  {\cal P}(A) \cap {\cal P}(B) $ gilt, muss $ {\cal P}(A\cap B) =  {\cal P}(A) \cap {\cal P}(B) $ gelten.

\end{frame}
%
%%
\begin{frame}\frametitle{Lösung: Teil d)}
Finden Sie zwei Mengen $A,B$, so dass
			$$
				{\cal P}(A) \cup {\cal P}(B) \neq {\cal P}(A\cup B).
			$$\pause
Wir benutzen hierfür zwei disjunkte Mengen, da deren Vereinigung ``viel'' größer ist als die ursprünglichen Mengen.\\
\quad\\
\pause
$A=\{1\}$ und $B=\{2\}$:\pause 
			$$
				{\cal P}(A) = \{\emptyset, \{1\} \}, \quad {\cal P}(B) = \{\emptyset, \{2\} \}.
			$$\pause
			$$
				{\cal P}(A\cup B) = \{\emptyset, \{1\}, \{2\}, \textcolor{blue}{\{1,2\}} \}.
			$$
\end{frame}
%
\end{document}