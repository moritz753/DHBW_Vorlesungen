\documentclass[
				a4paper,
				10pt
			]
			{scrartcl}

\parindent0mm

\usepackage{amsfonts}
\usepackage{amsmath}
\usepackage{amssymb}
\usepackage{amsthm}
\usepackage[ngerman]{babel}
\usepackage{graphicx}
\usepackage{xcolor}

\usepackage[
			pdftex,
			colorlinks,
			breaklinks,
			linkcolor=blue,
			citecolor=blue,
			filecolor=black,
			menucolor=black,
			urlcolor=black,
			pdfauthor={Andreas Weber},
			pdftitle={Aufgaben zu Analysis und Lineare Algebra},
			plainpages=false,
			pdfpagelabels,
			bookmarksnumbered=true
		   ]{hyperref}


%--------------------------------------%
% Math ------------------------------%
%--------------------------------------%

% Mengen (Zahlen)
\newcommand{\N}{\mathbb{N}}
\newcommand{\Q}{\mathbb{Q}}
\newcommand{\R}{\mathbb{R}}
\newcommand{\Z}{\mathbb{Z}}
\newcommand{\C}{\mathbb{C}}

% Mengen (allgemein)
\newcommand{\K}{\mathbb{K}}
\newcommand\PX{{\cal P}(X)}

% Zahlentheorie
\newcommand{\ggT}{\mathrm{ggT}}


% Ableitungen
\newcommand{\ddx}{\frac{d}{dx}}
\newcommand{\pddx}{\frac{\partial}{\partial x}}
\newcommand{\pddy}{\frac{\partial}{\partial y}}
\newcommand{\grad}{\text{grad}}

%--------------------------------------%
% Layout Colors ------------------%
%--------------------------------------%
\newcommand*{\highlightDef}[1]{{\color{lightBlue}#1}}
\newcommand*{\highlight}[1]{{\color{lightBlue}#1}}
% Color Definitions
\definecolor{dhbwRed}{RGB}{226,0,26} 
\definecolor{dhbwGray}{RGB}{61,77,77}
\definecolor{lightBlue}{RGB}{28,134,230}

%
\addtokomafont{section}{\color{dhbwGray}}
\addtokomafont{subsection}{\color{dhbwGray}}


%-------------------------------------------------------------------
\begin{document}

\vspace*{-20mm}
{
	%\usekomafont{title}
	\color{dhbwGray}
	Dr. Moritz Gruber	\hfill Version \today\\
	DHBW Karlsruhe\\
}

\vspace{10mm}
\begin{center}
	{
		\usekomafont{title}
		\color{lightBlue}
		{ \LARGE 	\"Ubungsaufgaben 7}\\[3mm]
		{\Large Lineare Abbildungen und Matrizen}
	}
\end{center}

\vspace{5mm}

%-------------------------------------------------------------------



%-------------------------
\section{Der Kern}
%%%
Es seien $V,W$ $K$-Vektorräume und $\Phi: V \to W$ ein Homomorphismus.
\begin{itemize}
\item[a)] Zeigen Sie, dass die Menge $Kern(\Phi):=\Phi^{-1}(\{0_W\})$ ein Untervektorraum von $V$ ist.
\item[b)] Zeigen Sie, dass $\Phi$ genau dann injektiv ist, wenn  $Kern(\Phi)=\{0_V\}$ gilt.
\end{itemize}
Es sei nun $A=\begin{pmatrix} 1&2&3 \\ 1&2&3 \end{pmatrix} \in \R^{2\times 3}$ und $\Phi_A: \R^3 \to \R^2, x \mapsto A \cdot x$.
\begin{itemize}
\item[c)] Bestimmen Sie $Kern(\Phi_A)$.
\end{itemize}

%-------------------------
\section{Drehung in $\R^2$}
%%%

Sei $\varphi \in [0,2\pi)$ und 
$
	D_\varphi=\begin{pmatrix}
		\cos(\varphi)	&-\sin(\varphi)	\\
		\sin(\varphi)	&\cos(\varphi)
	\end{pmatrix}
	\in \R^{2\times 2}$.
\begin{itemize}
\item[a)] Bestimmen Sie die Bilder $\Phi_{D_\varphi}(v)$ für die folgenden Vekoren:\\
i) $v=e_1$ \quad ii) $v=e_2$ \quad iii) $v=\begin{pmatrix} 2 \\ 1 \end{pmatrix}$
\item[b)] Zeigen Sie, dass für $\varphi, \psi \in [0,2\pi)$ gilt: $D_\varphi \cdot D_\psi = D_{\varphi+\psi}$.
\item[c)] Bestimmen Sie die inverse Matrix $D_\varphi^{-1}$.
\item[d)] Es sei nun $\varphi=\frac{\pi}{6}$. Bestimmen Sie die Abbildungsmatrix Verkettung $(\Phi_{D_{\frac{\pi}{6}}})^{1320}$.
\end{itemize}
%-------------------------
\section{Matrizen}
%%%

Seien
$$
	A
	=
	\begin{pmatrix}
		1	&0	\\
		1	&1	
	\end{pmatrix}
	\quad
	\text{und}
	\quad
	B
	=
	\begin{pmatrix}
		1	&2	\\
		3	&4	
	\end{pmatrix}.
$$

\begin{itemize}
	\item[(a)] Bestimmen Sie die zu $A$ inverse Matrix $A^{-1}$. 
	\item[(b)] Bestimmen Sie die Matrix $X\in \R^{2\times 2}$, f\"ur die gilt:
		$$
			X\cdot A = B.
		$$
\end{itemize}

%-------------------------
\newpage
\section{Spezielle Matrizen}
%%%

Zeigen Sie, dass die Matrizen $A\in \R^{2\times 2}$ invertierbar sind und bestimmen Sie die zugeh\"orige inverse Matrix $A^{-1}$.
Berechnen Sie ferner
		$$
			A
			\cdot
			\begin{pmatrix}
				a & b & c\\
				d & e  & f
			\end{pmatrix}.
		$$
\begin{itemize}
	\item[(a)] Additionsmatrix:	
		Sei $\alpha \in \R$. 
		$$
			A
			=
			\begin{pmatrix}
				1 & \alpha\\
				0 & 1
			\end{pmatrix}.
		$$

	\item[(b)] Vertauschungsmatrix:
		$$
			A
			=
			\begin{pmatrix}
				0 & 1\\
				1 & 0
			\end{pmatrix}.
		$$

	\item[(c)] Multiplikation einer Zeile mit einer Konstante:
		Seien $\alpha, \beta \in \R\backslash\{0\}$. 
		$$
			A
			=
			\begin{pmatrix}
				\alpha 	& 0\\
				0 		& \beta
			\end{pmatrix}.
		$$
	
\end{itemize}

%%%
\section{Abbildungsmatrix *}
%%%
Es sei $\Psi: \R^3 \to \R^3$ eine lineare Abbildung mit
$$
\Psi(\begin{pmatrix}1\\1\\1 \end{pmatrix}) = \begin{pmatrix}2 \\ 1 \\ 2 \end{pmatrix}, \quad \Psi(\begin{pmatrix} 2 \\ 0\\ 2 \end{pmatrix}) = \begin{pmatrix}0\\2\\4 \end{pmatrix} \quad \text{ und } \quad \Psi(\begin{pmatrix} 0\\1\\1\end{pmatrix} )= \begin{pmatrix} 2 \\ 0 \\ 2\end{pmatrix}.
$$
Bestimmen Sie eine Matrix $A \in \R^{3\times 3}$, sodass $\Psi=\Phi_A$ gilt. 


\end{document}
