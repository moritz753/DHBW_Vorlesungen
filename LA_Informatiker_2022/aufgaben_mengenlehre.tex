\documentclass[
				a4paper,
				10pt
			]
			{scrartcl}

\parindent0mm

\usepackage{amsfonts}
\usepackage{amsmath}
\usepackage{amssymb}
\usepackage{amsthm}
\usepackage[ngerman]{babel}

\usepackage[
			pdftex,
			colorlinks,
			breaklinks,
			linkcolor=blue,
			citecolor=blue,
			filecolor=black,
			menucolor=black,
			urlcolor=black,
			pdfauthor={Andreas Weber},
			pdftitle={Aufgaben zur Logik und Algebra},
			plainpages=false,
			pdfpagelabels,
			bookmarksnumbered=true
		   ]{hyperref}

 %%%%%%%%%%%%%%%Schriften%%%%%%%%%%%%%
\DeclareMathAlphabet{\lier}{U}{eur}{m}{n}  %% Gothisch/Fraktur - Roman



\newcommand{\M}{\mathbb{M}}
\newcommand{\E}{\mathbb{E}}
\newcommand{\Hy}{\mathbb{H}}
\newcommand{\N}{\mathbb{N}}
\newcommand{\Q}{\mathbb{Q}}
\newcommand{\R}{\mathbb{R}}
\newcommand{\Z}{\mathbb{Z}}
\newcommand{\C}{\mathbb{C}}
\newcommand{\K}{\mathbb{K}}
\newcommand*\ee{\mathrm{e}}
\newcommand*\e{\mathrm{e}}

\newcommand*\ii{\mathrm{i}}
\newcommand*\re{\mathrm{Re}}
\newcommand*\im{\mathrm{Im}}
\newcommand*\id{\mathrm{id}}
\newcommand*\rang{\mathrm{rang}}
\newcommand*\grad{\mathrm{grad}}
\newcommand*\dive{\mathrm{div}}
\newcommand*\sym{\mathrm{Sym}}
\newcommand*\spur{\mathrm{Spur}}
\newcommand*\isom{\mathrm{Isom}}
\newcommand*\vol{\mathrm{vol\,}}
\newcommand*\supp{\mathrm{supp}}
\newcommand*\inj{\mathrm{inj}}
\newcommand*\rank{\mathrm{rank}}
\newcommand*\qrank{\Q\mbox{-}\mathrm{rank}}
\newcommand*\rrank{\R\mbox{-}\mathrm{rank}}
\newcommand*\dom{\mathrm{dom}}
\newcommand*\tr{\mathrm{tr}}
\newcommand*\spa{\mathrm{span}}
\newcommand*\diam{\mathrm{diam}}

\newcommand*\ric{\mathrm{Ric}}

\newcommand*\con{\mathrm{con}}
\newcommand*\dis{\mathrm{dis}}

\newcommand\PX{{\cal P}(X)}
\newcommand\T{{\cal T}}
\newcommand\B{{\cal B}}



\newcommand\scp{\langle\cdot,\cdot\rangle}     %% Metric

%%%%%%%%%%%%Tilde%%%%%%%
\newcommand\tx{\tilde{x}}
\newcommand\ty{\tilde{y}}
\newcommand\tu{\tilde{u}}
\newcommand\tk{\tilde{k}}
\newcommand\td{\tilde{d}}
\newcommand\tD{\tilde{D}}
\newcommand\tX{\tilde{X}}
\newcommand\tY{\tilde{Y}}
\newcommand\tZ{\tilde{Z}}


%%%%%%Lie-Gruppen%%%%%%%%%
\newcommand\ad{\mathrm{ad}}
\newcommand\Ad{\mathrm{Ad}}
\newcommand{\kak}{K\exp\overline{\lier{a}^+}K}              %%%%Cartan-Zerlegung
\newcommand*\Rang{\mathrm{Rang}}
\newcommand*\glnr{\mathrm{\it GL}(n,\R)}
\newcommand*\glnc{\mathrm{\it GL}(n,\C)}
\newcommand*\slnr{\mathrm{\it SL}(n,\R)}
\newcommand*\on{\mathrm{\it O}(n)}
\newcommand*\son{\mathrm{\it SO}(n)}
\newcommand*\SLzr{\mathrm{\it SL}(2,\R)}
\newcommand*\SOzr{\mathrm{\it SO}(2,\R)}

%%%%%%%%%%%%%Algebraische Gruppen
\newcommand\bG{{\bf G}}
\newcommand\bT{{\bf T}}
\newcommand\bP{{\bf P}}
\newcommand\bN{{\bf N}}
\newcommand\bL{{\bf L}}
\newcommand\bS{{\bf S}}
\newcommand\bM{{\bf M}}

\newcommand\Mor{\mathrm{Mor}}



%%%%%%Geometry%%%%%%%%%%%
\newcommand{\Si}{\mathcal{S}}


%%%%%%Ableitungsoperatoren%%%%%%%%%%%
\newcommand*\pddt{\frac{\partial}{\partial t}}
\newcommand*\pddx{\frac{\partial}{\partial x}}
\newcommand*\pddxio{\frac{\partial}{\partial x^i}}
\newcommand*\pddxjo{\frac{\partial}{\partial x^j}}
\newcommand*\pddxko{\frac{\partial}{\partial x^k}}
\newcommand*\pddxlo{\frac{\partial}{\partial x^l}}

\newcommand*\pddy{\frac{\partial}{\partial y}}
\newcommand*\pddyio{\frac{\partial}{\partial y^i}}
\newcommand*\pddyjo{\frac{\partial}{\partial y^j}}
\newcommand*\pddyko{\frac{\partial}{\partial y^k}}
\newcommand*\pddylo{\frac{\partial}{\partial y^l}}

\newcommand*\pddyq{\frac{\partial^2}{\partial y^2}}
\newcommand*\pddyj{\frac{\partial}{\partial y_j}}
\newcommand*\pddyjq{\frac{\partial^2}{\partial y_j^2}}
\newcommand*\pddxiq{\frac{\partial^2}{\partial x_i^2}}
\newcommand*\pddxi{\frac{\partial}{\partial x_i}}
\newcommand*\ddt{\frac{d}{dt}}

\newcommand*\dx{\,dvol(x)}
\newcommand*\dy{\,dvol(y)}
\newcommand*\dty{\,dvol(\ty)}

\newcommand*\DMp{\Delta_{M,p}}                 %%%%Laplace-Operatoren
\newcommand*\DMq{\Delta_{M,q}}
\newcommand*\DM{\Delta_M}
\newcommand*\DX{\Delta_X}
\newcommand*\DXp{\Delta_{X,p}}
\newcommand*\DXq{\Delta_{X,q}}
\newcommand*\DAx{\Delta_{Ax_0}}
\newcommand*\Rad{\mathrm{Rad}}
\newcommand*\DXps{\Delta^{\#}_{X,p}}

\newcommand*\eDXps{\e^{-t(\Delta^{\#}_{X,p}-c)}} %%%%%% Semigroups
\newcommand*\LpsX{L^p_{\#}(X)}


%%%%%%%%%%%%%%%Komplexe Analysis
\newcommand*\Res{\mathrm{Res}}

%%%%%%%%%%%%%%%Definitionsmenge
\newcommand*\D{{\cal D}}







\author{Dr. Moritz Gruber\\ DHBW Karlsruhe}
\title{\"Ubungsaufgaben 2\\ 
	Mengenlehre und Abbildungen
}
\date{}

%%%%%%%%%
\begin{document}
%%%%%%%%%
\maketitle


%%%
\section{Gleichheit}
%%%
Gilt f\"ur die Mengen $A=B$?
\begin{itemize}
	\item[(a)] $A = \{2,4,6,8\}$, $B = \{2,8,6,2,4\}$.
	\item[(b)] $A = \{ x \in \N ~|~ x^2 = 4\}$, $B = \{ x\in \Z ~|~ x^2 = 4\}$.
\end{itemize}

%%%
\section{Elemente}
%%%
Geben Sie alle Elemente der folgenden Menge an:
$$
	A = \{ x \in \N ~|~ x \text{~gerade und~} x < 17\}.
$$

%%%
\section{Potenzmenge *}
%%%

\begin{itemize}
	\item[(a)]Sei $M = \{0,1,2,3\}.$ Bestimmen Sie ${\cal P}(M)$.
	\item[(b)] Was ist ${\cal P}(\emptyset)$?
	\item[(c)] Seien $A, B$ zwei Mengen. Zeigen Sie:
			$$
				{\cal P}(A) \cap {\cal P}(B) = {\cal P}(A\cap B).
			$$
	\item[(d)] Finden Sie zwei Mengen $A,B$, so dass
			$$
				{\cal P}(A) \cup {\cal P}(B) \neq {\cal P}(A\cup B).
			$$
\end{itemize}

%%%
\section{Mengenoperationen}
%%%

Seien $A = \{1,2,3,4,5 \}$, $B=\{1,3,5,7\}$, $C= \{ 1,2,4,7,9,11\}$. Bestimmen Sie:
\begin{itemize}
	\item[(a)] $A\cap B \cap C$
	\item[(b)] $A\cup( B\cap C)$
	\item[(c)] $(A\cup B) \cap C$.
	\item[(d)] $(A\cap \emptyset) \cup (B \cup \emptyset)$.
\end{itemize}

%%%

%%%
\section{De Morgan'sche Regel}
Beweisen Sie die zweite de Morgan'sche Regel:
$$
	(A \cap B)^{\mathsf{c}} = A^{\mathsf{c}} \cup B^{\mathsf{c}}.
$$
%%%

\section{Abbildungen}
Es seien die Abbildungen $f:\R \to \R^3, x \mapsto (-x, 0, 2x)$ und $g: \R^3 \to \R, (x,y,z) \mapsto x+y+z$ gegeben. 
\begin{itemize}
\item[(a)] Sind $f$ und $g$ injektiv? Begr\"unden Sie Ihre Antwort.
\item[(b)] Sind $f$ und $g$ surjektiv? Begr\"unden Sie Ihre Antwort.
\item[(c)] Bestimmen Sie $g \circ f$.
\end{itemize}


%%%
\section{Verkettung von Abbildungen (optional)}

Der Benzinverbrauch $B$ eines Fahrzeuges ist abh\"angig von der Geschwindigkeit $v$:
$$ 
	B(v) = 2 + 0.5v + 0.25v^2.
$$
Dabei ist $v$ in Meilen pro Stunde anzugeben und $B$ ist in (US-)Gallonen pro Meile abzulesen. 
Wandeln Sie diese Formel in eine Formel um, bei der die Geschwindigkeit in Kilometer pro Stunde angegeben wird und 
der Verbrauch in Liter pro Kilometer abgelesen werden kann.\\
(Eine Meile entspricht 1.60935 Kilometern, eine Gallone entspricht 3.7853 Litern.)

\end{document}