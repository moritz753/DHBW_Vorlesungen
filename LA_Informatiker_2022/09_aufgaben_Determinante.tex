\documentclass[
				a4paper,
				10pt
			]
			{scrartcl}

\parindent0mm

\usepackage{amsfonts}
\usepackage{amsmath}
\usepackage{amssymb}
\usepackage{amsthm}
\usepackage[ngerman]{babel}
\usepackage{graphicx}
\usepackage{xcolor}

\usepackage[
			pdftex,
			colorlinks,
			breaklinks,
			linkcolor=blue,
			citecolor=blue,
			filecolor=black,
			menucolor=black,
			urlcolor=black,
			pdfauthor={Andreas Weber},
			pdftitle={Aufgaben zu Analysis und Lineare Algebra},
			plainpages=false,
			pdfpagelabels,
			bookmarksnumbered=true
		   ]{hyperref}


%--------------------------------------%
% Math ------------------------------%
%--------------------------------------%

% Mengen (Zahlen)
\newcommand{\N}{\mathbb{N}}
\newcommand{\Q}{\mathbb{Q}}
\newcommand{\R}{\mathbb{R}}
\newcommand{\Z}{\mathbb{Z}}
\newcommand{\C}{\mathbb{C}}

% Mengen (allgemein)
\newcommand{\K}{\mathbb{K}}
\newcommand\PX{{\cal P}(X)}

% Zahlentheorie
\newcommand{\ggT}{\mathrm{ggT}}


% Ableitungen
\newcommand{\ddx}{\frac{d}{dx}}
\newcommand{\pddx}{\frac{\partial}{\partial x}}
\newcommand{\pddy}{\frac{\partial}{\partial y}}
\newcommand{\grad}{\text{grad}}

%--------------------------------------%
% Layout Colors ------------------%
%--------------------------------------%
\newcommand*{\highlightDef}[1]{{\color{lightBlue}#1}}
\newcommand*{\highlight}[1]{{\color{lightBlue}#1}}
% Color Definitions
\definecolor{dhbwRed}{RGB}{226,0,26} 
\definecolor{dhbwGray}{RGB}{61,77,77}
\definecolor{lightBlue}{RGB}{28,134,230}

%
\addtokomafont{section}{\color{dhbwGray}}
\addtokomafont{subsection}{\color{dhbwGray}}


%-------------------------------------------------------------------
\begin{document}

\vspace*{-20mm}
{
	%\usekomafont{title}
	\color{dhbwGray}
	Dr. Moritz Gruber	\hfill Version \today\\
	DHBW Karlsruhe\\
}

\vspace{10mm}
\begin{center}
	{
		\usekomafont{title}
		\color{lightBlue}
		{ \LARGE 	Übungsaufgaben 9}\\[3mm]
		{\Large Determinante}
	}
\end{center}

\vspace{5mm}

%-------------------------------------------------------------------


%-------------------------
\section{Invertierbar?}
%%%

Prüfen Sie mit Hilfe der Determinanten, ob die folgenden Matrizen invertierbar sind.

$$
A=\begin{pmatrix} 1 & 1 & 1 \\ 1 & 1 &2 \\ 0&0&1\end{pmatrix}, \quad 
B=\begin{pmatrix} 1 & 2 & 3 \\ 1 & 0 &2 \\ 0&0&1\end{pmatrix} \ \text{ und } \
C=\begin{pmatrix} 0 & 1 & 1 \\ 1 & 0 &1 \\ 1&1&0\end{pmatrix}
$$


%-------------------------
\section{Determinante}
%%%

Berechnen Sie die Determinante der Matrix
$$
	M=\begin{pmatrix}
		0	& 2	& 4	& -2 & 1	\\
		1	& 0	& 1	& 3	& 0	\\
		1	& 1	& 3	& 2	& 0\\
		0	& 1	& 2	& -1	& -1	\\
		3	& 2	& 7	& 7	& -1	&
	\end{pmatrix}.
$$

%-------------------------
\section{Invertierbarkeit}
%%%

Sei $\alpha \in \R$ und
$$
	A =
	\begin{pmatrix}
		1	&5	&\alpha	\\
		0	&-2	&1	\\
		-1	&1	&3
	\end{pmatrix},
$$

F\"ur welche Werte von $\alpha$ ist $A$ invertierbar?


%-------------------------
\section{Inverse bestimmen mittels Determinanten-Formel}
%%%

Sei 
$$
	D = 
	\begin{pmatrix}
		1	& x	&z\\
		0	&1&y\\		
		0	&0	&1
	\end{pmatrix}
	\in \R^{3\times 3}.
$$
Bestimmen Sie mit Hilfe der komplementären Matrix $\bar D$ die zu $D$ inverse Matrix.


%-------------------------
\section{Determinante einer dünn-besetzten Matrix *}
%%%
Berechnen Sie die Determinante der Matrix
$$
	M=\begin{pmatrix}
		0	& 2	&  0& 1	\\
		1	& 0	&  3	& 0	\\
		1	& 0	&  2	& 0\\
		0	& 1	&  -1	& -1	
	\end{pmatrix}.
$$



\end{document}
