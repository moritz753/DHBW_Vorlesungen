\documentclass{beamer}

\usepackage{beamerthemesplit}

\usepackage{amsfonts}
\usepackage{amsmath}
\usepackage{amssymb}
\usepackage{amsthm}
\usepackage{amscd}

\usepackage{stmaryrd} 					%\lightning
\usepackage{algorithm2e}


\usepackage[ngerman]{babel}

\usepackage[utf8]{inputenc}
\usepackage[T1]{fontenc}
\usepackage{textcomp}


% Color Definitions
\definecolor{dhbwRed}{RGB}{226,0,26} 
\definecolor{dhbwGray}{RGB}{61,77,77}
\definecolor{lightBlue}{RGB}{28,134,230}

% Basic Theme
\usetheme{Malmoe}

% Color Re-Definitions
\usecolortheme[named=lightBlue]{structure}
\setbeamercolor*{alerted  text}{fg=dhbwRed, bg=white}
\setbeamercolor*{subsection in toc}{fg=dhbwGray, bg=white}

%\setbeamercolor*{palette primary}{fg=white,bg=lightBlue}
%\setbeamercolor*{palette secondary}{fg=white,bg=gray}
%\setbeamercolor*{palette tertiary}{fg=white,bg=gray}
%\setbeamercolor*{palette quaternary}{fg=white,bg=dhbwRed}

% no navigation symbols
\setbeamertemplate{navigation symbols}{}

% headline, footline
\setbeamertemplate{footline}{\color{dhbwGray} \hfill\insertframenumber\hspace{5mm}\vspace{2mm}}
\setbeamertemplate{headline}{}

% Title Page
\newcommand*{\makeTitlePage}{
	
	\begin{frame}[plain]
		
		\vfill
		\vfill
		\begin{center}
			{
				\usebeamerfont{title}
				\usebeamercolor[fg]{title}
				\Large
				\inserttitle
			}\\[3mm]
			{	
				\usebeamerfont{subtitle}
				\usebeamercolor[fg]{subtitle}
				\large
				\insertsubtitle
			}
		\end{center}
		%
		\vfill
		\vfill
		\vfill
		\vfill
		%
		\begin{columns}
			\begin{column}{0.5\textwidth}
				\begin{flushleft}
					{
						\usebeamerfont{normal text}
						\color{dhbwGray!80}
						\scriptsize
						Dr. Moritz Gruber\\
						DHBW Karlsruhe\\
						
					}
				\end{flushleft}
			\end{column}
			%
			\begin{column}{0.5\textwidth}
				\begin{flushright}
					\includegraphics[scale=0.06]{../DHBW.png}
				\end{flushright}
			\end{column}
		\end{columns}
		%
		\vspace{1mm}
		\begin{columns}
			\begin{column}{0.5\textwidth}
				\begin{flushleft}
					{
						\usebeamerfont{normal text}
						\color{dhbwGray!80}
						\scriptsize
						Version \today
					}
				\end{flushleft}
			\end{column}
			%
			\begin{column}{0.5\textwidth}
				% nothing (just a placeholder to be in line with the columns above
			\end{column}
		\end{columns}
	\end{frame}

}

% Section Divider Page
\newcommand*{\makeSectionDividerPage}{

	\begin{frame}[plain]
		\begin{center}
			\begin{flushleft}
				{				
					\usebeamercolor[fg]{frametitle}
					{\Large \insertsection} \\[3mm]
					{\large \insertsubsection}
				}
			\end{flushleft}
		\end{center}
        \end{frame}
	
}

% itemize
\setbeamertemplate{itemize items}[circle]
\setbeamertemplate{enumerate item}{(\theenumi)}




%--------------------------------------%
% Math ------------------------------%
%--------------------------------------%

% Mengen (Zahlen)
\newcommand{\N}{\mathbb{N}}
\newcommand{\Q}{\mathbb{Q}}
\newcommand{\R}{\mathbb{R}}
\newcommand{\Z}{\mathbb{Z}}
\newcommand{\C}{\mathbb{C}}

% Mengen (allgemein)
\newcommand{\K}{\mathbb{K}}
\newcommand\PX{{\cal P}(X)}

% Zahlentheorie
\newcommand{\ggT}{\mathrm{ggT}}


% Ableitungen
\newcommand{\ddx}{\frac{d}{dx}}
\newcommand{\pddx}{\frac{\partial}{\partial x}}
\newcommand{\pddy}{\frac{\partial}{\partial y}}
\newcommand{\grad}{\text{grad}}

%--------------------------------------%
% Layout Colors ------------------%
%--------------------------------------%
\newcommand*{\highlightDef}[1]{{\color{lightBlue}#1}}
\newcommand*{\highlight}[1]{{\color{lightBlue}#1}} % after theme for colours

%----------------------------------------------------------------------------------------------------
%--------- Document Title ---------------------------------------------------------------------
\title{Lineare Algebra\\[3mm] 
	\large Determinanten-Aufgabe
}
\author{Dr. Moritz Gruber} 
\institute{DHBW Karlsruhe}
\date{2022}
%%%%%%%%%%%%%%
\begin{document}

\AtBeginSection[]{
	\begin{frame}				
		\usebeamercolor[fg]{frametitle}
		{\Large \insertsection} 
        \end{frame}
}

%
\begin{frame}[plain] 
 \titlepage
\end{frame}
%
\begin{frame}\frametitle{Aufgabe: Determinante}
%
Berechnen Sie die Determinante der Matrix
$$
	M=\begin{pmatrix}
		0	& 2	&  0& 1	\\
		1	& 0	&  3	& 0	\\
		1	& 0	&  2	& 0\\
		0	& 1	&  -1	& -1	
	\end{pmatrix}
$$
mit Hilfe der Laplace-Entwicklung.

%
\end{frame}
%
%
\begin{frame}\frametitle{Lösung (I):}
%	
Wir beginnen mit einer Umformung durch eine Additionsmatrix ($\det(A_{i,j}(\alpha))=1$) und entwickeln dann nach der ersten Spalte:\pause
\begin{align*}
\det(M)&=\det(A_{3,2}(-1))\cdot \det(M)\\
&=\det(A_{3,2}(-1)\cdot M)\\
&=\det(\begin{pmatrix}
		0	& 2	&  0& 1	\\
		1	& 0	&  3	& 0	\\
		0	& 0	&  -1	& 0\\
		0	& 1	&  -1	& -1	
	\end{pmatrix})\\
&=1\cdot (-1)^{1+2}\cdot\det(\begin{pmatrix}
		 2	&  0& 1	\\
		 0	&  -1	& 0\\
		1	&  -1	& -1	
	\end{pmatrix})\\
&=-1\cdot\det(\begin{pmatrix}
		 2	&  0& 1	\\
		 0	&  -1	& 0\\
		1	&  -1	& -1	
	\end{pmatrix})\\
\end{align*}
\end{frame}
%
%
\begin{frame}\frametitle{Lösung (II):}
%	
Wir fahren mit Entwickeln nach der zweiten Zeile:\pause
\begin{align*}
\det(M)&=-1\cdot\det(\begin{pmatrix}
		 2	&  0& 1	\\
		 0	&  -1	& 0\\
		1	&  -1	& -1	
	\end{pmatrix})\\
&=-1\cdot(-1)\cdot(-1)^{2+2}\cdot\det(\begin{pmatrix}
		 2	&   1	\\
		1	& -1	
	\end{pmatrix})\\
&=\det(\begin{pmatrix}
		 2	&   1	\\
		1	& -1	
	\end{pmatrix})\\
\end{align*}
\end{frame}
%
%
\begin{frame}\frametitle{Lösung (III):}
%	
Abschließend nutzen wir die Determinanten-Formel für $(2 \times 2)$-Matrizen:\pause
\begin{align*}
\det(M)&=\det(\begin{pmatrix}
		 2	&   1	\\
		1	& -1	
	\end{pmatrix})\\
&=(2\cdot(-1)) - (1\cdot 1)\\
&=-3
\end{align*}
\end{frame}
\end{document}