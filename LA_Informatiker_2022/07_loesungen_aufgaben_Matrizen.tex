\documentclass[
				a4paper,
				10pt
			]
			{scrartcl}

\parindent0mm

\usepackage{amsfonts}
\usepackage{amsmath}
\usepackage{amssymb}
\usepackage{amsthm}
\usepackage[ngerman]{babel}
\usepackage{graphicx}
\usepackage{xcolor}

\usepackage[
			pdftex,
			colorlinks,
			breaklinks,
			linkcolor=blue,
			citecolor=blue,
			filecolor=black,
			menucolor=black,
			urlcolor=black,
			pdfauthor={Andreas Weber},
			pdftitle={Aufgaben zu Analysis und Lineare Algebra},
			plainpages=false,
			pdfpagelabels,
			bookmarksnumbered=true
		   ]{hyperref}


%--------------------------------------%
% Math ------------------------------%
%--------------------------------------%

% Mengen (Zahlen)
\newcommand{\N}{\mathbb{N}}
\newcommand{\Q}{\mathbb{Q}}
\newcommand{\R}{\mathbb{R}}
\newcommand{\Z}{\mathbb{Z}}
\newcommand{\C}{\mathbb{C}}

% Mengen (allgemein)
\newcommand{\K}{\mathbb{K}}
\newcommand\PX{{\cal P}(X)}

% Zahlentheorie
\newcommand{\ggT}{\mathrm{ggT}}


% Ableitungen
\newcommand{\ddx}{\frac{d}{dx}}
\newcommand{\pddx}{\frac{\partial}{\partial x}}
\newcommand{\pddy}{\frac{\partial}{\partial y}}
\newcommand{\grad}{\text{grad}}

%--------------------------------------%
% Layout Colors ------------------%
%--------------------------------------%
\newcommand*{\highlightDef}[1]{{\color{lightBlue}#1}}
\newcommand*{\highlight}[1]{{\color{lightBlue}#1}}
% Color Definitions
\definecolor{dhbwRed}{RGB}{226,0,26} 
\definecolor{dhbwGray}{RGB}{61,77,77}
\definecolor{lightBlue}{RGB}{28,134,230}

%
\addtokomafont{section}{\color{dhbwGray}}
\addtokomafont{subsection}{\color{dhbwGray}}


%-------------------------------------------------------------------
\begin{document}

\vspace*{-20mm}
{
	%\usekomafont{title}
	\color{dhbwGray}
	Dr. Moritz Gruber	\hfill Version \today\\
	DHBW Karlsruhe\\
}

\vspace{10mm}
\begin{center}
	{
		\usekomafont{title}
		\color{lightBlue}
		{ \LARGE 	L\"osungen \"Ubungsaufgaben 7}\\[3mm]
		{\Large Lineare Abbildungen und Matrizen}
	}
\end{center}

\vspace{5mm}

%-------------------------------------------------------------------



%-------------------------
\section{Der Kern}
%%%
Es seien $V,W$ $K$-Vektorräume und $\Phi: V \to W$ ein Homomorphismus.
\begin{itemize}
\item[a)] Zeigen Sie, dass die Menge $Kern(\Phi):=\Phi^{-1}(\{0_W\})$ ein Untervektorraum von $V$ ist.
\item[b)] Zeigen Sie, dass $\Phi$ genau dann injektiv ist, wenn  $Kern(\Phi)=\{0_V\}$ gilt.
\end{itemize}
Es sei nun $A=\begin{pmatrix} 1&2&3 \\ 1&2&3 \end{pmatrix} \in \R^{2\times 3}$ und $\Phi_A: \R^3 \to \R^2, x \mapsto A \cdot x$.
\begin{itemize}
\item[c)] Bestimmen Sie $Kern(\Phi_A)$.
\end{itemize}


%-------------------------
\subsection*{L\"osung}

\begin{itemize}
\item[a)]
Wir benutzen das Untervektorraumkriterium:
\begin{itemize}
\item[i)] $Kern(\Phi) \ne \emptyset$, denn $0_V \in Kern(\Phi)$.
\item[ii)] Seien $u,v \in Kern(\Phi)$, dann gilt: $\Phi(u+v)=\Phi(u)+\Phi(v)=0_W+0_W=0_W$ und somit auch $u+v \in Kern(\Phi)$.
\item[iii)] Es sei $a \in K$ und $u\in Kern(\Phi)$, dann gilt: $\Phi(a\cdot u)=a\cdot \Phi(u)=a\cdot 0_W=0_W$ und somit auch $a\cdot u \in Kern(\Phi)$.
\end{itemize}
Damit ist $Kern(\Phi) \le V$ ein Untervektorraum.

\item[b)]
\begin{itemize}
\item[$\Rightarrow$:] Wenn $\Phi$ injektiv ist, dann gilt für alle $w\in W: 0\le \#\Phi^{-1}(\{w\}) \le 1$. Da $\Phi$ ein Homomorphismus ist, gilt $\Phi(0_V)=0_W$ und somit folgt $Kern(\Phi)=\Phi^{-1}(\{0_W\})=\{0_V\}$. 
\item[$\Leftarrow$:] Sei nun $Kern(\Phi)=\{0_V\}$. Seien weiter $u,v \in V$ mit $\Phi(u)=\Phi(v)$. Dann gilt:
$$0_W=\Phi(u)-\Phi(v)=\Phi(u-v)$$
und somit $u-v \in Kern(\Phi)$. Da aber $Kern(\Phi)=\{0_V\}$ muss daher $u-v=0_V$ und also auch $u=v$ gelten. Damit ist $\Phi$ injektiv.
\end{itemize}

\item[c)]
$$x \in Kern(\Phi_A) \Leftrightarrow A \cdot x =0 
\Leftrightarrow \begin{pmatrix} 1&2&3 \\ 1&2&3 \end{pmatrix} \cdot \begin{pmatrix} x_1\\x_2\\x_3 \end{pmatrix} = \begin{pmatrix} 0\\0 \end{pmatrix}
\Leftrightarrow   x_1+2\cdot x_2 + 3\cdot x_3=0$$
Um diese Gleichung zu erfüllen haben wir zwei freie Variablen $t$ und $s$ für die $x_1$- bzw. die $x_2$-Koordinate. Für die $x_3$-Koordinate ergibt sich dann
$$
t+2s+3x_3=0 \Leftrightarrow x_3= -\frac{1}{3}t-\frac{2}{3}s
$$
Damit erhalten wir
$$
Kern(\Phi_A)=\left\{t\cdot \begin{pmatrix}1\\0\\-\frac{1}{3}\end{pmatrix}+ s\cdot \begin{pmatrix}0\\1\\-\frac{2}{3}\end{pmatrix}\mid s,t \in \R \right\}
$$

\end{itemize}


\newpage
%-------------------------
\section{Drehung}
%%%
%%%

Sei $\varphi \in [0,2\pi)$ und 
$
	D_\varphi=\begin{pmatrix}
		\cos(\varphi)	&-\sin(\varphi)	\\
		\sin(\varphi)	&\cos(\varphi)
	\end{pmatrix}
	\in \R^{2\times 2}$.
\begin{itemize}
\item[a)] Bestimmen Sie die Bilder $\Phi_{D_\varphi}(v)$ für die folgenden Vekoren:\\
i) $v=e_1$ \quad ii) $v=e_2$ \quad iii) $v=\begin{pmatrix} 2 \\ 1 \end{pmatrix}$
\item[b)] Zeigen Sie, dass für $\varphi, \psi \in [0,2\pi)$ gilt: $D_\varphi \cdot D_\psi = D_{\varphi+\psi}$.
\item[c)] Bestimmen Sie die inverse Matrix $D_\varphi^{-1}$.
\item[d)] Es sei nun $\varphi=\frac{\pi}{6}$. Bestimmen Sie die Abbildungsmatrix Verkettung $(\Phi_{D_{\frac{\pi}{6}}})^{1320}$.
\end{itemize}

%-------------------------
\subsection*{L\"osung}
%%%
\begin{itemize}
\item[a)]
\begin{itemize}
\item[i)] $\Phi_{D_\varphi}(e_1)=\begin{pmatrix}
		\cos(\varphi)	&-\sin(\varphi)	\\
		\sin(\varphi)	&\cos(\varphi)
	\end{pmatrix} \cdot \begin{pmatrix} 1\\0\end{pmatrix}=\begin{pmatrix} \cos(\varphi)\cdot1 - \sin(\varphi) \cdot 0 \\\sin(\varphi)\cdot1  +\cos(\varphi) \cdot0\end{pmatrix} =\begin{pmatrix} \cos(\varphi)\\\sin(\varphi)\end{pmatrix}$
\item[ii)] $\Phi_{D_\varphi}(e_2)=\begin{pmatrix}
		\cos(\varphi)	&-\sin(\varphi)	\\
		\sin(\varphi)	&\cos(\varphi)
	\end{pmatrix} \cdot \begin{pmatrix} 0\\1\end{pmatrix}=\begin{pmatrix} \cos(\varphi)\cdot0 - \sin(\varphi) \cdot 1 \\\sin(\varphi)\cdot0  +\cos(\varphi) \cdot1\end{pmatrix} =\begin{pmatrix} -\sin(\varphi)\\\cos(\varphi)\end{pmatrix}$
\item[iii)] $\Phi_{D_\varphi}(\begin{pmatrix} 2 \\ 1 \end{pmatrix})=\Phi_{D_\varphi}(2e_1+e_2)=2\Phi_{D_\varphi}(e_1)+\Phi_{D_\varphi}(e_2)=2\begin{pmatrix} \cos(\varphi)\\\sin(\varphi)\end{pmatrix}+\begin{pmatrix} -\sin(\varphi)\\\cos(\varphi)\end{pmatrix}\\ =\begin{pmatrix} 2\cos(\varphi)-\sin(\varphi)\\2\sin(\varphi)+\cos(\varphi)\end{pmatrix}$
\end{itemize}
\item[b)]
Es gilt
\begin{align*}
D_\varphi \cdot D_\psi&=\begin{pmatrix}
		\cos(\varphi)	&-\sin(\varphi)	\\
		\sin(\varphi)	&\cos(\varphi)
	\end{pmatrix}\cdot\begin{pmatrix}
		\cos(\psi)	&-\sin(\psi)	\\
		\sin(\psi)	&\cos(\psi)
	\end{pmatrix}\\
	&= \begin{pmatrix}\cos(\varphi)\cos(\psi)-\sin(\varphi)\sin(\psi) &\cos(\varphi)\sin(-\psi)-\sin(\varphi)\cos(\psi) \\\sin(\varphi)\cos(\psi)+\cos(\varphi)\sin(\psi) &\sin(\varphi)\sin(-\psi)+\cos(\varphi)\cos(\psi)\end{pmatrix}
\end{align*}
Wir benötigen nun zwei der Additionstheorem für Sinus und Cosinus:
\begin{align*}
\cos(\alpha+\beta)&=\cos(\alpha)\cos(\beta)-\sin(\alpha)\sin(\beta)\\
\sin(\alpha+\beta)&=\cos(\alpha)\sin(\beta)+\sin(\alpha)\cos(\beta)
\end{align*}
sowie die Eigenschaft $\sin(-\alpha)=-\sin(\alpha)$. Damit folgt die Behauptung.
\item[c)] 
Die inverse Matrix ${D_\varphi}^{-1}$ ist die eindeutig bestimmte Matrix in $\R^{2\times2}$ für die gilt 
$$D_\varphi\cdot {D_\varphi}^{-1}=\begin{pmatrix} 1 & 0 \\ 0 & 1 \end{pmatrix}=D_0$$
%Sei also $A=\begin{pmatrix} a & b \\ c & d \end{pmatrix} \in \R^{2\times2}$ mit 
%$$\begin{pmatrix} 1 & 0 \\ 0 & 1 \end{pmatrix}=D_\varphi\cdot A =\begin{pmatrix}\cos(\varphi)\cdot a -\sin(\varphi)\cdot c & \cos(\varphi)\cdot b -\sin(\varphi)\cdot d \\ \sin(\varphi)\cdot a +\cos(\varphi)\cdot c & \sin(\varphi)\cdot b + \cos(\varphi)\cdot d  \end{pmatrix}$$
%Damit erhalten wir die Gleichungen
%\begin{align*}
%\cos(\varphi)\cdot a -\sin(\varphi)\cdot c =1\\
% \cos(\varphi)\cdot b -\sin(\varphi)\cdot d =0\\
% \sin(\varphi)\cdot a +\cos(\varphi)\cdot c=0\\
%\sin(\varphi)\cdot b + \cos(\varphi)\cdot d=1
%\end{align*}
%Mit den Eigenschaften $\cos(-\alpha)=\cos(\alpha)$, $\sin(-\alpha)=-\sin(\alpha)$ und $\sin(\alpha)^2+\cos(\alpha)^2=1$ finden wir mit
%\begin{align*}
%a&= \cos(-\varphi)\\
%b&=-\sin(-\varphi)\\
%c&=\sin(-\varphi)\\
%d&=\cos(-\varphi)
%\end{align*}
%eine Lösung des Gleichungssystems.\\
Mit Aufgabenteil b) ist die inverse Matrix somit gegeben durch
$$
{D_\varphi}^{-1}=\begin{pmatrix}
		\cos(-\varphi)	&-\sin(-\varphi)	\\
		\sin(-\varphi)	&\cos(-\varphi)
	\end{pmatrix}
$$
\textbf{Bemerkung:} Geometrisch beschreibt die Matrix $D_\varphi$ eine Drehung um den Winkel $\varphi$. Die inverse Matrix ist beschreibt daher eine Drehung um den Winkel $-\varphi$. Mehr dazu in der Vorlesung.
\item[d)]
Mit Aufgabenteil b) gilt $({D_\frac{\pi}{6}})^{12}={D_{2\pi}}=D_0 =\begin{pmatrix} 1 & 0 \\ 0 & 1\end{pmatrix}$.\\
Da $1320=110 \cdot 12$ folgt 
$$
({D_\frac{\pi}{6}})^{1320}=\left(({D_\frac{\pi}{6}})^{12}\right)^{110}=\begin{pmatrix} 1 & 0 \\ 0 & 1\end{pmatrix}^{110}=\begin{pmatrix} 1 & 0 \\ 0 & 1\end{pmatrix}.
$$
\end{itemize}

%-------------------------
 \newpage
\section{Matrizen}
%%%

Seien
$$
	A
	=
	\begin{pmatrix}
		1	&0	\\
		1	&1	
	\end{pmatrix}
	\quad
	\text{und}
	\quad
	B
	=
	\begin{pmatrix}
		1	&2	\\
		3	&4	
	\end{pmatrix}.
$$

\begin{itemize}
	\item[(a)] Bestimmen Sie die zu $A$ inverse Matrix $A^{-1}$. 
	\item[(b)] Bestimmen Sie die Matrix $X\in \R^{2\times 2}$, f\"ur die gilt:
		$$
			X\cdot A = B.
		$$
\end{itemize}

%-------------------------
\subsection*{L\"osung}
%%%

\begin{itemize}
	\item[(a)] 
		Gesucht ist 
		$$
			A^{-1} =
			\begin{pmatrix}
				a	&b	\\
				c	&d
			\end{pmatrix}
		$$
		mit
		$$
			A\cdot A^{-1} = E.
		$$
		Das hei{\ss}t
		$$
			\begin{pmatrix}
				1	&0	\\
				1	&1	
			\end{pmatrix}
			\cdot
			\begin{pmatrix}
				a	&b	\\
				c	&d
			\end{pmatrix}
			=
			\begin{pmatrix}
				a	&b	\\
				a+c	&b+d
			\end{pmatrix}	
			=
			\begin{pmatrix}
				1	&0	\\
				0	&1
			\end{pmatrix}.	
		$$
		Es folgt:
		$$
			a=1,\quad b=0,\quad c=-1,\quad d=1
		$$
		bzw.
		$$
			A^{-1} =
			\begin{pmatrix}
				1	&0	\\
				-1	&1
			\end{pmatrix}.
		$$
	\item[(b)]
		Es gilt
		$$
			X 
			= 
			B\cdot A^{-1}
			=
			\begin{pmatrix}
				1	&2	\\
				3	&4	
			\end{pmatrix}
			\cdot
			\begin{pmatrix}
				1	&0	\\
				-1	&1
			\end{pmatrix}
			=
			\begin{pmatrix}
				-1	&2	\\
				-1	&4
			\end{pmatrix}.									 
		$$
\end{itemize}

%-------------------------
\newpage
\section{Spezielle Matrizen}
%%%

Zeigen Sie, dass die Matrizen $A\in \R^{2\times 2}$ invertierbar sind und bestimmen Sie die zugeh\"orige inverse Matrix $A^{-1}$.
Berechnen Sie ferner
		$$
			A
			\cdot
			\begin{pmatrix}
				a & b & c\\
				d & e  & f
			\end{pmatrix}.
		$$
\begin{itemize}
	\item[(a)] Additionsmatrix:	
		Sei $\alpha \in \R$. 
		$$
			A
			=
			\begin{pmatrix}
				1 & \alpha\\
				0 & 1
			\end{pmatrix}.
		$$

	\item[(b)] Vertauschungsmatrix:
		$$
			A
			=
			\begin{pmatrix}
				0 & 1\\
				1 & 0
			\end{pmatrix}.
		$$

	\item[(c)] Multiplikation einer Zeile mit einer Konstante:
		Seien $\alpha, \beta \in \R\backslash\{0\}$. 
		$$
			A
			=
			\begin{pmatrix}
				\alpha 	& 0\\
				0 		& \beta
			\end{pmatrix}.
		$$
\end{itemize}

%-------------------------
\subsection*{L\"osung}
%%%

\begin{itemize}
	\item[(a)] 
		$$
			A^{-1}
			=
			\begin{pmatrix}
				1 & -\alpha\\
				0 & 1
			\end{pmatrix}.
		$$

		$$
			\begin{pmatrix}
				1 & \alpha\\
				0 & 1
			\end{pmatrix}
			\cdot
			\begin{pmatrix}
				a & b & c\\
				d & e  & f
			\end{pmatrix}
			=
			\begin{pmatrix}
				a +\alpha\cdot d	& b + \alpha\cdot e	& c + \alpha\cdot f\\
				d 				& e  				& f
			\end{pmatrix}.			
		$$
	\item[(b)] 
		$$
			A^{-1}
			=
			A
			=
			\begin{pmatrix}
				0 & 1\\
				1 & 0
			\end{pmatrix}.
		$$
		
		$$
			\begin{pmatrix}
				0 & 1\\
				1 & 0
			\end{pmatrix}
			\cdot
			\begin{pmatrix}
				a & b & c\\
				d & e  & f
			\end{pmatrix}
			=
			\begin{pmatrix}
				d & e & f\\
				a & b  & c
			\end{pmatrix}.		
		$$
	\item[(c)] 
		$$
			A^{-1}
			=
			\begin{pmatrix}
				1/\alpha 	& 0\\
				0 		& 1/\beta
			\end{pmatrix}.
		$$
		
		$$
			\begin{pmatrix}
				\alpha 	& 0\\
				0 		& \beta
			\end{pmatrix}
			\cdot
			\begin{pmatrix}
				a & b & c\\
				d & e  & f
			\end{pmatrix}
			=
			\begin{pmatrix}
				\alpha \cdot a 	& \alpha \cdot b & \alpha \cdot c\\
				\beta \cdot d	& \beta \cdot e	 & \beta \cdot f
			\end{pmatrix}.		
		$$
\end{itemize}

%%%
\section{Abbildungsmatrix}
%%%
Es sei $\Psi: \R^3 \to \R^3$ eine lineare Abbildung mit
$$
\Psi(\begin{pmatrix}1\\1\\1 \end{pmatrix}) = \begin{pmatrix}2 \\ 1 \\ 2 \end{pmatrix}, \quad \Psi(\begin{pmatrix} 2 \\ 0\\ 2 \end{pmatrix}) = \begin{pmatrix}0\\2\\4 \end{pmatrix} \quad \text{ und } \quad \Psi(\begin{pmatrix} 0\\1\\1\end{pmatrix} )= \begin{pmatrix} 2 \\ 0 \\ 2\end{pmatrix}.
$$
Bestimmen Sie eine Matrix $A \in \R^{3\times 3}$, sodass $\Psi=\Phi_A$ gilt. 

%-------------------------
\subsection*{L\"osung}
%%%
Um die Abbildungsmatrix $A$ zu bestimmen, benötigen wir die Bilder von $e_1,e_2$ und $e_3$ unter $\Psi$:
\begin{itemize}
\item $\Psi(e_1)=\Psi(\begin{pmatrix} 1 \\ 0\\0\end{pmatrix}=\Psi(\begin{pmatrix}1\\1\\1 \end{pmatrix}-\begin{pmatrix} 0\\1\\1\end{pmatrix})=\Psi(\begin{pmatrix}1\\1\\1 \end{pmatrix})-\Psi(\begin{pmatrix} 0\\1\\1\end{pmatrix})=\begin{pmatrix}2 \\ 1 \\ 2 \end{pmatrix}-\begin{pmatrix} 2 \\ 0 \\ 2\end{pmatrix}\\=\begin{pmatrix} 0 \\ 1 \\ 0\end{pmatrix}$
\item $\Psi(e_2)=\Psi(\begin{pmatrix} 0 \\ 1\\0\end{pmatrix}=\Psi(\begin{pmatrix}1\\1\\1 \end{pmatrix}-\frac{1}{2}\begin{pmatrix} 2\\0\\2\end{pmatrix})=\Psi(\begin{pmatrix}1\\1\\1 \end{pmatrix})-\frac{1}{2}\Psi(\begin{pmatrix} 2\\0\\2\end{pmatrix})\\=\begin{pmatrix}2 \\ 1 \\ 2 \end{pmatrix}-\frac{1}{2} \begin{pmatrix}0\\2\\4 \end{pmatrix}= \begin{pmatrix}2\\0\\0\end{pmatrix}$
\item $\Psi(e_3)=\Psi(\begin{pmatrix} 0 \\ 0\\1\end{pmatrix}=\Psi(\begin{pmatrix}0\\1\\1 \end{pmatrix}-\begin{pmatrix} 0\\1\\0\end{pmatrix})=\Psi(\begin{pmatrix}0\\1\\1 \end{pmatrix})-\Psi(\begin{pmatrix} 0\\1\\0\end{pmatrix})= \begin{pmatrix} 2 \\ 0 \\ 2\end{pmatrix}- \begin{pmatrix}2\\0\\0\end{pmatrix}\\= \begin{pmatrix}0\\0\\2\end{pmatrix}$
\end{itemize}

Damit erhalten wie die Abbildungsmatrix $A=\begin{pmatrix} \Psi(e_1)\mid \Psi(e_2) \mid \Psi(e_3)\end{pmatrix}=\begin{pmatrix}0&2&0\\1&0&0\\0&0&2 \end{pmatrix}$.


\end{document}
