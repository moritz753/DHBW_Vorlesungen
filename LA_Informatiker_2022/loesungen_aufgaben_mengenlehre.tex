\documentclass[
				a4paper,
				10pt
			]
			{scrartcl}

\parindent0mm

\usepackage{amsfonts}
\usepackage{amsmath}
\usepackage{amssymb}
\usepackage{amsthm}
\usepackage[ngerman]{babel}

\usepackage[
			pdftex,
			colorlinks,
			breaklinks,
			linkcolor=blue,
			citecolor=blue,
			filecolor=black,
			menucolor=black,
			urlcolor=black,
			pdfauthor={Andreas Weber},
			pdftitle={Aufgaben zur Logik und Algebra},
			plainpages=false,
			pdfpagelabels,
			bookmarksnumbered=true
		   ]{hyperref}

 %%%%%%%%%%%%%%%Schriften%%%%%%%%%%%%%
\DeclareMathAlphabet{\lier}{U}{eur}{m}{n}  %% Gothisch/Fraktur - Roman



\newcommand{\M}{\mathbb{M}}
\newcommand{\E}{\mathbb{E}}
\newcommand{\Hy}{\mathbb{H}}
\newcommand{\N}{\mathbb{N}}
\newcommand{\Q}{\mathbb{Q}}
\newcommand{\R}{\mathbb{R}}
\newcommand{\Z}{\mathbb{Z}}
\newcommand{\C}{\mathbb{C}}
\newcommand{\K}{\mathbb{K}}
\newcommand*\ee{\mathrm{e}}
\newcommand*\e{\mathrm{e}}

\newcommand*\ii{\mathrm{i}}
\newcommand*\re{\mathrm{Re}}
\newcommand*\im{\mathrm{Im}}
\newcommand*\id{\mathrm{id}}
\newcommand*\rang{\mathrm{rang}}
\newcommand*\grad{\mathrm{grad}}
\newcommand*\dive{\mathrm{div}}
\newcommand*\sym{\mathrm{Sym}}
\newcommand*\spur{\mathrm{Spur}}
\newcommand*\isom{\mathrm{Isom}}
\newcommand*\vol{\mathrm{vol\,}}
\newcommand*\supp{\mathrm{supp}}
\newcommand*\inj{\mathrm{inj}}
\newcommand*\rank{\mathrm{rank}}
\newcommand*\qrank{\Q\mbox{-}\mathrm{rank}}
\newcommand*\rrank{\R\mbox{-}\mathrm{rank}}
\newcommand*\dom{\mathrm{dom}}
\newcommand*\tr{\mathrm{tr}}
\newcommand*\spa{\mathrm{span}}
\newcommand*\diam{\mathrm{diam}}

\newcommand*\ric{\mathrm{Ric}}

\newcommand*\con{\mathrm{con}}
\newcommand*\dis{\mathrm{dis}}

\newcommand\PX{{\cal P}(X)}
\newcommand\T{{\cal T}}
\newcommand\B{{\cal B}}



\newcommand\scp{\langle\cdot,\cdot\rangle}     %% Metric

%%%%%%%%%%%%Tilde%%%%%%%
\newcommand\tx{\tilde{x}}
\newcommand\ty{\tilde{y}}
\newcommand\tu{\tilde{u}}
\newcommand\tk{\tilde{k}}
\newcommand\td{\tilde{d}}
\newcommand\tD{\tilde{D}}
\newcommand\tX{\tilde{X}}
\newcommand\tY{\tilde{Y}}
\newcommand\tZ{\tilde{Z}}


%%%%%%Lie-Gruppen%%%%%%%%%
\newcommand\ad{\mathrm{ad}}
\newcommand\Ad{\mathrm{Ad}}
\newcommand{\kak}{K\exp\overline{\lier{a}^+}K}              %%%%Cartan-Zerlegung
\newcommand*\Rang{\mathrm{Rang}}
\newcommand*\glnr{\mathrm{\it GL}(n,\R)}
\newcommand*\glnc{\mathrm{\it GL}(n,\C)}
\newcommand*\slnr{\mathrm{\it SL}(n,\R)}
\newcommand*\on{\mathrm{\it O}(n)}
\newcommand*\son{\mathrm{\it SO}(n)}
\newcommand*\SLzr{\mathrm{\it SL}(2,\R)}
\newcommand*\SOzr{\mathrm{\it SO}(2,\R)}

%%%%%%%%%%%%%Algebraische Gruppen
\newcommand\bG{{\bf G}}
\newcommand\bT{{\bf T}}
\newcommand\bP{{\bf P}}
\newcommand\bN{{\bf N}}
\newcommand\bL{{\bf L}}
\newcommand\bS{{\bf S}}
\newcommand\bM{{\bf M}}

\newcommand\Mor{\mathrm{Mor}}



%%%%%%Geometry%%%%%%%%%%%
\newcommand{\Si}{\mathcal{S}}


%%%%%%Ableitungsoperatoren%%%%%%%%%%%
\newcommand*\pddt{\frac{\partial}{\partial t}}
\newcommand*\pddx{\frac{\partial}{\partial x}}
\newcommand*\pddxio{\frac{\partial}{\partial x^i}}
\newcommand*\pddxjo{\frac{\partial}{\partial x^j}}
\newcommand*\pddxko{\frac{\partial}{\partial x^k}}
\newcommand*\pddxlo{\frac{\partial}{\partial x^l}}

\newcommand*\pddy{\frac{\partial}{\partial y}}
\newcommand*\pddyio{\frac{\partial}{\partial y^i}}
\newcommand*\pddyjo{\frac{\partial}{\partial y^j}}
\newcommand*\pddyko{\frac{\partial}{\partial y^k}}
\newcommand*\pddylo{\frac{\partial}{\partial y^l}}

\newcommand*\pddyq{\frac{\partial^2}{\partial y^2}}
\newcommand*\pddyj{\frac{\partial}{\partial y_j}}
\newcommand*\pddyjq{\frac{\partial^2}{\partial y_j^2}}
\newcommand*\pddxiq{\frac{\partial^2}{\partial x_i^2}}
\newcommand*\pddxi{\frac{\partial}{\partial x_i}}
\newcommand*\ddt{\frac{d}{dt}}

\newcommand*\dx{\,dvol(x)}
\newcommand*\dy{\,dvol(y)}
\newcommand*\dty{\,dvol(\ty)}

\newcommand*\DMp{\Delta_{M,p}}                 %%%%Laplace-Operatoren
\newcommand*\DMq{\Delta_{M,q}}
\newcommand*\DM{\Delta_M}
\newcommand*\DX{\Delta_X}
\newcommand*\DXp{\Delta_{X,p}}
\newcommand*\DXq{\Delta_{X,q}}
\newcommand*\DAx{\Delta_{Ax_0}}
\newcommand*\Rad{\mathrm{Rad}}
\newcommand*\DXps{\Delta^{\#}_{X,p}}

\newcommand*\eDXps{\e^{-t(\Delta^{\#}_{X,p}-c)}} %%%%%% Semigroups
\newcommand*\LpsX{L^p_{\#}(X)}


%%%%%%%%%%%%%%%Komplexe Analysis
\newcommand*\Res{\mathrm{Res}}

%%%%%%%%%%%%%%%Definitionsmenge
\newcommand*\D{{\cal D}}







\author{Dr. Moritz Gruber\\ DHBW Karlsruhe}
\title{L\"osungen \"Ubungsaufgaben 2\\ 
	Mengenlehre
}
\date{}

%%%%%%%%%
\begin{document}
%%%%%%%%%
\maketitle


%%%
\section{Gleichheit}
%%%
Gilt f\"ur die Mengen $A=B$?
\begin{itemize}
	\item[(a)] $A = \{2,4,6,8\}$, $B = \{2,8,6,2,4\}$.
	\item[(b)] $A = \{ x \in \N ~|~ x^2 = 4\}$, $B = \{ x\in \Z ~|~ x^2 = 4\}$.
\end{itemize}

%%%
\paragraph{L\"osung:}
%%%
\begin{itemize}
	\item[(a)] Ja, da $A\subseteq B$ und $B\subseteq A$.
	\item[(b)] Nein: $-2 \in B$ aber $-2 \notin A$.
\end{itemize}


%%%
\section{Elemente}
%%%
Geben Sie alle Elemente der folgenden Menge an:
$$
	A = \{ x \in \N ~|~ x \text{~gerade und~} x < 17\}.
$$

%%%
\paragraph{L\"osung:}
%%%
$A = \{2,4,6,8,10,12,14,16 \}$.


%%%
\newpage
\section{Potenzmenge}
%%%

\begin{itemize}
	\item[(a)]Sei $M = \{0,1,2,3\}.$ Bestimmen Sie ${\cal P}(M)$.
	\item[(b)] Was ist ${\cal P}(\emptyset)$?
	\item[(c)] Seien $A, B$ zwei Mengen. Zeigen Sie:
			$$
				{\cal P}(A) \cap {\cal P}(B) = {\cal P}(A\cap B).
			$$
	\item[(d)] Finden Sie zwei Mengen $A,B$, so dass
			$$
				{\cal P}(A) \cup {\cal P}(B) \neq {\cal P}(A\cup B).
			$$
\end{itemize}
%%%
\paragraph{L\"osung:}
%%%
\begin{itemize}
	\item[(a)] ${\cal P}(M) = $\\
			$\{ \emptyset,\\ 
				\{0\}, \{1\}, \{2\}, \{3\},\\ 
				\{0,1\}, \{0,2\}, \{0,3\}, \{1,2\}, \{1,3\}, \{2,3\},\\ 
				\{0,1,2\}, \{0,1,3\}, \{0,2,3\}, \{1,2,3\},\\ 
				\{0,1,2,3\}  \}.$
	\item[(b)] ${\cal P}(\emptyset) = \{ \emptyset \}$.
	\item[(c)] 1. Schritt:\\ 
			Wir zeigen ${\cal P}(A) \cap {\cal P}(B) \subseteq  {\cal P}(A\cap B)$:\\
			Sei $X \in {\cal P}(A) \cap {\cal P}(B).$ Dann folgt:
			$$
				X \subseteq A\quad \land \quad X \subseteq B.
			$$
			Das hei{\ss}t
			$$
				X \subseteq A \cap B
			$$
			und somit
			$$
				X \in  {\cal P}(A\cap B).
			$$
			
			2. Schritt:\\
			Wir zeigen $ {\cal P}(A\cap B) \subseteq  {\cal P}(A) \cap {\cal P}(B) $:\\
			Sei $X \in {\cal P}(A\cap B)$. Dann folgt:
			$$
				X \subseteq A\cap B.
			$$
			Das hei{\ss}t
			$$
				X \subseteq A \quad \land \quad X \subseteq B
			$$
			und somit
			$$
				X \in {\cal P}(A) \cap {\cal P}(B).
			$$
			
	\item[(d)] $A=\{1\}$ und $B=\{2\}$: 
			$$
				{\cal P}(A) = \{\emptyset, \{1\} \}, \quad {\cal P}(B) = \{\emptyset, \{2\} \}.
			$$
			$$
				{\cal P}(A\cup B) = \{\emptyset, \{1\}, \{2\}, \{1,2\} \}.
			$$
\end{itemize}

%%%
\section{Mengenoperationen}
%%%

Seien $A = \{1,2,3,4,5 \}$, $B=\{1,3,5,7\}$, $C= \{ 1,2,4,7,9,11\}$. Bestimmen Sie:
\begin{itemize}
	\item[(a)] $A\cap B \cap C$
	\item[(b)] $A\cup( B\cap C)$
	\item[(c)] $(A\cup B) \cap C$.
	\item[(d)] $(A\cap \emptyset) \cup (B \cup \emptyset)$.
\end{itemize}

%%%
\paragraph{L\"osung:}
%%%
\begin{itemize}
	\item[(a)] $A\cap B \cap C = \{ 1 \}$
	\item[(b)] $A\cup( B\cap C) = \{ 1,2,3,4,5, 7 \}$
	\item[(c)] $(A\cup B) \cap C = \{ 1,2,4,7\}$.
	\item[(d)] $(A\cap \emptyset) \cup (B \cup \emptyset) = \emptyset \cup B = B$.
\end{itemize}


%%%
\newpage
\section{De Morgan'sche Regel}
Beweisen Sie die zweite de Morgan'sche Regel:
$$
	(A \cap B)^{\mathsf{c}} = A^{\mathsf{c}} \cup B^{\mathsf{c}}.
$$

%%%
\paragraph{L\"osung:}
%%%
$A,B$ seien Teilmenge von $M$ und das Komplement beziehe sich auf die Menge $M$.\\

Sei $x \in (A \cap B)^{\mathsf{c}}$. Dann gilt:
$$
	(x \in M) \land (x \notin (A\cap B) ).
$$
Das ist \"aquivalent zu
$$
	(x \in M) \land \neg( (x \in A)\land (x\in B) )
$$
bzw. zu (de Morgan'sche Regel der Aussagenlogik)
$$
	(x \in M) \land ( \neg(x \in A)\lor \neg(x\in B) )
$$
Diese Aussage ist wiederum \"aquivalent zur folgenden (Distributivgesetz der Aussagenlogik)
$$
	( (x \in M) \land ( \neg(x \in A)) \lor ( (x \in M)  \land (\neg(x\in B)) )
$$
bzw
$$
	( (x \in M) \land ( x \notin A)) \lor ( (x \in M)  \land (x\notin B) ).
$$
Dies ist wiederum \"aquivalent zu
$$
	(x \in A^{\mathsf{c}}) \lor (x\in B^{\mathsf{c}})
$$
bzw
$$
	x \in A^{\mathsf{c}}\cup B^{\mathsf{c}}.
$$

Da wir nur \"Aquivalenzumformungen vorgenommen haben, gilt $x \in (A \cap B)^{\mathsf{c}}$ genau dann, wenn $x \in A^{\mathsf{c}}\cup B^{\mathsf{c}}$. Somit folgt die Behauptung.

%%%
\newpage
\section{Abbildungen}
Es seien die Abbildungen $f:\R \to \R^3, x \mapsto (-x, 0, 2x)$ und $g: \R^3 \to \R, (x,y,z) \mapsto x+y+z$ gegeben. 
\begin{itemize}
\item[(a)] Sind $f$ und $g$ injektiv? Begr\"unden Sie Ihre Antwort.
\item[(b)] Sind $f$ und $g$ surjektiv? Begr\"unden Sie Ihre Antwort.
\item[(c)] Bestimmen Sie $g \circ f$.
\end{itemize}


%%%
\paragraph{L\"osung:}
%%%
\begin{itemize}
\item[(a)]
\underline{$f$ ist injektiv:} Seien $x_1, x_2 \in \R$ mit $f(x_1)=f(x_2)$. Dann folgt $$(-x_1,0,2x_1)=(-x_2,0,2x_2) \ \text{ und damit } \ x_1=x_2$$
\underline{$g$ ist nicht injektiv:} Aus $g((x_1,y_1,z_1))=g((x_2,y_2,z_2))$ folgt nicht zwingend $(x_1,y_1,z_1)=(x_2,y_2,z_2)$, denn z.B. $g((1,0,0))=1=g((0,1,0))$.
 
\item[(b)] 
\underline{$f$ ist nicht surjektiv:} Es gilt $Bild(f) \ne \R^3$, da z.B. $(0,1,0) \notin Bild(f)$. Damit ist $f$ nicht surjektiv.\\
\underline{$g$ ist surjektiv:} Sei $a \in \R$, dann gilt: $(a,0,0) \in \R^3$ und $a=g((a,0,0))$. Somit ist $g$ surjektiv.
\item[(c)]
Es ist
$$ g\circ f : \R \to \R, x \mapsto x$$
denn 
$$g(f(x))=g((-x,0,2x))=-x+0+2x=x$$ 
\end{itemize}

%%%
\newpage
\section{Verkettung von Abbildungen (optional)}
%%%

Der Benzinverbrauch $B$ eines Fahrzeuges ist abh\"angig von der Geschwindigkeit $v$:
$$ 
	B(v) = 2 + 0.5v + 0.25v^2.
$$
Dabei ist $v$ in Meilen pro Stunde anzugeben und $B$ ist in (US-)Gallonen pro Meile abzulesen. 
Wandeln Sie diese Formel in eine Formel um, bei der die Geschwindigkeit in Kilometer pro Stunde angegeben wird und 
der Verbrauch in Liter pro Kilometer abgelesen werden kann.\\
(Eine Meile entspricht 1.60935 Kilometern, eine Gallone entspricht 3.7853 Litern.)

%%%
\paragraph{L\"osung:}
%%%

Seien $v_K$ die Geschwindigkeit in $km/h$ und $B_L$ der Benzinverbrauch in $L/km$.
Wegen 
$$
	1 \frac{\text{Meile}}{h} = 1.60935 \frac{km}{h}  
$$
folgt
$$
	v \frac{\text{Meile}}{h} = 1.60935\cdot v \frac{km}{h}  = v_K  \frac{km}{h}  
$$
Die Abbildung
$$
	h(v_K) = \frac{v_K}{1.60935}
$$
bildet somit eine Geschwindigkeit in $km/h$ in Meilen/h ab.\\
Wegen
$$
	1 \frac{\text{Gallone}}{\text{Meile}} = \frac{3.7853}{1.60935}\frac{L}{km}
$$
folgt
$$
 	B_L \frac{\text{Gallone}}{\text{Meile}} =  \frac{1.60935}{3.7853}\cdot B_L \frac{L}{km} = B \frac{L}{km}
$$
Die Abbildung
$$
	g(B) = \frac{3.7853}{1.60935}\cdot B
$$
bildet somit einen Verbrauch in Gallonen/Meile in $L/km$ ab.\\

Es folgt:
\begin{align*}
	B_L(v_K) &= g\circ B \circ h(v_K)	\\ 
			&= g( 2 + 0.5\cdot v_K/1.60935 + 0.25\cdot (v_K/1.60935)^2  )\\
			&= \frac{3.7853}{1.60935}\cdot ( 2 + 0.5\cdot v_K/1.60935 + 0.25\cdot (v_K/1.60935)^2  )	\\
			&\approx 4.7 + 0.73\cdot v_K + 0.23\cdot v_K^2 
\end{align*}
\end{document}